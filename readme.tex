\documentclass{minimal}
\usepackage{forest}
\def\getforestyear#1/#2/#3 v#4 #5\getforestyear{#1}
\edef\forestyear{\expandafter\expandafter\expandafter\getforestyear\csname ver@forest.sty\endcsname\getforestyear}
\begin{document}
\newwrite\readme
\immediate\openout\readme=README.txt
\immediate\write\readme{%
LaTeX package: forest [\csname ver@forest.sty\endcsname]^^J%
^^J%
Copyright (c) \forestyear\space Saso Zivanovic^^J%
\space\space\space\space\space\space\space\space\space\space\space\space\space\space\space\space\space\space\space(Sa\string\v{s}o \string\v{Z}ivanovi\string\'{c})^^J%
saso.zivanovic@guest.arnes.si^^J%
^^J%
^^J%
ABSTRACT^^J%
^^J%
`forest' is a pgf/tikz-based package for drawing linguistic (and other^^J%
kinds of) trees.  Its main features are:^^J%
- a packing algorithm which can produce very compact trees;^^J%
- a user-friendly interface consisting of the familiar bracket^^J%
  encoding of trees plus the key--value interface to option-setting;^^J%
- many tree-formatting options, with control over option values of^^J%
  individual nodes and mechanisms for their manipulation;^^J%
- a powerful mechanism for traversing the tree;^^J%
- the possibility to decorate the tree using the full power of pgf/tikz;^^J%
- an externalization mechanism sensitive to code-changes.^^J%
^^J%
^^J%
LICENSE^^J%
^^J%
This work may be distributed and/or modified under the^^J%
conditions of the LaTeX Project Public License, either version 1.3^^J%
of this license or (at your option) any later version.^^J%
The latest version of this license is in^^J%
^^J%
  http://www.latex-project.org/lppl.txt^^J%
^^J%
and version 1.3 or later is part of all distributions of LaTeX^^J%
version 2005/12/01 or later.^^J%
^^J%
^^J%
For the list of files constituting the work see the main source file^^J%
of the package, `forest.dtx', or the derived `forest.sty'.^^J%
}
\immediate\closeout\readme
\end{document}

%%% Local Variables: 
%%% mode: latex
%%% TeX-master: t
%%% End: 
