\documentclass{article}
\usepackage[margin=1cm,landscape,twocolumn]{geometry}
\usepackage{forest}

\usepackage{titlesec}
\newcommand{\sectionbreak}{\clearpage}

\parindent 0pt

\makeatletter

\def\testmacro#1#2{% #1 = result, #2 = expected
  \def\test@result{#1}%
  \def\test@expected{#2}%
  \ifx\test@result\test@expected
    {#1}%
  \else
    {\textcolor{red}{#1} (expected: #2)}%
  \fi
}

\newcount\test@result@count
\newcount\test@expected@count
\def\testcount#1#2{% #1 = result, #2 = expected
  \test@result@count=#1\relax
  \test@expected@count=#2\relax
  \ifnum\test@result@count=\test@expected@count
    {#1}%
  \else
    {\textcolor{red}{#1} (expected: #2)}%
  \fi
}


\newdimen\test@result@dimen
\newdimen\test@expected@dimen
\def\testdimen#1#2{% #1 = result, #2 = expected
  \test@result@dimen=#1\relax
  \test@expected@dimen=#2\relax
  \ifdim\test@result@dimen=\test@expected@dimen
    {#1}%
  \else
    {\textcolor{red}{#1} (expected: #2)}%
  \fi
}


\begin{document}

\section{Is it a number or dimension?}
\label{sec:test-num-dim}

\def\testnumdim#1#2#3{%
 % #1 = test for a number or dimension (num/dim)
 % #2 = expression to be tested
 % #3 = correct result (true/false)
  is \texttt{\detokenize{"#2"}} a #1?
  \begingroup
  %\tracingall
  \csname forest@is#1\endcsname{#2}{\forest@temptrue}{\forest@tempfalse}% test
  \csname testnumdim@collect@\ifforest@temp true\else false\fi @#1\endcsname
  \ifforest@temp\endgroup\forest@temptrue\else\endgroup\forest@tempfalse\fi
  \csname if#3\endcsname
    % "true" is the correct result
    \csname testnumdim@texeval@#1\endcsname{#2}% store correct num/dim result into \forest@temp
    expecting yes (\forest@temp),
    \ifforest@temp
      \ifx\forest@temp\forest@global@temp
        got it!
      \else
        but got yes (\textcolor{red}{\forest@global@temp}).
      \fi
    \else
      but got \textcolor{red}{no} (\forest@global@temp).
    \fi
  \else
    % "false" is the correct result
    expecting no,
    \ifforest@temp
      but got \textcolor{red}{yes} (\forest@global@temp).
    \else
      got it!
    \fi
  \fi
}
\def\testnumdim@collect@true@num{\xdef\forest@global@temp{\the\forest@isnum@count}}%
\def\testnumdim@collect@false@num{\xdef\forest@global@temp{\the\forest@isnum@count}}%
\def\testnumdim@texeval@num#1{\count0 #1\relax\edef\forest@temp{\the\count0}}%

\def\testnumdim@collect@true@dim{\xdef\forest@global@temp{\the\forest@isdim@dimen}}%
\def\testnumdim@collect@false@dim{\xdef\forest@global@temp{\the\forest@isnum@count.\forest@isdim@leadingzeros\the\forest@isdim@nonintpart\forest@isdim@unit}}%
\def\testnumdim@texeval@dim#1{\dimen0 #1\relax\edef\forest@temp{\the\dimen0}}%

\let\testnumdim@collect@true@baredim\testnumdim@collect@true@dim
\let\testnumdim@collect@false@baredim\testnumdim@collect@false@dim
\def\testnumdim@texeval@baredim#1{\dimen0 #1pt\relax\edef\forest@temp{\the\dimen0}}%
\let\forest@isbaredim\forest@isdim


\testnumdim{num}{42}{true}
\\\newcount\mynum\testnumdim{num}{\the\mynum}{true}
\\\testnumdim{num}{42 }{true}
\\\testnumdim{num}{ 42}{true}
\\\testnumdim{num}{ 42  }{true}
\\\testnumdim{num}{0}{true}
\\\testnumdim{num}{00}{true}
\\\testnumdim{num}{01}{true}
\\\testnumdim{num}{-0}{true}
\\\testnumdim{num}{a}{false}
\\\testnumdim{num}{a42}{false}
\\\testnumdim{num}{42a}{false}
\\\testnumdim{num}{-a}{false}
\\\testnumdim{num}{a42a}{false}
\\\testnumdim{num}{+42a}{false}
\\\testnumdim{num}{+42}{true}
\\\testnumdim{num}{-42}{true}
\\\testnumdim{num}{--42}{true}
\\\testnumdim{num}{- -42}{true}
\\\testnumdim{num}{- -+ +42}{true}
\\\testnumdim{num}{42!}{false}
\\\testnumdim{num}{!42!}{false}
\\\testnumdim{num}{!42}{false}
\\\testnumdim{num}{42!a}{false}
\\\testnumdim{num}{42a!}{false}
\\\testnumdim{num}{042}{true}
\\\testnumdim{num}{-042}{true}
\\\testnumdim{num}{- 0a}{false}
\\\testnumdim{num}{+ 0}{true}
\\\testnumdim{num}{- 0}{true}
\\\testnumdim{num}{0}{true}
\\\testnumdim{num}{2147483647}{true}
\\\testnumdim{num}{-2147483647}{true}
\\\testnumdim{num}{-- - + 002147483647}{true}
%\\\testnumdim{num}{2147483648}{} % number yes, but too big
%\\\testnumdim{num}{-2147483648}{} % number yes, but too big
\\\testnumdim{num}{0\space}{true}
\\\testnumdim{num}{\space0}{true}
\\\testnumdim{num}{\space0}{true}
\\\testnumdim{num}{0\space\space\space}{false}
\\\testnumdim{num}{0 }{true}
\\\testnumdim{num}{0 \space}{false}
\\\testnumdim{num}{\space0}{true}
\\\testnumdim{num}{\space\space0}{true}
\\\testnumdim{num}{\space\space\space0}{true}
\\\testnumdim{num}{\the\count0}{true}
\\\testnumdim{num}{\space0\the\count0}{true}
\\\testnumdim{dim}{4.2pt}{true}
\\\testnumdim{dim}{4.2p}{false}
\\\testnumdim{dim}{4.cm}{true}
\\\testnumdim{dim}{.42mm}{true}
\\\testnumdim{dim}{.42mm  }{true}
\\\testnumdim{dim}{.42mm \space}{false}
\\\testnumdim{dim}{.42mm a}{false}
\\\testnumdim{baredim}{.42}{true}
\\\testnumdim{dim}{cm}{false}
\\\testnumdim{dim}{.cm}{true}
\\\testnumdim{baredim}{.}{true}% pgfmath is inconsistent here: accepts this as a dimension but not as a count
\\\testnumdim{dim}{ 1.  pt}{true}
\\\testnumdim{baredim}{42}{true}
\\\testnumdim{baredim}{42.0}{true}
\\\testnumdim{dim}{42cm}{true}
\\\testnumdim{dim}{42 cm}{true}
\\\testnumdim{dim}{16383.99998pt}{true}
\\\testnumdim{dim}{16383.99998pt}{true}
\\\testnumdim{dim}{-16383.99998pt}{true}
\\\testnumdim{dim}{11PT}{true}
\\\testnumdim{dim}{11PT }{true}
\\\testnumdim{dim}{11ptt}{false}
\\\testnumdim{dim}{11pt }{true}
\\\testnumdim{dim}{21cm+21cm}{false}
\\\testnumdim{dim}{4. 2cm}{false}
\\\testnumdim{dim}{  4.2  cm }{true}

\section{Aggregate functions}
\label{sec:aggregate-functions}

\begin{forest}(stages={
    for root'={
      process keylist=given options,
      TeX={%
        \def\forestdebugtypeouttreenodeinfo{\forestoption{content}}%
        {\tt\forestdebug@typeouttree\forest@cn\temp\temp\\}
      },
      %
      tempcounta/.count=children,
      TeX/.process=Rw{tempcounta}{Number of root's children: \testcount{##1}{3}\\},
      %
      tempcounta/.sum={>O+n{content}}{tree},
      TeX/.process=Rw{tempcounta}{Sum of all nodes: \testcount{##1}{21}\\},
      %
      tempcounta/.average={>O+n{content}}{fake=2,1n},
      TeX/.process=Rw{tempcounta}{Average (rounded) of bottom nodes: \testcount{##1}{5}\\},
      %
      tempcounta/.product={>O+n{content}}{fake={21},current and ancestors},
      TeX/.process=Rw{tempcounta}{Product from root to 5: \testcount{##1}{15}\\},
      %
      tempcounta/.min={>O+n{content}}{tree},
      TeX/.process=Rw{tempcounta}{Minimum in tree: \testcount{##1}{1}\\},
      %
      tempcounta/.max={>O+n{content}}{tree},
      TeX/.process=Rw{tempcounta}{Maximum in tree: \testcount{##1}{6}\\},
      %
      TeX={Again: now interpret content as dimensions in points.\\},
      tempdima/.sum={>O+n{content}}{tree},
      TeX/.process=Rw{tempdima}{Sum of all nodes: \testdimen{##1}{21pt}\\},
      %
      tempdima/.average={>O+n{content}}{fake=2,1n},
      TeX/.process=Rw{tempdima}{Average (rounded) of bottom nodes: \testdimen{##1}{5.5pt}\\},
      %
      tempdima/.product={>O+n{content}}{fake={21},current and ancestors},
      TeX/.process=Rw{tempdima}{Product from root to 5: \testdimen{##1}{15pt}\\},
      %
      tempdima/.min={>O+n{content}}{tree},
      TeX/.process=Rw{tempdima}{Minimum in tree: \testdimen{##1}{1pt}\\},
      %
      tempdima/.max={>O+n{content}}{tree},
      TeX/.process=Rw{tempdima}{Maximum in tree: \testdimen{##1}{6pt}\\},
      %
      TeX={Sums again. Now sum using pgfmath:\\},
      temptoksa/.sum={content}{tree},
      TeX/.process=Rw{temptoksa}{Assign to toks: \testmacro{##1}{21.0}\\},
      tempdima/.sum={content}{tree},
      TeX/.process=Rw{tempdima}{Assign to dimen: \testdimen{##1}{21pt}\\},
      tempcounta/.sum={content}{tree},
      TeX/.process=Rw{tempcounta}{Assign to count: \testcount{##1}{21}\\},
    }
  })
  [1
    [2]
    [3
      [5]
      [6]
    ]
    [4]
  ]
\end{forest}


\end{document}
