% \CheckSum{12819}
% \iffalse meta-comment
% forest.dtx
%% `forest' is a `pgf/tikz'-based package for drawing (linguistic) trees.
%%
%% Copyright (c) 2013 Saso Zivanovic
%%                    (Sa\v{s}o \v{Z}ivanovi\'{c})
%% saso.zivanovic@guest.arnes.si
%%
%% This work may be distributed and/or modified under the
%% conditions of the LaTeX Project Public License, either version 1.3
%% of this license or (at your option) any later version.
%% The latest version of this license is in
%% 
%%   http://www.latex-project.org/lppl.txt
%%   
%% and version 1.3 or later is part of all distributions of LaTeX
%% version 2005/12/01 or later.
%%
%% This work has the LPPL maintenance status `maintained'.
%% 
%% The Current Maintainer of this work is Saso Zivanovic.
%%
%% This work consists of the files forest.dtx and forest.ins
%% and the derived file forest.sty.
%%
%
%<*driver>
\documentclass[a4paper]{ltxdoc}
\usepackage{fullpage}

\usepackage[external]{forest}
%\tikzexternalize
\tikzset{
  external/prefix={forest.for.dir/},
  external/system call={
    pdflatex \tikzexternalcheckshellescape -halt-on-error -interaction=nonstopmode
    -jobname "\image" "\texsource"},
}
%\usepackage[trace]{trace-pgfkeys}

\usepackage[colorlinks=true,linkcolor=blue,citecolor=blue,hyperindex=false]{hyperref}
\usepackage{url}
\usepackage[numbers]{natbib}
\usepackage[multiple]{footmisc}
\usepackage{tipa}
\usepackage{paralist}
\usepackage{printlen}

\makeatletter
\DeleteShortVerb\|
\newcommand\OR{\ensuremath{\,|\,}}%

%%%%%%%% 
%\usepackage{lstdoc} %%% copy/paste most of the file, but omit and adjust some stuff like
%section-modifications 
\usepackage{listings}
\def\lst@sampleInput{%
    \MakePercentComment\catcode`\^^M=10\relax
    \small\lst@sample
    {\setkeys{lst}{SelectCharTable=\lst@ReplaceInput{\^\^I}%
                                  {\lst@ProcessTabulator}}%
     \leavevmode \input{\jobname.tmp}}\MakePercentIgnore}
\definecolor{darkgreen}{rgb}{0,0.5,0}
\def\rstyle{\color{red}}
\def\advise{\par\list\labeladvise
    {\advance\linewidth\@totalleftmargin
     \@totalleftmargin\z@
     \@listi
     \let\small\footnotesize \small\sffamily
     \parsep \z@ \@plus\z@ \@minus\z@
     \topsep6\p@ \@plus1\p@\@minus2\p@
     \def\makelabel##1{\hss\llap{##1}}}}
\let\endadvise\endlist
\def\advisespace{\hbox{}\qquad}
\def\labeladvise{$\to$}
\newenvironment{syntax}
   {\list{}{\itemindent-\leftmargin
    \def\makelabel##1{\hss\lst@syntaxlabel##1,,,,\relax}}}
   {\endlist}
\def\lst@syntaxlabel#1,#2,#3,#4\relax{%
    \llap{\scriptsize\itshape#3}%
    \def\lst@temp{#2}%
    \expandafter\lst@syntaxlabel@\meaning\lst@temp\relax
    \rlap{\hskip-\itemindent\hskip\itemsep\hskip\linewidth
          \llap{\ttfamily\lst@temp}\hskip\labelwidth
          \def\lst@temp{#1}%
          \ifx\lst@temp\lstdoc@currversion#1\fi}}
\def\lst@syntaxlabel@#1>#2\relax
    {\edef\lst@temp{\zap@space#2 \@empty}}
\newcommand*\syntaxnewline{\newline\hbox{}\kern\labelwidth}
\newcommand*\syntaxor{\qquad or\qquad}
\newcommand*\syntaxbreak
    {\hfill\kern0pt\discretionary{}{\kern\labelwidth}{}}
\let\syntaxfill\hfill
\def\alternative#1{\lst@true \alternative@#1,\relax,}
\def\alternative@#1,{%
    \ifx\relax#1\@empty
        \expandafter\@gobble
    \else
        \ifx\@empty#1\@empty\else %\if
            \lst@if \lst@false \else $\vert$\fi
            \textup{\texttt{#1}}%
        \fi
    \fi
    \alternative@}
\lst@RequireAspects{writefile}
\lst@InstallKeywords{p}{point}{pointstyle}\relax{keywordstyle}{}ld
\def\pstyle{\color{darkgreen}}
\lstset{language={[LaTeX]TeX},tabsize=4,gobble=4,%
  basicstyle=\small\ttfamily,basewidth=0.51em,boxpos=t,pointstyle=\pstyle,moredelim=[is][\pstyle]{~}{~}}%
\newbox\sampleoutputbox
\newbox\lst@samplebox
\newdimen\forestexample@code
\newdimen\forestexample@sample
\newdimen\forestexample@hsep
\forestexample@hsep=1em
\lst@Key{hsep}\relax{\forestexample@hsep=#1}%
\pgfqkeys{/forestexample}{%
  samplebox/.code={\let\sampleoutputbox#1},
  codebox/.code={\let\lst@samplebox#1},
  pos/.initial=l, % example is left of the code
  before/.code={\gdef\lst@sample{#1}},
  labelformat/.initial={\def\@currentlabel{#1}},
  no numbering/.code={\addtocounter{lstlisting}{-1}\pgfkeysalso{labelformat={}}},
  .unknown/.code={\lstset{\pgfkeyscurrentname={#1}}},
  ekeynames/.code={\def\myindex@for@temp##1{\ekeyname[example]{##1}}\forcsvlist\myindex@for@temp{#1}},
  ecmdnames/.code={\forcsvlist{\ecmdname[example]}{#1}},
  filename/.initial={},
}
\lstnewenvironment{forestexample}[1][]{%
  \global\let\lst@intname\@empty
  \def\@currentlabel{(\arabic{lstlisting})}%
  \addtocounter{lstlisting}{1}%
  \gdef\lst@sample{}%
  \pgfqkeys{/forestexample}{#1}%
  \setbox\lst@samplebox=\hbox\bgroup
  \xdef\samplebox@baselineskip{\the\baselineskip}%
  \catcode`~=9\relax
  \lst@BeginAlsoWriteFile{\jobname.tmp}%
}{%
  \lst@EndWriteFile\egroup
  \immediate\write18{cat \jobname.tmp}%
  \pgfkeysgetvalue{/forestexample/pos}\fe@pos
  \if x\fe@pos   %%%%%%%%  user position: boxes are stored in cs given in samplebox and codebox args
    \forest@temp@count=\@listdepth
    \pgfutil@tempdima=0pt
    \loop
    \ifnum\forest@temp@count>0
      \advance\pgfutil@tempdima\csname leftmargin\romannumeral\the\forest@temp@count\endcsname\relax
      \advance\forest@temp@count-1
    \repeat
    \global\setbox\lst@samplebox=\hbox{\hskip-\pgfutil@tempdima\box\lst@samplebox\hskip\pgfutil@tempdima}%
    \global\setbox\sampleoutputbox=\hbox{\lst@sampleInput}%
  \else
    \if l\fe@pos %%%% example is left of the code
  % move the code left for each list's \leftmargin ... have no idea why this must be done
      \forest@temp@count=\@listdepth
      \pgfutil@tempdima=0pt
      \loop
      \ifnum\forest@temp@count>0
        \advance\pgfutil@tempdima\csname leftmargin\romannumeral\the\forest@temp@count\endcsname\relax
        \advance\forest@temp@count-1
      \repeat
      \setbox\lst@samplebox=\hbox{\hskip-\pgfutil@tempdima\box\lst@samplebox\hskip\pgfutil@tempdima}%
      \setbox\sampleoutputbox=\hbox{\lst@sampleInput}%
      \pgfutil@tempdima=\wd\sampleoutputbox
      \advance\pgfutil@tempdima\wd\lst@samplebox
      \advance\pgfutil@tempdima\forestexample@hsep
      \ifdim\pgfutil@tempdima>\linewidth
        \forestexample@code=\linewidth
        \advance\forestexample@code-\wd\lst@samplebox
        \forestexample@sample=\forestexample@code
        \advance\forestexample@sample-\forestexample@hsep
        \advance\forestexample@sample-\wd\sampleoutputbox
      \else
        \pgfutil@tempdima=\wd\sampleoutputbox
        \advance\pgfutil@tempdima\forestexample@hsep
        \ifdim\pgfutil@tempdima>.5\linewidth
          \forestexample@sample=0pt
          \forestexample@code=\wd\sampleoutputbox
          \advance\forestexample@code\forestexample@hsep
        \else
          \pgfutil@tempdima=\wd\lst@samplebox
          \advance\pgfutil@tempdima\forestexample@hsep
          \ifdim\pgfutil@tempdima>.5\linewidth
            \forestexample@code=\linewidth
            \advance\forestexample@code-\wd\lst@samplebox
            \forestexample@sample=0pt
          \else
            \forestexample@sample=0pt
            \forestexample@code=.5\linewidth
          \fi
        \fi
      \fi
      \begin{trivlist}\item\relax
        $%
          \vcenter{
            \hbox{%
              \hbox to 0pt{\hskip\linewidth\llap{\@currentlabel}}%
              \hbox to 0pt{%
                \hskip\forestexample@code
                \raise\samplebox@baselineskip\box\lst@samplebox
              }%
            }%
          }%
          \vcenter{%
            \hbox to 0pt{%
              \hskip\forestexample@sample
              \box\sampleoutputbox
            }%
          }%
        $%
      \end{trivlist}%
    \else
      \if t\fe@pos %%%% example is above the code
        \forest@temp@count=\@listdepth
        \pgfutil@tempdima=0pt
        \loop
        \ifnum\forest@temp@count>0
          \advance\pgfutil@tempdima\csname leftmargin\romannumeral\the\forest@temp@count\endcsname\relax
          \advance\forest@temp@count-1
        \repeat
        \setbox\lst@samplebox=\hbox{\hskip-\pgfutil@tempdima\box\lst@samplebox\hskip\pgfutil@tempdima}%
        \setbox\sampleoutputbox=\hbox{\lst@sampleInput}%
        \begin{trivlist}%
        \item
          \hfil\box\sampleoutputbox\hfil
        \item
          \hbox{%
            \hbox to 0pt{\hskip\linewidth\llap{\@currentlabel}}%
            \hbox to 0pt{%
              \raise\samplebox@baselineskip\box\lst@samplebox
            }%
          }%
        \end{trivlist}%
      \else
        \if b\fe@pos %%% example is below the code
          \forest@temp@count=\@listdepth
          \pgfutil@tempdima=0pt
          \loop
          \ifnum\forest@temp@count>0
            \advance\pgfutil@tempdima\csname leftmargin\romannumeral\the\forest@temp@count\endcsname\rel                                    ax
            \advance\forest@temp@count-1
          \repeat
          \setbox\lst@samplebox=\hbox{\hskip-\pgfutil@tempdima\box\lst@samplebox\hskip\pgfutil@tempdima}%
          \setbox\sampleoutputbox=\hbox{\lst@sampleInput}%
          \begin{trivlist}%
          \item
            \hbox{%
              \hbox to 0pt{\hskip\linewidth\llap{\@currentlabel}}%
              \hbox to 0pt{%
                \raise\samplebox@baselineskip\box\lst@samplebox
              }%
            }%
          \item
            \hfil\box\sampleoutputbox\hfil
          \end{trivlist}%
        \else %%% insert other pos here.... 
        \fi
      \fi
    \fi
  \fi
}%
\def\myisaspect#1#2#3{% #1=aspect id, #2=aspect display, #3=entry ids
  \csdef{myaspect@display@#1}{#2}%
  \edef\myisaspect@##1{%
    \csdef{myaspect@of@##1}{#1}%
  }%
  \forcsvlist\myisaspect@{#3}%
}
\def\my@index#1#2#3#4{% #1=entry id,#2=entry display,#3=aspect id,#4=pagestyle
  \ifstrempty{#3}{%
    \edef\mytemp{%
      \noexpand\index{#1=\unexpanded{#2}#4}%
    }%
  }{%
    \edef\mytemp{%
      \noexpand\index{%
        #1=\unexpanded{#2}\protect\noexpand\space
        {\protect\noexpand\scriptsize
          \expandafter\expandafter\expandafter\unexpanded
          \expandafter\expandafter\expandafter
          {\csname myaspect@display@#3\endcsname}%
        }%
        #4%
      }%
      \noexpand\index{#3=\expandafter\expandafter\expandafter\unexpanded
        \expandafter\expandafter\expandafter
        {\csname myaspect@display@#3\endcsname}\levelchar
        #1=\unexpanded{#2}%
        #4%
      }%
    }%
  }%
  \mytemp
}%
\newcommand\myindex[1]{%
  \@bsphack
  \pgfqkeys{/myindex}{#1}%
  \pgfkeysgetvalue{/myindex/id}\myindex@temp@entryid
  \pgfkeysgetvalue{/myindex/display}\myindex@temp@entrydisplay
  \pgfkeysgetvalue{/myindex/aspect}\myindex@temp@aspectid
  \pgfkeysgetvalue{/myindex/pagestyle}\myindex@temp@pagestyle
  \edef\myindex@temp{\noexpand\my@index
    {\expandonce{\myindex@temp@entryid}}{\expandonce{\myindex@temp@entrydisplay}}{\expandonce{\myindex@temp@aspectid}}{\myindex@temp@pagestyle}%
  }\myindex@temp
  \@esphack
  \pgfkeysvalueof{/myindex/text}%
}
\def\stripfirst#1#2\stripfirst{#2}%
\pgfqkeys{/myindex}{%
  /handlers/.wrap/.code={%
    \edef\myindex@currentpath{\pgfkeyscurrentpath}%
    \pgfkeysgetvalue{\myindex@currentpath}\myindex@keyvalue
    \forest@def@with@pgfeov\myindex@wrap@code{#1}%
    \expandafter\edef\expandafter\myindex@wrapped@value\expandafter{\expandafter\expandonce\expandafter{\expandafter\myindex@wrap@code\myindex@keyvalue\pgfeov}}%
    \pgfkeysalso{\myindex@currentpath/.expand once=\myindex@wrapped@value}%
  },
  /handlers/.ewrap/.code={% not used!
    \edef\myindex@currentpath{\pgfkeyscurrentpath}%
    \pgfkeysgetvalue{\myindex@currentpath}\myindex@keyvalue
    \forest@def@with@pgfeov\myindex@wrap@code{#1}%
    \edef\myindex@wrapped@value{\expandafter\myindex@wrap@code\expandafter{\myindex@keyvalue}\pgfeov}%
    \pgfkeysalso{\myindex@currentpath/.expand once=\myindex@wrapped@value}%
  },
  id/.code={%
    \pgfkeyssetvalue{/myindex/id}{#1}%
    \pgfkeysgetvalue{/myindex/id}\myindex@temp
    \pgfkeyslet{/myindex/display}\myindex@temp
    \pgfkeyslet{/myindex/text}\myindex@temp
    \pgfkeyssetvalue{/myindex/pagestyle}{}%
    \pgfkeyssetvalue{/myindex/version}{}%
    \ifcsname myaspect@of@#1\endcsname
      \pgfkeysalso{aspect/.expand once=\csname myaspect@of@#1\endcsname}%
    \else
      \pgfkeyssetvalue{/myindex/aspect}{}%
      \pgfkeyssetvalue{/myindex/margin}{}%
    \fi
  },
  id'/.code={\pgfkeyssetvalue{/myindex/id}{#1}},
  .unknown/.code={%
    \edef\myindex@temp{%
      \noexpand\pgfkeysalso{id={\pgfkeyscurrentname}}%
    }\myindex@temp
  },
  display/.initial={},
  pagestyle/.initial={},
  text/.initial={},
  aspect/.code={%
    \edef\myindex@temp{%
      \noexpand\pgfkeyssetvalue{/myindex/aspect}{#1}%
    }\myindex@temp
    \ifcsname myaspect@display@#1\endcsname
      \edef\myindex@temp{%
        \noexpand\pgfkeyslet{/myindex/margin}\expandonce{\csname myaspect@display@#1\endcsname}%
      }\myindex@temp
    \else
      \pgfkeyssetvalue{/myindex/margin}{}%
    \fi
  },
  nfc/.style={% no first char (in id)
    id'/.expanded={\expandafter\stripfirst\romannumeral-`0\pgfkeysvalueof{/myindex/id}\stripfirst}
  },
  rstyle/.style={text/.wrap={\begingroup\rstyle##1\endgroup}},
  example/.style={pagestyle=|indextextexample},
  def/.style={pagestyle=|indextextdef},
  normal/.style={pagestyle=|indextextnormal},
  item/.style={% #1=default
    text/.wrap={% ##1=current text
      \item[\pgfkeysvalueof{/myindex/version},#1,\pgfkeysvalueof{/myindex/margin}]{##1}%
    }
  },
  item/.default={},
  version/.initial={},
  margin/.initial={},
}
\newcommand\indextextexample[1]{\hyperlink{page.#1}{\textcolor{darkgreen}{#1}}}
\newcommand\indextextdef[1]{\hyperlink{page.#1}{\textcolor{red}{#1}}}
\newcommand\indextextnormal[1]{\hyperlink{page.#1}{\textcolor{blue}{#1}}}
\let\keyname\texttt
\newcommand\rkeyname[2][]{\myindex{%
    #2,
    display/.wrap=\protect\texttt{##1},
    text/.wrap=\hypertarget{\pgfkeysvalueof{/myindex/id}}{{\rstyle\keyname{##1}}},
    def,
    #1
  }}
\newcommand\ikeyname[2][]{\myindex{%
    #2,
    display/.wrap=\protect\texttt{##1},
    text/.wrap=\hyperlink{\pgfkeysvalueof{/myindex/id}}{\keyname{##1}},
    normal,
    #1
  }}
\newcommand\ekeyname[2][]{\myindex{%
    #2,
    display/.wrap=\protect\texttt{##1},
    text={},
    normal,
    #1
  }}
\newcommand\rmeta[2][]{\myindex{%
    #2,
    display/.wrap=\protect\meta{##1},
    %text/.wrap=\begingroup\rstyle\meta{##1}\endgroup,
    text/.wrap=\hypertarget{\pgfkeysvalueof{/myindex/id}}{{\rstyle\meta{##1}}},
    def,
    #1
  }}
\newcommand\imeta[2][]{\myindex{%
    #2,
    display/.wrap=\protect\meta{##1},
    text/.wrap=\hyperlink{\pgfkeysvalueof{/myindex/id}}{\meta{##1}},
    normal,
    #1
  }}
\newcommand\cmdname[1]{\expandafter\texttt\expandafter{\expandafter\string\csname#1\endcsname}}
\newcommand\rcmdname[2][]{\myindex{%
    #2,
    id'/.expanded=\pgfkeysvalueof{/myindex/id} macro,
    display/.wrap=\protect\cmdname{##1},
    text/.wrap=\hypertarget{\pgfkeysvalueof{/myindex/id}}{{\rstyle\cmdname{##1}}},
    def,
    #1
  }}
\newcommand\icmdname[2][]{\myindex{%
    #2,
    id'/.expanded=\pgfkeysvalueof{/myindex/id} macro,
    text/.wrap=\hyperlink{\pgfkeysvalueof{/myindex/id}}{\cmdname{##1}},
    normal,
    #1
  }}
\newcommand\ecmdname[2][]{\myindex{%
    #2,
    id'/.expanded=\pgfkeysvalueof{/myindex/id} macro,
    display/.wrap=\protect\cmdname{##1},
    text={},
    normal,
    #1
  }}
\makeatother
\myisaspect{environment}{environment}{forest}
\myisaspect{option}{option}{align,content,content format,node format,base,node options,phantom,anchor,calign,calign primary angle,calign secondary angle,calign primary child,calign secondary child,fit,grow,ignore,ignore edge,reversed,l,s,l sep,s sep,tier,x,y,child anchor,edge,edge label,edge path,parent anchor,name,tikz,anchor,level,n,n',n children,id,max x,max y,min x,min y}
\myisaspect{propagator}{propagator}{for,if,where,for tree,repeat,delay,delay n,if have delayed,for ancestors,for ancestors',for children,for descendants,for descendants',for all next,for all previous,for previous siblings,before typesetting nodes,before packing,before computing xy,before drawing tree,repeat}
\myisaspect{type}{type}{toks,autowrapped toks,keylist,dimen,count,boolean}%relative node name,node walk,step}
\myisaspect{handler}{handler}{.pgfmath,.wrap value,.wrap pgfmath arg,.wrap $n$ pgfmath args,.wrap 2 pgfmath args,.wrap 3 pgfmath args,.wrap 4 pgfmath args,.wrap 5 pgfmath args,.wrap 6 pgfmath args,.wrap 7 pgfmath args,.wrap 8 pgfmath args,.wrap 9 pgfmath args}
\myisaspect{key prefix}{key prefix}{if in ,where in ,if ,where ,not ,for }
\myisaspect{key suffix}{key suffix}{',+,-,*,:,'+,'-,'*,':}
\myisaspect{key}{}{afterthought,baseline,label,pin,alias,TeX,TeX',TeX'',no edge,typeset node,repeat,use as bounding box,use as bounding box',draw tree box}
\myisaspect{style}{style}{stages,typeset nodes stage,pack stage,compute xy stage,draw tree
  stage,math content}
\myisaspect{stage}{stage}{typeset nodes,typeset nodes',pack,compute xy,draw tree,draw tree'}
\myisaspect{package option}{package option}{external,tikzcshack,tikzinstallkeys}
\myisaspect{dynamic tree}{dynamic tree}{create,remove,prepend,append,insert after,insert before,set
  root,replace by,prepend',append',insert after',insert before',replace by',prepend'',append'',insert after'',insert before'',replace by'',copy name template}
\myisaspect{forest cs}{forest cs}{}

\myisaspect{calign}{\keyname{calign} value}{}%{child,first,last,child edge,midpoint,center,edge midpoint,fixed angles,fixed edge angles}
\myisaspect{align}{\keyname{align} value}{}%{left,center,right}
\myisaspect{fit}{\keyname{fit} value}{}%{tight,rectangle,band}
\myisaspect{base}{\keyname{base} value}{}%{top,bottom}

\myisaspect{step}{\meta{step}}{current,next,previous,parent,sibling,previous leaf,next leaf,
  linear next,linear previous,first leaf,last leaf,to tier,next on tier,previous on tier,
  root,embed,trip,group,first,last
  %,n,n',name,id,   % these equal option names
}
\myisaspect{short step}{\meta{short step}}{1,2,3,4,5,6,7,8,9,u,p,%
  %,n,l,s equal option names
  P,N,F,L,<,%> is a level char
  c,r}
\myisaspect{generic anchor}{generic anchor}{}%
{\catcode`\|=12 \gdef\myindexgt{\texttt{>}}}
{\makeatletter % an dirty patch: \lst@nolig can sneak in the name...
\gdef\myexampleindex#1{{\def\lst@nolig{}\lstaspectindex{#1}{}}}
}
\lstset{indexstyle={[1]\myexampleindex}}
\makeindex
%%% end lst-related stuff

\EnableCrossrefs         
 %\DisableCrossrefs     % Say \DisableCrossrefs if index is ready
 %\CodelineIndex
 %\RecordChanges          % Gather update information
 %\OnlyDescription      % comment out for implementation details
\setlength\hfuzz{15pt}  % dont make so many
\hbadness=7000          % over and under full box warnings

\def\partname{Part}

\def\TikZ;{{\rm Ti\emph{k}Z}}\def\PGF;{\textsc{pgf}}\def\foRest;{\textsc{Forest}}\def\FoRest;{\textsc{Forest}}
\usetikzlibrary{intersections}
\tikzset{>=latex}
\forestset{
  background tree/.style={
    for tree={text opacity=0.2,draw opacity=0.2,edge={draw opacity=0.2}}}
}

\def\getforestversion#1/#2/#3 v#4 #5\getforestversion{v#4}
\edef\forestversion{\expandafter\expandafter\expandafter\getforestversion\csname ver@forest.sty\endcsname\getforestversion}

\def\getforestdate#1/#2/#3 v#4 #5\getforestdate{#1/#2/#3}
\edef\forestdate{\expandafter\expandafter\expandafter\getforestdate\csname ver@forest.sty\endcsname\getforestdate}

\title{\FoRest;: a \PGF;/\TikZ;-based package for drawing linguistic trees\\\normalsize\forestversion}
\author{Sa\v so \v Zivanovi\'c\footnote{e-mail:
    \href{mailto:saso.zivanovic@guest.arnes.si}{saso.zivanovic@guest.arnes.si};
    web:
    \href{http://spj.ff.uni-lj.si/zivanovic/}{http://spj.ff.uni-lj.si/zivanovic/}}}
\def\settodayfromforestdateA#1/#2/#3 v#4 #5\settodayfromforestdateA{\def\year{#1}\def\month{#2}\def\day{#3}}
\def\settodayfromforestdate{\expandafter\expandafter\expandafter\settodayfromforestdateA\csname ver@forest.sty\endcsname\settodayfromforestdateA}

\begin{document}
  \DocInput{forest.dtx}
\end{document}
%</driver>
% \fi
%
% ^^A short verbatim: | (changes spaces into _)
% \DeleteShortVerb\|
% {\catcode`\_=12 \def\marshal{^^A
% \lstMakeShortInline[basicstyle=\ttfamily,literate={_}{ }1 {__}{_}1]}^^A
% \expandafter}\marshal |
% 
% \newbox\treebox
% \newbox\codebox
%
%
%
% {\settodayfromforestdate\maketitle}
%
% \begin{abstract}
% \FoRest; is a \PGF;/\TikZ;-based package for drawing linguistic (and
% other kinds of) trees.  Its main features are 
% \begin{inparaenum}[(i)]
% \item a packing algorithm which can produce very compact trees;
% \item a user-friendly interface consisting of the familiar bracket encoding of trees plus the
%   key--value interface to option-setting;
% \item many tree-formatting options, with control over option values of individual nodes and
%   mechanisms for their manipulation;
% \item the possibility to decorate the tree using the full power of \PGF;/\TikZ;;
% \item an externalization mechanism sensitive to code-changes.
% \end{inparaenum}
% \end{abstract}
%
% {\lstset{basicstyle=\ttfamily\scriptsize}^^A
% \begin{forestexample}[samplebox=\treebox,codebox=\codebox,pos=x,ekeynames={content,{id=.pgfmath,nfc},if,repeat,append,before drawing tree,where,y,alias,for ,name,for children,edge,before typesetting nodes,for tree,s sep,l,+,,for ancestors',typeset node}]
%   \pgfmathsetseed{14285}
%   \begin{forest}
%     random tree/.style n args={3}{% #1=max levels, #2=max children, #3=max content
%       content/.pgfmath={random(0,#3)},
%       if={#1>0}{repeat={random(0,#2)}{append={[,random tree={#1-1}{#2}{#3}]}}}{}},
%     for deepest/.style={before drawing tree={
%         alias=deepest,
%         where={y()<y("deepest")}{alias=deepest}{},
%         for name={deepest}{#1}}},
%     colorone/.style={fill=yellow,for children=colortwo}, colortwo/.style={fill=green,for children=colorone},
%     important/.style={draw=red,line width=1.5pt,edge={red,line width=1.5pt,draw}},
%     before typesetting nodes={colorone, for tree={draw,s sep=2pt,rotate={int(30*rand)},l+={5*rand}}},
%     for deepest={for ancestors'={important,typeset node}}
%     [,random tree={9}{3}{100}]
%   \end{forest}
% \end{forestexample}%
% \begin{center}\mbox{}\box\treebox\\\box\codebox\end{center}}
% \newpage
% {\parskip 0pt                ^^A We have to reset \parskip
%                              ^^A (bug in \LaTeX)
% \tableofcontents
% }
%
% \newpage
% \part{User's Guide}
% \section{Introduction}
%
% Over several years, I had been a grateful user of various packages
% for typesetting linguistic trees.  My main experience was with
% |qtree| and |synttree|, but as far as I can tell, all of the tools
% on the market had the same problem: sometimes, the trees were just
% too wide.  They looked something like the tree on the left,
% while I wanted something like the tree on the right.
% \begin{center}
%   \begin{forest}
%     baseline,
%     for tree={parent anchor=south,child anchor=north,l=7ex,s sep=10pt},
%     for children={fit=rectangle}
%     [CP
%       [DP
%         [D][NP[N][CP[C][TP[T][vP[v][VP[DP][V'[V][DP]]]]]]]]
%       [TP
%         [T][vP[v][VP[DP][V'[V][DP]]]]]
%     ]
%   \end{forest}
%   \hfill
%   \begin{forest}
%     baseline,
%     for tree={parent anchor=south,child anchor=north,l=7ex,s sep=10pt},
%     [CP
%       [DP
%         [D][NP[N][CP[C][TP[T][vP[v][VP[DP][V'[V][DP]]]]]]]]
%       [TP
%         [T][vP[v][VP[DP][V'[V][DP]]]]]
%     ]
%   \end{forest}
% \end{center}
%
% Luckily, it was possible to tweak some parameters by hand to get a
% narrower tree, but as I quite dislike constant manual adjustments, I
% eventually started to develop \foRest;.  It started out as
% xyforest, but lost the xy prefix as I became increasingly fond
% of \PGF;/\TikZ;, which offered not only a drawing package but also a
% `programming paradigm.'  It is due to the awesome power of the
% supplementary facilities of \PGF;/\TikZ; that \foRest; is now, I
% believe, the most flexible tree typesetting package for \LaTeX\ you can get. 
% 
% After all the advertising, a disclaimer.  Although the present version
% is definitely usable (and has been already used), the package and
% its documentation are still under development: comments, criticism,
% suggestions and code are all very welcome!
%
% \FoRest; is \href{http://www.ctan.org/pkg/forest}{available} at \href{http://www.ctan.org}{CTAN},
% and I have also started a \href{https://github.com/sasozivanovic/forest-styles}{style repository}
% at \href{https://github.com}{GitHub}.
% 
% \section{Tutorial}
% \label{sec:tutorial}
%
% This short tutorial progresses from basic through useful to
% obscure \dots
% 
% \subsection{Basic usage}
% \label{sec:basic-usage}
%
% A tree is input by enclosing its specification in a \ikeyname{forest}
% environment.  The tree is encoded by \emph{the bracket syntax}:
% every node is enclosed in square brackets; the children of a
% node are given within its brackets, after its content.
% {\lstdefinelanguage[my]{TeX}[LaTeX]{TeX}{keywords=forest,
%   otherkeywords={[,]},keywordstyle=\pstyle,texcsstyle={}}^^A
% \lstset{language={[my]TeX}}^^A
% \begin{forestexample}
%   \begin{forest}
%     [VP
%       [DP]
%       [V'
%         [V]
%         [DP]
%       ]
%     ]
%   \end{forest}
% \end{forestexample}}
% Binary trees are nice, but not the only thing this package can draw.
% Note that by default, the children are vertically centered with
% respect to their parent, i.e.\ the parent is vertically aligned with the midpoint between the
% first and the last child.
% \begin{forestexample}
%   \begin{forest}
%     [VP
%       [DP[John]]
%       [V'
%         [V[sent]]
%         [DP[Mary]]
%         [DP[D[a]][NP[letter]]]
%       ]
%     ]
%   \end{forest}
% \end{forestexample}
% Spaces around brackets are ignored --- format your code as you
% desire!
% \begin{forestexample}
%   \begin{forest}
%     [VP[DP][V'[V][DP]]]
%   \end{forest}
%   \quad
%   \begin{forest}[VP
%     [DP ] [ V'[V][ DP]]
%     ]\end{forest}
% \end{forestexample}
% If you need a square bracket as part of a node's content, use
% braces.  The same is true for the other characters which have a
% special meaning in the \foRest; package: comma
% |,| and equality sign |=|.
% \begin{forestexample}
%   \begin{forest}
%     [V{P,}
%       [{[DP]}]
%       [V'
%         [V]
%         [{===DP===}]]]
%   \end{forest}
% \end{forestexample}
% Macros in a node specification will be expanded when the node is
% drawn --- you can freely use formatting commands inside nodes!
% \begin{forestexample}
%   \begin{forest}
%     [VP
%       [{~\textbf~{DP}}]
%       [V'
%         [V]
%         [DP]]]
%   \end{forest}
% \end{forestexample}
%
% \newbox\GPone
% \begin{forestexample}[pos=x,samplebox=\treebox,codebox=\GPone]
%   \newbox\standardnodestrutbox
%   \setbox\standardnodestrutbox=\hbox to 0pt{\phantom{\forestOve{standard node}{content}}}
%   \def\standardnodestrut{\copy\standardnodestrutbox}
%   \forestset{
%     ~GP1~/.style 2 args={
%       for n={1}{baseline},
%       s sep=0pt, l sep=0pt,
%       for descendants={
%         l sep=0pt, l={#1},
%         anchor=base,calign=first,child anchor=north,
%         inner xsep=1pt,inner ysep=2pt,outer sep=0pt,s sep=0pt,
%       },
%       delay={
%         if content={}{phantom}{for children={no edge}},
%         for tree={
%           if content={O}{tier=OR}{},
%           if content={R}{tier=OR}{},
%           if content={N}{tier=N}{},
%           if content={x}{
%             tier=x,content={$\times$},outer xsep={#2},
%             for tree={calign=center},
%             for descendants={content format={\standardnodestrut\forestoption{content}}},
%             before drawing tree={outer xsep=0pt,delay={typeset node}},
%             s sep=4pt
%           }{},
%         },
%       },
%       before drawing tree={where content={}{parent anchor=center,child anchor=center}{}},
%     },
%     GP1/.default={5ex}{8.0pt},
%     associate/.style={%
%       tikz+={\draw[densely dotted](!)--(!#1);}},
%     spread/.style={
%       before drawing tree={tikz+={\draw[dotted](!)--(!#1);}}},
%     govern/.style={
%       before drawing tree={tikz+={\draw[->](!)--(!#1);}}},
%     p-govern/.style={
%       before drawing tree={tikz+={\draw[->](.north) to[out=150,in=30] (!#1.north);}}},
%     no p-govern/.style={
%       before drawing tree={tikz+={\draw[->,loosely dashed](.north) to[out=150,in=30] (!#1.north);}}},
%     encircle/.style={before drawing tree={circle,draw,inner sep=0pt}},
%     fen/.style={pin={[font=\footnotesize,inner sep=1pt,pin edge=<-]10:\textsc{Fen}}},
%     el/.style={content=\textsc{\textbf{##1}}},
%     head/.style={content=\textsc{\textbf{\underline{##1}}}}
%   }
% \end{forestexample}
% \input{\jobname.tmp}
% 
% All the examples given above produced top-down trees with centered
% children.  The other sections of this manual explain how various
% properties of a tree can be changed, making it possible to typeset
% radically different-looking trees.  However, you don't have to learn
% everything about this package to profit from its power.  Using
% styles, you can draw predefined types of trees with ease.  For
% example, a phonologist can use the \ikeyname{GP1} style from \S\ref{sec:gallery} to easily typeset
% (Government Phonology) phonological 
% representations.  The style is applied simply by writing its name
% before the first (opening) bracket of the tree.
% \begin{forestexample}[label=ex:gp1-frost]
%   \begin{forest} ~GP1~ [
%     [O[x[f]][x[r]]]
%     [R[N[x[o]]][x[s]]]
%     [O[x[t]]]
%     [R[N[x]]]
%   ]\end{forest}
% \end{forestexample}
% Of course, someone needs to develop the style --- you, me, your
% local \TeX nician \dots\@ Furtunately, designing styles is not very
% difficult once you know your \foRest; options.  If you write one,
% please contribute!
%
% I have started a \href{https://github.com/sasozivanovic/forest-styles}{style repository} at
% GitHub.  Hopefully, it will grow \dots\@ Check it out, download the styles \dots\ and contribute
% them! 
%
% \subsection{Options}
% \label{sec:options}
% 
% A node can be given various options, which control various
% properties of the node and the tree.  For example, at the end of
% section~\ref{sec:basic-usage}, we have seen that the \ikeyname{GP1} style
% vertically aligns the parent with the first
% child.  This is achieved by setting option \ikeyname{calign} (for
% \emph{c}hild-\emph{align}ment) to \ikeyname{first,aspect=calign} (child).
%
% Let's try.  Options are given inside the brackets, following the
% content, but separated from it by a comma.  (If multiple options are
% given, they are also separated by commas.)  A single option
% assignment takes the form \meta{option name}|=|\meta{option value}.  (There are
% also options which do not require a value or have a default value:
% these are given simply as \meta{option name}.)
% \begin{forestexample}[label=ex:numerals-simple,ekeynames={calign,{first,aspect=calign}}]
%   \begin{forest}
%     [\LaTeX\ numerals, ~calign=first~
%       [arabic[1][2][3][4]]
%       [roman[i][ii][iii][iv]]
%       [alph[a][b][c][d]]
%     ]
%   \end{forest}
% \end{forestexample}
%
% The experiment has succeeded only partially.  The root node's
% children are aligned as desired (so \ikeyname{calign}|=|\ikeyname{first,aspect=calign} applied to the
% root node), but the value of the \ikeyname{calign} option didn't get
% automatically assigned to the root's children! \emph{An option given
% at some node applies only to that node.} In \foRest;, the options
% are passed to the node's relatives via special options, called
% \emph{propagators}.  (We'll
% call the options that actually change some property of the node
% \emph{node options}.) What we need above is the \ikeyname{for tree} propagator.  Observe:
% \begin{forestexample}[label=ex:numerals-manual]
%   \begin{forest}
%     [\LaTeX\ numerals,
%          ~for tree~={calign=first}
%       [arabic[1][2][3][4]]
%       [roman[i][ii][iii][iv]]
%       [alph[a][b][c][d]]
%     ]
%   \end{forest}
% \end{forestexample}
% The value of propagator \ikeyname{for tree} is the option string that we
% want to process.  This option string is propagated to all the nodes in
% the subtree\footnote{It might be more precise to call this option
% \texttt{for subtree} \dots\@ but this name at least saves some typing.}
% rooted in the current node (i.e.\ the node where \ikeyname{for tree} was
% given), including the node itself.  (Propagator \ikeyname{for descendants} is
% just like \ikeyname{for tree}, only that it excludes the node itself.  There
% are many other \ikeyname{id={{for }}}|...| propagators; for the complete list, see
% sections~\ref{ref:propagators} and \ref{ref:node-walk}.)
%
% Some other useful options are \ikeyname{parent anchor}, \ikeyname{child anchor}
% and \ikeyname{tier}.  The \ikeyname{parent anchor} and \ikeyname{child anchor} options tell
% where the parent's and child's endpoint of the edge between them
% should be, respectively: usually, the value is either empty
% (meaning a smartly determined border point \citep[see][\S16.11]{tikzpgf2.10};  this is the default)
% or a compass direction \citep[see][\S16.5.1]{tikzpgf2.10}.  (Note: the \ikeyname{parent anchor}
% determines where the
% edge from the child will arrive to this node, not where the node's
% edge to its parent will start!)
%
% Option \ikeyname{tier} is what makes the
% skeletal points $\times$ in example \ref{ex:gp1-frost} align horizontally although they
% occur at different levels in the logical structure of the tree.
% Using option \ikeyname{tier} is very simple: just set |tier=tier_name| at
% all the nodes that you want to align horizontally.  Any tier name
% will do, as long as the tier names of different tiers are
% different \dots\@ (Yes, you can have multiple tiers!)
% \begin{forestexample}[point={tier},ekeynames={parent anchor,child anchor,tier},label=ex:tier-manual]
%   \begin{forest} 
%     [VP, for tree={~parent anchor~=south, ~child anchor~=north}
%       [DP[John,tier=word]]
%       [V'
%         [V[sent,tier=word]]
%         [DP[Mary,tier=word]]
%         [DP[D[a,tier=word]][NP[letter,tier=word]]]
%       ]
%     ]
%   \end{forest}
% \end{forestexample}
% Before discussing the variety of \foRest;'s options, it is worth
% mentioning that \foRest;'s node accepts all options \citep[see
% \S16]{tikzpgf2.10} that \TikZ;'s node does --- mostly, it just passes
% them on to \TikZ;.  For example, you can easily encircle a node like
% this:\footnote{If option \texttt{draw} was not given, the shape of the node
% would still be circular, but the edge would not be drawn.  For
% details, see \cite[\S16]{tikzpgf2.10}.}
% \begin{forestexample}
%   \begin{forest}
%     [VP,~circle~,~draw~
%       [DP][V'[V][DP]]
%     ]
%   \end{forest}
% \end{forestexample}
% 
% Let's have another look at example \ref{ex:gp1-frost}.  You will note that the skeletal
% positions were input by typing |x|s, while the result looks like
% this: $\times$ (input as |\times| in math mode). Obviously, the
% content of the node can be changed.  Even more, it can be
% manipulated: added to, doubled, boldened, emphasized, etc.  We will
% demonstrate this by making example \ref{ex:numerals-manual} a bit
% fancier: we'll write the input in the arabic numbers and have
% \LaTeX\ convert it to the other formats.  We'll start with the
% easiest case of roman numerals: to get them, we can use the (plain)
% \TeX\ command |\romannumeral|.  To change the content of the node,
% we use option \ikeyname{content}.  When specifying its new value, we can use
% |#1| to insert the current content.\footnote{This mechanism is called
% \emph{wrapping}. \ikeyname{content} is the only option where wrapping works implicitely (simply
% because I assume that wrapping will be almost exclusively used with this option).  To wrap values
% of other options, use handler \ikeyname{id=.wrap value,nfc}; see~\S\ref{ref:handlers}.}
% \begin{forestexample}[point={content,delay},ekeynames={for children,content,delay},label=ex:romannumeral]
%   \begin{forest}
%     [roman, delay={for children={content=\romannumeral#1}}
%       [1][2][3][4]
%     ]
%   \end{forest}
% \end{forestexample}
% This example introduces another option: \ikeyname{delay}.  Without it, the
% example wouldn't work: we would get arabic numerals.  This is so
% because of the order in which the options are processed.  The
% processing proceeds through the tree in a depth-first, parent-first fashion (first
% the parent is processed, and then its children, recursively).  The option string of a node is
% processed linearly, in the order they were given.  (Option \keyname{content}
% is specified implicitely and is always the first.) If a propagator
% is encountered, the options given as its value are propagated \emph{immediately}.  The net effect
% is that if the 
% above example contained simply |roman,for_children={content=...}|, the
% \keyname{content} option given there would be processed \emph{before} the
% implicit content options given to the children (i.e.\ numbers |1|,
% |2|, |3| and |4|).  Thus, there would be nothing for the
% |\romannumeral| to change --- it would actually crash; more generally, the content assigned
% in such a way would get overridden by the implicit content.  Option
% \ikeyname{delay} is true to its name.  It delays the processing of its option
% string argument until the whole tree was processed.  In other words,
% it introduces cyclical option processing.  Whatever is delayed in
% one cycle, gets processed in the next one.  The number of cycles is
% not limited --- you can nest \ikeyname{delay}s as deep as you need.
%
% Unlike \ikeyname{id={{for }}}|_...| options we have met before, option \ikeyname{delay} is not a
% spatial, but a temporal propagator.  Several other temporal propagators options exist, see
% \S\ref{ref:stages}.
%
% We are now ready to learn about simple conditionals.  Every node option has the corresponding
% \ikeyname{id={{if }}}|...| and \ikeyname{id={{where }}}|...| keys.
% \ikeyname{id={{if }}}\meta{option}|=|\meta{value}\meta{true options}\meta{false options} checks whether
% the value of \meta{option} equals \meta{value}.  If so, \meta{true options} are
% processed, otherwise \meta{false options}.  The \ikeyname{id={{where }}}|_...| keys are
% the same, but do this for the every node in the subtree; informally
% speaking, |where| = |for_tree| + |if|.  To see this in action,
% consider the rewrite of the \ikeyname{tier} example \ref{ex:tier-manual} from above.  We don't set
% the tiers manually, but rather put the terminal nodes (option
% \ikeyname{n children} is a read-only option containing the number
% of children) on tier \keyname{word}.\footnote{We could omit the braces around \texttt{0} because
% it is a single character. If we were hunting for nodes with 42 children, we'd have to write 
% \texttt{where n children=\{42\}...}.}
% \begin{forestexample}[ekeynames={tier,where ,n children}]
%   \begin{forest}
%     ~where n children~=0{tier=word}{}
%     [VP
%       [DP[John]]
%       [V'
%         [V[sent]]
%         [DP[Mary]]
%         [DP[D[a]][NP[letter]]]
%       ]
%     ]
%   \end{forest}
% \end{forestexample}
%
% Finally, let's talk about styles. Styles are simply collections of
% options.  (They are not actually defined in the \foRest; package, but
% rather inherited from |pgfkeys|.)  If you often want to have non-default 
% parent/child anchors, say south/north as in example \ref{ex:tier-manual}, you would save some
% typing by defining a style.  Styles are defined using \PGF;'s handler 
% |.style|. (In the example below, style |ns_edges| is first defined and then used.)
% \begin{forestexample}[ekeynames={tier,parent anchor,child anchor}]
%   \begin{forest}
%     ~sn edges~/~.style~={for tree={
%             parent anchor=south, child anchor=north}},
%     ~sn edges~
%     [VP, 
%       [DP[John,tier=word]]
%       [V'
%         [V[sent,tier=word]]
%         [DP[Mary,tier=word]]
%         [DP[D[a,tier=word]][NP[letter,tier=word]]]]]
%   \end{forest}
% \end{forestexample}
% If you want to use a style in more than one tree, you have to define it outside the \ikeyname{forest}
% environment.  Use macro \icmdname{forestset} to do this.
% \begin{lstlisting}
%   ~\forestset~{
%     sn edges/.style={for tree={parent anchor=south, child anchor=north}},
%     background tree/.style={for tree={
%                 text opacity=0.2,draw opacity=0.2,edge={draw opacity=0.2}}}
%   }
% \end{lstlisting}
%
% You might have noticed that the last two examples contain options (actually, keys) even before the
% first opening bracket, contradicting was said at the beginning of this section.  This is mainly
% just syntactic sugar (it can separate the design and the content): such preamble
% keys behave as if they were given in the root node, the only difference (which often does not
% matter) being that they get processed before all other root node options, even the implicit
% content.
% 
% \subsection{Decorating the tree}
% \label{sec:decorating}
%
% The tree can be decorated (think movement arrows) with arbitrary
% \TikZ; code.
% \begin{forestexample}
%   \begin{forest}
%     [XP
%       [specifier]
%       [X$'$
%         [X$^0$]
%         [complement]
%       ]
%     ]
%     ~\node at (current bounding box.south)
%       [below=1ex,draw,cloud,aspect=6,cloud puffs=30]
%       {\emph{Figure 1: The X' template}};~
%   \end{forest}
% \end{forestexample}
% 
% However, decorating the tree would make little sense if one could
% not refer to the nodes.  The simplest way to do so is to give them a
% \TikZ; name using the \ikeyname{name} option, and then use this name in \TikZ;
% code as any other (\TikZ;) node name.
% \begin{forestexample}[point=name,ekeynames={phantom,name}]
%   \begin{forest}
%     [CP
%       [DP,name=spec CP]
%       [\dots
%         [,phantom]
%         [VP
%           [DP]
%           [V'
%             [V]
%             [DP,name=object]]]]]
%     \draw[->,dotted] ~(object)~ to[out=south west,in=south] ~(spec CP)~;
%   \end{forest}
% \end{forestexample}
% 
% It gets better than this, however! In the previous examples, we put
% the \TikZ; code after the tree specification, i.e.\ after the closing
% bracket of the root node.  In fact, you can put \TikZ; code after
% \emph{any} closing bracket, and \foRest; will know what the current
% node is. (Putting the code after a node's bracket is actually just a
% special way to provide a value for option \ikeyname{tikz} of that node.)  To
% refer to the current node, simply use an empty node name.  This works both with and without
% anchors \citep[see][\S16.11]{tikzpgf2.10}: below, |(.south east)| and |()|. 
% \begin{forestexample}[ekeynames={phantom,name}]
%   \begin{forest}
%     [CP
%       [DP,name=spec CP]
%       [\dots
%         [,phantom]
%         [VP
%           [DP]
%           [V'
%             [V]
%             [DP,draw] ~{~
%               \draw[->,dotted] ~()~ to[out=south west,in=south] (spec CP);
%               \draw[<-,red] ~(.south east)~--++(0em,-4ex)--++(-2em,0pt)
%                    node[anchor=east,align=center]{This guy\\has moved!};
%                  ~}~
%           ]]]]
%   \end{forest}
% \end{forestexample}
%
% Important: \emph{the \TikZ; code should usually be enclosed in braces} to hide
% it from the bracket parser.  You don't want all the bracketed code
% (e.g.\ |[->,dotted]|) to become tree nodes, right? (Well, they
% probably wouldn't anyway, because \TeX\ would spit out a thousand
% errors.)
%
% \bigskip
% 
% Finally, the most powerful tool in the node reference toolbox:
% \emph{relative nodes}.  It is possible to refer to other nodes which stand
% in some (most often geometrical) relation to the current node.  To
% do this, follow the node's name with a |!| and a \emph{node walk}
% specification.
%
% A node walk is a concise\footnote{Actually, \foRest; distinguishes two kinds of
% steps in node walks: long and short steps.  This section introduces only short steps.  See 
% \S\ref{ref:node-walk}.} way of expressing node
% relations.  It is simply a string of steps, which are represented by single
% characters, where: \ikeyname{u} stands for the parent node (up); \ikeyname{p} for the
% previous sibling; \ikeyname{n,aspect=short step} for the next sibling; \ikeyname{s,aspect=short step} for \emph{the}
% sibling (useful only in binary trees);\ekeyname{3} \ikeyname{1}, \ikeyname{2},
% \ekeyname{3}\ekeyname{4}\ekeyname{5}\ekeyname{6}\ekeyname{7}\ekeyname{8}\dots\
% \ikeyname{9} for first,
% second, \dots\ ninth child; \ikeyname{l,aspect=short step}, for the last child, etc.  For the
% complete specification, see section~\ref{ref:node-walk}.
%
% To see the node walk in action, consider the following examples.
% In the first example, the agree arrow connects the V node, specified
% simply as |()|, since the \TikZ; code follows |[V]|, and the DP node,
% which is described as ``a sister of V's parent'': |!us| = up +
% sibling. 
% \begin{forestexample}
%   \begin{forest}
%     [VP
%       [DP] 
%       [V'
%         [V] {\draw[<->] ~()~
%              .. controls +(left:1cm) and +(south west:0.4cm) ..
%              node[very near start,below,sloped]{\tiny agree}
%              ~(!us)~;}
%         [DP]
%       ]
%     ]
%   \end{forest}
% \end{forestexample}
%
% {\footnotesize\begin{forestexample}[ekeynames={phantom,tikz,fit to tree},samplebox=\treebox,codebox=\codebox,pos=x,basicstyle=\footnotesize\ttfamily]
%   \begin{forest}
%     [CP
%       [DP$_1$]
%       [\dots
%         [,phantom]
%         [VP,tikz={\node [draw,red,~fit to tree~]{};}
%           [DP$_2$] 
%           [V'
%             [V]
%             [DP$_3$]
%     ]]]]
%   \end{forest}
% \end{forestexample}}
% The second example uses \TikZ;'s fitting library to compute the
% smallest rectangle containing node VP, its first child (DP$_2$) and its last grandchild (DP$_3$).
% The example also illustrates that the \TikZ; code 
% can be specified via the ``normal'' option syntax, i.e.\ as a value
% to option \ikeyname{tikz}.\footnote{\label{fn:fit-to-tree}Actually, there's a simpler way to do this: use \ikeyname{fit to tree}!\\\raisebox{\dimexpr-\dp\codebox+1ex\relax}{\box\treebox}\hfill\box\codebox}
% \begin{forestexample}[point=tikz,ekeynames={phantom,tikz}]
%   \begin{forest}
%     [CP
%       [DP$_1$]
%       [\dots
%         [,phantom]
%         [VP,tikz={\node [draw,red,fit=~()(!1)(!ll)~] {};}
%           [DP$_2$] 
%           [V'
%             [V]
%             [DP$_3$]
%     ]]]]
%   \end{forest}
% \end{forestexample}
% 
%
% \subsection{Node positioning}
% \label{sec:node-positioning}
%
% \FoRest; positions the nodes by a recursive bottom-up algorithm which, for every non-terminal node,
% computes the positions of the node's children relative to their parent.  By default, all the
% children will be aligned horizontally some distance down from their parent: the ``normal'' tree
% grows down.  More generally, however, the direction of growth can change from node to node; this is
% controlled by option \ikeyname{grow}=\meta{direction}.\footnote{The direction can be specified either in
%   degrees (following the standard mathematical convention that $0$ degrees is to the right, and that
%   degrees increase counter-clockwise) or by the compass directions: \texttt{east}, \texttt{north east},
%   \texttt{north}, etc.}  The system thus computes and stores the positions of children using a
% coordinate system dependent on the parent, called an \emph{ls-coordinate system}: the origin is the
% parent's anchor; l-axis is in the direction of growth in the parent; s-axis is orthogonal to the
% l-axis (positive side in the counter-clockwise direction from $l$-axis); l stands for \emph{l}evel,
% s for \emph{s}ibling.  The example shows the ls-coordinate system for a node with |grow=45|.
%
% \begin{forestexample}[point=grow,ekeynames=grow]
%   \begin{forest} background tree
%     [parent, grow=45
%       [child 1][child 2][child 3][child 4][child 5]
%     ]
%     \draw[,->](-135:1cm)--(45:3cm) node[below]{$l$};
%     \draw[,->](-45:1cm)--(135:3cm) node[right]{$s$};
%   \end{forest}
% \end{forestexample}
%
% \begin{forestexample}[basicstyle=\scriptsize\ttfamily,samplebox=\treebox,codebox=\codebox,pos=x]
%   \newcommand\measurexdistance[5][####1]{\measurexorydistance{#2}{#3}{#4}{#5}{\x}{-|}{(5pt,0)}{#1}}
%   \newcommand\measureydistance[5][####1]{\measurexorydistance{#2}{#3}{#4}{#5}{\y}{|-}{(0,5pt)}{#1}}
%   \tikzset{dimension/.style={<->,>=latex,thin,every rectangle node/.style={midway,font=\scriptsize}},
%     guideline/.style=dotted}
%   \newdimen\absmd
%   \def\measurexorydistance#1#2#3#4#5#6#7#8{%
%     \path #1 #3 #6 coordinate(md1) #1; \draw[guideline] #1 --  (md1);
%     \path (md1) #6 coordinate(md2) #2; \draw[guideline] #2 -- (md2);
%     \path let \p1=($(md1)-(md2)$), \n1={abs(#51)} in \pgfextra{\xdef\md{#51}\global\absmd=\n1\relax};
%     \def\distancelabelwrapper##1{#8}%
%     \ifdim\absmd>5mm
%       \draw[dimension] (md1)--(md2) node[#4]{\distancelabelwrapper{\uselengthunit{mm}\rndprintlength\absmd}};
%     \else
%       \ifdim\md>0pt
%         \draw[dimension,<-] (md1)--+#7; \draw[dimension,<-] let \p1=($(0,0)-#7$) in (md2)--+(\p1);
%       \else
%         \draw[dimension,<-] let \p1=($(0,0)-#7$) in (md1)--+(\p1); \draw[dimension,<-] (md2)--+#7;
%       \fi
%       \draw[dimension,-] (md1)--(md2) node[#4]{\distancelabelwrapper{\uselengthunit{mm}\rndprintlength\absmd}};
%     \fi}
% \end{forestexample}
% \input{\jobname.tmp}
% 
% The l-coordinate of children is (almost) completely under your control, i.e.\ you set what is
% often called the level distance by yourself.  Simply set option \ikeyname{l} to change the
% distance of a node from its parent.  More precisely, \ikeyname{l}, and the related option
% \ikeyname{s}, control the distance between the (node) anchors of a node and its parent.  The
% anchor of a node can be changed using option \ikeyname{anchor}: by default, nodes are anchored at
% their base; see \cite[\S16.5.1]{tikzpgf2.10}.)  In the example below, positions of the anchors are
% shown by dots: observe that anchors of nodes with the same \ikeyname{l} are aligned and that the
% distances between the anchors of the children and the parent are as specified in the
% code.\footnote{Here are the definitons of the macros for measuring distances.  Args: the x or y
% distance between points \#2 and \#3 is measured; \#4 is where the distance line starts (given as an
% absolute coordinate or an offset to \#2); \#5 are node options; the optional arg \#1 is the format of
% label. (Lengths are printed using package \texttt{printlen}.)
%
% \vskip-2ex \box\codebox}
% \begin{forestexample}[pos=t,ekeynames={for tree,tikz,l,anchor}]
%   \begin{forest} background tree,
%     for tree={draw,tikz={\fill[](.anchor)circle[radius=1pt];}}
%     [parent
%       [child 1, ~l~=10mm, ~anchor~=north west]
%       [child 2, ~l~=10mm, ~anchor~=south west]
%       [child 3, ~l~=12mm, ~anchor~=south]
%       [child 4, ~l~=12mm, ~anchor~=base east]
%     ]
%     \measureydistance[\texttt{l(child)}=#1]{(!2.anchor)}{(.anchor)}{(!1.anchor)+(-5mm,0)}{left}
%     \measureydistance[\texttt{l(child)}=#1]{(!3.anchor)}{(.anchor)}{(!4.anchor)+(5mm,0)}{right}
%     \measurexdistance[\texttt{s sep(parent)}=#1]{(!1.south east)}{(!2.south west)}{+(0,-5mm)}{below}
%     \measurexdistance[\texttt{s sep(parent)}=#1]{(!2.south east)}{(!3.south west)}{+(0,-5mm)}{below}
%     \measurexdistance[\texttt{s sep(parent)}=#1]{(!3.south east)}{(!4.south west)}{+(0,-8mm)}{below}
%   \end{forest}
% \end{forestexample}
%
% Positioning the chilren in the s-dimension is the job and \emph{raison d'etre} of the package.  As a
% first approximation: the children are positioned so that the distance between them is at least the
% value of option \ikeyname{s sep} (s-separation), which defaults to double \PGF;'s |inner_xsep| (and this
% is 0.3333em by default).  As you can see from the example above, s-separation is the distance
% between the borders of the nodes, not their anchors!
%
% A fuller story is that \ikeyname{s sep} does not control the s-distance between two siblings, but rather
% the distance between the subtrees rooted in the siblings.  When the green and the yellow child of
% the white node are s-positioned in the example below, the horizontal
% distance between the green and the yellow subtree is computed.  It can be seen with the naked eye
% that the closest nodes of the subtrees are the TP and the DP with a red border.  Thus, the children
% of the root CP (top green DP and top yellow TP) are positioned so that the horizontal distance
% between the red-bordered TP and DP equals \ikeyname{s sep}.
% \begin{forestexample}[ekeynames={for tree,s sep}]
%   \begin{forest}
%     important/.style={name=#1,draw={red,thick}}
%     [CP, ~s sep~=3mm, for tree=draw
%       [DP, for tree={fill=green}
%         [D][NP[N][CP[C][TP,important=left
%         [T][vP[v][VP[DP][V'[V][DP]]]]]]]]
%       [TP,for tree={fill=yellow}
%         [T][vP[v][VP[DP,important=right][V'[V][DP]]]]]
%     ]
%     \measurexdistance[\texttt{s sep(root)}=#1]
%       {(left.north east)}{(right.north west)}{(.north)+(0,3mm)}{above}
%   \end{forest}
% \end{forestexample}
%
% Note that \foRest; computes the same distances between nodes
% regardless of whether the nodes are filled or not, or whether their
% border is drawn or not.  Filling the node or drawing its border does
% not change its size.  You can change the size by adjusting \TikZ;'s
% |inner_sep| and |outer_sep| \citep[\S16.2.2]{tikzpgf2.10}, as shown
% below:
% \begin{forestexample}[ekeynames={for tree,s sep}]
%   \begin{forest}
%     important/.style={name=#1,draw={red,thick}}
%     [CP, s sep=3mm, for tree=draw
%       [DP, for tree={fill=green,~inner sep~=0}
%         [D][NP,important=left[N][CP[C][TP[T][vP[v]
%         [VP[DP][V'[V][DP]]]]]]]]
%       [TP,for tree={fill=yellow,~outer sep~=2pt}
%         [T,important=right][vP[v][VP[DP][V'[V][DP]]]]]
%     ]
%     \measurexdistance[\texttt{s sep(root)}=#1]
%       {(left.north east)}{(right.north west)}{(.north)+(0,3mm)}{above}
%   \end{forest}
% \end{forestexample}
% (This looks ugly!) Observe that having increased |outer sep| makes the edges stop touching
% borders of the nodes. By (\PGF;'s) default, the |outer sep| is exactly half of the border
% line width, so that the edges start and finish precisely at the border.
%
% Let's play a bit and change the \ikeyname{l} of the root of the yellow subtree.  Below, we set the
% vertical 
% distance of the yellow TP to its parent to 3\,cm: and the yellow submarine sinks diagonally \dots\@
% Now, the closest nodes are the higher yellow DP and the green VP.
% \begin{forestexample}[ekeynames={l,s sep,for tree}]
%   \begin{forest}
%     important/.style={name=#1,draw={red,thick}}
%     [CP, s sep=3mm, for tree=draw
%       [DP, for tree={fill=green}
%         [D][NP[N][CP[C][TP
%         [T][vP[v][VP,important=left[DP][V'[V][DP]]]]]]]]
%       [TP,for tree={fill=yellow}, l=3cm
%         [T][vP[v][VP[DP,important=right][V'[V][DP]]]]]
%     ]
%     \measurexdistance[\texttt{s sep(root)}=#1]
%       {(left.north east)}{(right.north west)}{(.north)+(0,3mm)}{above}
%   \end{forest}
% \end{forestexample}
%
% Note that the yellow and green nodes are not vertically aligned anymore.  The positioning algorithm
% has no problem with that.  But you, as a user, might have, so here's a neat trick.  (This only works
% in the ``normal'' circumstances, which are easier to see than describe.)
% \begin{forestexample}[label=ex:l*,ekeynames={l,*,phantom,for tree}]
%   \begin{forest}
%     [CP, for tree=draw
%       [DP, for tree={fill=green},~l*~=3
%         [D][NP]]
%       [TP,for tree={fill=yellow}
%         [T][VP[DP][V'[V][DP]]]]
%     ]
%   \end{forest}
% \end{forestexample}
% We have changed DP's \ikeyname{l}'s value via ``augmented assignment'' known from
% many programming languages: above, we have used |l*=3| to triple
% \ekeyname{l}'s value; we could have also said |l+=5mm| or |l-=5mm| to
% increase or decrease its value by 5\,mm, respectively.  This
% mechanism works for every numeric and dimensional option in \foRest;.
%
% Let's now play with option \ikeyname{s sep}.
% \begin{forestexample}[ekeynames={s sep,l,*,for tree}]
%   \begin{forest}
%     [CP, for tree=draw, ~s sep~=0
%       [DP, for tree={fill=green},l*=3
%         [D][NP]]
%       [TP,for tree={fill=yellow}
%         [T][VP[DP][V'[V][DP]]]]
%     ]
%   \end{forest}
% \end{forestexample}
% Surprised? You shouldn't be. The value of \ikeyname{s sep} at a given node controls the s-distance
% \emph{between the subtrees rooted in the children of that node}!  It has no influence over the
% internal geometry of these subtrees. In the above example, we have set |s_sep=0| only for the root
% node, so the green and the yellow subtree are touching, although internally, their nodes are not.
% Let's play a bit more.  In the following example, we set the \ikeyname{s sep} to: $0$ at the last
% branching level (level 3; the root is level 0), to 2\,mm at level 2, to 4\,mm at level 1 and to
% 6\,mm at level 0.
% 
% \begin{forestexample}[label=ex:spread-s,point={level},ekeynames={level,for tree,s sep}]
%   \begin{forest}
%     for tree={~s sep~=(3-level)*2mm}
%     [CP, for tree=draw
%       [DP, for tree={fill=green},l*=3
%         [D][NP]]
%       [TP,for tree={fill=yellow}
%         [T][VP[DP][V'[V][DP]]]]
%     ]
%     \measurexdistance{(!11.south east)}{(!12.south west)}{+(0,-5mm)}{below}
%     \path(md2)-|coordinate(md)(!221.south east);
%     \measurexdistance{(!221.south east)}{(!222.south west)}{(md)}{below}
%     \measurexdistance{(!21.north east)}{(!22.north west)}{+(0,2cm)}{above}
%     \measurexdistance{(!1.north east)}{(!221.north west)}{+(0,-2.4cm)}{below}
%   \end{forest}
% \end{forestexample}
% As we go up the tree, the nodes ``spread.'' At the lowest level, V and DP are touching.  In the
% third level, the \ikeyname{s sep} of level 2 applies, so DP and V' are 2\,mm apart.  At the second
% level we 
% have two pairs of nodes, D and NP, and T and TP: they are 4\,mm apart.  Finally, at level 1, the
% \ikeyname{s sep} of level 0 applies, so the green and yellow DP are 6\,mm apart.  (Note that D and NP are
% at level 2, not 4! Level is a matter of structure, not geometry.)
%
% As you have probably noticed, this example also demostrated that we can compute the value of an
% option using an (arbitrarily complex) formula.  This is thanks to \PGF;'s module |pgfmath|.
% \FoRest; provides an interface to |pgfmath| by defining |pgfmath| functions for every node option,
% and some other information, like the \ikeyname{level} we have used above, the number of children
% \ikeyname{n children}, the sequential number of the child \ikeyname{n}, etc.  For details, see
% \S\ref{ref:pgfmath}. 
%
% The final separation parameter is \ikeyname{l sep}.  It determines the minimal
% separation of a 
% node from its descendants.  It the value of \ikeyname{l} is too small, then \emph{all} the
% children (and thus their subtrees)
% are pushed 
% away from the parent (by increasing their \ikeyname{l}s), so that the distance between the node's
% and each child's subtree
% boundary is at least \ikeyname{l sep}.  The initial \ikeyname{l} can be too small for
% two reasons: either 
% some child is too high, or the parent is too deep.  The first problem is easier to see: we force the
% situation using a bottom-aligned multiline node.  (Multiline nodes can be easily created using |\\|
% as a line-separator.  However, you must first specify the horizontal alignment using option
% \ikeyname{align} (see \S\ref{ref:node-appearance}). 
% Bottom vertical alignment is achieved by setting \ikeyname{base}|=|\ikeyname{bottom,aspect=base};
% the default, unlike in \TikZ;, is \ikeyname{base}|=|\ikeyname{top,aspect=base}). 
% \begin{forestexample}[point={align,base},ekeynames={align,base}]
%   \begin{forest}
%     [parent
%       [child]
%       [child]
%       [a very\\tall\\child, align=center, base=bottom]
%     ]
%   \end{forest}
% \end{forestexample}
%
% The defaults for \ikeyname{l} and \ikeyname{l sep} are set so that they ``cooperate.''
% What this 
% means and why it is necessary is a complex issue explained in \S\ref{sec:defaults}, which you will
% hopefully never have to read \dots\@ You might be out of luck, however.  What if you 
% needed to decrease the level distance? And nothing happened, like below on the left?  Or, what if
% you used lots of parenthesis in your nodes?  And got a strange vertical misalignment, like below
% on the right?  Then rest assured that these (at least) are features not bugs and read
% \S\ref{sec:defaults}.
% \begin{forestexample}[pos=t,label=ex:misalignments,ekeynames={phantom,for children,fit,for,baseline,edge,for descendants,content,{id=.pgfmath,nfc}}]
%   \begin{forest}
%     [,phantom,for children={l sep=1ex,fit=band,
%       for=1{edge'=,l=0},baseline}
%       [{l+=5mm},for descendants/.pgfmath=content
%         [AdjP[AdvP][Adj'[Adj][PP]]]]
%       [default
%         [AdjP[AdvP][Adj'[Adj][PP]]]]
%       [{l-=5mm},for descendants/.pgfmath=content
%         [AdjP[AdvP][Adj'[Adj][PP]]]]
%     ]
%     \path (current bounding box.west)|-coordinate(l1)(!212.base);
%     \path (current bounding box.west)|-coordinate(l2)(!2121.base);
%     \path (current bounding box.east)|-coordinate(r1)(!212.base);
%     \path (current bounding box.east)|-coordinate(r2)(!2121.base);
%     \draw[dotted] (l1)--(r1) (l2)--(r2);
%   \end{forest}
%   \hspace{4cm}
%   \raisebox{0pt}[\height][0pt]{\begin{forest}
%     [x forest, baseline
%       [x[x[x[x[x[x[x[x[x[x[x[x[x]]]]]]]]]]]]]
%       [(x)[(x)[(x)[(x)[(x)[(x)[(x)[(x)[(x)[(x)[(x)[(x)[(x)]]]]]]]]]]]]]
%     ]
%   \end{forest}}
% \end{forestexample}
% 
% \subsubsection{The defaults, or the hairy details of vertical alignment}
% \label{sec:defaults}
% 
% In this section we discuss the default values of options controlling the l-alignment of the nodes.
% The defaults are set with top-down trees in mind, so l-alignment is actually vertical alignment.
% There are two desired effects of the defaults.  First, the spacing between the nodes of a tree
% should adjust to the current font size.  Second, the nodes of a given level should be vertically
% aligned (at the base), if possible.
% 
% Let us start with the base alignment: \TikZ;'s default is to anchor the nodes at their center,
% while \foRest;, given the usual content of nodes in linguistic representations, rather anchors them
% at the base \cite[\S16.5.1]{tikzpgf2.10}.  The difference is particularly clear for a ``phonological''
% representation:
% \begin{forestexample}[ekeynames={for tree,anchor}]
%   \begin{forest} for tree={anchor=center}
%     [maybe[m][a][y][b][e]]
%   \end{forest}\quad
%   \begin{forest}
%     [maybe[m][a][y][b][e]]
%   \end{forest}
% \end{forestexample}
% The following example shows that the vertical distance between nodes depends on the current font size.
% \begin{forestexample}
%   \hbox{\small A small tree
%   \begin{forest} baseline
%     [VP[DP][V'[V][DP]]]
%   \end{forest}
%   \normalsize and
%   \large
%   a large tree
%   \begin{forest} baseline
%     [VP[DP][V'[V][DP]]]
%   \end{forest}}
% \end{forestexample}
% Furthermore, the distance between nodes also depends on the value of \PGF;'s |inner_sep| (which
% also depends on the font size by default: it equals 0.3333\,em).
% \[\ikeyname{l sep}=\mbox{height}(\mbox{strut})+\mbox{\texttt{inner ysep}}\]
% The default value of \ikeyname{s sep} depends on |inner_xsep|: more precisely, it equals double
% |inner_xsep|). 
% \begin{forestexample}[ekeynames={baseline,for tree}]
%   \begin{forest} baseline,for tree=draw
%     [VP[DP][V'[V][DP]]]
%   \end{forest}    
%   \pgfkeys{/pgf/inner sep=0.6666em}
%   \begin{forest} baseline,for tree=draw
%     [VP[DP][V'[V][DP]]]
%   \end{forest}    
% \end{forestexample}
% Now a hairy detail: the formula for the default \ikeyname{l}.
% \[\ikeyname{l}=\ikeyname{l sep}+2\cdot\mbox{\texttt{outer ysep}}+\mbox{total
% height}(\mbox{`dj'})\]
%
% To understand what this is all about we must first explain why it is necessary to set the default
% \ikeyname{l} at all?  Wouldn't it be enough to simply set \ikeyname{l sep} (leaving
% \ikeyname{l} at 0)?
% The problem is that not all letters have the same height and depth.  A tree where the vertical
% position of the nodes would be controlled solely by (a constant) \ikeyname{l sep} could
% result in a ragged tree (although the height of the child--parent edges would be constant).
% \begin{forestexample}[ekeynames={baseline,for children,no edge,name,for descendants,l}]
%   \begin{forest}
%     [default,baseline,for children={no edge}
%       [DP
%         [AdjP[Adj]]
%         [D'[D][NP,name=np]]]]
%     \path (current bounding box.west)|-coordinate(l)(np.base);
%     \path (current bounding box.east)|-coordinate(r)(np.base);
%     \draw[dotted] (l)--(r);
%   \end{forest}
%   \begin{forest}
%     [{l=0},baseline,for children={no edge}    
%       [DP,for descendants={l=0}
%         [AdjP[Adj]]
%         [D'[D][NP,name=np]]]]
%     \path (current bounding box.west)|-coordinate(l)(np.base);
%     \path (current bounding box.east)|-coordinate(r)(np.base);
%     \draw[dotted] (l)--(r);
%   \end{forest}
% \end{forestexample}
% The vertical misalignment of Adj in the right tree is a consequence of the fact that letter j is the
% only letter with non-zero depth in the tree.  Since only \ikeyname{l sep} (which is constant
% throughout the tree) controls the vertical positioning, Adj, child of Ad\emph{j}P, is pushed lower
% than the other nodes on level 2.  If the content of the nodes is variable enough (various heights
% and depths), the cumulative effect can be quite strong, see the right tree of example
% \ref{ex:misalignments}.
% 
% Setting only a default \ikeyname{l sep} thus does not work well enough in general.  The same
% is true for the reverse possibility, setting a default \ikeyname{l} (and leaving \ikeyname{l sep} at 0).  In the example below, the depth of the multiline node (anchored at the top
% line) is such that the child--parent edges are just too short if the level distance is kept constant.
% Sometimes, misalignment is much preferred \dots
% \begin{forestexample}[ekeynames={align,{center,aspect=align},for tree,l sep}]
%   \mbox{}\begin{forest}
%     [default
%       [first child[a][b][c]]
%       [second child\\\scriptsize(a copy),
%        align=center[a][b][c]]
%     ]
%   \end{forest}\\
%   \begin{forest} for tree={l sep=0}
%     [{\texttt{l sep}=0}
%       [first child[a][b][c]]
%       [second child\\\scriptsize(a copy),
%                    align=center[a][b][c]]
%     ]
%   \end{forest}
% \end{forestexample}
% 
% Thus, the idea is to make \ikeyname{l} and \ikeyname{l sep} work as a team:
% \ikeyname{l} prevents 
% misalignments, if possible, while \ikeyname{l sep} determines the minimal vertical distance
% between levels.  Each of the two options deals with a certain kind of a ``deviant'' node, i.e.\ a
% node which is too high or too deep, or a node which is not high or deep enough, so we need to
% postulate what a \emph{standard} node is, and synchronize them so that their effect on standard
% nodes is the same.
% 
% By default, \foRest; sets the standard node to be a node containing letters d and j.  Linguistic
% representations consist mainly of letters, and in the \TeX's default Computer Modern font, d is the
% highest letter (not character!), and j the deepest, so this decision guarantees that trees
% containing only letters will look nice.  If the tree contains many parentheses, like the right
% tree of example \ref{ex:misalignments}, the default will of course fail
% and the standard node needs to be modified.  But for many applications, including nodes with
% indices, the default works.
% 
% The standard node can be changed using macro \icmdname{forestStandardNode};
% see \ref{ref:standard-node}. 
% 
% \subsection{Advanced option setting}
% \label{sec:advanced-option-setting}
%
% We have already seen that the value of options can be manipulated: in \ref{ex:romannumeral} we have
% converted numeric content from arabic into roman numerals using the \emph{wrapping} mechanism
% |content=\romannumeral#1|; in \ref{ex:l*}, we have tripled the value of |l|
% by saying |l*=3|.  In this section, we will learn about the mechanisms for setting and
% referring to option values offered by \foRest;.
%
% One other way to access an option value is using macro \icmdname{forestoption}.  The macro takes a
% single argument: an option name.  (For details, see \S\ref{ref:options-and-keys}.)  In the
% following example, the node's child sequence number is appended to the existing content. (This is
% therefore also an example of wrapping.)
% \begin{forestexample}[label=ex:forestoption,ekeynames={phantom,delay,for descendants,content,n},ecmdnames={forestoption}]
%   \begin{forest}
%     [,phantom,delay={for descendants={
%       content=#1$_{~\forestoption~{n}}$}}
%     [c][o][u][n][t]]
%   \end{forest}
% \end{forestexample}
%
% However, only options of the current node can be accessed using \icmdname{forestoption}.  To
% access option values of other nodes, \foRest;'s extensions to the \PGF;'s mathematical library
% |pgfmath|, documented in \citep[part VI]{tikzpgf2.10}, must be used.  To see |pgfmath| in action,
% first take a look at the crazy tree on the title page, and observe how the nodes are
% rotated: the value given to (\TikZ;) option \texttt{rotate} is a full-fledged |pgfmath| expression
% yielding an integer
% in the range from $-30$ to $30$.  Similiarly, \ikeyname{l}\ikeyname{+} adds a random float
% in the $[-5,5]$ range to the current value of \ikeyname{l}.
%
% Example \ref{ex:spread-s} demonstrated that information about
% the node, like the node's level, can be accessed within |pgfmath| expressions.  All
% options are accessible in this way, i.e.\ every option has a corresponding |pgfmath| function.
% For example, we could rotate the node based on its content:
% \begin{forestexample}[ekeynames={delay,for tree,rotate,content}]
%   \begin{forest}
%     delay={for tree={~rotate=content~}}
%     [30[-10[5][0]][-90[180]][90[-60][90]]]
%   \end{forest}
% \end{forestexample}
% 
% All numeric, dimensional and boolean options of \foRest; automatically pass the given value
% through |pgfmath|.  If you need pass the value through |pgfmath| 
% for a string option, use the \ikeyname{id=.pgfmath,nfc} handler.  The following example sets the node's
% content to its child sequence number (the root has child sequence number 0).
% \begin{forestexample}[ekeynames={delay,for tree,content,n,{id=.pgfmath,nfc}}]
%   \begin{forest}
%     delay={for tree={content/~.pgfmath~=int(n)}}
%     [[[][][]][[][]]]
%   \end{forest}
% \end{forestexample}
%
% As mentioned above, using |pgfmath| it is possible to access options of non-current nodes.  This
% is achieved by providing the option function with a \imeta{relative node name}
% (see~\S\ref{ref:relative-node-names}) argument.\footnote{The form without
% parentheses \texttt{option\string_name} that we have been using until now to refer to an option of
% the 
% current node is just a short-hand notation for \texttt{option\string_name()} --- note that in some
% contexts, like preceding \texttt{+} or \texttt{-}, the short form does not work! (The same
% seems to be true for all pgfmath functions with ``optional'' arguments.)}  In the next example, we
% rotate the node based on the content of its parent.
% \begin{forestexample}[ekeynames={delay,for tree,rotate,content,u}]
%   \begin{forest}
%     delay={for descendants={rotate=content~("!u")~}}
%     [30[-10[5][0]][-90[180]][90[-60][90]]]
%   \end{forest}
% \end{forestexample}
% Note that the argument of the option function is surrounded by double quotation marks: this is
% to prevent evaluation of the relative node name as a |pgfmath| function --- which it is not.
%
% Handlers \ikeyname{id=.wrap pgfmath arg,nfc} and \ikeyname{id=.wrap $n$ pgfmath args,nfc}
% (for $n=2,\dots,8$) combine the wrapping mechanism with the |pgfmath| evaluation.  The
% idea is to compute (most often, just access option values) arguments using |pgfmath| and then
% wrap them with the given macro.  Below, this is used to include the number of parent's children in
% the index. 
% \begin{forestexample}[ekeynames={phantom,delay,for descendants,content,n,n children,{id=.wrap 3 pgfmath args,nfc}}]
%   \begin{forest} [,phantom,delay={for descendants={
%               ~content/.wrap 3 pgfmath args=
%               {#1$_{#2/#3}$}{content}{n}{n_children("!u")}~}}
%     [c][o][u][n][t]]
%   \end{forest}
% \end{forestexample}
% Note the underscore |__| character in |n__children|: in |pgfmath| function names, spaces,
% apostrophes and other non-alphanumeric characters from option names are all replaced by
% underscores. 
% 
% As another example, let's make the numerals example \ref{ex:numerals-simple} a bit fancier.
% The numeral type is read off the parent's content and used to construct the appropriate control
% sequence (|\@arabic|, |\@roman| and |\@alph|).  (Also, the numbers are not specified in content
% anymore: we simply read the sequence number \ikeyname{n}.  And, to save some horizontal space for the
% code, each child of the root is pushed further down.)
% \begin{forestexample}[ekeynames={delay,where ,level,content,n,for children,l,{id=.wrap 2 pgfmath args,nfc}}]
%   \begin{forest}
%     delay={where level={2}{~content/.wrap 2 pgfmath args=
%         {\csname @#1\endcsname{#2}}{content("!u")}{n}~}{}},
%     for children={l*=n},
%     [\LaTeX numerals,
%       [arabic[][][][]]
%       [roman[][][][]]
%       [alph[][][][]]
%     ]
%   \end{forest}
% \end{forestexample}
%
% The final way to use |pgfmath| expressions in \foRest;: \ikeyname{if} clauses.  In
% section~\ref{sec:options}, we have seen that every option has a corresponding \ikeyname{id={{if }}}|...|
% (and \ikeyname{id={{where }}}|...|) option.  However, these are just a matter of convenience.  The full
% power resides 
% in the general \ikeyname{if} option, which takes three arguments: 
% |if=|\meta{condition}\meta{true options}\meta{false options}, where \meta{condition} can be any
% |pgfmath| expression 
% (non-zero means true, zero means false).  (Once again, option \ikeyname{where} is an abbreviation
% for \ikeyname{for tree}|={|\ikeyname{if}|=...}|.)  In the following example, \ikeyname{if} option
% is used to orient the 
% arrows from the smaller number to the greater, and to color the odd and even numbers differently.
%
% \forestset{random tree/.style n args={3}{^^A #1=max levels, #2=max children, #3=max content
%       content/.pgfmath={random(0,#3)},
%       if={#1>0}{repeat={random(0,#2)}{append={[,random tree={#1-1}{#2}{#3}]}}}{}}}
% \begin{forestexample}[ekeynames={before typesetting nodes,for descendants,if,content,edge,edge label,for tree,if},point=if]
%   \pgfmathsetseed{314159}
%   \begin{forest}
%     before typesetting nodes={
%       for descendants={
%         if={content()>content("!u")}{edge=->}{
%           if={content()<content("!u")}{edge=<-}{}},
%           edge label/.wrap pgfmath arg=
%             {node[midway,above,sloped,font=\scriptsize]{+#1}}
%             {int(abs(content()-content("!u")))}
%       },
%       for tree={circle,if={mod(content(),2)==0}
%                           {fill=yellow}{fill=green}}
%     }
%     [,random tree={3}{3}{100}]
%   \end{forest}
% \end{forestexample}
%
% This exhausts the ways of using |pgfmath| in forest.  We continue by introducing \emph{relative
% node setting}: write \imeta{relative node name}|.|\meta{option}|=|\meta{value} to set the
% value of \meta{option} of the specified relative node. Important: computation (pgfmath or wrap) of
% the value is 
% done in the context of the original node.  The following example defines style \keyname{move} which
% not only draws an arrow from the source to the target, but also moves the content of the source
% to the target (leaving a trace).  Note the difference between |#1| and |##1|: |#1| is the argument
% of the style \keyname{move}, i.e.\ the given node walk, while |##1| is the original option value
% (in this case, content).
% \begin{forestexample}[ekeynames={for tree,calign,tikz,delay,content}]
%   \begin{forest}
%     for tree={calign=fixed edge angles},
%     move/.style={
%       tikz={\draw[->] () to[out=south west,in=south] (#1);},
%       delay={~#1.content~={##1},content=$t$}},
%     [CP[][C'[C][\dots[,phantom][VP[DP][V'[V][DP,move=!r1]]]]]]
%   \end{forest}
% \end{forestexample}
%
% In the following example, the content of the branching nodes is computed by \foRest;: a branching
% node is a sum of its 
% children.  Besides the use of the relative node setting, this example notably uses a recursive
% style: for each child of the node, style \keyname{calc} first applies itself to the child and then
% adds the result to the node; obviously, recursion is made to stop at terminal nodes. 
% \begin{forestexample}[ekeynames={id={{if }},n children,content,for children,delay,{id=.pgfmath,nfc}}]
%   \begin{forest}
%     calc/.style={if n children={0}{}{content=0,for children={
%           calc,~!u.content~/.pgfmath=int(content("!u")+content())}}},
%     delay=calc,
%     [[[3][4][5]][[3][9]][8][[[1][2][3]]]]
%   \end{forest}
% \end{forestexample}
% 
% 
% \subsection{Externalization}
% \label{tut:externalization}
%
% \FoRest; can be quite slow, due to the slowness of both \PGF;/\TikZ; and its own computations.
% However, using \emph{externalization}, the amount of time spent in \foRest; in everyday life can
% be reduced dramatically.  The idea is to typeset the trees only once, saving them in separate
% PDFs, and then, on the subsequent compilations of the document, simply include these PDFs instead
% of doing the lenghty tree-typesetting all over again.
%
% \FoRest;'s externalization mechanism is built on top of \TikZ;'s |external| library.  It
% enhances it by automatically detecting the code and context changes: the tree is recompiled if and
% only if either the code in the \ikeyname{forest} environment or the context (arbitrary parameters; by
% default, the parameters of the standard node) changes.
%
% To use \foRest;'s externalization facilities, say:\footnote{When you switch on
% the externalization for a document containing many \keyname{forest} environments, the first
% compilation can take quite a while, much more than the compilation without externalization. (For
% example, more than ten minutes for the document you are reading!)  Subsequent compilations,
% however, will be very fast.}\ekeyname{external}
% \begin{lstlisting}[point=external]
%   \usepackage[external]{forest}
%   ~\tikzexternalize~
% \end{lstlisting}
%
% If your \ikeyname{forest} environment contains some macro, you will probably want the externalized
% tree to be recompiled when the definition of the macro changes.  To achieve this, use
% \icmdname{forestset}|{|\ikeyname{id={external/depends on macro}}|=|\cmdname{macro}|}|.  The effect is
% local to the \TeX\ group.
%
% \TikZ;'s externalization library promises a |\label| inside the externalized graphics to work
% out-of-box, while |\ref| inside the externalized graphics should work only if the externalization
% is run manually or by |make| \citep[\S32.4.1]{tikzpgf2.10}.  A bit surprisingly perhaps, the
% situation is roughly reversed in \foRest;.  |\ref| inside the externalized graphics will work
% out-of-box.  |\label| inside the externalized graphics will not work at all.  Sorry.  (The reason
% is that \foRest; prepares the node content in advance, before merging it in the whole tree, which
% is when \TikZ;'s externalization is used.)
%
% \subsection{Expansion control in the bracket parser}
% \label{tut:bracket}
%
% By default, macros in the bracket encoding of a tree are not
% expanded until nodes are being drawn --- this way, node
% specification can contain formatting instructions, as illustrated in
% section~\ref{sec:basic-usage}.  However, sometimes it is useful to
% expand macros while parsing the bracket representation, for example to
% define tree templates such as the X-bar template, familiar
% to generative grammarians:\footnote{Honestly, dynamic node creation might be a better way to do
% this; see~\S\ref{ref:dynamic}.}
% \begin{forestexample}[ecmdnames=bracketset,ekeynames={action character}]
%   ~\bracketset{action character=@}~
%   \def\XP#1#2#3{#1P[#2][#1'[#1][#3]]}
%   \begin{forest}
%     [~@~\XP T{DP}{~@~\XP V{DP}{DP}}]
%   \end{forest}
% \end{forestexample}
% In the above example, the |\XP| macro is preceded by the \emph{action character} |@|: as
% the result, the token following the action character was expanded before the parsing proceeded.
%
% The action character is not hard coded into \foRest;.  Actually, there is no action character by
% default.  (There's enough special characters in \foRest; already, anyway, and the situations where
% controlling the expansion is preferable to using the pgfkeys interface are not numerous.)  It is
% defined at the top of the example by processing key \ikeyname{action character} in the
% \ikeyname{id={/bracket},nfc} path; the definition is local to the \TeX\ group.
% 
% Let us continue with the description of the expansion control facilities of the bracket parser.
% The expandable token following the
% action character is expanded only once.  Thus, if one defined macro
% |\VP| in terms of the general |\XP| and tried to use it in the same
% fashion as |\XP| above, he would fail.  The correct way is to follow
% the action character by a braced expression: the braced expression
% is fully expanded before bracket-parsing is resumed.
% \begin{forestexample}[ecmdnames=bracketset,ekeynames=action character]
%   \bracketset{action character=@}
%   \def\XP#1#2#3{#1P[#2][#1'[#1][#3]]}
%   \def\VP#1#2{\XP V{#1}{#2}}
%   \begin{forest}
%     [@\XP T{DP}{~@{~\VP{DP}{DP}~}~}]
%   \end{forest}
% \end{forestexample}
%
% In some applications, the need for macro expansion might be much
% more common than the need to embed formatting instructions.
% Therefore, the bracket parser provides commands |@+| and |@-|: |@+|
% switches to full expansion mode --- all tokens are fully expanded
% before parsing them; |@-| switches back to the default mode, where
% nothing is automatically expanded.
% \begin{forestexample}[ecmdnames=bracketset,ekeynames=action character]
%   \bracketset{action character=@}
%   \def\XP#1#2#3{#1P[#2][#1'[#1][#3]]}
%   \def\VP#1#2{\XP V{#1}{#2}}
%   \begin{forest} ~@+~
%     [\XP T{DP}{\VP{DP}{DP}}]
%   \end{forest}
% \end{forestexample}
% 
% All the action commands discussed above were dealing only with
% \TeX's macro expansion.  There is one final action command, |@@|,
% which yields control to the user code and expects it to call
% |\bracketResume| to resume parsing.  This is useful to e.g.\
% implement automatic node enumeration:
% \begin{forestexample}[ecmdnames=bracketset,ekeynames={action character,phantom,for
% ,n,baseline,delay,where ,level,content}]
%   \bracketset{action character=@}
%   \newcount\xcount
%   \def\x#1{~@@~\advance\xcount1
%     \edef\xtemp{[$\noexpand\times_{\the\xcount}$[#1]]}%
%     \expandafter\bracketResume\xtemp
%   }
%   \begin{forest}
%     phantom,
%     delay={where level=1{content={\strut #1}}{}}
%     ~@+~
%     [\x{f}\x{o}\x{r}\x{e}\x{s}\x{t}]
%   \end{forest}
% \end{forestexample}
% This example is fairly complex, so let's discuss how it works.  |@+| switches to the full
% expansion mode, so that macro |\x| can be easily run.  The real magic hides in this macro.  In
% order to be able to advance the node counter |\xcount|, the macro takes control from \foRest; by
% the |@@| command.  Since we're already in control, we can use |\edef| to define the node content.
% Finally, the |\xtemp| macro containing the node specification is expanded with the resume command
% sticked in front of the expansion.
%
% \section{Reference}
% \label{sec:reference}
%
% \subsection{Environments}
% \label{ref:environments}
%
% \begin{syntax}
% \item[,,environment]|\begin{|\rkeyname{forest}|}|\meta{tree}|\end{|\rkeyname{forest}|}|
% \rcmdname[item]{Forest}[*]\marg{tree}
%
% The environment and the starless version of the macro introduce a group; the starred macro does
% not, so the created nodes can be used afterwards.  (Note that this will leave a lot of temporary
% macros lying around.  This shouldn't be a problem, however, since all of them reside in the
% |\forest| namespace.)
% \end{syntax}
%
% \subsection{The bracket representation}
% \label{ref:bracket}
%
% A bracket representation of a tree is a token list with the following syntax:
% \begin{eqnarray*}
%   \meta{tree}&=&\left[\meta{preamble}\right]\meta{node}\\
%   \meta{node}&=&\texttt{[}\left[\meta{content}\right]\left[\texttt{,}\meta{keylist}\right]
%                  \left[\meta{children}\right]\texttt{]}\meta{afterthought}\\
%   \meta{preamble}&=&\meta{keylist}\\
%   \meta{keylist}&=&\meta{key--value}\left[,\meta{keylist}\right]\\
%   \meta{key--value}&=&\meta{key}\OR\meta{key}\texttt{=}\meta{value}\\
%   \meta{children}&=&\meta{node}\left[\meta{children}\right]
% \end{eqnarray*}
% 
% The actual input might be different, though, since expansion may have occurred during the input
% reading.  Expansion control sequences of \foRest;'s bracket parser are shown below.
% \begin{center}
%   \begin{tabular}{ll}
%     \rstyle\meta{action character}\texttt{-}&no-expansion mode (default): nothing is expanded\\
%     \rstyle\meta{action character}\texttt{+}&expansion mode: everything is fully expanded\\
%     \rstyle\meta{action character}\texttt{}\meta{token}&expand \meta{token}\\
%     \rstyle\meta{action character}\texttt{}\meta{\TeX-group}&fully expand \meta{\TeX-group}\\
%     \rstyle\meta{action character}\meta{action character}&yield control;\\&upon finishing its job,
%     user's code should call \texttt{\string\bracketResume}
%   \end{tabular}
% \end{center}
% 
% \paragraph{Customization} To customize the bracket parser, call
% \rcmdname{bracketset}\meta{keylist}, where the keys can be the following. 
% \begin{syntax}
% \rkeyname[item={[}]{opening bracket}|=|\meta{character}
% \rkeyname[item={{{{]}}}}]{closing bracket}|=|\meta{character}
% \rkeyname[item=none]{action character}|=|\meta{character}
% \end{syntax}
% 
% By redefining the following two keys, the bracket parser can be used outside \foRest;.
% \begin{syntax}
% \rkeyname[item]{new node}|=|\meta{preamble}\meta{node specification}\meta{csname}.
%   Required semantics: create a new node given the preamble (in the case of a new
%   root node) and the node specification and store the new node's id into \meta{csname}. 
% \rkeyname[item]{set afterthought}|=|\meta{afterthought}\meta{node id}.
%   Required semantics: store the afterthought in the node with given id.
% \end{syntax}
%
% \subsection{Options and keys}
% \label{ref:option-types}
% \label{ref:options-and-keys}
% 
% The position and outlook of nodes is controlled by \emph{options}.  Many options can be set for a
% node.  \emph{Each node's options are set independently of other nodes:} in particular, setting an
% option of a node does \emph{not} set this option for the node's descendants.  
%
% Options are set using \PGF;'s key management utility |pgfkeys| \citep[\S55]{tikzpgf2.10}.  In the
% bracket representation of a tree (see~\S\ref{ref:bracket}), each node can be given a
% \meta{keylist}.  After parsing the representation of the tree, the keylists of the 
% nodes are processed (recursively, in a depth-first, parent-first fashion).  The preamble is
% processed first, in 
% the context of the root node.\footnote{The value of a key (if it is given) is interpreted as one
% or more arguments to the key command. 
% If there is only one argument, the situation is simple: the whole value is the argument.  When the
% key takes more than one argument, each argument should be enclosed in braces, unless, as usual in
% \TeX, the argument is a single token.  (The pairs of braces can be separated by whitespace.)  An
% argument should also be enclosed in braces if it contains a special character: a comma \texttt{,}, an
% equal sign \texttt{=} or a bracket \texttt{[]}.}
%
% The node whose keylist is being processed is the \emph{current node}.  During the processing of
% the keylist, the current node can temporarily change.  This mainly happens when propagators
% (\S\ref{ref:propagators}) are being processed.
% 
% Options can be set in various ways, depending on the option type (the types are listed below).
% The most straightforward way is to use the key with the same name as the option:
% \begin{syntax}
% \item \meta{option}|=|\meta{value}  Sets the value of \meta{option} of the current node to
% \meta{value}.
%
%   Notes: (i) Obviously, this does not work for read-only options.  (ii) Some option types override
%   this behaviour.
% \end{syntax}
% It is also possible to set a non-current option:
% \begin{syntax}
% \item
%   \imeta{relative node name}|.|\meta{option}|=|\meta{value}  Sets the value of
%   \meta{option} of the node specified by \meta{relative node name} to \meta{value}.
%
%   Notes: \begin{inparaenum}[(i)]
%   \item\emph{\meta{value} is evaluated in the context of the current node.}
%   \item In general, the resolution of \meta{relative node name} depends on the
%     current node; see \S\ref{ref:relative-node-names}.
%   \item \meta{option} can also be an ``augmented operator'' (see below) or an additional
%     option-setting key defined for a specific option.
%   \end{inparaenum}
% \end{syntax}
% The option values can be not only set, but also read.
% \begin{itemize}
% \item Using macros \rcmdname{forestoption}|{|\meta{option}|}| and
%   \rcmdname{foresteoption}|{|\meta{option}|}|, options of the current node can be accessed in \TeX\
%   code.  (``\TeX\ code'' includes \meta{value} expressions!).
%
%   In the context of |\edef| or \PGF;'s handler |.expanded| \citep[\S55.4.6]{tikzpgf2.10},
%   \cmdname{forestoption} expands precisely to the token list of the option value, while
%   \cmdname{foresteoption} allows the option value to be expanded as well.
% \item Using |pgfmath| functions defined by \foRest;, options of both current and non-current nodes
%   can be accessed.  For details, see \S\ref{ref:pgfmath}.
% \end{itemize}
%
% We continue with listing of all keys defined for every option.  The set of defined keys and their
% meanings depends on the option type.  Option types and the type-specific keys can be found in the
% list below.  Common to all types are two simple conditionals, \ikeyname{id={{if }}}\meta{option}
% and \ikeyname{id={{where }}}\meta{option}, which are  
% defined for every \meta{option}; for details, see \S\ref{ref:conditionals}.  
% 
% \begin{syntax}
% \rmeta[item]{toks} contains \TeX's \meta{balanced text} \citep[275]{texbook}.
%
%   A toks \meta{option} additionally defines the following keys:
%   \begin{syntax}
%   \item {\rstyle\meta{option}}\rkeyname{+}|=|\meta{toks} appends the given \meta{toks} to the
%     current value of the option.
%     
%   \item {\rstyle\meta{option}}\rkeyname{-}|=|\meta{toks} prepends the given \meta{toks} to the
%     current value of the option.
%     
%     \rkeyname[margin={},item]{id={{if in }}}{\rstyle\meta{option}}|=|\meta{toks}\meta{true
%     keylist}\meta{false keylist} checks if \meta{toks} occurs in the option value; if it does,
%     \meta{true keylist} are executed, otherwise \meta{false keylist}.
%     
%     \rkeyname[margin={},item]{id={{where in }}}\meta{option}|=|\meta{toks}\meta{true
%     keylist}\meta{false keylist} is a style equivalent to \ikeyname{for tree}|={|\keyname{if in }\meta{option}=\meta{toks}\meta{true keylist}\meta{false keylist}|}|: for every node in
%     the subtree rooted in the current node, \keyname{if in }\meta{option} is executed in
%     the context of that node.
%   \end{syntax}
% 
% \rmeta[item]{autowrapped toks} is a subtype of \imeta{toks} and contains \TeX's \meta{balanced
% text} \citep[275]{texbook}. 
%
%   {\rstyle\meta{option}}|=|\meta{toks} of an autowrapped \meta{option} is equivalent to
%   \meta{option}|/|\ikeyname{id=.wrap value,nfc}|=|\meta{toks} of a normal \meta{toks} option.
%   
%   Keyvals {\rstyle\meta{option}}\rkeyname{+}|=|\meta{toks} and
%   {\rstyle\meta{option}\rkeyname{-}}|=|\meta{toks} are equivalent to
%   \meta{option}\keyname{+}|/|\ikeyname{id=.wrap value,nfc}|=|\meta{toks} and
%   \meta{option}\keyname{-}|/|\ikeyname{id=.wrap value,nfc}|=|\meta{toks}, respectively.  The
%   normal toks behaviour can be accessed via keys {\rstyle\meta{option}|'|},
%   {\rstyle\meta{option}|+'|} and {\rstyle\meta{option}|-'|}.  
%
% \rmeta[item]{keylist} is a subtype of \imeta{toks} and contains a comma-separated list of
%   \meta{key}[|=|\meta{value}] pairs.
% 
%   Augmented operators {\rstyle\meta{option}\keyname{+}} and {\rstyle\meta{option}\keyname{-}} automatically
%   insert a comma before/after the appended/prepended material.
%
%   {\rstyle\meta{option}}|=|\meta{keylist} of a keylist option is equivalent to
%   \meta{option}\keyname{+}|=|\meta{keylist}.  In other words, keylists behave additively by
%   default.  The rationale is that one usually wants to add keys to a keylist.  The usual,
%   non-additive behaviour can be accessed by {\rstyle\meta{option}\rkeyname{'}}|=|\meta{keylist}.
%
% \rmeta[item]{dimen} contains a dimension.
%
%   The value given to a dimension option is automatically evaluated by pgfmath.  In other words:
%
%   {\rstyle\meta{option}}|=|\meta{pgfmath} is an implicit \meta{option}|/.pgfmath=|\meta{pgfmath}.
%
%   For a \meta{dimen} option \meta{option}, the following additional keys (``augmented
%   assignments'') are defined:
%   \begin{itemize}
%   \item {\rstyle\meta{option}\rkeyname{+}}|=|\meta{value} is equivalent to \meta{option}|=|\meta{option}|()+|\meta{value}
%   \item {\rstyle\meta{option}\rkeyname{-}}|=|\meta{value} is equivalent to \meta{option}|=|\meta{option}|()-|\meta{value}
%   \item {\rstyle\meta{option}\rkeyname{*}}|=|\meta{value} is equivalent to \meta{option}|=|\meta{option}|()*|\meta{value}
%   \item {\rstyle\meta{option}\rkeyname{:}}|=|\meta{value} is equivalent to \meta{option}|=|\meta{option}|()/|\meta{value}
%   \end{itemize}
% 
%   The evaluation of \meta{pgfmath} can be quite slow.  There are two tricks to speed things up
%   \emph{if} the \meta{pgfmath} expression is simple, i.e.\ just a \TeX\ \meta{dimen}:
%   \begin{enumerate}
%   \item |pgfmath| evaluation of simple values can be sped up by prepending \ikeyname{+} to the value
%     \citep[\S62.1]{tikzpgf2.10};
%   \item use the key {\rstyle\meta{option}\rkeyname{'}}|=|\meta{value} to invoke a normal \TeX\ assignment.
%   \end{enumerate}
%
%   The two above-mentioned speed-up tricks work for the augmented assignments as well.  The keys
%   for the second, \TeX-only trick are: {\rstyle\meta{option}\rkeyname{'+}},
%   {\rstyle\meta{option}\rkeyname{'-}}, {\rstyle\meta{option}\rkeyname{'*}} and
%   {\rstyle\meta{option}\rkeyname{':}} --- note that for the latter two, the value should be an
%   integer.
% 
% \rmeta[item]{count} contains an integer.
%
%   The additional keys and their behaviour are the same as for the \meta{dimen} options.
%
% \rmeta[item]{boolean} contains $0$ (false) or $1$ (true).
%
%   In the general case, the value given to a \meta{boolean} option is automatically
%   parsed by pgfmath (just as for \meta{count} and \meta{dimen}): if the computed value is
%   non-zero, $1$ is stored; otherwise, $0$ is stored.  Note that |pgfmath| recognizes constants
%   |true| and |false|, so it is possible to write \meta{option}|=true| and
%   \meta{option}|=false|.
%
%   If key \meta{option} is given no argument, pgfmath evaluation does not apply and a true value is
%   set.  To quickly set a false value, use key {\rstyle\rkeyname{id={{not }}}\meta{option}} (with
%   no arguments). 
% \end{syntax}
% 
% The following subsections are a complete reference to the part of the user interface residing in
% the |pgfkeys|' path \keyname{/forest}.  In plain language, they list all the options known to
% \foRest;.  More precisely, however, not only options are listed, but also other keys, such as
% propagators, conditionals, etc.
%
% Before listing the keys, it is worth mentioning that users can also define their own keys.  The
% easiest way to do this is by using \emph{styles}.  Styles are a feature of the |pgfkeys| package.
% They are 
% named keylists, whose usage ranges from mere abbreviations through templates to devices
% implementing recursion.  To define a style, use \PGF;'s handler \keyname{.style}
% \citep[\S55.4.4]{tikzpgf2.10}: \meta{style name}|/.style=|\meta{keylist}.
%
% Using the following keys, users can also declare their own options.  The new options will behave
% exactly like the predefined ones.
% \begin{syntax}
% \rkeyname[item]{declare toks}|=|\meta{option name}\meta{default value}  Declares a \meta{toks} option.
% \rkeyname[item]{declare autowrapped toks}|=|\meta{option name}\meta{default value} Declares an
% \meta{autowrapped toks} option.
% \rkeyname[item]{declare keylist}|=|\meta{option name}\meta{default value} Declares a
% \meta{keylist} option. 
% \rkeyname[item]{declare dimen}|=|\meta{option name}\meta{default value} Declares a \meta{dimen} option.
% \rkeyname[item]{declare count}|=|\meta{option name}\meta{default value} Declares a \meta{count} option.
% \rkeyname[item]{declare boolean}|=|\meta{option name}\meta{default value} Declares a
% \meta{boolean} option.
% \end{syntax}
%
% The style definitions and option declarations given
% among the other keys in the bracket specification are local to the current tree.  To define
% globally accessible styles and options (well, definitions are always local to the current \TeX\
% group), use macro |\forestset| outside the
% \ikeyname{forest} environment:\footnote{\cmdname{forestset}\meta{keylist} is equivalent to
% \cmdname{pgfkeys}\texttt{\{}/forest,\meta{keylist}\texttt{\}}.}
% \begin{syntax}
% \rcmdname[item]{forestset}\marg{keylist}
%   
%   Execute \meta{keylist} with the default path set to \keyname{/forest}.
%   \begin{advise}
%   \item Usually, no current node is set when this macro is called.  Thus, executing node options
%     in this place will \emph{fail}. However, if you have some nodes lying around, you can use
%     propagator \ikeyname{for name}|=|\meta{node name} to set the node with the given name as
%     current.
%   \end{advise}
% \end{syntax}
%
% \subsubsection{Node appearance}
% \label{ref:node-appearance}
%
% The following options apply at stage \ikeyname{typesetting nodes}. Changing them
% afterwards has no effect in the normal course of events. 
%
% \begin{syntax}
%   \rkeyname[item={{{{{}}}}}]{align}|=|\keyname{left,aspect=align}\OR\keyname{center,aspect=align}\OR\keyname{right,aspect=align}\OR\meta{toks: tabular header} 
%
%   Creates a left/center/right-aligned multiline node, or a tabular node.  In the
%   \ikeyname{content} option, the lines of the node should separated by |\\| and the columns (if
%   any) by |&|, as usual.
%
%   The vertical alignment of the multiline/tabular node can be specified by option \ikeyname{base}.
%
% \begin{forestexample}[ekeynames={l sep,align,base}]
%   \begin{forest} l sep+=2ex
%     [special value&actual value\\\hline
%       \rkeyname{left,aspect=align}&||\texttt{@\{\}l@\{\}}\\
%       \rkeyname{center,aspect=align}&||\texttt{@\{\}c@\{\}}\\
%       \rkeyname{right,aspect=align}&||\texttt{@\{\}r@\{\}}\\
%       ,~align~=ll,draw
%       [top base\\right aligned, ~align~=right,~base~=top]
%       [left aligned\\bottom base, ~align~=left,~base~=bottom]
%     ]
%   \end{forest}
% \end{forestexample}
% 
%   Internally, setting this option has two effects:
%   \begin{enumerate}
%   \item The option value (a |tabular| environment header specification) is set.  The special
%     values \keyname{left}, \keyname{center} and \keyname{right} invoke styles setting the actual
%     header to the value shown in the above example.
%     \begin{advise}
%     \item If you know that the \keyname{align} was set with a special value, you can easily check
%       the value using \ikeyname{id={{if in }}}\ikeyname{align}.
%     \end{advise}
%   \item Option \ikeyname{content format} is set to the following value:
%     \begin{lstlisting}
%       \noexpand\begin{tabular}[\forestoption{base}]{\forestoption{align}}%
%         \forestoption{content}%
%       \noexpand\end{tabular}%
%     \end{lstlisting}
%     As you can see, it is this value that determines that options \keyname{base}, \keyname{align} and
%     \keyname{content} specify the vertical alignment, header and content of the table.
%   \end{enumerate}
% 
% \rkeyname[item=t]{base}|=|\meta{toks: vertical alignment}
%
% This option controls the vertical alignment of multiline (and in general, \texttt{tabular}) nodes
% created with \ikeyname{align}.  Its value becomes the optional argument to the \texttt{tabular}
% environment.  Thus, sensible values are \rkeyname{t,aspect=base} (the top line of the table will
% be the baseline) and \rkeyname{b,aspect=base} (the bottom line of the table will be the baseline).
% Note that this will only have effect if the node is anchored on a baseline, like in the default
% case of \ikeyname{anchor}|=base|.
%
% For readability, you can use \rkeyname{top,aspect=base} and \rkeyname{bottom,aspect=base} instead
% of \keyname{t} and \keyname{b}.  (\keyname{top} and \keyname{bottom} are still stored as
% \keyname{t} and \keyname{b}.) 
%
%   \rkeyname[item={{{{{}}}}}]{content}|=|\meta{autowrapped toks} The content of the node.
%
%   Normally, the value of option \keyname{content} is given implicitely by virtue of the special
%   (initial) position of content in the bracket representation (see~\S\ref{ref:bracket}).  However,
%   the option also be set explicitely, as any other option.
%
% \begin{forestexample}[ekeynames={for tree,id={{if }},n,n'},point={content,delay},ekeynames={content,delay}]
%   \begin{forest}
%     delay={for tree={
%         if n=1{content=L}
%              {if n'=1{content=R}
%                       {content=C}}}}
%     [[[][][]][[][][]]]
%   \end{forest}
% \end{forestexample}
%   Note that the execution of the \keyname{content} option should usually be delayed: otherwise, the
%   implicitely given content (in the example below, the empty string) will override the explicitely
%   given content.
%
% \begin{forestexample}[ekeynames={for tree,id={{if }},n,n',content},point={content}]
%   \begin{forest}
%     for tree={
%         if n=1{content=L}
%              {if n'=1{content=R}
%                       {content=C}}}
%     [[[][][]][[][][]]]
%   \end{forest}
% \end{forestexample}
%
%   \rkeyname[item=\forestoption{content}]{content format}|=|\meta{toks}  
%
%   When typesetting the node under the default conditions (see option \ikeyname{node format}), the
%   value of this option is passed to the \TikZ; \texttt{node} operation as its \meta{text} argument
%   \citep[\S16.2]{tikzpgf2.10}.  The default value of the option simply puts the content in the
%   node.
% 
%   This is a fairly low level option, but sometimes you might still want to change its value.  If
%   you do so, take care of what is expanded when.  For details, read the documentation of option
%   \ikeyname{node format} and macros \icmdname{forestoption} and \icmdname{foresteoption}; for an
%   example, see option \ikeyname{align}.
%
% \rkeyname[item]{math content}  The content of the node will be typeset in a math environment.
%
% This style is just an abbreviation for \ikeyname{content
% format}|={\ensuremath{\forestoption{content}}}|. 
%
% \rkeyname[item]{node format}|=|\meta{toks}
%   \hfill|\noexpand\node|\\
%   \mbox{}\hfill|[\forestoption{node options},anchor=\forestoption{anchor}]|\\
%   \mbox{}\hfill|(\forestoption{name}){\foresteoption{content format}};|
%
%   The node is typeset by executing the expansion of this option's value in a |tikzpicture|
%   environment.
%
%   Important: the value of this option is first expanded using |\edef| and only then executed. Note
%   that in its default value, \ikeyname{content format} is fully expanded using
%   \icmdname{foresteoption}: this is necessary for complex content formats, such as |tabular|
%   environments.
%
%   This is a low level option.  Ideally, there should be no need to change its value.  If you do,
%   note that the \TikZ; node you create should be named using the value of option \ikeyname{name};
%   otherwise, parent--child edges can't be drawn, see option \ikeyname{edge path}. 
%
%   \rkeyname[item={{{{{}}}}}]{node options}|=|\meta{keylist}
% 
%   When the node is being typeset under the default conditions (see option \ikeyname{node format}),
%   the content of this option is passed to \TikZ; as options to the 
%   \TikZ; |node| operation \citep[\S16]{tikzpgf2.10}.
%
%   This option is rarely manipulated manually: almost all options unknown to \foRest; are
%   automatically appended to \keyname{node options}.  Exceptions are (i) \ikeyname{label} and
%   \ikeyname{pin}, which require special attention in order to work; and (ii) \ikeyname{anchor},
%   which is saved in order to retain the information about the selected anchor.
%
% \begin{forestexample}[ekeynames={for descendants,anchor,child anchor,parent anchor,grow,l sep,for tree,where,delay,content,node options,rotate,{id=.pgfmath,nfc}}]
%   \begin{forest}
%     for descendants={anchor=east,child anchor=east},
%     grow=west,anchor=north,parent anchor=north,
%     l sep=1cm,
%     for tree={~fill=yellow~},where={n()>3}{~draw=red~}{},
%     delay={for tree={content/.pgfmath=~node_options~}}
%     [root,rotate=90,
%       [,~fill=white~]
%       [,~node options'~]
%       []
%       []
%       [,~node options~={~ellipse~}]
%     ]
%   \end{forest}
% \end{forestexample} 
%
%
%
%   \rkeyname[item=false]{phantom}|=|\meta{boolean}
%
%   A phantom node and its surrounding edges are taken into account when packing, but not
%   drawn. (This option applies in stage \ikeyname{draw tree}.)
% \begin{forestexample}[point=phantom,ekeynames=phantom]
%   \begin{forest}
%     [VP[DP][V',phantom[V][DP]]]
%   \end{forest}
% \end{forestexample}
%
% \end{syntax}
%
%
%
% \subsubsection{Node position}
% \label{ref:ref-node-position}
%
% Most of the following options apply at stage \ikeyname{pack}. Changing them
% afterwards has no effect in the normal course of events.  (Options \ikeyname{l},
% \ikeyname{s}, \ikeyname{x}, \ikeyname{y} and \ikeyname{anchor} are exceptions; see their documentation for
% details).
%
% \begin{syntax}
%
%   \rkeyname[item=base]{anchor}|=|\meta{toks: \TikZ; anchor name}
%
%   This is essentially a \TikZ; option \citep[see][\S16.5.1]{tikzpgf2.10} --- it is passed to
%   \TikZ; as a node option when the node is typeset (this option thus applies in stage
%   \ikeyname{typeset nodes}) --- but it is also saved by \foRest;.
%
%   The effect of this option is only observable when a node has a sibling: the anchors of all
%   siblings are s-aligned (if their \ikeyname{l}s have not been modified after packing).
%
%   In the \TikZ; code, you can refer to the node's anchor using the generic anchor
%   \rkeyname{anchor,aspect=generic anchor}.
%
%   \rkeyname[item=center]{calign}|=|\alternative{child,child edge,midpoint,edge midpoint,fixed
%   angles,fixed edge angles}\\\alternative{first,last,center}.
%   
%   The packing algorithm positions the children so that they don't overlap, effectively computing
%   the minimal distances between the node anchors of the children.  This option (\keyname{calign}
%   stands for child alignment) specifies how the children are positioned
%   with respect to the parent (while respecting the above-mentioned minimal distances).
%
%   The child alignment methods refer to the primary and the secondary child, and to the primary and
%   the secondary angle.  These are set using the keys described just after \keyname{calign}.
%
%   \let\outerleftmargin\leftmargin
%   \begin{syntax}
%   \item\keyname{calign}|=|\rkeyname{child,aspect=calign} s-aligns the node anchors of the parent and
%     the primary child.
%   \item\keyname{calign}|=|\rkeyname{child edge,aspect=calign} s-aligns the parent anchor of the parent 
%     and the child anchor of the primary child.
%   \item \keyname{calign}|=|\rkeyname{first,aspect=calign} is an abbreviation for
%     |calign=child,calign_child=1|.
%   \item \keyname{calign}|=|\rkeyname{last,aspect=calign} is an abbreviation for
%     |calign=child,calign_child=-1|.
%   \item\keyname{calign}|=|\rkeyname{midpoint,aspect=calign} s-aligns the parent's node anchor and the
%     midpoint between the primary and the secondary child's node anchor.
%   \item\keyname{calign}|=|\rkeyname{edge midpoint,aspect=calign} s-aligns the parent's parent anchor
%     and the midpoint between the primary and the secondary child's child anchor.
%   \item \keyname{calign}|=|\rkeyname{center,aspect=calign} is an abbreviation for\\
%     |calign=midpoint,| |calign_primary_child=1,| |calign_secondary_child=-1|. 
% \begin{forestexample}
%   \begin{forest}
%     [center,calign=center[1]
%       [first,calign=first[A][B][C]][3][4][5][6]
%       [last,calign=last[A][B][C]][8]]
%   \end{forest}
% \end{forestexample}
%   \item\keyname{calign}|=|\rkeyname{fixed angles,aspect=calign}: The angle between the direction of
%   growth at the current node (specified by option \ikeyname{grow}) and the line through the node
%   anchors of the parent and the primary/secondary child will equal the primary/secondary angle.
%
%   To achieve this, the block of children might be spread or further distanced from the parent.
%   \item\keyname{calign}|=|\rkeyname{fixed edge angles,aspect=calign}: The angle between the direction of
%   growth at the current node (specified by option \ikeyname{grow}) and the line through the
%   parent's parent anchor and the primary/secondary child's child anchor will equal the
%   primary/secondary angle.
%   
%   To achieve this, the block of children might be spread or further distanced from the parent.
% \begin{forestexample}[point=calign,ekeynames={calign,fixed edge angles,calign primary angle,calign secondary angle,for tree,l}]
%   \begin{forest}
%     calign=fixed edge angles,
%     calign primary angle=-30,calign secondary angle=60,
%     for tree={l=2cm}
%     [CP[C][TP]]
%     \draw[dotted] (!1) -| coordinate(p) () (!2) -| ();
%     \path ()--(p) node[pos=0.4,left,inner sep=1pt]{-30};
%     \path ()--(p) node[pos=0.1,right,inner sep=1pt]{60};
%   \end{forest}
% \end{forestexample}
%   \end{syntax}
% \rkeyname[item]{calign child}|=|\meta{count} is an abbreviation for \ikeyname{calign primary
%   child}|=|\meta{count}.
% \rkeyname[item=1]{calign primary child}|=|\meta{count} Sets the primary child.  
%   (See \ikeyname{calign}.)
%
%   \meta{count} is the child's sequence number.  Negative numbers start counting at the last child.
% \rkeyname[item=-1]{calign secondary child}|=|\meta{count} Sets the secondary child. 
%   (See \ikeyname{calign}.)
%
%   \meta{count} is the child's sequence number.  Negative numbers start counting at the last child.
% \rkeyname[item]{calign angle}|=|\meta{count} is an abbreviation for \ikeyname{calign primary
%   angle}|=-|\meta{count}, \ikeyname{calign secondary angle}|=|\meta{count}.
% \rkeyname[item=-35]{calign primary angle}|=|\meta{count} Sets the primary angle.
%   (See \ikeyname{calign}.) 
% \rkeyname[item=35]{calign secondary angle}|=|\meta{count} Sets the secondary angle. 
%   (See \ikeyname{calign}.)
% \rkeyname[item]{calign with current} s-aligns the node anchors of the current node and its
%   parent.  This key is an abbreviation for:\\   
%   |for_parent/.wrap_pgfmath_arg={calign=child,calign primary child=##1}{n}|.
% \rkeyname[item]{calign with current edge} s-aligns the child anchor of the current node and the
%   parent anchor of its parent.  This key is an abbreviation for:\\
%   |for_parent/.wrap_pgfmath_arg={calign=child edge,calign primary child=##1}{n}|.
%
%   \rkeyname[item=tight]{fit}|=|\alternative{tight,rectangle,band}
%
% \begin{forestexample}[pos=x,samplebox=\treebox,codebox=\codebox,basicstyle=\footnotesize\ttfamily]
%   \makeatletter\tikzset{use path/.code={\tikz@addmode{\pgfsyssoftpath@setcurrentpath#1}
%     \appto\tikz@preactions{\let\tikz@actions@path#1}}}\makeatother
%   \forestset{show boundary/.style={
%     before drawing tree={get min s tree boundary=\minboundary, get max s tree boundary=\maxboundary},
%     tikz+={\draw[red,use path=\minboundary]; \draw[red,use path=\maxboundary];}}}
% \end{forestexample}
% \input{\jobname.tmp}
%
% This option sets the type of the (s-)boundary that will be computed for the subtree rooted in the
% node, thereby determining how it will be packed into the subtree rooted in the node's parent.
% There are three choices:\footnote{Below is the definition of style \keyname{show boundary}. The
% \keyname{use path} trick is adjusted from \TeX\ Stackexchange question
% \href{http://tex.stackexchange.com/questions/26382/calling-a-previously-named-path-in-tikz}{Calling
% a previously named path in tikz}.
%
% \vskip-2ex \box\codebox}
%   \begin{itemize}
%   \item\keyname{fit}|=|\rkeyname{tight,aspect=fit}: an exact boundary of the node's subtree is computed,
%     resulting in a compactly packed tree.  Below, the boundary of subtree L is drawn.
% \begin{forestexample}[point={fit,tight},ekeynames={fit,{tight,aspect=fit},delay,for tree,name,content,{id=.pgfmath,nfc}}]
%   \begin{forest}
%     delay={for tree={name/.pgfmath=content}}
%     [root
%       [L,fit=tight, % default
%          show boundary
%         [L1][L2][L3]]
%       [R]
%     ]
%   \end{forest}
% \end{forestexample}
% \makeatletter\tikzset{use path/.code={%
%   \tikz@addmode{\pgfsyssoftpath@setcurrentpath#1}%
%   \appto\tikz@preactions{\let\tikz@actions@path#1}%
%   }}\makeatother
% \item\keyname{fit}|=|\rkeyname{rectangle,aspect=fit}: puts the node's subtree in a rectangle and effectively
%   packs this rectangle; the resulting tree will usually be wider.
% \begin{forestexample}[point={fit,rectangle},ekeynames={fit,{rectangle,aspect=fit},delay,for tree,name,content,{id=.pgfmath,nfc}}]
%   \begin{forest}
%     delay={for tree={name/.pgfmath=content}}
%     [root
%       [L,fit=rectangle,
%          show boundary
%         [L1][L2][L3]]
%       [R]
%     ]
%   \end{forest}
% \end{forestexample}
% \item\keyname{fit}|=|\rkeyname{band,aspect=fit}: puts the node's subtree in a rectangle of ``infinite
%   depth'': the space under the node and its descendants will be kept clear.
% \begin{forestexample}[point={fit,band},ekeynames={fit,{band,aspect=fit},delay,for tree,name,content,{id=.pgfmath,nfc}}]
%   \begin{forest}
%     delay={for tree={name/.pgfmath=content}}
%     [root
%       [L[L1][L2][L3]]
%       [C,fit=band]
%       [R[R1][R2][R3]]
%     ]
%     \draw[thin,red]
%       (C.south west)--(C.north west)
%       (C.north east)--(C.south east);
%     \draw[thin,red,dotted]
%       (C.south west)--+(0,-1)
%       (C.south east)--+(0,-1);
%   \end{forest}
% \end{forestexample}
%   \end{itemize}
%
%   \rkeyname[item=270]{grow}|=|\meta{count}  The direction of the tree's growth at the node.
%
%   The growth direction is understood as in \TikZ;'s tree library \citep[\S18.5.2]{tikzpgf2.10}
%   when using the default growth method: the (node anchor's of the) children of the node are placed
%   on a line orthogonal to the current direction of growth. (The final result might be different,
%   however, if \ikeyname{l} is changed after packing or if some child undergoes tier alignment.)
%
%   This option is essentially numeric (|pgfmath| function \keyname{grow} will always return an
%   integer), but there are some twists.  The growth direction can be specified either numerically
%   or as a compass direction (|east|, |north east|, \dots).  Furthermore, like in \TikZ;, setting
%   the growth direction using key \keyname{grow} additionally sets the value of option
%   \ikeyname{reversed} to |false|, while setting it with \rkeyname{grow'} sets it to |true|; to
%   change the growth direction without influencing \ikeyname{reversed}, use key \rkeyname{grow''}.
%
%   Between stages \ikeyname{pack} and \ikeyname{compute xy}, the value of \keyname{grow} should not
%   be changed.
%
% \begin{forestexample}[ekeynames={delay,id={{where in }},content,for ,current,grow,grow',grow'',{id=.pgfmath,nfc}}]
%   \begin{forest}
%     delay={where in content={~grow~}{
%         for current/.pgfmath=content,
%         content=\texttt{#1}
%       }{}
%     }
%     [{~grow~=south}
%       [{~grow'~=west}[1][2][3]
%         [{~grow''~=90}[1][2][3]]]
%       [2][3][4]
%       [{~grow~=east}[1][2][3]
%         [{~grow''~=90}[1][2][3]]]]
%   \end{forest}
% \end{forestexample}
%
% \rkeyname[item=false]{ignore}|=|\meta{boolean}
%
% If this option is set, the packing mechanism ignores the node, i.e.\ it pretends that the node has
% no boundary. Note: this only applies to the node, not to the tree.
%
% Maybe someone will even find this option useful for some reason \dots
% 
% \rkeyname[item=false]{ignore edge}|=|\meta{boolean}
%
% If this option is set, the packing mechanism ignores the edge from the node to the parent, i.e.\
% nodes and other edges can overlap it. (See \S\ref{sec:bugs} for some problematic situations.)
%
% \begin{forestexample}[ekeynames={ignore edge,l,*}]
%   \begin{forest}
%     [A[B[B][B][B][B]][C
%       [\texttt{not ignore edge},l*=2]]]
%   \end{forest}
%   \begin{forest}
%     [A[B[B][B][B][B]][C
%       [\texttt{ignore edge},l*=2,~ignore edge~]]]
%   \end{forest}
% \end{forestexample}
%
% \rkeyname[item]{l}|=|\meta{dimen} The l-position of the node, in the parent's ls-coordinate system.  (The
% origin of a node's ls-coordinate system is at its (node) anchor.  The l-axis points in the
% direction of the tree growth at the node, which is given by option \ikeyname{grow}. The s-axis is
% orthogonal to the l-axis; the positive side is in the counter-clockwise direction from |l| axis.)
%
% The initial value of \keyname{l} is set from the standard node.  By default, it equals:
% \[\ikeyname{l sep}+2\cdot\mbox{\texttt{outer ysep}}+\mbox{total
% height(standard node)}\]
%
% The value of \keyname{l} can be changed at any point, with different effects.  
% \begin{itemize}
% \item The value of \keyname{l} at the beginning of stage \ikeyname{pack} determines the minimal
%   l-distance between the anchors of the node and its parent.  Thus, changing \keyname{l} before
%   packing will influence this process.  (During packing, \keyname{l} can be increased due to
%   parent's \ikeyname{l sep}, tier alignment, or \ikeyname{calign} method \keyname{fixed (edge)
%   angles}\ekeyname{fixed angles},\ekeyname{fixed edge angles}.)
%
% \item Changing \keyname{l} after packing but before stage \ikeyname{compute xy} will result in a
%   manual adjustment of the computed position.  (The augmented operators can be useful here.)
%
% \item Changing \keyname{l} after the absolute positions have been computed has no effect in the
%   normal course of events.
% \end{itemize}
% 
% \rkeyname[item]{l sep}|=|\meta{dimen} The minimal l-distance between the node and its
% descendants.
%
% This option determines the l-distance between the \emph{boundaries} of the node and its descendants,
% not node anchors.  The final effect is that there will be a \keyname{l sep} wide band,
% in the l-dimension, between the node and all its descendants.
%
% The initial value of \keyname{l sep} is set from the standard node and equals
% \[\mbox{height}(\mbox{strut})+\mbox{\texttt{inner ysep}}\]
%
% Note that despite the similar name, the semantics of \keyname{l sep} and \keyname{s sep} are
% quite different.
%
% \rkeyname[item=false]{reversed}|=|\meta{boolean}
%
% If |false|, the children are positioned around the node in the counter-clockwise direction; if
% |true|, in the clockwise direction.  See also \ikeyname{grow}. 
%
% \rkeyname[item]{s}|=|\meta{dimen} The s-position of the node, in the parent's ls-coordinate system.
% (The origin of a node's ls-coordinate system is at its (node) anchor.  The l-axis points in the
% direction of the tree growth at the node, which is given by option \ikeyname{grow}.  The s-axis is
% orthogonal to the l-axis; the positive side is in the counter-clockwise direction from |l| axis.)
%
% The value of \keyname{s} is computed by the packing mechanism.  Any value given before packing is
% overridden.  In short, it only makes sense to (inspect and) change this option after stage
% \ikeyname{pack}, which can be useful for manual corrections, like below.  (B is closer to A than C
% because packing proceeds from the first to the last child --- the position of B would be the same
% if there was no C.)  Changing the value of \keyname{s} after stage \ikeyname{compute xy} has no
% effect.
% \begin{forestexample}[point=s,ekeynames={before computing xy,s}]
%   \begin{minipage}{.5\linewidth}
%   \begin{forest}
%     [no manual correction of B
%       [A[1][2][3][4]]
%       [B]
%       [C[1][2][3][4]]
%     ]
%   \end{forest}
%   
%    \begin{forest}
%     [manual correction of B
%       [A[1][2][3][4]]
%       [B,before computing xy={s=(s("!p")+s("!n"))/2}]
%       [C[1][2][3][4]]
%     ]
%   \end{forest}
%   \end{minipage}
% \end{forestexample}
%
% \rkeyname[item]{s sep}|=|\meta{dimen}
%
% The subtrees rooted in the node's children will be kept at least \keyname{s sep} apart in the
% s-dimension.  Note that \keyname{s sep} is about the minimal distance between node
% \emph{boundaries}, not node anchors.
%
% The initial value of \keyname{s sep} is set from the standard node and equals
% $2\cdot\mbox{\texttt{inner xsep}}$. 
%
% Note that despite the similar name, the semantics of \keyname{s sep} and \keyname{l sep} are
% quite different.
%
% \rkeyname[item={{{{{}}}}}]{tier}|=|\meta{toks}
%
% Setting this option to something non-empty ``puts a node on a tier.''  All the nodes on the same
% tier are aligned in the l-dimension.
%
% Tier alignment across changes in growth direction is impossible.  In the case of incompatible
% options, \foRest; will yield an error.
%
% Tier alignment also does not work well with \ikeyname{calign}|=|\keyname{fixed (edge)
% angles}\ekeyname{fixed angles}\ekeyname{fixed edge angles}, because these child alignment methods
% may change the l-position of the children.  When this might happen, \foRest; will yield a warning.
%
% \rkeyname[item]{x}=\meta{dimen}
% \vspace{-\parskip}
% \rkeyname[item]{y}=\meta{dimen}
%
% \keyname{x} and \keyname{y} are the coordinates of the node in the ``normal'' (paper) coordinate
% system, relative to the root of the tree that is being drawn.  So, essentially, they are absolute
% coordinates.
%
% The values of \keyname{x} and \keyname{y} are computed in stage \ikeyname{compute xy}.  It only
% makes sense to inspect and change them (for manual adjustments) afterwards (normally, in the
% \ikeyname{before drawing tree} hook, see \S\ref{ref:stages}.)
% \begin{forestexample}[label=ex:adjustxy,ekeynames={y,-,grow',l,for tree,before drawing tree}]
%   \begin{forest}
%     for tree={grow'=45,l=1.5cm}
%     [A[B][C][D,before drawing tree={~y-~=4mm}[1][2][3][4][5]][E][F]]
%   \end{forest}
% \end{forestexample}
% 
% \end{syntax}
% 
% \subsubsection{Edges}
% \label{ref:ref-edge}
%
% These options determine the shape and position of the edge from a node to its parent.  They apply
% at stage \ikeyname{draw tree}.
%
% \begin{syntax}
% \rkeyname[item={{{{{}}}}}]{child anchor}|=|\meta{toks} See \ikeyname{parent anchor}.
% 
% \rkeyname[item=draw]{edge}|=|\meta{keylist}
%
%   When \ikeyname{edge path} has its default value, the value of this option is passed as options to
%   the \TikZ; |\path| expression used to draw the edge between the node and its parent.
%
%   Also see key \ikeyname{no edge}.
%
% \begin{forestexample}[point=edge,ekeynames={edge,no edge,for tree,grow',l,anchor,child anchor}]
%   \begin{forest} for tree={grow'=0,l=2cm,anchor=west,child anchor=west},
%     [root
%       [normal]
%       [none,~no~ edge]
%       [dotted,edge=dotted]
%       [dashed,edge=dashed]
%       [dashed,edge={dashed,red}]
%     ]
%   \end{forest}
% \end{forestexample}
%
% \rkeyname[item={{{{{}}}}}]{edge label}|=|\meta{toks: \TikZ; code}
%
% When \ikeyname{edge path} has its default value, the value of this option is used at the end of
% the edge path specification to typeset a node (or nodes) along the edge.
% 
% The packing mechanism is not sensitive to edge labels.
% 
% \begin{forestexample}[ekeynames={edge label}]
%   \begin{forest}
%     [VP
%       [V,~edge label~={node[midway,left,font=\scriptsize]{head}}]
%       [DP,~edge label~={node[midway,right,font=\scriptsize]{complement}}]
%     ]
%   \end{forest}
% \end{forestexample}
%
% \rkeyname[item]{edge path}|=|\meta{toks: \TikZ; code}
%  \hfill |\noexpand\path[\forestoption{edge}]|\\
%  \mbox{}\hfill |(!u.parent anchor)--(.child anchor)\forestoption{edge label};|
%
% This option contains the code that draws the edge from the node to its parent.  By default, it
% creates a path consisting of a single line segment between the node's \ikeyname{child anchor} and
% its parent's \ikeyname{parent anchor}. Options given by \ikeyname{edge} are passed to the path; by
% default, the path is simply drawn.  Contents of \ikeyname{edge label} are used to potentially place
% a node (or nodes) along the edge.
%
% When setting this option, the values of options \ikeyname{edge} and \ikeyname{edge label} can be
% used in the edge path specification to include the values of options \ikeyname{edge} and \ikeyname{edge
% node}.  Furthermore, two generic anchors, \ikeyname{parent anchor,aspect=generic anchor} and \ikeyname{child anchor,aspect=generic anchor}, are defined, 
% to facilitate access to options \ikeyname{parent anchor} and \ikeyname{child anchor} from the \TikZ; code.
%
% The node positioning algorithm is sensitive to edges, i.e.\ it will avoid a node overlapping an
% edge or two edges overlapping. However, the positioning algorithm always behaves as if the
% \keyname{edge path} had the default value --- \emph{changing the \keyname{edge path} does not
% influence the packing!}  Sorry.  (Parent--child edges can be ignored, however: see option
% \ikeyname{ignore edge}.)
%
% \rkeyname[item={{{{{}}}}}]{parent anchor}|=|\meta{toks: \TikZ; anchor} (Information also applies to
% option \ikeyname{child anchor}.)
%
% \FoRest; defines generic anchors \rkeyname{parent anchor,aspect=generic anchor} and
% \rkeyname{child anchor,aspect=generic anchor} (which work only for \foRest; and not also \TikZ;
% nodes, of course) to facilitate reference to the desired endpoints of child--parent edges.
% Whenever one of these anchors is invoked, it looks up the value of the \keyname{parent anchor} or
% \keyname{child anchor} of the node named in the coordinate specification, and forwards the request
% to the (\TikZ;) anchor given as the value.
%
% The indented use of the two anchors is chiefly in \ikeyname{edge path} specification, but they can
% used in any \TikZ; code.
% \begin{forestexample}[ekeynames={parent anchor,child anchor,for tree}]
%   \begin{forest}
%     for tree={~parent anchor~=south,~child anchor~=north}
%     [VP[V][DP]]
%     \path[fill=red] (.parent anchor) circle[radius=2pt]
%                     (!1.child anchor) circle[radius=2pt]
%                     (!2.child anchor) circle[radius=2pt];
%   \end{forest}
% \end{forestexample}
%
% The empty value (which is the default) is interpreted as in \TikZ;: as an edge to the appropriate
% border point.
%
%
% \rkeyname[item]{no edge} Clears the edge options (\ikeyname{edge}|'={}|) and sets \ikeyname{ignore
% edge}.
%
% \rkeyname[item]{triangle} Makes the edge to parent a triangular roof.  Works only for south-growing
% trees.  Works by changing the value of \ikeyname{edge path}.
% 
% \end{syntax}
%
% \subsubsection{Readonly}
% \label{ref:readonly-options}
%
% The values of these options provide various information about the tree and its nodes.
% 
% \begin{syntax}
% \rkeyname[item]{id=id}|=|\meta{count}) The internal id of the node.
%
% \rkeyname[item]{level}|=|\meta{count}  The hierarchical level of the node.  The root is on level $0$. 
%
% \rkeyname[item]{max x}|=|\meta{dimen} \vspace{-\parskip}
% \rkeyname[item]{max y}|=|\meta{dimen} \vspace{-\parskip}
% \rkeyname[item]{min x}|=|\meta{dimen} \vspace{-\parskip}
% \rkeyname[item]{min y}|=|\meta{dimen}
%   Measures of the node, in the shape's coordinate system
%   \citep[see][\S16.2,\S48,\S75]{tikzpgf2.10} shifted so that the node anchor is at the origin.
%
%   In |pgfmath| expressions, these options are accessible as |max__x|, |max__y|, |min__x| and |min__y|.
%
% \rkeyname[item]{n}|=|\meta{count}  The child's sequence number in the list of its parent's
%   children. 
%
%   The enumeration starts with 1.  For the root node, \keyname{n} equals $0$. 
%
% \rkeyname[item]{n'}|=|\meta{count}  Like \ikeyname{n}, but starts counting at the last child.
%
%   In |pgfmath| expressions, this option is accessible as |n__|.
%
% \rkeyname[item]{n children}|=|\meta{count} The number of children of the node.
%   
%   In |pgfmath| expressions, this option is accessible as |n__children|.
% \end{syntax}
% 
% \subsubsection{Miscellaneous}
% \label{ref:miscellaneous}
%
% \begin{syntax}
%   \rkeyname[item]{afterthought}|=|\meta{toks}  Provides the afterthought explicitely.
%
%   This key is normally not used by the end-user, but rather called by the bracket parser.  By
%   default, this key is a style defined by |afterthought/.style={tikz+={#1}}|: afterthoughts are
%   interpreted as (cumulative) \TikZ; code.  If you'd like to use afterthoughts for some other
%   purpose, redefine the key --- this will take effect even if you do it in the tree preamble.
%
% \rkeyname[item]{alias}|=|\meta{toks} Sets the alias for the node's name.
%
%   Unlike \ikeyname{name}, \keyname{alias} is \emph{not} an option: you cannot e.g.\ query it's
%   value via a |pgfmath| expression.
%
%   Aliases can be used as the \meta{forest node name} part of a relative node name and as the
%   argument to the \ikeyname{name,aspect=step} step of a node walk.  The latter includes the usage
%   as the argument of the \ikeyname{id={{for }}}\ikeyname{name} propagator.
%
%   Technically speaking, \foRest; alias is \emph{not} a \TikZ; alias!  However, you can still use
%   it as a ``node name'' in \TikZ; coordinates, since \foRest; hacks \TikZ;'s implicit node
%   coordinate system to accept relative node names; see \S\ref{ref:forest-cs}.
%
%   \rkeyname[item]{baseline} The node's anchor becomes the baseline of the whole tree
%   \citep[cf.][\S69.3.1]{tikzpgf2.10}.
%
%   In plain language, when the tree is inserted in your (normal \TeX) text, it will be vertically
%   aligned to the anchor of the current node.
%
%   Behind the scenes, this style sets the alias of the current node to \keyname{forest@baseline@node}.
% \begin{forestexample}[ekeynames={baseline,use as bounding box'}]
%   {\tikzexternaldisable
%   Baseline at the
%   \begin{forest}
%     [parent,~baseline~,use as bounding box'
%       [child]]
%   \end{forest}
%   and baseline at the
%   \begin{forest}
%     [parent
%       [child,~baseline~,use as bounding box']]
%   \end{forest}.}
% \end{forestexample}
%
% \rkeyname[item=\begin{tikzpicture}]{begin draw}|/.code=|\meta{toks: \TeX\ code}  \vspace{-\parskip}
% \rkeyname[item=\end{tikzpicture}]{end draw}|/.code=|\meta{toks: \TeX\ code}
%
% The code produced by \ikeyname{draw tree} is put in the environment specified by \keyname{begin
% draw} and \keyname{end draw}.  Thus, it is this environment, normally a |tikzpicture|, that does
% the actual drawing.
%
% A common use of these keys might be to enclose the |tikzpicture| environment in a |center|
% environment, thereby automatically centering all trees; or, to provide the \TikZ; code to execute
% at the beginning and/or end of the picture.
%
% Note that \keyname{begin draw} and \keyname{end draw} are \emph{not} node options: they are
% |\pgfkeys|' code-storing keys \citep[\S55.4.3--4]{tikzpgf2.10}.
%
%
% \rkeyname[item={{{{{}}}}}]{begin forest}|/.code=|\meta{toks: \TeX\ code}  \vspace{-\parskip}
% \rkeyname[item={{{{{}}}}}]{end forest}|/.code=|\meta{toks: \TeX\ code}
%
% The code stored in these (|\pgfkeys|) keys is executed at the beginning and end of the
% \ikeyname{forest} environment / \icmdname{Forest} macro.  
%
% Using these keys is only effective \emph{outside} the \ikeyname{forest} environment, and the
% effect lasts until the end of the current \TeX\ group.
%
% For example, executing \icmdname{forestset}|{begin forest/.code=\small}| will typeset all trees (and only
% trees) in the small font size.  
%
%
% \rkeyname[item]{fit to tree}  Fits the \TikZ; node to the current node's subtree.
%
% This key should be used like \keyname{/tikz/fit} of the \TikZ;'s fitting library
% \citep[see][\S34]{tikzpgf2.10}: as an option to \emph{\TikZ;'s} |node| operation, the obvious
% restriction being that \keyname{fit to tree} must be used in the context of some \foRest; node.
% For an example, see footnote~\ref{fn:fit-to-tree}.
%
% This key works by calling \keyname{/tikz/fit} and providing it with the the coordinates of the
% subtree's boundary.
%
% \rkeyname[item]{get min s tree boundary}|=|\meta{cs} \vspace{-\parskip}
% \rkeyname[item]{get max s tree boundary}|=|\meta{cs}
%
% Puts the boundary computed during the packing process into the given \meta{cs}.  The boundary is
% in the form of \PGF; path.  The |min| and |max| versions give the two sides of the node.  For an
% example, see how the boundaries in the discussion of \ikeyname{fit} were drawn.
%
%   \rkeyname[item]{label}|=|\meta{toks: \TikZ; node}  The current node is labelled by a \TikZ; node.
%
%   The label is specified as a \TikZ; option \texttt{label} \citep[\S16.10]{tikzpgf2.10}.
%   Technically, the value of this option is passed to \TikZ;'s as a late option
%   \citep[\S16.14]{tikzpgf2.10}.  (This is so because \foRest; must first typeset the nodes
%   separately to measure them (stage \ikeyname{typeset nodes}); the preconstructed nodes are inserted
%   in the big picture later, at stage \ikeyname{draw tree}.)  Another option with the same
%   technicality is \ikeyname{pin}. 
%
%   \rkeyname[item]{name}|=|\meta{toks} Sets the name of the node.\hfill\texttt{node@}\meta{id}
%
%   The expansion of \meta{toks} becomes the \meta{forest node name} of the node.  Node names must
%   be unique.  The \TikZ; node created from the \foRest; node will get the name specified by this
%   option.
%
% \rkeyname[item]{node walk}|=|\meta{node walk}  This key is the most general way to use a \meta{node
%   walk}.
%
%   Before starting the \meta{node walk}, key \rkeyname{id={node walk/before walk}} is processed.
%   Then, the \meta{step}s composing the \meta{node walk} are processed: making a step (normally)
%   changes the current node.  After every step, key \rkeyname{id={node walk/every step}} is
%   processed.  After the walk, key \rkeyname{id={node walk/after walk}} is processed.
%
%   \keyname{node walk/before walk}, \keyname{node walk/every step} and \keyname{node walk/after
%   walk} are processed with \keyname{/forest} as the default path: thus, \foRest;'s options and
%   keys described in \S\ref{ref:options-and-keys} can be used normally inside their definitions.
%
%   \begin{advise}
%   \item Node walks can be tail-recursive, i.e.\ you can call another node walk from \keyname{node
%     walk/after walk} --- embedding another node walk in \keyname{node walk/before walk} or
%     \keyname{node walk/every step} will probably fail, because the three node walk styles are not
%     saved and restored (a node walk doesn't create a \TeX\ group).
%   \item \keyname{every step} and \keyname{after walk} can be redefined even during the walk.
%     Obviously, redefining \keyname{before walk} during the walk has no effect (in the current
%     walk).
%   \end{advise}
%
%   \rkeyname[item]{pin}|=|\meta{toks: \TikZ; node}  The current node gets a pin, see
%   \citep[\S16.10]{tikzpgf2.10}. 
%
%   The technical details are the same as for \ikeyname{label}.
%
%   \rkeyname[item]{use as bounding box} The current node's box is used as a bounding box for the
%   whole tree.
%
%   \rkeyname[item]{use as bounding box'} Like \ikeyname{use as bounding box}, but subtracts the
%   (current) inner and outer sep from the node's box.  For an example, see \ikeyname{baseline}.
%   
%   \rkeyname[item]{TeX}|=|\meta{toks: \TeX\ code} The given code is executed immediately.
%
%   This can be used for e.g.\ enumerating nodes:
% \begin{forestexample}[point=TeX,ekeynames={TeX,delay,where ,tier,content,GP1},label=ex:enumerate]
%   \newcount\xcount
%   \begin{forest} GP1,
%     delay={TeX={\xcount=0},
%       where tier={x}{TeX={\advance\xcount1},
%          content/.expanded={##1$_{\the\xcount}$}}{}}
%     [
%       [O[x[f]]]
%       [R[N[x[o]]]]
%       [O[x[r]]]
%       [R[N[x[e]]][x[s]]]
%       [O[x[t]]]
%       [R[N[x]]]
%     ]
%   \end{forest}
% \end{forestexample}
%
% \rkeyname[item]{TeX'}|=|\meta{toks: \TeX\ code} This key is a combination of keys \ikeyname{TeX}
% and \ikeyname{TeX''}: the given code is both executed and externalized.
%
% \rkeyname[item]{TeX''}|=|\meta{toks: \TeX\ code} The given code is externalized, i.e.\ it will be
% executed when the externalized images are loaded.
%
% The image-loading and \keyname{TeX'(')} produced code are intertwined.
%
%   \rkeyname[item={{{{{}}}}}]{tikz}|=|\meta{toks: \TikZ; code}  ``Decorations.''
%
%   The code given as the value of this option will be included in the |tikzpicture| environment
%   used to draw the tree.  The code given to various nodes is appended in a depth-first,
%   parent-first fashion.  The code is included after all nodes of the tree have been drawn, so it
%   can refer to any node of the tree.  Furthermore, relative node names can be used to refer to
%   nodes of the tree, see \S\ref{ref:relative-node-names}.
%
%   By default, bracket parser's afterthoughts feed the value of this option.  See
%   \ikeyname{afterthought}. 
%
% \end{syntax}
% 
% \subsubsection{Propagators}
% \label{ref:propagators}
% 
% Propagators pass the given \meta{keylist} to other node(s), delay their processing, or cause them
% to be processed only under certain conditions.
%
% A propagator can never fail --- i.e.\ if you use \keyname{for next} on the last child of some node,
% no error will arise: the \meta{keylist} will simply not be passed to any node. (The generic
% node walk propagator \keyname{for} is
% an exception.  While it will not fail if the final node of the walk does not exist (is null), its node walk
% can fail when trying to walk away from the null node.)
%
% \paragraph{Spatial propagators}
% pass the given \meta{keylist} to other node(s) in the tree.  (\keyname{for} and \keyname{for
% }\meta{step} always pass the \meta{keylist} to a single node.)
%
% \begin{syntax}
% \rkeyname[item]{for}|=|\meta{node walk}\meta{keylist}  Processes \meta{keylist} in the context of the final
%   node in the \meta{node walk} starting at the current node.
%
% \rkeyname[item]{id={{for }}}\meta{step}|=|\meta{keylist}  Walks a single-step node-walk
% \meta{step} from the current node and passes the given \meta{keylist} to the final (i.e.\ second) node.
%
% \meta{step} must be a long node walk step; see \S\ref{ref:node-walk}. \keyname{for
% }\meta{step}|=|\meta{keylist} is equivalent to \ikeyname{for}|=|\meta{step}{keylist}.
%
% Examples: |for_parent={l_sep+=3mm}|, |for_n=2{circle,draw}|.
% 
% \rkeyname[item]{for ancestors}|=|\meta{keylist}
% \rkeyname[item]{for ancestors'}|=|\meta{keylist}  Passes the \meta{keylist} to itself, too.
% \begin{forestexample}[ekeynames={for ancestors',delay,content,edge}]
%   \pgfkeys{/forest,
%     inptr/.style={%
%       red,delay={content={\textbf{##1}}},
%       edge={draw,line width=1pt,red}},
%     ptr/.style={~for ancestors'~=inptr}
%   }
%   \begin{forest}
%     [x
%       [x[x[x][x]][x[x,ptr][x]]]
%       [x[x[x][x]][x[x][x]]]]
%   \end{forest}
% \end{forestexample}
%
%   \rkeyname[item]{for all next}|=|\meta{keylist} Passes the \meta{keylist} to all the following siblings.
%   
%   \rkeyname[item]{for all previous}|=|\meta{keylist} Passes the \meta{keylist} to all the preceding siblings.
%
%   \rkeyname[item]{for children}|=|\meta{keylist}
% 
%   \rkeyname[item]{for descendants}|=|\meta{keylist}
%     
%   \rkeyname[item]{for tree}|=|\meta{keylist}
%
%   Passes the key to the current node and its the descendants.
%
%   This key should really be named \keyname{for subtree} \dots  
%
% \end{syntax}
%
% \paragraph{Conditionals}
% \label{ref:conditionals}
% 
% For all conditionals, both the true and the false keylist are obligatory! Either keylist can be
% empty, however --- but don't omit the braces!
% 
% \begin{syntax}
% \rkeyname[item]{if}|=|\meta{pgfmath condition}\meta{true keylist}\meta{false keylist}
%
% If \meta{pgfmath condition} evaluates to |true| (non-zero), \meta{true keylist} is processed (in
% the context of the current node); otherwise, \meta{false keylist} is processed.
%
% For a detailed description of
% |pgfmath| expressions, see \cite[part VI]{tikzpgf2.10}.  (In short: write the usual mathematical
% expressions.)
% 
% \rkeyname[item]{id={{if }}}\meta{option}|=|\meta{value}\meta{true keylist}\meta{false keylist}
%
% A simple conditional is defined for every \meta{option}: if \meta{value} equals the value of the
% option at the current node, \meta{true keylist} is executed; otherwise, \meta{false keylist}.
%
% \rkeyname[item]{where}|=|\meta{value}\meta{true keylist}\meta{false keylist}
%
% Executes conditional \ikeyname{if} for every node in the current subtree.
%
% \rkeyname[item]{id={{where }}}\meta{option}|=|\meta{value}\meta{true keylist}\meta{false keylist}
%
% Executes simple conditional \ikeyname{id={{if }}}\meta{option} for every node in the current subtree.
%
% \rkeyname[item]{id={{if in }}}\meta{option}|=|\meta{toks}\meta{true keylist}\meta{false
%   keylist}
%
%   Checks if \meta{toks} occurs in the option value; if it does, \meta{true keylist} are executed,
%   otherwise \meta{false keylist}.
%
%   This conditional is defined only for \meta{toks} options, see \S\ref{ref:options-and-keys}.
%   
% \rkeyname[item]{id={{where in }}}\meta{toks option}|=|\meta{toks}\meta{true keylist}\meta{false keylist}
% 
% A style equivalent to \ikeyname{for tree}|=|\ikeyname{id={{if in }}}\meta{option}=\meta{toks}\meta{true
% keylist}\meta{false keylist}: for every node in the subtree rooted in the current node,
% \ikeyname{id={{if in }}}\meta{option} is executed in the context of that node.  
% 
% This conditional is defined only for \meta{toks} options, see \S\ref{ref:options-and-keys}.
% \end{syntax}
%
% \paragraph{Temporal propagators}
% There are two kinds of temporal propagators.  The |before_...| propagators defer the processing of
% the given keys to a hook just before some stage in the computation.  The \keyname{delay}
% propagator is ``internal'' to the current hook (the first hook, the given options, is
% implicit): the keys in a hook are processed cyclically, and \keyname{delay} delays the
% processing of the given options until the next cycle.  All these keys can be nested without
% limit. For details, see~\S\ref{ref:stages}.
% \begin{syntax}
% \rkeyname[item]{delay}|=|\meta{keylist} Defers the processing of the \meta{keylist} until the next
%   cycle.
% \rkeyname[item]{delay n}|=|\meta{integer}\meta{keylist} Defers the processing of the
%   \meta{keylist} for $n$ cycles.  $n$ may be $0$, and it may be given as a |pgfmath| expression.
% \rkeyname[item]{if have delayed}|=|\meta{true keylist}\meta{false keylist}  If any options were
%   delayed in the current cycle (more precisely, up to the point of the execution of this key),
%   process \meta{true keylist}, otherwise process \meta{false keylist}.  (\ikeyname{delay n} will
%   trigger ``true'' for the intermediate cycles.)
% \rkeyname[item]{before typesetting nodes}|=|\meta{keylist}  Defers the processing of the
%   \meta{keylist} to until just before the nodes are typeset.
% \rkeyname[item]{before packing}|=|\meta{keylist}  Defers the processing of the
%   \meta{keylist} to until just before the nodes are packed.
% \rkeyname[item]{before computing xy}|=|\meta{keylist}  Defers the processing of the
%   \meta{keylist} to until just before the absolute positions of the nodes are computed.
% \rkeyname[item]{before drawing tree}|=|\meta{keylist}  Defers the processing of the
%   \meta{keylist} to until just before the tree is drawn.
% \end{syntax}
%
% \paragraph{Other propagators}
% \begin{syntax}
%   \rkeyname[item]{repeat}|=|\meta{number}\meta{keylist}  The \meta{keylist} is processed \meta{number}
%   times.
%
%   The \meta{number} expression is evaluated using |pgfmath|.  Propagator \keyname{repeat} also
%   works in node walks.
% \end{syntax}
%
% \subsubsection{Stages}
% \label{ref:stages}
%
% \FoRest; does its job in several steps.  The normal course of events is the following:
% \begin{enumerate}
% \item\label{step:parsing-bracket} The bracket representation of the tree if parsed and stored in a
%   data structure.
% \item\label{step:given-options} The given options are processed, including the options in the
%   preamble, which are processed first (in the context of the root node).
% \item\label{step:typeset-nodes} Each node is typeset in its own |tikzpicture| environment, saved
%   in a box and its measures are taken.
% \item\label{step:pack} The nodes of the tree are \emph{packed}, i.e.\ the relative positions of the nodes are
%   computed so that the nodes don't overlap.  That's difficult.  The result: option \ikeyname{s} is
%   set for all nodes.  (Sometimes, the value of \ikeyname{l} is adjusted as well.)
% \item\label{step:compute-xy} Absolute positions, or rather, positions of the nodes relative to the
%   root node are computed.  That's easy.  The result: options \ikeyname{x} and \ikeyname{y} are
%   set.
% \item\label{step:draw-tree} The \TikZ; code that will draw the tree is produced.  (The nodes are
%   drawn by using the boxes typeset in step~\ref{step:typeset-nodes}.)
% \end{enumerate}
%
% Steps~\ref{step:parsing-bracket} and \ref{step:given-options} collect user input and are thus
% ``fixed''.  However, the other steps, which do the actual work, are under user's control.
%
% First, hooks exist which make it possible (and easy) to change node's properties between the
% processing stages.  For a simple example, see example~\ref{ex:adjustxy}: the manual adjustment of
% \ikeyname{y} can only be done after the absolute positions have been computed, so the processing
% of this option is deferred by \ikeyname{before drawing tree}.  For a more realistic example, see
% the definition of style \ikeyname{GP1}: before packing, \texttt{outer xsep} is set to a high (user
% determined) value to keep the $\times$s uniformly spaced; before drawing the tree, the
% \texttt{outer xsep} is set to \texttt{0pt} to make the arrows look better.
% 
% Second, the execution of the processing stages \ref{step:typeset-nodes}--\ref{step:draw-tree} is
% \emph{completely} under user's control.  To facilitate adjusting the processing flow, the approach
% is twofold.  The outer level: \foRest; initiates the processing by executing style
% \keyname{stages}, which by default executes the processing stages
% \ref{step:typeset-nodes}--\ref{step:draw-tree}, preceding the execution of each stage by
% processing the options embedded in temporal propagators \keyname{before ...} (see
% \S\ref{ref:propagators}).  The inner level: each processing step is the sole resident of a
% stage-style, which makes it easy to adjust the workings of a single step.  What follows is the
% default content of style \keyname{stages}, including the default content of the individual
% stage-styles.  
% \begin{syntax}
%   \rkeyname[item]{stages}
%   \begin{syntax}
%     \item \ikeyname{process keylist}|=|\ikeyname{before typesetting nodes}
%     \rkeyname[item]{typeset nodes stage}\hfill
%       |{|\ikeyname{id={{for }}}\ikeyname{root'}|=|\ikeyname{typeset nodes}|}|
%     \item \ikeyname{process keylist}|=|\ikeyname{before packing}
%     \rkeyname[item]{pack stage}\hfill
%       |{|\ikeyname{id={{for }}}\ikeyname{root'}|=|\ikeyname{pack}|}|
%     \item \ikeyname{process keylist}|=|\ikeyname{before computing xy}
%     \rkeyname[item]{compute xy stage}\hfill
%       |{|\ikeyname{id={{for }}}\ikeyname{root'}|=|\ikeyname{compute xy}|}|
%     \item \ikeyname{process keylist}|=|\ikeyname{before drawing tree}
%     \rkeyname[item]{draw tree stage}\hfill
%       |{|\ikeyname{id={{for }}}\ikeyname{root'}|=|\ikeyname{draw tree}|}|
%   \end{syntax}
% \end{syntax}
%
% Both style \keyname{stages} and the individual stage-styles may be freely modified by the user.
% Obviously, a style must be redefined before it is processed, so it is safest to do so either
% outside the \ikeyname{forest} environment (using macro \icmdname{forestset}) or in the preamble
% (in a non-deferred fashion).
% 
% Here's the list of keys used either in the default processing or useful in an alternative
% processing flow.
% \begin{syntax}
% \rkeyname[item]{typeset nodes} Typesets each node of the current node's subtree in its own
%   |tikzpicture| environment.  The result is saved in a box and its measures are taken.
% 
% \rkeyname[item]{typeset nodes'}  Like \ikeyname{typeset nodes}, but the node box's content is not
%   overwritten if the box already exists.
%   
% \rkeyname[item]{typeset node}  Typesets the \emph{current} node, saving the result in the node box.
%
%   This key can be useful also in the default \ikeyname{stages}.  If, for example, the node's content
%   is changed and the node retypeset just before drawing the tree, the node will be positioned as if
%   it contained the ``old'' content, but have the new content: this is how the constant distance
%   between $\times$s is implemented in the \ikeyname{GP1} style.
%
% \rkeyname[item]{pack} The nodes of the tree are \emph{packed}, i.e.\ the relative positions of
%   the nodes are computed so that the nodes don't overlap.  The result: option \ikeyname{s} is set
%   for all nodes; sometimes (in tier alignment and for some values of \ikeyname{calign}), the value
%   of some nodes' \ikeyname{l} is adjusted as well.
%
% \rkeyname[item]{pack'} ``Non-recursive'' packing: packs the children of the current node only.
%   (Experimental, use with care, especially when combining with tier alignment.)
%
% \rkeyname[item]{compute xy} Computes the positions of the nodes relative to the (formal) root
%   node.  The results are stored into options \ikeyname{x} and \ikeyname{y}.
%
% \rkeyname[item]{draw tree} Produces the \TikZ; code that will draw the tree.  First, the nodes
%   are drawn (using the boxes typeset in step~\ref{step:typeset-nodes}), followed by edges and
%   custom code (see option \ikeyname{tikz}).
%   
% \rkeyname[item]{draw tree'} Like \ikeyname{draw tree}, but the node boxes are included in the
%   picture using \cmdname{copy}, not \cmdname{box}, thereby preserving them.     
%   
% \rkeyname[item]{draw tree box}|=|[\meta{\TeX\ box}]  The picture drawn by the subsequent
%   invocations of \ikeyname{draw tree} and \ikeyname{draw tree'} is put into \meta{\TeX\ box}.  If
%   the argument is omitted, the subsequent pictures are typeset normally (the default).
%
% \rkeyname[item]{process keylist}|=|\meta{keylist option name}  Processes the keylist saved in
%   option \meta{keylist option name} for all the nodes in the \emph{whole} tree.
%
%   This key is not sensitive to the current node: it processes the keylists for the whole tree.
%   The calls of this key should \emph{not} be nested.
%
%   Keylist-processing proceeds in cycles.  In a given cycle, the value of option \meta{keylist
%   option name} is processed for every node, in a recursive (parent-first, depth-first) fashion.
%   During a cycle, keys may be \emph{delayed} using key \ikeyname{delay}.  (Keys of the dynamically
%   created nodes are automatically delayed.) Keys delayed in a cycle are processed in the next
%   cycle.  The number of cycles in unlimited.  When no keys are delayed in a cycle, the processing
%   of a hook is finished.
% \end{syntax}
%
% \subsubsection{Dynamic tree}
% \label{ref:dynamic}
%
% The following keys can be used to change the geometry of the tree by creating new nodes and
% integrating them into the tree, moving and copying nodes around the tree, and removing nodes from
% the tree.
% 
% The node that will be (re)integrated into the tree can be specified in the following ways:
% \begin{syntax}
% \item \meta{empty}: uses the last (non-integrated, i.e.\ created/removed/replaced) node.
% \item \meta{node}: a new node is created using the given bracket representation (the node may
%   contain children, i.e.\ a tree may be specified), and used as the argument to the key.
%
%   The bracket representation must be enclosed in brackets, which will usually be enclosed in
%   braces to prevent them being parsed while parsing the ``host tree.''
% \item \imeta{relative node name}: the node \meta{relative node name} resolves to will be used.
% \end{syntax}
%
% Here is the list of dynamic tree keys:
% 
% \begin{syntax}
% \rkeyname[item]{append}|=|\meta{empty}\OR|[|\meta{node}|]|\OR\meta{relative node name}
% 
%   The specified node becomes the new final child of the current node.  If the specified node had a
%   parent, it is first removed from its old position.
%   
% \begin{forestexample}[label=ex:append,point=append,ekeynames={append,delay,for tree,n,content,n',repeat}]
%   \begin{forest}
%     before typesetting nodes={for tree={
%       if n=1{content=L}
%            {if n'=1{content=R}
%                     {content=C}}}}
%     [,repeat=2{append={[
%       ,repeat=3{append={[]}}
%     ]}}]
%   \end{forest}
% \end{forestexample}
%
% \rkeyname[item]{create}|=[|\meta{node}|]|
%
% Create a new node. The new node becomes the last node.
% 
% \rkeyname[item]{insert after}|=|\meta{empty}\OR|[|\meta{node}|]|\OR\meta{relative node name}
% 
%   The specified node becomes the new following sibling of the current node.  If the specified node had a
%   parent, it is first removed from its old position.
%   
% \rkeyname[item]{insert before}|=|\meta{empty}\OR|[|\meta{node}|]|\OR\meta{relative node name}
% 
%   The specified node becomes the new previous sibling of the current node.  If the specified node had a
%   parent, it is first removed from its old position.
%   
% \rkeyname[item]{prepend}|=|\meta{empty}\OR|[|\meta{node}|]|\OR\meta{relative node name}
% 
%   The specified node becomes the new first child of the current node.  If the specified node had a
%   parent, it is first removed from its old position.
%   
% \rkeyname[item]{remove}
%
% The current node is removed from the tree and becomes the last node.
%
% The node itself is not deleted: it is just not integrated in the tree anymore.  Removing the root
% node has no effect.
% 
% \rkeyname[item]{replace by}|=|\meta{empty}\OR|[|\meta{node}|]|\OR\meta{relative node name}
% 
%   The current node is replaced by the specified node.  The current node becomes the last node.
%
%   It the specified node is a new node containing a dynamic tree key, it can refer to the replaced
%   node by the \meta{empty} specification.  This works even if multiple replacements are made.
%   
%   If \keyname{replace by} is used on the root node, the ``replacement'' becomes the root node
%   (\ikeyname{set root} is used). 
%
%   \rkeyname[item]{set root}
%
% The current node becomes the new \emph{formal} root of the tree.
%
% Note: If the current node has a parent, it is \emph{not} removed from it.  The node becomes the
% root only in the sense that the default implementation of stage-processing will consider it a
% root, and thus typeset/pack/draw the (sub)tree rooted in this root.  The processing of keys such
% as \ikeyname{for parent} and \ikeyname{for root} is not affected: \ikeyname{for root} finds the
% real, geometric root of the current node.  To access the formal root, use node walk step
% \ikeyname{root'}, or the corresponding propagator \ikeyname{id={{for }}}\ikeyname{root'}.
% \end{syntax}
%
% If given an existing node, most of the above keys \emph{move} this node
% (and its subtree, of course).  Below are the versions of these operations which rather \emph{copy}
% the node: either the whole subtree (|'|) or just the node itself (|''|).
% \begin{syntax}
% \rkeyname[item]{append'}, \rkeyname{insert after'}, \rkeyname{insert before'}, \rkeyname{prepend'},
%   \rkeyname{replace by'}
%
%   Same as versions without |'| (also the same arguments), but it is the copy of the specified node
%   and its subtree that is integrated in the new place.
% \rkeyname[item]{append''}, \rkeyname{insert after''}, \rkeyname{insert before''}, \rkeyname{prepend''},
%   \rkeyname{replace by''}
%
%   Same as versions without |''| (also the same arguments), but it is the copy of the specified node
%   (without its subtree) that is integrated in the new place.
% \rkeyname[item]{copy name template}|=|\meta{empty}\OR\meta{macro definition} \hfill\meta{empty}
% 
% Defines a template for constructing the \ikeyname{name} of the copy from the name of the
% original.  \meta{macro definition} should be either empty (then, the \ikeyname{name} is
% constructed from the \ikeyname{id=id}, as usual), or an expandable macro taking one argument (the
% name of the original).
% \end{syntax}
%
% \begin{advise}
% \item You might want to \ikeyname{delay} the processing of the copying operations, giving the
%   original nodes the chance to process their keys first!
% \end{advise}
% 
% \begin{forestexample}
%   \begin{forest}
%     copy name template={copy of #1}
%     [CP,delay={prepend'=subject}
%       [VP[DP,name=subject[D][NP]][V'[V][DP]]]]
%     \draw[->,dotted] (subject)--(copy of subject);
%   \end{forest}
% \end{forestexample}
%
% A dynamic tree operation is made in two steps:
% \begin{itemize}
% \item If the argument is given by a \meta{node} argument, the new node is created immediately,
%   i.e.\ while the dynamic tree key is being processed.  Any options of the new node are
%   implicitely \ikeyname{delay}ed. 
% \item The requested changes in the tree structure are actually made between the cycles of keylist
%   processing.
% \end{itemize}
% 
% \begin{advise}
% \item Such a two-stage approach is employed because changing the tree structure during the dynamic
%   tree key processing would lead to an unmanageable order of keylist processing.
% \item A consequence of this approach is that nested dynamic tree keys take several cycles to
%   complete.  Therefore, be careful when using \ikeyname{delay} and dynamic tree keys
%   simultaneously: in such a case, it is often safer to use \ikeyname{before typesetting nodes}
%   instead of \ikeyname{delay}, see example \ref{ex:append}.
% \item Further examples: title page (in style |random tree|), \ref{ex:xlist}.
% \end{advise}
%
% \subsection{Handlers}
% \label{ref:handlers}
%
% \begin{syntax}
% \rkeyname[item]{id=.pgfmath,nfc}|=|\meta{pgfmath expression}
%
% The result is the evaluation of \meta{pgfmath expression} in the context of the current node.
% 
% \rkeyname[item]{id=.wrap value,nfc}|=|\meta{macro definition}
%
% The result is the (single) expansion of the given
%   \meta{macro definition}.  The defined macro takes one parameter.  The current value of the
%   handled option will be passed as that parameter.
%   
% \rkeyname[item]{id=.wrap $n$ pgfmath args,nfc}|=|\meta{macro definition}\meta{arg $1$}\dots\meta{arg $n$}
% 
%   The result is the (single) expansion of the given \meta{macro definition}.  The defined macro
%   takes $n$ parameters, where $n\in\{2,\dots,8\}$.  Expressions \meta{arg $1$} to \meta{arg $n$}
%   are evaluated using |pgfmath| and passed as arguments to the defined macro.
%   
% \rkeyname[item]{id=.wrap pgfmath arg,nfc}|=|\meta{macro definition}\meta{arg}
%
% Like \ikeyname{id=.wrap $n$ pgfmath args,nfc} for $n=1$. 
% \end{syntax}
% 
% \subsection{Relative node names}
% \label{ref:relative-node-names}
%
% \begin{syntax}
% \item\rmeta{relative node name}|=|[\meta{forest node name}][|!|\meta{node walk}]
%
%   \meta{relative node name} refers to the \foRest; node at the end of the \meta{node walk}
%   starting at node named \meta{forest node name}.  If \meta{forest node name} is omitted, the walk
%   starts at the current node.  If \meta{node walk} is omitted, the ``walk'' ends at the start
%   node.   (Thus, an empty \meta{relative node name} refers to the current node.)
% \end{syntax}
%
% Relative node names can be used in the following contexts:
% \begin{itemize}
% \item \FoRest;'s |pgfmath| option functions (\S\ref{ref:pgfmath}) take a relative node name as
%   their argument, e.g.\ |content("!u")| and |content("!parent")| refer to the content of the
%   parent node.
% \item An option of a non-current node can be set by \meta{relative node name}|.|\meta{option
%   name}|=|\meta{value}, see \S\ref{ref:options-and-keys}.
% \item The |forest| coordinate system, both explicit and implicit; see \S\ref{ref:forest-cs}.
% \end{itemize}
%
% \subsubsection{Node walk}
% \label{ref:node-walk}
%
% A \rmeta{node walk} is a sequence of \rmeta{step}s describing a path through the tree.  
% The primary use of node walks is in relative node names.  However, they can also be used in a
% ``standalone'' way, using key \ikeyname{node walk}; see \S\ref{ref:miscellaneous}.
%
% Steps are keys in the \keyname{/forest/node walk} path.  (\foRest; always sets this path as
% default when a node walk is to be used, so step keynames can be used.)  Formally, a \meta{node
% walk} is thus a keylist, and steps must be separated by commas.  There is a twist, however.  Some
% steps also have \emph{short} names, which consist of a single character.  The comma between two
% adjacent short steps can be omitted.  Examples:
% \begin{itemize}
% \item |parent,parent,n=2| or |uu2|: the grandparent's second child (of the current node)
% \item |first leaf,uu|: the grandparent of the first leaf (of the current node)
% \end{itemize}
% The list of long steps:
% \newcommand\nwritem[1]{\rkeyname[item]{#1,aspect=step}\ekeyname{for #1,aspect=propagator,def}}
% \begin{syntax}
% \nwritem{current} an ``empty'' step: the current node remains the same\footnote{While it
%   might at first sight seem stupid to have an empty step, this is not the case.  For example,
%   using propagator \ikeyname{for current} derived from this step, one can process a \meta{keylist}
%   constructed using \texttt{.wrap (n) pgfmath arg(s)}\ekeyname{id=.wrap 
%   pgfmath arg,nfc}\ekeyname{id=.wrap $n$ pgfmath args,nfc} or \ikeyname{id=.wrap value,nfc}.} 
% \nwritem{first} the primary child
% \nwritem{first leaf} the first leaf (terminal node)
% \rkeyname[item]{group,aspect=step}|=|\meta{node walk}  treat the given \meta{node walk} as a single step
% \nwritem{last} the last child
% \nwritem{last leaf} the last leaf
% \nwritem{id=id}|=|\meta{id} the node with the given id
% \nwritem{linear next} the next node, in the processing order 
% \nwritem{linear previous} the previous node, in the processing order 
% \nwritem{n}|=|$n$ the $n$th child; counting starts at $1$ (not $0$)
% \nwritem{n'}|=|$n$ the $n$th child, starting the count from the last child
% \nwritem{name} the node with the given name
% \nwritem{next} the next sibling 
% \nwritem{next leaf} the next leaf
%
% (the current node need not be a leaf) 
% \nwritem{next on tier} the next node on the same tier as the current node 
% \rkeyname[item]{node walk,aspect=step}|=|\meta{node walk}  embed the given \meta{node walk}
%
% (the \ikeyname{id={node walk/before walk}} and \ikeyname{id={node walk/after walk}} are processed)
% \nwritem{parent} the parent 
% \nwritem{previous} the previous sibling 
% \nwritem{previous leaf} the previous leaf
%
% (the current node need not be a leaf) 
% \nwritem{previous on tier} the next node on the same tier as the current node 
% \rkeyname[item]{repeat}|=|$n$\meta{node walk} repeat the given \meta{node walk} $n$ times
%
% (each step in every repetition counts as a step)
% \nwritem{root} the root node
% \nwritem{root'} the formal root node (see \ikeyname{set root} in \S\ref{ref:dynamic})
% \nwritem{sibling} the sibling
%
% (don't use if the parent doesn't have exactly two children \dots)
% \nwritem{to tier}|=|\meta{tier} the first ancestor of the current node on the given \meta{tier}
% \rkeyname[item]{trip,aspect=step}|=|\meta{node walk}  after walking the embedded \meta{node walk}, return to the
%   current node; the return does not count as a step
% \end{syntax}
%
% For each long \meta{step} except \keyname{node walk}, \keyname{group}, \keyname{trip} and
% \keyname{repeat}, propagator \ikeyname{id={{for }}}\meta{step} is also defined.  Each such
% propagator takes a \meta{keylist} argument.  If the step takes an argument, then so does its
% propagator; this argument precedes the \meta{keylist}.  See also \S\ref{ref:propagators}.
% 
% Short steps are single-character keys in the \keyname{/forest/node walk} path.  They are defined
% as styles resolving to long steps, e.g.\ |1/.style={n=1}|.  The list of predefined short steps
% follows. 
% \begin{syntax}
% \rkeyname[item]{1},
%   \rkeyname{2},
%   \rkeyname{3},
%   \rkeyname{4},
%   \rkeyname{5},
%   \rkeyname{6},
%   \rkeyname{7},
%   \rkeyname{8},
%   \rkeyname{9} the first, \dots, ninth child
% \rkeyname[item]{l,aspect=short step} the last child
% \rkeyname[item]{u} the parent (up)
% \rkeyname[item]{p} the previous sibling
% \rkeyname[item]{n,aspect=short step} the next sibling
% \rkeyname[item]{s,aspect=short step} the sibling
% \rkeyname[item]{P} the previous leaf
% \rkeyname[item]{N} the next leaf
% \rkeyname[item]{F} the first leaf
% \rkeyname[item]{L} the last leaf
% \rkeyname[item]{id=<<<,display=\protect\myindexgt,text=>,aspect=short step}
%   the next node on the current tier
% \rkeyname[item]{<} the previous node on the current tier
% \rkeyname[item]{c} the current node
% \rkeyname[item]{r} the root node
% \end{syntax}
% \begin{advise}
% \item You can define your own short steps, or even redefine predefined short steps!
% \end{advise}
% 
% \subsubsection{The \texttt{forest} coordinate system}
% \label{ref:forest-cs}
% 
% Unless package options \ikeyname{tikzcshack} is set to |false|, \TikZ;'s implicit node coordinate
% system \citep[\S13.2.3]{tikzpgf2.10} is hacked to accept relative node names.\footnote{Actually,
% the hack can be switched on and off on the fly, using \cmdname{i}\keyname{fforesttikzcshack}.}.
% 
% The explicit \texttt{forest} coordinate system is called simply |forest| and used like this:
% |(forest_cs:|\meta{forest cs spec}|)|; see \citep[\S13.2.5]{tikzpgf2.10}.  \meta{forest cs spec}
% is a keylist; the following keys are accepted.
% 
% \begin{syntax}
% \rkeyname[item]{name,aspect=forest cs}|=|\meta{node name}  The node with the given name becomed the current node.  The
% resulting point is its (node) anchor.
% \rkeyname[item]{id=id,aspect=forest cs}|=|\meta{node id}  The node with the given name becomed the current node. The
% resulting point is its (node) anchor.
% \rkeyname[item]{go,aspect=forest cs}|=|\meta{node walk}  Walk the given node walk, starting at the current node.  The node
% at the end of the walk becomes the current node.  The resulting point is its (node) anchor.
% \rkeyname[item]{anchor,aspect=forest cs}|=|\meta{anchor}  The resulting point is the given anchor of the current node.
% \rkeyname[item]{l,aspect=forest cs}|=|\meta{dimen} \vspace{-\parskip}
% \rkeyname[item]{s,aspect=forest cs}|=|\meta{dimen}  Specify the \ikeyname{l} and \ikeyname{s}
% coordinate of the resulting point.
%
% The coordinate system is the node's ls-coordinate system: its origin is at its (node) anchor; the
% l-axis points in the direction of the tree growth at the node, which is given by option
% \ikeyname{grow};  the s-axis is orthogonal to the l-axis; the positive side is in the
% counter-clockwise direction from |l| axis.
%
% The resulting point is computed only after both \ikeyname{l} and \ikeyname{s} were given. 
% \item Any other key is interpreted as a \imeta{relative node name}[.\meta{anchor}].
% \end{syntax}
% 
% \subsection{New \texttt{pgfmath} functions}
% \label{ref:pgfmath}
%
% For every option, \foRest; defines a pgfmath function with the same name, with the
% proviso that all non-alphanumeric characters in the option name are replaced by an underscore
% |__| in the pgfmath function name.
%
% Pgfmath functions corresponding to options take one argument, a \imeta{relative node name}
% (see~\S\ref{ref:relative-node-names}) expression, making it possible to refer to option values of
% non-current nodes.  The \meta{relative node name} expression must be enclosed in double quotes in
% order to 
% prevent pgfmath evaluation: for example, to refer to the content of the parent, write
% \ikeyname{content}|("!u")|.  To refer to the option of the current node, use empty parentheses:
% \ikeyname{content}|()|.\footnote{In most cases, the parentheses are optional, so \texttt{content}
% is ok.  A known case where this doesn't work is preceding an operator: \texttt{l+1cm} will fail.}
% 
% Three string functions are also added to |pgfmath|: \rkeyname{strequal} tests the equality of
% its two arguments; \rkeyname{instr} tests if the first string is a substring of the second one;
% \rkeyname{strcat} joins an arbitrary number of strings.
%
% Some random notes on |pgfmath|: \begin{inparaenum}[(i)]
% \item |&&|, \verb!||! and |!| are boolean ``and'', ``or'' and ``not'', respectively.
% \item The equality operator (for numbers and dimensions) is |==|, \emph{not} |=|.
% \end{inparaenum}  And some examples:
% 
% \begin{forestexample}[pos=t,ekeynames={for tree,grow',calign,l,l sep,child
% anchor,anchor,fit,tier,level,delay,before typesetting nodes,content,{id=.wrap 2 pgfmath args,nfc},{id=.pgfmath,nfc}}]
%   \begin{forest}
%     for tree={grow'=0,calign=first,l=0,l sep=2em,child anchor=west,anchor=base
%       west,fit=band,tier/.pgfmath=~level~()},
%     fullpath/.style={if n=0{}{content/.wrap 2
%         pgfmath args={##1/##2}{~content~("!u")}{~content~()}}},
%     delay={for tree=fullpath,content=/},
%     before typesetting nodes={for tree={content=\strut#1}}
%     [
%       [home
%         [joe
%           [\TeX]]
%         [saso
%           [\TeX]]
%         [a user with a long name
%           [\TeX]]]
%       [usr]]
%   \end{forest}
% \end{forestexample}
%
% \begin{forestexample}[point=instr,ekeynames={delay,for tree,if,content,n children}]
%   \begin{forest}
%     delay={for tree={if=
%       {!instr("!P",~content~) && ~n_children~==0}
%       {fill=yellow}
%       {}
%     }}
%     [CP[DP][C'[C][TP[DP][T'[T][VP[DP][V'[V][DP]]]]]]]
%   \end{forest}
% \end{forestexample}
%
% \begin{forestexample}[point=instr,ekeynames={where ,n children,tier,content,no edge,tikz}]
%   \begin{forest}
%     where n children=0{tier=word,
%       if={~instr~("!P",~content~("!u"))}{no edge,
%         tikz={\draw (!.north west)--
%         (!.north east)--(!u.south)--cycle;
%       }}{}
%     }{},
%     [VP[DP[John]][V'[V[loves]][DP[Mary]]]]
%   \end{forest}
% \end{forestexample}
%
%
% \subsection{Standard node}
% \label{ref:standard-node}
%
% \begin{syntax}
% \item\rcmdname{forestStandardNode}\meta{node}\meta{environment fingerprint}\meta{calibration
%   procedure}\meta{exported options}
%
%   This macro defines the current \emph{standard node}.  The standard node declares some options as
%   \emph{exported}.  When a new node is created, the values of the exported options are initialized
%   from the standard node.  At the beginning of every \ikeyname{forest} environment, it is checked whether
%   the \emph{environment fingerprint} of the standard node has changed.  If it did, the standard
%   node is \emph{calibrated}, adjusting the values of exported options.  The \emph{raison d'etre} for
%   such a system is given in \S\ref{sec:defaults}.
%
%   In \meta{node}, the standard node's content and possibly other options are specified, using the
%   usual bracket representation.  The \meta{node}, however, \emph{must not contain children}.  The
%   default: \texttt{[dj]}.
%
%   The \meta{environment fingerprint} must be an expandable macro definition.  It's expansion
%   should change whenever the calibration is necessary.
%
%   \meta{calibration procedure} is a keylist (processed in the |/forest| path) which calculates the
%   values of exported options.
%
%   \meta{exported options} is a comma-separated list of exported options.
%
%   This is how the default standard node is created:
% \begin{lstlisting}
%   \forestStandardNode[dj]
%     {%
%       \forestOve{\csname forest@id@of@standard node\endcsname}{content},%
%       \the\ht\strutbox,\the\pgflinewidth,%
%       \pgfkeysvalueof{/pgf/inner ysep},\pgfkeysvalueof{/pgf/outer ysep},%
%       \pgfkeysvalueof{/pgf/inner xsep},\pgfkeysvalueof{/pgf/outer xsep}%
%     }
%     {
%       l sep={\the\ht\strutbox+\pgfkeysvalueof{/pgf/inner ysep}},
%       l={l_sep()+abs(max_y()-min_y())+2*\pgfkeysvalueof{/pgf/outer ysep}},
%       s sep={2*\pgfkeysvalueof{/pgf/inner xsep}}
%     }
%     {l sep,l,s sep}
%   \end{lstlisting}
% \end{syntax}
%
% \subsection{Externalization}
% \label{ref:externalization}
%
% Externalized tree pictures are compiled only once. The result of the compilation is saved into a
% separate |.pdf| file and reused on subsequent compilations of the document.  If the code of the
% tree (or the context, see below) is changed, the tree is automatically recompiled.
%
% Externalization is enabled by:
% \begin{lstlisting}
%   \usepackage[~external~]{forest}
%   ~\tikzexternalize~
% \end{lstlisting}
% Both lines are necessary.  \TikZ;'s externalization library is automatically loaded if necessary.
%
% \begin{syntax}
% \rkeyname[item]{id={external/optimize}} Parallels \keyname{/tikz/external/optimize}: if |true| (the
% default), the processing of non-current trees is skipped during the embedded compilation.
% \rkeyname[item]{id={external/context}} If the expansion of the macro stored in
% this option changes, the tree is recompiled.
% \rkeyname[item]{id={external/depends on macro}}|=|\meta{cs} Adds the definition of macro \meta{cs} to
% \keyname{external/context}.  Thus, if the definition of \meta{cs} is changed, the tree will be
% recompiled. 
% \end{syntax}
%
% \foRest; respects or is compatible with several (not all) keys and commands of \TikZ;'s
% externalization library.  In particular, the following keys and commands might be useful; see
% \cite[\S32]{tikzpgf2.10}.
% \begin{itemize}
% \item\keyname{/tikz/external/remake next}
% \item\keyname{/tikz/external/prefix}
% \item\keyname{/tikz/external/system call}
% \item\cmdname{tikzexternalize}
% \item\cmdname{tikzexternalenable}
% \item\cmdname{tikzexternaldisable}
% \end{itemize}
% \FoRest; does not disturbe the externalization of non-\foRest; pictures. (At least it
% shouldn't \dots)
%
% The main auxiliary file for externalization has suffix |.for|.  The externalized pictures have
% suffices |-forest-|$n$ (their prefix can be set by \keyname{/tikz/external/prefix}, e.g.\ to a
% subdirectory).  Information on all trees that were ever externalized in the document (even if
% they were changed or deleted) is kept.  If you need a ``clean'' |.for| file, delete it and
% recompile.  Deleting |-forest-|$n$|.pdf| will result in recompilation of a specific tree.
%
% Using \keyname{draw tree} and \keyname{draw tree'} multiple times \emph{is} compatible with
% externalization, as is drawing the tree in the box (see \ikeyname{draw tree box}).  If you are
% trying to externalize a \ikeyname{forest} environment which utilizes \ikeyname{TeX} to produce a
% visible effect, you will probably need to use \ikeyname{TeX'} and/or \ikeyname{TeX''}.
% 
% \subsection{Package options}
% \label{ref:package-options}
%
% \begin{syntax}
% \rkeyname[item=false]{external}|=|\alternative{true,false}
%
% Enable/disable externalization, see \S\ref{ref:externalization}.
% \rkeyname[item=true]{tikzcshack}|=|\alternative{true,false}
%
% Enable/disable the hack into \TikZ;'s implicite coordinate syntax hacked, see
% \S\ref{ref:relative-node-names}.
%
% \rkeyname[item=true]{tikzinstallkeys}|=|\alternative{true,false}
%
% Install certain keys into the \keyname{/tikz} path.  Currently: \ikeyname{fit to tree}.
% \end{syntax}
%
% \section{Gallery}
% \label{sec:gallery}
%
%
% \subsection{Styles}
% \label{sec:gallery-styles}
%
% \paragraph{\rkeyname{GP1}}
% For Government Phonology (v1) representations. Here, the big trick
% is to evenly space $\times$s by having a large enough |outer_xsep|
% (adjustable), and then, before drawing (timing control option
% |before_drawing_tree|), setting |outer_xsep| back to 0pt.  The last step
% is important, otherwise the arrows between $\times$s won't draw!
%
% \box\GPone
%
% An example of an ``embedded'' |GP1| style:
% \begin{forestexample}[pos=b,ekeynames={where ,tier,for children,content,tikz,l,+,no edge}]
%   \begin{forest}
%     myGP1/.style={
%       ~GP1~,
%       delay={where tier={x}{
%           for children={content=\textipa{##1}}}{}},
%       tikz={\draw[dotted](.south)--
%             (!1.north west)--(!l.north east)--cycle;},
%       for children={l+=5mm,no edge}
%     }
%     [VP[DP[John,tier=word,myGP1
%             [O[x[dZ]]]
%             [R[N[x[6]]]]
%             [O[x[n]]]
%             [R[N[x]]]
%     ]][V'[V[loves,tier=word,myGP1
%              [O[x[l]]]
%              [R[N[x[a]]]]
%              [O[x[v]]]
%              [R[N[x]]]
%              [O[x[z]]]
%              [R[N[x]]]
%     ]][DP[Mary,tier=word,myGP1
%            [O[x[m]]]
%            [R[N[x[e]]]]
%            [O[x[r]]]
%            [R[N[x[i]]]]
%     ]]]]
%   \end{forest}%
% \end{forestexample}
%
% And an example of annotations.
% \begin{forestexample}
%   \begin{forest}[,phantom,s sep=1cm
%     [{[ei]}, GP1
%       [R[N[x[A,~el~[I,~head~,~associate=N~]]][x]]]
%     ]
%     [{[mars]}, GP1
%       [O[x[m]]]
%       [R[N[x[a]]][x,~encircle~,densely dotted[r]]]
%       [O[x,~encircle~,~govern=<~[s]]]
%       [R,~fen~[N[x]]]
%     ]
%   ]\end{forest} 
% \end{forestexample}
%
%
% \paragraph{rlap and llap}  The \foRest; versions of \TeX's \cmdname{rlap}\ and \cmdname{llap}: the
% ``content'' added by these styles will influence neither the packing algorithm nor the anchor
% positions. 
% \begin{forestexample}[pos=b,point={rlap,llap},ekeynames={TeX,delay,where ,tier,content,GP1}]
%   \forestset{
%     llap/.style={tikz+={
%         \edef\forest@temp{\noexpand\node[\forestoption{node options},
%           anchor=base east,at=(.base east)]}
%         \forest@temp{#1\phantom{\forestoption{content format}}};
%       }},
%     rlap/.style={tikz+={
%         \edef\forest@temp{\noexpand\node[\forestoption{node options},
%           anchor=base west,at=(.base west)]}
%         \forest@temp{\phantom{\forestoption{content format}}#1};
%       }}
%   }
%   \newcount\xcount
%   \begin{forest} GP1,
%     delay={
%       TeX={\xcount=0},
%       where tier={x}{TeX={\advance\xcount1},rlap/.expanded={$_{\the\xcount}$}}{}
%     }
%     [
%       [O[x[f]]]
%       [R[N[x[o]]]]
%       [O[x[r]]]
%       [R[N[x[e]]][x[s]]]
%       [O[x[t]]]
%       [R[N[x]]]
%     ]
%   \end{forest}
% \end{forestexample}
%
% \paragraph{xlist} This style makes it easy to put ``separate''
% trees in a picture and enumerate them.  For an example, see the |nice_empty_nodes|
% style.
% \begin{forestexample}[pos=t,label=ex:xlist]
%   \makeatletter
%   \forestset{
%     xlist/.style={
%       phantom,
%       for children={no edge,replace by={[,append,
%           delay={content/.wrap pgfmath arg={\@alph{##1}.}{n()+#1}}
%           ]}}
%     },
%     xlist/.default=0
%   }
%   \makeatother
% \end{forestexample}
% \input{\jobname.tmp}
% 
% \paragraph{nice empty nodes} 
% We often need empty nodes: tree (a) shows how they look like by
% default: ugly.
%
% First, we don't want the gaps: we change the shape of empty nodes to coordinate.  We get tree (b).
%
% Second, the empty nodes seem too close
% to the other (especially empty) nodes (this is a result of a small
% default |s_sep|).  We could use a greater \ikeyname{s sep}, but a better solution seems
% to be to use |calign=node_angle|.  The result is shown in (c). 
%
% However, at the transitions from empty to non-empty nodes, tree (d)
% above seems to zigzag (although the base points of the spine nodes
% are perfectly in line), and the edge to the empty node left to VP
% seems too long (it reaches to the level of VP's base, while we'd
% prefer it to stop at the same level as the edge to VP itself).  The
% first problem is solved by substituting |node_angle| for
% |edge_angle|; the second one, by anchoring siblings of
% empty nodes at north.
% \begin{forestexample}[pos=b,ekeynames={fixed angles,fixed edge angles,calign,for tree,delay,where
% ,content,for ,parent,for children,anchor}]
%   \forestset{
%     ~nice empty nodes~/.style={
%       for tree={calign=fixed edge angles},
%       delay={where content={}{shape=coordinate,for parent={for children={anchor=north}}}{}}
%     }}
%   \begin{forest}
%     [,~xlist~
%       [CP,                                                               %(a)
%         [][[][[][VP[DP[John]][V'[V[loves]][DP[Mary]]]]]]]
%       [CP, delay={where content={}{shape=coordinate}{}}                  %(b)
%         [][[][[][VP[DP[John]][V'[V[loves]][DP[Mary]]]]]]]
%       [CP, for tree={calign=fixed angles},                                 %(c)
%            delay={where content={}{shape=coordinate}{}}
%         [][[][[][VP[DP[John]][V'[V[loves]][DP[Mary]]]]]]]
%       [CP, ~nice empty nodes~                                              %(d)
%         [][[][[][VP[DP[John]][V'[V[loves]][DP[Mary]]]]]]]
%     ]
%   \end{forest}
% \end{forestexample}
%
%
% \subsection{Examples}
% \label{sec:examples}
%
% The following example was inspired by a question on \TeX\ Stackexchange:
% \href{http://tex.stackexchange.com/questions/39103/how-to-change-the-level-distance-in-tikz-qtree-for-one-level-only}{How to change the level distance in tikz-qtree for one level only?}.  The question is about |tikz-qtree|: how to adjust the level distance for the first level only, in order to avoid first-level labels crossing the parent--child edge.  While this example solves the problem (by manually shifting the offending labels; see \texttt{elo} below), it does more: the preamble is setup so that inputing the tree is very easy.
%
% \begin{forestexample}[pos=t,ekeynames={id={{if }},n,no edge,tikz,strequal,strcat,child anchor,parent
% anchor,anchor,calign,for tree,s sep,l,n children,declare toks,delay,content,before typesetting nodes,for descendants,+,{id=.wrap pgfmath arg,nfc},{id=.wrap 2 pgfmath args,nfc}}]
%   \def\getfirst#1;#2\endget{#1}
%   \def\getsecond#1;#2\endget{#2}
%   \forestset{declare toks={elo}{}} % edge label options
%   \begin{forest}
%     anchors/.style={anchor=#1,child anchor=#1,parent anchor=#1},
%     for tree={
%       s sep=0.5em,l=8ex,
%       if n children=0{anchors=north}{
%         if n=1{anchors=south east}{anchors=south west}},
%       content format={$\forestoption{content}$}
%     },
%     anchors=south, outer sep=2pt,
%     nomath/.style={content format=\forestoption{content}},
%     dot/.style={tikz+={\fill (.child anchor) circle[radius=#1];}},
%     dot/.default=2pt,
%     dot=3pt,for descendants=dot,
%     decision edge label/.style n args=3{
%       edge label/.expanded={node[midway,auto=#1,anchor=#2,\forestoption{elo}]{\strut$#3$}}
%     },
%     decision/.style={if n=1
%       {decision edge label={left}{east}{#1}}
%       {decision edge label={right}{west}{#1}}
%     },
%     delay={for descendants={
%         decision/.expanded/.wrap pgfmath arg={\getsecond#1\endget}{content},
%         content/.expanded/.wrap pgfmath arg={\getfirst#1\endget}{content},
%     }},
%     [N,nomath
%       [I;{p_1=0.5},nomath,elo={yshift=4pt}
%         [{5,1};a]
%         [II;b,nomath
%           [{1,2};m]
%           [{2,3};n]
%         ]
%       ]
%       [II;{p_2=0.5},nomath,elo={yshift=4pt}
%         [;c
%           [{1,0};z]
%           [{2,2};t]
%         ]
%         [;d
%           [{3,1};z]
%           [{0,0};t]
%         ]
%       ] {\draw[dashed](!1.anchor)--(!2.anchor) node[pos=0.5,above]{I};}
%     ]
%   \end{forest}
% \end{forestexample}
% 
%
% \section{Known bugs}
% \label{sec:bugs}
%
% If you find a bug (there are bound to be some \dots), please contact
% me at \href{mailto:saso.zivanovic@guest.arnes.si}{saso.zivanovic@guest.arnes.si}.
%
% \paragraph{System requirements} This package requires \LaTeX\ and e\TeX.  If you use something
% else: sorry.
%
% The requirement for \LaTeX\ might be dropped in the future, when I get some time and energy for a
% code-cleanup (read: to remedy the consequences of my bad programming practices and general
% disorganization).
%
% The requirement for e\TeX\ will probably stay.  If nothing else, \foRest; is heavy on boxes: every
% node requires its own \dots\ and consequently, I have freely used e\TeX\ constructs in the code
% \dots
%
% \paragraph{\PGF; internals} \FoRest; relies on some details of \PGF; implementation, like the name
% of the ``not yet positioned'' nodes.  Thus, a new bug might appear with the development of \PGF;.
% If you notice one, please let me know.
%
% \paragraph{Edges cutting through sibling nodes}
% \label{sec:cutting-edge}
%
% In the following example, the R--B edge crosses the AAA node, although \ikeyname{ignore edge} is
% set to the default |false|.
% \begin{forestexample}[ekeynames={calign,{first,aspect=calign},align,{center,aspect=align},base,{bottom,aspect=base}}]
%   \begin{forest}
%     calign=first
%     [R[AAAAAAAAAA\\AAAAAAAAAA\\AAAAAAAAAA,align=center,base=bottom][B]]
%   \end{forest}
% \end{forestexample}
% This happens because s-distances between the adjacent children are
% computed before child alignment  (which is obviously the correct order in the general case), but
% child alignment non-linearly influences the edges.  Observe that the with a different value of
% \ikeyname{calign}, the problem does not arise. 
% \begin{forestexample}[ekeynames={calign,{last,aspect=calign},align,{center,aspect=align},base,{bottom,aspect=base}}]
%   \begin{forest}
%     calign=last
%     [R[AAAAAAAAAA\\AAAAAAAAAA\\AAAAAAAAAA,align=center,base=bottom][B]]
%   \end{forest}
% \end{forestexample}
% While it would be possible to fix the situation after child alignment (at least for some child
% alignment methods), I have decided against that, since the distances between siblings would soon
% become too large.  If the AAA node in the example above was large enough, B could easily be pushed
% off the paper.  The bottomline is, please use manual adjustment to fix such situations. 
%
% \paragraph{Orphans}
% \label{sec:orphans}
%
% If the \ikeyname{l} coordinates of adjacent children are too different (as a result of manual adjustment or
% tier alignment), the packing algorithm might have nothing so say about the desired distance
% between them: in this sense, node C below is an ``orphan.'' 
% \begin{forestexample}[ekeynames={for tree,s sep,l,*}]
%   \begin{forest}
%     for tree={s sep=0,draw},
%     [R[A][B][C,l*=2][D][E]]
%   \end{forest}
% \end{forestexample}
% To prevent orphans from ending up just anywhere, I have decided to vertically align them with
% their preceding sibling --- although I'm not certain that's really the best solution.  In other
% words, you can rely that the sequence of s-coordinates of siblings is non-decreasing.
%
% The decision also incluences a similar situation, illustrated below.  The packing algorithm puts
% node E immediately next to B (i.e.\ under C): however, the monotonicity-retaining mechanism then
% vertically aligns it with its preceding sibling, D.
% \begin{forestexample}[ekeynames={for tree,s sep,tier}]
%   \begin{forest}
%     for tree={s sep=0,draw},
%     [R[A[B,tier=bottom]][C][D][E,tier=bottom]]
%   \end{forest}
% \end{forestexample}
%
% Obviously, both examples also create the situation of an edge crossing some sibling node(s).
% Again, I don't think anything sensible can be done about this, in general.
%
% \section{Changelog}
% 
% \begin{description}
% \item[v1.04 (2013/10/17)] \mbox{}
%   \begin{compactitem}
%   \item Fixed an \href{http://tex.stackexchange.com/questions/138986/error-using-tikzexternalize-with-forest/139145}{externalization bug}.
%   \end{compactitem}
% \item[v1.03 (2013/01/28)] \mbox{}
%   \begin{compactitem}
%   \item Bugfix: options of dynamically created nodes didn't get processed.
%   \item Bugfix: the bracket parser was losing spaces before opening braces.
%   \item Bugfix: a family of utility macros dealing with affixing token lists was not expanding
%     content correctly.
%   \item Added style \ikeyname{math content}.
%   \item Replace key \keyname{tikz preamble} with more general \ikeyname{begin draw} and
%     \ikeyname{end draw}.
%   \item Add keys \ikeyname{begin forest} and \ikeyname{end forest}.
%   \end{compactitem}
% \item[v1.02 (2013/01/20)] \mbox{}
%   \begin{compactitem}
%   \item Reworked style \ikeyname{stages}: it's easier to modify the processing flow now.
%   \item Individual stages must now be explicitely called in the context of some (usually root)
%     node. 
%   \item Added \ikeyname{delay n} and \ikeyname{if have delayed}.
%   \item Added (experimental) \ikeyname{pack'}.
%   \item Added reference to the \href{https://github.com/sasozivanovic/forest-styles}{style
%     repository}.
%   \end{compactitem}
% \item[v1.01 (2012/11/14)] \mbox{}
% 
%   \begin{compactitem}
%   \item Compatibility with the |standalone| package: temporarily disable the effect of
%     |standalone|'s package option |tikz| while typesetting nodes.
%   \item Require at least the [2010/08/21] (v2.0) release of package |etoolbox|.
%   \item Require version [2010/10/13] (v2.10, rcs-revision 1.76) of \PGF;/\TikZ;.  Future
%     compatibility: adjust to the change of the ``not yet positioned'' node name (2.10 |@|
%     $\rightarrow$ 2.10-csv |PGFINTERNAL|).
%   \item Add this changelog.
%   \end{compactitem}
% \item[v1.0 (2012/10/31)] First public version
% \end{description}
%
% \paragraph{Acknowledgements} Many thanks to the people who have reported bugs! In the
% chronological order: Markus P\"ochtrager, Timothy Dozat, Ignasi Furio.\footnote{If you're in the
% list but don't want to be, my apologies and please let me know about it!}
%
% \newpage
% \part{Implementation}
%
% A disclaimer: the code could've been much cleaner and better-documented \dots
% 
% Identification.
%    \begin{macrocode}
\ProvidesPackage{forest}[2013/10/17 v1.04 Drawing (linguistic) trees]

\RequirePackage{tikz}[2010/10/13]
\usetikzlibrary{shapes}
\usetikzlibrary{fit}
\usetikzlibrary{calc}
\usepgflibrary{intersections}

\RequirePackage{pgfopts}
\RequirePackage{etoolbox}[2010/08/21]
\RequirePackage{environ}

 %\usepackage[trace]{trace-pgfkeys}
%    \end{macrocode}
%
% |/forest| is the root of the key hierarchy.
%    \begin{macrocode}
\pgfkeys{/forest/.is family}
\def\forestset#1{\pgfqkeys{/forest}{#1}}
%    \end{macrocode}
%
% \section{Patches}
% These patches apply to pgf/tikz 2.10.
% 
% Serious: forest cannot load if this is not patched; disable
% \texttt{/handlers/.wrap n pgfmath} for n=6,7,8 if you cannot patch.
%    \begin{macrocode}
\long\def\forest@original@pgfkeysdefnargs@#1#2#3#4{%
  \ifcase#2\relax
	  \pgfkeyssetvalue{#1/.@args}{}%
  \or
	  \pgfkeyssetvalue{#1/.@args}{##1}%
  \or
	  \pgfkeyssetvalue{#1/.@args}{##1##2}%
  \or
	  \pgfkeyssetvalue{#1/.@args}{##1##2##3}%
  \or
	  \pgfkeyssetvalue{#1/.@args}{##1##2##3##4}%
  \or
	  \pgfkeyssetvalue{#1/.@args}{##1##2##3##4##5}%
  \or
	  \pgfkeyssetvalue{#1/.@args}{##1##2##3##4##5##6}%
  \or
	  \pgfkeyssetvalue{#1/.@args}{##1##2##3##4##5##6}%
  \or
	  \pgfkeyssetvalue{#1/.@args}{##1##2##3##4##5##6##7}%
  \or
	  \pgfkeyssetvalue{#1/.@args}{##1##2##3##4##5##6##7##8}%
  \or
	  \pgfkeyssetvalue{#1/.@args}{##1##2##3##4##5##6##7##8##9}%
  \else
	  \pgfkeys@error{\string\pgfkeysdefnargs: expected  <= 9 arguments, got #2}%
  \fi
  \pgfkeysgetvalue{#1/.@args}\pgfkeys@tempargs
  \def\pgfkeys@temp{\expandafter#4\csname pgfk@#1/.@@body\endcsname}%
  \expandafter\pgfkeys@temp\pgfkeys@tempargs{#3}%
  % eliminate the \pgfeov at the end such that TeX gobbles spaces
  % by using
  % \pgfkeysdef{#1}{\pgfkeysvalueof{#1/.@@body}##1}
  % (with expansion of '#1'):
  \edef\pgfkeys@tempargs{\noexpand\pgfkeysvalueof{#1/.@@body}}%
  \def\pgfkeys@temp{\pgfkeysdef{#1}}%
  \expandafter\pgfkeys@temp\expandafter{\pgfkeys@tempargs##1}%
  \pgfkeyssetvalue{#1/.@body}{#3}%
}

\long\def\forest@patched@pgfkeysdefnargs@#1#2#3#4{%
  \ifcase#2\relax
	  \pgfkeyssetvalue{#1/.@args}{}%
  \or
	  \pgfkeyssetvalue{#1/.@args}{##1}%
  \or
	  \pgfkeyssetvalue{#1/.@args}{##1##2}%
  \or
	  \pgfkeyssetvalue{#1/.@args}{##1##2##3}%
  \or
	  \pgfkeyssetvalue{#1/.@args}{##1##2##3##4}%
  \or
	  \pgfkeyssetvalue{#1/.@args}{##1##2##3##4##5}%
  \or
	  \pgfkeyssetvalue{#1/.@args}{##1##2##3##4##5##6}%
  %%%%% removed:
  %%%%% \or
  %%%%% \pgfkeyssetvalue{#1/.@args}{##1##2##3##4##5##6}%
  \or
	  \pgfkeyssetvalue{#1/.@args}{##1##2##3##4##5##6##7}%
  \or
	  \pgfkeyssetvalue{#1/.@args}{##1##2##3##4##5##6##7##8}%
  \or
	  \pgfkeyssetvalue{#1/.@args}{##1##2##3##4##5##6##7##8##9}%
  \else
	  \pgfkeys@error{\string\pgfkeysdefnargs: expected  <= 9 arguments, got #2}%
  \fi
  \pgfkeysgetvalue{#1/.@args}\pgfkeys@tempargs
  \def\pgfkeys@temp{\expandafter#4\csname pgfk@#1/.@@body\endcsname}%
  \expandafter\pgfkeys@temp\pgfkeys@tempargs{#3}%
  % eliminate the \pgfeov at the end such that TeX gobbles spaces
  % by using
  % \pgfkeysdef{#1}{\pgfkeysvalueof{#1/.@@body}##1}
  % (with expansion of '#1'):
  \edef\pgfkeys@tempargs{\noexpand\pgfkeysvalueof{#1/.@@body}}%
  \def\pgfkeys@temp{\pgfkeysdef{#1}}%
  \expandafter\pgfkeys@temp\expandafter{\pgfkeys@tempargs##1}%
  \pgfkeyssetvalue{#1/.@body}{#3}%
}
\ifx\pgfkeysdefnargs@\forest@original@pgfkeysdefnargs@
  \let\pgfkeysdefnargs@\forest@patched@pgfkeysdefnargs@
\fi
%    \end{macrocode}
%
% Minor: a leaking space in the very first line.
%    \begin{macrocode}
\def\forest@original@pgfpointintersectionoflines#1#2#3#4{%
  {
    % 
    % Compute orthogonal vector to #1--#2
    % 
    \pgf@process{#2}%
    \pgf@xa=\pgf@x%
    \pgf@ya=\pgf@y%
    \pgf@process{#1}%
    \advance\pgf@xa by-\pgf@x%
    \advance\pgf@ya by-\pgf@y%
    \pgf@ya=-\pgf@ya%
    % Normalise a bit
    \c@pgf@counta=\pgf@xa%
    \ifnum\c@pgf@counta<0\relax%
      \c@pgf@counta=-\c@pgf@counta\relax%
    \fi%
    \c@pgf@countb=\pgf@ya%
    \ifnum\c@pgf@countb<0\relax%
      \c@pgf@countb=-\c@pgf@countb\relax%
    \fi%
    \advance\c@pgf@counta by\c@pgf@countb\relax%
    \divide\c@pgf@counta by 65536\relax%
    \ifnum\c@pgf@counta>0\relax%
      \divide\pgf@xa by\c@pgf@counta\relax%
      \divide\pgf@ya by\c@pgf@counta\relax%
    \fi%
    %
    % Compute projection
    %
    \pgf@xc=\pgf@sys@tonumber{\pgf@ya}\pgf@x%
    \advance\pgf@xc by\pgf@sys@tonumber{\pgf@xa}\pgf@y%
    % 
    % The orthogonal vector is (\pgf@ya,\pgf@xa)
    % 
    % 
    % Compute orthogonal vector to #3--#4
    % 
    \pgf@process{#4}%
    \pgf@xb=\pgf@x%
    \pgf@yb=\pgf@y%
    \pgf@process{#3}%
    \advance\pgf@xb by-\pgf@x%
    \advance\pgf@yb by-\pgf@y%
    \pgf@yb=-\pgf@yb%
    % Normalise a bit
    \c@pgf@counta=\pgf@xb%
    \ifnum\c@pgf@counta<0\relax%
      \c@pgf@counta=-\c@pgf@counta\relax%
    \fi%
    \c@pgf@countb=\pgf@yb%
    \ifnum\c@pgf@countb<0\relax%
      \c@pgf@countb=-\c@pgf@countb\relax%
    \fi%
    \advance\c@pgf@counta by\c@pgf@countb\relax%
    \divide\c@pgf@counta by 65536\relax%
    \ifnum\c@pgf@counta>0\relax%
      \divide\pgf@xb by\c@pgf@counta\relax%
      \divide\pgf@yb by\c@pgf@counta\relax%
    \fi%
    %
    % Compute projection
    %
    \pgf@yc=\pgf@sys@tonumber{\pgf@yb}\pgf@x%
    \advance\pgf@yc by\pgf@sys@tonumber{\pgf@xb}\pgf@y%
    % 
    % The orthogonal vector is (\pgf@yb,\pgf@xb)
    % 
    % Setup transformation matrx (this is just to use the matrix
    % inversion)
    % 
    \pgfsettransform{{\pgf@sys@tonumber\pgf@ya}{\pgf@sys@tonumber\pgf@yb}{\pgf@sys@tonumber\pgf@xa}{\pgf@sys@tonumber\pgf@xb}{0pt}{0pt}}%
    \pgftransforminvert%
    \pgf@process{\pgfpointtransformed{\pgfpoint{\pgf@xc}{\pgf@yc}}}%
  }%
}
\def\forest@patched@pgfpointintersectionoflines#1#2#3#4{%
  {% added the percent sign in this line
    % 
    % Compute orthogonal vector to #1--#2
    % 
    \pgf@process{#2}%
    \pgf@xa=\pgf@x%
    \pgf@ya=\pgf@y%
    \pgf@process{#1}%
    \advance\pgf@xa by-\pgf@x%
    \advance\pgf@ya by-\pgf@y%
    \pgf@ya=-\pgf@ya%
    % Normalise a bit
    \c@pgf@counta=\pgf@xa%
    \ifnum\c@pgf@counta<0\relax%
      \c@pgf@counta=-\c@pgf@counta\relax%
    \fi%
    \c@pgf@countb=\pgf@ya%
    \ifnum\c@pgf@countb<0\relax%
      \c@pgf@countb=-\c@pgf@countb\relax%
    \fi%
    \advance\c@pgf@counta by\c@pgf@countb\relax%
    \divide\c@pgf@counta by 65536\relax%
    \ifnum\c@pgf@counta>0\relax%
      \divide\pgf@xa by\c@pgf@counta\relax%
      \divide\pgf@ya by\c@pgf@counta\relax%
    \fi%
    %
    % Compute projection
    %
    \pgf@xc=\pgf@sys@tonumber{\pgf@ya}\pgf@x%
    \advance\pgf@xc by\pgf@sys@tonumber{\pgf@xa}\pgf@y%
    % 
    % The orthogonal vector is (\pgf@ya,\pgf@xa)
    % 
    % 
    % Compute orthogonal vector to #3--#4
    % 
    \pgf@process{#4}%
    \pgf@xb=\pgf@x%
    \pgf@yb=\pgf@y%
    \pgf@process{#3}%
    \advance\pgf@xb by-\pgf@x%
    \advance\pgf@yb by-\pgf@y%
    \pgf@yb=-\pgf@yb%
    % Normalise a bit
    \c@pgf@counta=\pgf@xb%
    \ifnum\c@pgf@counta<0\relax%
      \c@pgf@counta=-\c@pgf@counta\relax%
    \fi%
    \c@pgf@countb=\pgf@yb%
    \ifnum\c@pgf@countb<0\relax%
      \c@pgf@countb=-\c@pgf@countb\relax%
    \fi%
    \advance\c@pgf@counta by\c@pgf@countb\relax%
    \divide\c@pgf@counta by 65536\relax%
    \ifnum\c@pgf@counta>0\relax%
      \divide\pgf@xb by\c@pgf@counta\relax%
      \divide\pgf@yb by\c@pgf@counta\relax%
    \fi%
    %
    % Compute projection
    %
    \pgf@yc=\pgf@sys@tonumber{\pgf@yb}\pgf@x%
    \advance\pgf@yc by\pgf@sys@tonumber{\pgf@xb}\pgf@y%
    % 
    % The orthogonal vector is (\pgf@yb,\pgf@xb)
    % 
    % Setup transformation matrx (this is just to use the matrix
    % inversion)
    % 
    \pgfsettransform{{\pgf@sys@tonumber\pgf@ya}{\pgf@sys@tonumber\pgf@yb}{\pgf@sys@tonumber\pgf@xa}{\pgf@sys@tonumber\pgf@xb}{0pt}{0pt}}%
    \pgftransforminvert%
    \pgf@process{\pgfpointtransformed{\pgfpoint{\pgf@xc}{\pgf@yc}}}%
  }%
}

\ifx\pgfpointintersectionoflines\forest@original@pgfpointintersectionoflines
  \let\pgfpointintersectionoflines\forest@patched@pgfpointintersectionoflines
\fi

 % hah: hacking forest --- it depends on some details of PGF implementation
\def\forest@pgf@notyetpositioned{not yet positionedPGFINTERNAL}%
\expandafter\ifstrequal\expandafter{\pgfversion}{2.10}{%
  \def\forest@pgf@notyetpositioned{not yet positioned@}%
}{}
%    \end{macrocode}
% 
% \section{Utilities}
%
% Escaping |\if|s.
%    \begin{macrocode}
\long\def\@escapeif#1#2\fi{\fi#1}
\long\def\@escapeifif#1#2\fi#3\fi{\fi\fi#1}
%    \end{macrocode}
% 
% A factory for creating |\...loop...| macros.
%    \begin{macrocode}
\def\newloop#1{%
  \count@=\escapechar
  \escapechar=-1
  \expandafter\newloop@parse@loopname\string#1\newloop@end
  \escapechar=\count@
}%
{\lccode`7=`l \lccode`8=`o \lccode`9=`p
  \lowercase{\gdef\newloop@parse@loopname#17889#2\newloop@end{%
      \edef\newloop@marshal{%
        \noexpand\csdef{#1loop#2}####1\expandafter\noexpand\csname #1repeat#2\endcsname{%
          \noexpand\csdef{#1iterate#2}{####1\relax\noexpand\expandafter\expandafter\noexpand\csname#1iterate#2\endcsname\noexpand\fi}%
          \expandafter\noexpand\csname#1iterate#2\endcsname
          \let\expandafter\noexpand\csname#1iterate#2\endcsname\relax
        }%
      }%
      \newloop@marshal
    }%
  }%
}%
%    \end{macrocode}
%    
% Additional loops (for embedding).
%    \begin{macrocode}
\newloop\forest@loop
\newloop\forest@loopa
\newloop\forest@loopb
\newloop\forest@loopc
\newloop\forest@sort@loop
\newloop\forest@sort@loopA
%    \end{macrocode}
% New counters, dimens, ifs.
%    \begin{macrocode}
\newdimen\forest@temp@dimen
\newcount\forest@temp@count
\newcount\forest@n
\newif\ifforest@temp
\newcount\forest@temp@global@count
%    \end{macrocode}
%
% Appending and prepending to token lists.
%    \begin{macrocode}
\def\apptotoks#1#2{\expandafter#1\expandafter{\the#1#2}}
\long\def\lapptotoks#1#2{\expandafter#1\expandafter{\the#1#2}}
\def\eapptotoks#1#2{\edef\pot@temp{#2}\expandafter\expandafter\expandafter#1\expandafter\expandafter\expandafter{\expandafter\the\expandafter#1\pot@temp}}
\def\pretotoks#1#2{\toks@={#2}\expandafter\expandafter\expandafter#1\expandafter\expandafter\expandafter{\expandafter\the\expandafter\toks@\the#1}}
\def\epretotoks#1#2{\edef\pot@temp{#2}\expandafter\expandafter\expandafter#1\expandafter\expandafter\expandafter{\expandafter\pot@temp\the#1}}
\def\gapptotoks#1#2{\expandafter\global\expandafter#1\expandafter{\the#1#2}}
\def\xapptotoks#1#2{\edef\pot@temp{#2}\expandafter\expandafter\expandafter\global\expandafter\expandafter\expandafter#1\expandafter\expandafter\expandafter{\expandafter\the\expandafter#1\pot@temp}}
\def\gpretotoks#1#2{\toks@={#2}\expandafter\expandafter\expandafter\global\expandafter\expandafter\expandafter#1\expandafter\expandafter\expandafter{\expandafter\the\expandafter\toks@\the#1}}
\def\xpretotoks#1#2{\edef\pot@temp{#2}\expandafter\expandafter\expandafter\global\expandafter\expandafter\expandafter#1\expandafter\expandafter\expandafter{\expandafter\pot@temp\the#1}}
%    \end{macrocode}
%    
% Expanding number arguments.
%    \begin{macrocode}
\def\expandnumberarg#1#2{\expandafter#1\expandafter{\number#2}}
\def\expandtwonumberargs#1#2#3{%
  \expandafter\expandtwonumberargs@\expandafter#1\expandafter{\number#3}{#2}}
\def\expandtwonumberargs@#1#2#3{%
  \expandafter#1\expandafter{\number#3}{#2}}
\def\expandthreenumberargs#1#2#3#4{%
  \expandafter\expandthreenumberargs@\expandafter#1\expandafter{\number#4}{#2}{#3}}
\def\expandthreenumberargs@#1#2#3#4{%
  \expandafter\expandthreenumberargs@@\expandafter#1\expandafter{\number#4}{#2}{#3}}
\def\expandthreenumberargs@@#1#2#3#4{%
  \expandafter#1\expandafter{\number#4}{#2}{#3}}
%    \end{macrocode}
%
% A macro converting all non-letters in a string to |__|.  |#1| =
% string, |#2| = receiving macro.  Used for declaring pgfmath
% functions. 
%    \begin{macrocode}
\def\forest@convert@others@to@underscores#1#2{%
  \def\forest@cotu@result{}%
  \forest@cotu#1\forest@end
  \let#2\forest@cotu@result
}
\def\forest@cotu{%
  \futurelet\forest@cotu@nextchar\forest@cotu@checkforspace
}
\def\forest@cotu@checkforspace{%
  \expandafter\ifx\space\forest@cotu@nextchar
    \let\forest@cotu@next\forest@cotu@havespace
  \else
    \let\forest@cotu@next\forest@cotu@nospace
  \fi
  \forest@cotu@next
}
\def\forest@cotu@havespace#1{%
  \appto\forest@cotu@result{_}%
  \forest@cotu#1%
}
\def\forest@cotu@nospace{%
  \ifx\forest@cotu@nextchar\forest@end
    \@escapeif\@gobble
  \else
    \@escapeif\forest@cotu@nospaceB
  \fi
}
\def\forest@cotu@nospaceB{%
  \ifcat\forest@cotu@nextchar a%
    \let\forest@cotu@next\forest@cotu@have@alphanum
  \else
    \ifcat\forest@cotu@nextchar 0%
      \let\forest@cotu@next\forest@cotu@have@alphanum
    \else
      \let\forest@cotu@next\forest@cotu@haveother
    \fi
  \fi
  \forest@cotu@next
}
\def\forest@cotu@have@alphanum#1{%
  \appto\forest@cotu@result{#1}%
  \forest@cotu
}
\def\forest@cotu@haveother#1{%
  \appto\forest@cotu@result{_}%
  \forest@cotu
}
%    \end{macrocode}
%
% Additional list macros.
%    \begin{macrocode}
\def\forest@listedel#1#2{% #1 = list, #2 = item
  \edef\forest@marshal{\noexpand\forest@listdel\noexpand#1{#2}}%
  \forest@marshal
}
\def\forest@listcsdel#1#2{%
  \expandafter\forest@listdel\csname #1\endcsname{#2}%
}
\def\forest@listcsedel#1#2{%
  \expandafter\forest@listedel\csname #1\endcsname{#2}%
}
\edef\forest@restorelistsepcatcode{\noexpand\catcode`|\the\catcode`|\relax}%
\catcode`\|=3
\gdef\forest@listdel#1#2{%
  \def\forest@listedel@A##1|#2|##2\forest@END{%
    \forest@listedel@B##1|##2\forest@END%|
  }%
  \def\forest@listedel@B|##1\forest@END{%|
    \def#1{##1}%
  }%
  \expandafter\forest@listedel@A\expandafter|#1\forest@END%|
}
\forest@restorelistsepcatcode
%    \end{macrocode}
%
% Strip (the first level of) braces from all the tokens in the argument.
%    \begin{macrocode}
\def\forest@strip@braces#1{%
  \forest@strip@braces@A#1\forest@strip@braces@preend\forest@strip@braces@end
}
\def\forest@strip@braces@A#1#2\forest@strip@braces@end{%
  #1\ifx\forest@strip@braces@preend#2\else\@escapeif{\forest@strip@braces@A#2\forest@strip@braces@end}\fi
}
%    \end{macrocode}
%
% \subsection{Sorting}
%
% Macro |\forest@sort| is the user interface to sorting.
%
% The user should prepare the data in an arbitrarily encoded
% array,\footnote{In forest, arrays are encoded as families of
% macros. An array-macro name consists of the (optional, but
% recommended) prefix, the index, and the (optional) suffix (e.g.\
% \texttt{$\backslash$forest@42x}). Prefix establishes the ``namespace'',
% while using more than one suffix simulates an array of named tuples.
% The length of the array is stored in macro \texttt{$\backslash$<prefix>n}.}
% and provide the sorting macro (given in |#1|) and the array let
% macro (given in |#2|): these are the only ways in which sorting
% algorithms access the data.  Both user-given macros should take two
% parameters, which expand to array indices.  The comparison macro
% should compare the given array items and call |\forest@sort@cmp@gt|,
% |\forest@sort@cmp@lt| or |\forest@sort@cmp@eq| to signal that the
% first item is greater than, less than, or equal to the second item.
% The let macro should ``copy'' the contents of the second item onto
% the first item.
% 
% The sorting direction is be given in |#3|: it can one of
% |\forest@sort@ascending| and |\forest@sort@descending|.  |#4| and
% |#5| must expand to the lower and upper (both inclusive) indices of
% the array to be sorted.
%
% |\forest@sort| is just a wrapper for the central sorting macro
% |\forest@@sort|, storing the comparison macro, the array let macro
% and the direction.  The central sorting macro and the
% algorithm-specific macros take only two arguments: the array bounds.
%    \begin{macrocode}
\def\forest@sort#1#2#3#4#5{%
  \let\forest@sort@cmp#1\relax
  \let\forest@sort@let#2\relax
  \let\forest@sort@direction#3\relax
  \forest@@sort{#4}{#5}%
}
%    \end{macrocode}
% The central sorting macro.  Here it is decided which sorting
% algorithm will be used: for arrays at least
% |\forest@quicksort@minarraylength| long, quicksort is used;
% otherwise, insertion sort.
%    \begin{macrocode}
\def\forest@quicksort@minarraylength{10000}
\def\forest@@sort#1#2{%
  \ifnum#1<#2\relax\@escapeif{%
    \forest@sort@m=#2
    \advance\forest@sort@m -#1
    \ifnum\forest@sort@m>\forest@quicksort@minarraylength\relax\@escapeif{%
      \forest@quicksort{#1}{#2}%
    }\else\@escapeif{%
      \forest@insertionsort{#1}{#2}%
    }\fi
  }\fi
}
%    \end{macrocode}
% Various counters and macros needed by the sorting algorithms.
%    \begin{macrocode}
\newcount\forest@sort@m\newcount\forest@sort@k\newcount\forest@sort@p
\def\forest@sort@ascending{>}
\def\forest@sort@descending{<}
\def\forest@sort@cmp{%
  \PackageError{sort}{You must define forest@sort@cmp function before calling
    sort}{The macro must take two arguments, indices of the array
    elements to be compared, and return '=' if the elements are equal
    and '>'/'<' if the first is greater /less than the secong element.}%
}
\def\forest@sort@cmp@gt{\def\forest@sort@cmp@result{>}}
\def\forest@sort@cmp@lt{\def\forest@sort@cmp@result{<}}
\def\forest@sort@cmp@eq{\def\forest@sort@cmp@result{=}}
\def\forest@sort@let{%
  \PackageError{sort}{You must define forest@sort@let function before calling
    sort}{The macro must take two arguments, indices of the array:
    element 2 must be copied onto element 1.}%
}
%    \end{macrocode}
% Quick sort macro (adapted from
% \href{http://www.ctan.org/pkg/laansort}{laansort}).
%    \begin{macrocode}
\def\forest@quicksort#1#2{%
%    \end{macrocode}
% Compute the index of the middle element (|\forest@sort@m|).
%    \begin{macrocode}
  \forest@sort@m=#2
  \advance\forest@sort@m -#1
  \ifodd\forest@sort@m\relax\advance\forest@sort@m1 \fi
  \divide\forest@sort@m 2
  \advance\forest@sort@m #1
%    \end{macrocode}
% The pivot element is the median of the first, the middle and the
% last element.
%    \begin{macrocode}
  \forest@sort@cmp{#1}{#2}%
  \if\forest@sort@cmp@result=%
    \forest@sort@p=#1
  \else
    \if\forest@sort@cmp@result>%
      \forest@sort@p=#1\relax
    \else
      \forest@sort@p=#2\relax
    \fi
    \forest@sort@cmp{\the\forest@sort@p}{\the\forest@sort@m}%
    \if\forest@sort@cmp@result<%
    \else
      \forest@sort@p=\the\forest@sort@m
    \fi    
  \fi
%    \end{macrocode}
% Exchange the pivot and the first element.
%    \begin{macrocode}
  \forest@sort@xch{#1}{\the\forest@sort@p}%
%    \end{macrocode}
% Counter |\forest@sort@m| will hold the final location of the pivot
% element.
%    \begin{macrocode}
  \forest@sort@m=#1\relax
%    \end{macrocode}
% Loop through the list.
%    \begin{macrocode}
  \forest@sort@k=#1\relax
  \forest@sort@loop
  \ifnum\forest@sort@k<#2\relax
    \advance\forest@sort@k 1
%    \end{macrocode}
% Compare the pivot and the current element.
%    \begin{macrocode}
    \forest@sort@cmp{#1}{\the\forest@sort@k}%
%    \end{macrocode}
% If the current element is smaller (ascending) or greater
% (descending) than the pivot element, move it into the first part of
% the list, and adjust the final location of the pivot.
%    \begin{macrocode}
    \ifx\forest@sort@direction\forest@sort@cmp@result
      \advance\forest@sort@m 1
      \forest@sort@xch{\the\forest@sort@m}{\the\forest@sort@k}
    \fi
  \forest@sort@repeat
%    \end{macrocode}
% Move the pivot element into its final position.
%    \begin{macrocode}
  \forest@sort@xch{#1}{\the\forest@sort@m}%
%    \end{macrocode}
% Recursively call sort on the two parts of the list: elements before
% the pivot are smaller (ascending order) / greater (descending order)
% than the pivot; elements after the pivot are greater (ascending
% order) / smaller (descending order) than the pivot.
%    \begin{macrocode}
  \forest@sort@k=\forest@sort@m
  \advance\forest@sort@k -1
  \advance\forest@sort@m 1
  \edef\forest@sort@marshal{%
    \noexpand\forest@@sort{#1}{\the\forest@sort@k}%
    \noexpand\forest@@sort{\the\forest@sort@m}{#2}%
  }%
  \forest@sort@marshal
}
% We defines the item-exchange macro in terms of the (user-provided)
% array let macro.
%    \begin{macrocode}
\def\forest@sort@xch#1#2{%
  \forest@sort@let{aux}{#1}%
  \forest@sort@let{#1}{#2}%
  \forest@sort@let{#2}{aux}%
}
%    \end{macrocode}
% Insertion sort.
%    \begin{macrocode}
\def\forest@insertionsort#1#2{%
  \forest@sort@m=#1
  \edef\forest@insertionsort@low{#1}%
  \forest@sort@loopA
  \ifnum\forest@sort@m<#2
    \advance\forest@sort@m 1
    \forest@insertionsort@Qbody
  \forest@sort@repeatA
}
\newif\ifforest@insertionsort@loop
\def\forest@insertionsort@Qbody{%
  \forest@sort@let{aux}{\the\forest@sort@m}%  
  \forest@sort@k\forest@sort@m
  \advance\forest@sort@k -1
  \forest@insertionsort@looptrue
  \forest@sort@loop
  \ifforest@insertionsort@loop
    \forest@insertionsort@qbody
  \forest@sort@repeat
  \advance\forest@sort@k 1
  \forest@sort@let{\the\forest@sort@k}{aux}%
}
\def\forest@insertionsort@qbody{%
  \forest@sort@cmp{\the\forest@sort@k}{aux}%
  \ifx\forest@sort@direction\forest@sort@cmp@result\relax
    \forest@sort@p=\forest@sort@k
    \advance\forest@sort@p 1
    \forest@sort@let{\the\forest@sort@p}{\the\forest@sort@k}%
    \advance\forest@sort@k -1
    \ifnum\forest@sort@k<\forest@insertionsort@low\relax
      \forest@insertionsort@loopfalse
    \fi
  \else
    \forest@insertionsort@loopfalse
  \fi
}
%    \end{macrocode}
%
% Below, several helpers for writing comparison macros are
% provided. They take take two (pairs of) control sequence names and
% compare their contents.
% 
% Compare numbers.
%    \begin{macrocode}
\def\forest@sort@cmpnumcs#1#2{%
  \ifnum\csname#1\endcsname>\csname#2\endcsname\relax
   \forest@sort@cmp@gt
 \else
   \ifnum\csname#1\endcsname<\csname#2\endcsname\relax
     \forest@sort@cmp@lt
   \else
     \forest@sort@cmp@eq
   \fi
 \fi
}
%    \end{macrocode}
% Compare dimensions.
%    \begin{macrocode}
\def\forest@sort@cmpdimcs#1#2{%
  \ifdim\csname#1\endcsname>\csname#2\endcsname\relax
   \forest@sort@cmp@gt
 \else
   \ifdim\csname#1\endcsname<\csname#2\endcsname\relax
     \forest@sort@cmp@lt
   \else
     \forest@sort@cmp@eq
   \fi
 \fi
}
%    \end{macrocode}
% Compare points (pairs of dimension) |(#1,#2)| and |(#3,#4)|.
%    \begin{macrocode}
\def\forest@sort@cmptwodimcs#1#2#3#4{%
  \ifdim\csname#1\endcsname>\csname#3\endcsname\relax
   \forest@sort@cmp@gt
 \else
   \ifdim\csname#1\endcsname<\csname#3\endcsname\relax
     \forest@sort@cmp@lt
   \else
     \ifdim\csname#2\endcsname>\csname#4\endcsname\relax
       \forest@sort@cmp@gt
     \else
       \ifdim\csname#2\endcsname<\csname#4\endcsname\relax
         \forest@sort@cmp@lt
       \else
         \forest@sort@cmp@eq
       \fi
     \fi
   \fi
 \fi  
}
%    \end{macrocode}
%    
% The following macro reverses an array. The arguments: |#1| is
% the array let macro; |#2| is the start index (inclusive), and
% |#3| is the end index (exclusive).
%    \begin{macrocode}
\def\forest@reversearray#1#2#3{%
  \let\forest@sort@let#1%
  \c@pgf@countc=#2
  \c@pgf@countd=#3
  \advance\c@pgf@countd -1
  \forest@loopa
  \ifnum\c@pgf@countc<\c@pgf@countd\relax
    \forest@sort@xch{\the\c@pgf@countc}{\the\c@pgf@countd}%
    \advance\c@pgf@countc 1
    \advance\c@pgf@countd -1
  \forest@repeata
}
%    \end{macrocode}
%
% \section{The bracket representation parser}
% \label{imp:bracket}
%
% \subsection{The user interface macros}
%
% Settings.
%    \begin{macrocode}
\def\bracketset#1{\pgfqkeys{/bracket}{#1}}%
\bracketset{%
  /bracket/.is family,
  /handlers/.let/.style={\pgfkeyscurrentpath/.code={\let#1##1}},
  opening bracket/.let=\bracket@openingBracket,
  closing bracket/.let=\bracket@closingBracket,
  action character/.let=\bracket@actionCharacter,
  opening bracket=[,
  closing bracket=],
  action character,
  new node/.code n args={3}{% #1=preamble, #2=node spec, #3=cs receiving the id
    \forest@node@new#3%
    \forestOset{#3}{given options}{content'=#2}%
    \ifblank{#1}{}{%
      \forestOpreto{#3}{given options}{#1,}%
    }%
  },
  set afterthought/.code 2 args={% #1=node id, #2=afterthought
    \ifblank{#2}{}{\forestOappto{#1}{given options}{,afterthought={#2}}}%
  }
}
%    \end{macrocode}
%
% |\bracketParse| is the macro that should be called to parse a
% balanced bracket representation.  It takes five parameters: |#1| is the code that will be run
% after parsing the bracket; |#2| is a control sequence that will receive the id of the root of the
% created tree structure. (The bracket representation should follow (after optional spaces), but is
% is not a formal parameter of the macro.)
%    \begin{macrocode}
\newtoks\bracket@content
\newtoks\bracket@afterthought
\def\bracketParse#1#2={%
  \def\bracketEndParsingHook{#1}%
  \def\bracket@saveRootNodeTo{#2}%
%    \end{macrocode}
% Content and afterthought will be appended to these macros.  (The |\bracket@afterthought| toks register is
% abused for storing the preamble as well --- that's ok, the preamble comes before any afterhoughts.)
%    \begin{macrocode}
  \bracket@content={}%
  \bracket@afterthought={}%
%    \end{macrocode}
% The parser can be in three states: in content (0), in afterthought
% (1), or starting (2).  While in the content/afterthought state, the
% parser appends all non-control tokens to the content/afterthought macro.
%    \begin{macrocode}
  \let\bracket@state\bracket@state@starting
  \bracket@ignorespacestrue
%    \end{macrocode}
% By default, don't expand anything.
%    \begin{macrocode}
  \bracket@expandtokensfalse
%    \end{macrocode}
% We initialize several control sequences that are used to store some
% nodes while parsing.
%    \begin{macrocode}
  \def\bracket@parentNode{0}%
  \def\bracket@rootNode{0}%
  \def\bracket@newNode{0}%
  \def\bracket@afterthoughtNode{0}%
%    \end{macrocode}
% Finally, we start the parser.
%    \begin{macrocode}
  \bracket@Parse
}
%    \end{macrocode}
% The other macro that an end user (actually a power user) can use, is
% actually just a synonym for |\bracket@Parse|. It should be used to
% resume parsing when the action code has finished its work.
%    \begin{macrocode}    
\def\bracketResume{\bracket@Parse}%
%    \end{macrocode}
%    
% \subsection{Parsing}
% 
% We first check if the next token is a space. Spaces need special
% treatment because they are eaten by both the |\romannumeral| trick
% and \TeX s (undelimited) argument parsing algorithm. If a space is
% found, remember that, eat it up, and restart the parsing.
%    \begin{macrocode}
\def\bracket@Parse{%
  \futurelet\bracket@next@token\bracket@Parse@checkForSpace
}
\def\bracket@Parse@checkForSpace{%
  \expandafter\ifx\space\bracket@next@token\@escapeif{%
    \ifbracket@ignorespaces\else
      \bracket@haveSpacetrue
    \fi
    \expandafter\bracket@Parse\romannumeral-`0%
  }\else\@escapeif{%
    \bracket@Parse@maybeexpand
  }\fi
}
%    \end{macrocode}
% 
% We either fully expand the next token (using a popular \TeX nical
% trick \dots) or don't expand it at all, depending on the state of
% |\ifbracket@expandtokens|.
%    \begin{macrocode}
\newif\ifbracket@expandtokens
\def\bracket@Parse@maybeexpand{%
  \ifbracket@expandtokens\@escapeif{%
    \expandafter\bracket@Parse@peekAhead\romannumeral-`0%
  }\else\@escapeif{%
    \bracket@Parse@peekAhead
  }\fi
}
%    \end{macrocode}
% We then look ahead to see what's coming.
%    \begin{macrocode}
\def\bracket@Parse@peekAhead{%
  \futurelet\bracket@next@token\bracket@Parse@checkForTeXGroup
}
%    \end{macrocode}
% If the next token is a begin-group token, we append the whole group to
% the content or afterthought macro, depending on the state.
%    \begin{macrocode}
\def\bracket@Parse@checkForTeXGroup{%
  \ifx\bracket@next@token\bgroup%
    \@escapeif{\bracket@Parse@appendGroup}%
  \else
    \@escapeif{\bracket@Parse@token}%
  \fi
}
%    \end{macrocode}
% This is easy: if a control token is found, run the appropriate
% macro; otherwise, append the token to the content or afterthought
% macro, depending on the state.
%    \begin{macrocode}
\long\def\bracket@Parse@token#1{%
  \ifx#1\bracket@openingBracket
    \@escapeif{\bracket@Parse@openingBracketFound}%
  \else
    \@escapeif{%
      \ifx#1\bracket@closingBracket
        \@escapeif{\bracket@Parse@closingBracketFound}%
      \else
        \@escapeif{%
          \ifx#1\bracket@actionCharacter
            \@escapeif{\futurelet\bracket@next@token\bracket@Parse@actionCharacterFound}%
          \else
            \@escapeif{\bracket@Parse@appendToken#1}%
          \fi
        }%
      \fi
    }%
  \fi
}
%    \end{macrocode}
% Append the token or group to the content or afterthought macro. If a
% space was found previously, append it as well.
%    \begin{macrocode}
\newif\ifbracket@haveSpace
\newif\ifbracket@ignorespaces
\def\bracket@Parse@appendSpace{%
  \ifbracket@haveSpace
    \ifcase\bracket@state\relax
      \eapptotoks\bracket@content\space
    \or
      \eapptotoks\bracket@afterthought\space
    \or
      \eapptotoks\bracket@afterthought\space
    \fi    
    \bracket@haveSpacefalse
  \fi
}
\long\def\bracket@Parse@appendToken#1{%
  \bracket@Parse@appendSpace
  \ifcase\bracket@state\relax
    \lapptotoks\bracket@content{#1}%
  \or
    \lapptotoks\bracket@afterthought{#1}%
  \or
    \lapptotoks\bracket@afterthought{#1}%
  \fi
  \bracket@ignorespacesfalse
  \bracket@Parse
}
\def\bracket@Parse@appendGroup#1{%
  \bracket@Parse@appendSpace
  \ifcase\bracket@state\relax
    \apptotoks\bracket@content{{#1}}%
  \or
    \apptotoks\bracket@afterthought{{#1}}%
  \or
    \apptotoks\bracket@afterthought{{#1}}%
  \fi
  \bracket@ignorespacesfalse
  \bracket@Parse
}
%    \end{macrocode}
% Declare states.
%    \begin{macrocode}
\def\bracket@state@inContent{0}
\def\bracket@state@inAfterthought{1}
\def\bracket@state@starting{2}
%    \end{macrocode}
%    
% Welcome to the jungle.  In the following two macros, new nodes are
% created, content and afterthought are sent to them, parents and
% states are changed\dots\@  Altogether, we distinguish six cases, as
% shown below: in the schemas, we have just crossed the symbol after
% the dots. (In all cases, we reset the |\if| for spaces.)
%    \begin{macrocode}
\def\bracket@Parse@openingBracketFound{%
  \bracket@haveSpacefalse
  \ifcase\bracket@state\relax% in content [ ... [
%    \end{macrocode}
% |[...[|: we have just finished gathering the content and are about
% to begin gathering the content of another node. We create a
% new node (and put the content (\dots) into 
% it). Then, if there is a parent node, we append the new node to the
% list of its children. Next, since we have just crossed an opening
% bracket, we declare the newly created node to be the parent of the
% coming node.  The state does not change. Finally, we continue parsing.
%    \begin{macrocode}
    \@escapeif{%
      \bracket@createNode
      \ifnum\bracket@parentNode=0 \else
        \forest@node@Append{\bracket@parentNode}{\bracket@newNode}%
      \fi
      \let\bracket@parentNode\bracket@newNode
      \bracket@Parse
    }%
  \or % in afterthought   ] ... [
%    \end{macrocode}
% |]...[|: we have just finished gathering the afterthought and are
% about to begin gathering the content of another node. We add the
% afterthought (\dots) to the ``afterthought node'' and change into the
% content state. The parent does not change. Finally, we continue
% parsing.
%    \begin{macrocode}
    \@escapeif{%  
      \bracket@addAfterthought
      \let\bracket@state\bracket@state@inContent
      \bracket@Parse
    }%
  \else % starting
%    \end{macrocode}
% |{start}...[|: we have just started. Nothing to do yet (we couldn't
% have collected any content yet), just get into the content state and
% continue parsing.
%    \begin{macrocode}
    \@escapeif{%
      \let\bracket@state\bracket@state@inContent
      \bracket@Parse
    }%
  \fi 
}
\def\bracket@Parse@closingBracketFound{%
  \bracket@haveSpacefalse
  \ifcase\bracket@state\relax % in content [ ... ]
%    \end{macrocode}
% |[...]|: we have just finished gathering the content of a node and
% are about to begin gathering its afterthought. We create a new node
% (and put the content (\dots) into it). If there is no parent node,
% we're done with parsing. Otherwise, we set the newly created
% node to be the ``afterthought node'', i.e.\ the node that will
% receive the next afterthought, change into the afterthought mode,
% and continue parsing.
%    \begin{macrocode}
    \@escapeif{%
      \bracket@createNode
      \ifnum\bracket@parentNode=0
        \@escapeif\bracketEndParsingHook
      \else
        \@escapeif{%
          \let\bracket@afterthoughtNode\bracket@newNode
          \let\bracket@state\bracket@state@inAfterthought
          \forest@node@Append{\bracket@parentNode}{\bracket@newNode}%
          \bracket@Parse
        }%
      \fi
    }%
  \or % in afterthought ] ... ]
%    \end{macrocode}
% |]...]|: we have finished gathering an afterthought of some node and
% will begin gathering the afterthought of its parent. We first add
% the afterthought to the afterthought node and set the current parent
% to be the next afterthought node.  We change the parent to the
% current parent's parent and check if that node is null.  If it is,
% we're done with parsing (ignore the trailing spaces), otherwise we continue.
%    \begin{macrocode}
    \@escapeif{%
      \bracket@addAfterthought
      \let\bracket@afterthoughtNode\bracket@parentNode
      \edef\bracket@parentNode{\forestOve{\bracket@parentNode}{@parent}}%
      \ifnum\bracket@parentNode=0
        \expandafter\bracketEndParsingHook
      \else
        \expandafter\bracket@Parse
      \fi
    }%
  \else % starting
%    \end{macrocode}
% |{start}...]|: something's obviously wrong with the input here\dots
%    \begin{macrocode}
    \PackageError{forest}{You're attempting to start a bracket representation
      with a closing bracket}{}%
  \fi
}
%    \end{macrocode}
%
% The action character code.  What happens is determined by the next token.
%    \begin{macrocode}
\def\bracket@Parse@actionCharacterFound{%
%    \end{macrocode}
% If a braced expression follows, its contents will be fully expanded.
%    \begin{macrocode}
  \ifx\bracket@next@token\bgroup\@escapeif{%
    \bracket@Parse@action@expandgroup
  }\else\@escapeif{%
    \bracket@Parse@action@notagroup
  }\fi
}
\def\bracket@Parse@action@expandgroup#1{%
  \edef\bracket@Parse@action@expandgroup@macro{#1}%
  \expandafter\bracket@Parse\bracket@Parse@action@expandgroup@macro
}
\let\bracket@action@fullyexpandCharacter+
\let\bracket@action@dontexpandCharacter-
\let\bracket@action@executeCharacter!
\def\bracket@Parse@action@notagroup#1{%
%    \end{macrocode}
% If + follows, tokens will be fully expanded from this point on.
%    \begin{macrocode}
  \ifx#1\bracket@action@fullyexpandCharacter\@escapeif{%
    \bracket@expandtokenstrue\bracket@Parse
  }\else\@escapeif{%
%    \end{macrocode}
% If - follows, tokens will not be expanded from this point on. (This is the default behaviour.)
%    \begin{macrocode}
    \ifx#1\bracket@action@dontexpandCharacter\@escapeif{%
      \bracket@expandtokensfalse\bracket@Parse
    }\else\@escapeif{%
%    \end{macrocode}
% Inhibit expansion of the next token.
%    \begin{macrocode}
      \ifx#10\@escapeif{%
        \bracket@Parse@appendToken
      }\else\@escapeif{%
%    \end{macrocode}
% If another action characted follows, we yield the control.  The user is
% expected to resume the parser manually, using |\bracketResume|.
%    \begin{macrocode}
        \ifx#1\bracket@actionCharacter
        \else\@escapeif{%
%    \end{macrocode}
% Anything else will be expanded once.
%    \begin{macrocode}
          \expandafter\bracket@Parse#1%
        }\fi
      }\fi
    }\fi
  }\fi
}
%    \end{macrocode}
%
% \subsection{The tree-structure interface}
%
% This macro creates a new node and sets its content (and preamble, if it's a root node).  Bracket
% user must define a 3-arg key |/bracket/new node=|\meta{preamble}\meta{node
% specification}\meta{node cs}.  User's key must define \meta{node cs} to be a macro holding the
% node's id. 
%    \begin{macrocode}
\def\bracket@createNode{%
  \ifnum\bracket@rootNode=0
    % root node
    \bracketset{new node/.expanded=%
      {\the\bracket@afterthought}%
      {\the\bracket@content}%
      \noexpand\bracket@newNode
    }%
    \bracket@afterthought={}%
    \let\bracket@rootNode\bracket@newNode
    \expandafter\let\bracket@saveRootNodeTo\bracket@newNode
  \else
    % other nodes
    \bracketset{new node/.expanded=%
      {}%
      {\the\bracket@content}%
      \noexpand\bracket@newNode
    }%
  \fi
  \bracket@content={}%
}
%    \end{macrocode}
%
% This macro sets the afterthought.  Bracket user must define a 2-arg key
% |/bracket/set_afterthought=|\meta{node id}\meta{afterthought}.
%    \begin{macrocode}
\def\bracket@addAfterthought{%
  \bracketset{%
    set afterthought/.expanded={\bracket@afterthoughtNode}{\the\bracket@afterthought}%
  }%
  \bracket@afterthought={}%
}
%    \end{macrocode}
%
%
% \section{Nodes} 
%
% Nodes have numeric ids. The node option values of node $n$ are saved in the |\pgfkeys| tree in
% path |/forest/@node/|$n$.
%    
% \subsection{Option setting and retrieval}
%
% Macros for retrieving/setting node options of the current node.
%    \begin{macrocode}
 % full expansion expands precisely to the value
\def\forestov#1{\expandafter\expandafter\expandafter\expandonce
  \pgfkeysvalueof{/forest/@node/\forest@cn/#1}}
 % full expansion expands all the way
\def\forestove#1{\pgfkeysvalueof{/forest/@node/\forest@cn/#1}}
 % full expansion expands to the cs holding the value
\def\forestom#1{\expandafter\expandonce\expandafter{\pgfkeysvalueof{/forest/@node/\forest@cn/#1}}}\def\forestoget#1#2{\pgfkeysgetvalue{/forest/@node/\forest@cn/#1}{#2}}
\def\forestoget#1#2{\pgfkeysgetvalue{/forest/@node/\forest@cn/#1}{#2}}
\def\forestolet#1#2{\pgfkeyslet{/forest/@node/\forest@cn/#1}{#2}}
\def\forestoset#1#2{\pgfkeyssetvalue{/forest/@node/\forest@cn/#1}{#2}}
\def\forestoeset#1#2{%
  \edef\forest@option@temp{%
    \noexpand\pgfkeyssetvalue{/forest/@node/\forest@cn/#1}{#2}%
  }\forest@option@temp
}
\def\forestoappto#1#2{%
  \forestoeset{#1}{\forestov{#1}\unexpanded{#2}}%
}
\def\forestoifdefined#1#2#3{%
  \pgfkeysifdefined{/forest/@node/\forest@cn/#1}{#2}{#3}%
}
%    \end{macrocode}
% User macros for retrieving node options of the current node.
%    \begin{macrocode}
\let\forestoption\forestov
\let\foresteoption\forestove
%    \end{macrocode}
% Macros for retrieving node options of a node given by its id.
%    \begin{macrocode}
\def\forestOv#1#2{\expandafter\expandafter\expandafter\expandonce
  \pgfkeysvalueof{/forest/@node/#1/#2}}
\def\forestOve#1#2{\pgfkeysvalueof{/forest/@node/#1/#2}}
 % full expansion expands to the cs holding the value
\def\forestOm#1#2{\expandafter\expandonce\expandafter{\pgfkeysvalueof{/forest/@node/#1/#2}}}
\def\forestOget#1#2#3{\pgfkeysgetvalue{/forest/@node/#1/#2}{#3}}
\def\forestOget#1#2#3{\pgfkeysgetvalue{/forest/@node/#1/#2}{#3}}
\def\forestOlet#1#2#3{\pgfkeyslet{/forest/@node/#1/#2}{#3}}
\def\forestOset#1#2#3{\pgfkeyssetvalue{/forest/@node/#1/#2}{#3}}
\def\forestOeset#1#2#3{%
  \edef\forestoption@temp{%
    \noexpand\pgfkeyssetvalue{/forest/@node/#1/#2}{#3}%
  }\forestoption@temp
}
\def\forestOappto#1#2#3{%
  \forestOeset{#1}{#2}{\forestOv{#1}{#2}\unexpanded{#3}}%
}
\def\forestOeappto#1#2#3{%
  \forestOeset{#1}{#2}{\forestOv{#1}{#2}#3}%
}
\def\forestOpreto#1#2#3{%
  \forestOeset{#1}{#2}{\unexpanded{#3}\forestOv{#1}{#2}}%
}
\def\forestOepreto#1#2#3{%
  \forestOeset{#1}{#2}{#3\forestOv{#1}{#2}}%
}
\def\forestOifdefined#1#2#3#4{%
  \pgfkeysifdefined{/forest/@node/#1/#2}{#3}{#4}%
}
\def\forestOletO#1#2#3#4{% option #2 of node #1 <-- option #4 of node #3
  \forestOget{#3}{#4}\forestoption@temp
  \forestOlet{#1}{#2}\forestoption@temp}
\def\forestOleto#1#2#3{%
  \forestoget{#3}\forestoption@temp
  \forestOlet{#1}{#2}\forestoption@temp}
\def\forestoletO#1#2#3{%
  \forestOget{#2}{#3}\forestoption@temp
  \forestolet{#1}\forestoption@temp}
\def\forestoleto#1#2{%
  \forestoget{#2}\forestoption@temp
  \forestolet{#1}\forestoption@temp}
%    \end{macrocode}
% Node initialization.  Node option declarations append to |\forest@node@init|.
%    \begin{macrocode}
\def\forest@node@init{%
  \forestoset{@parent}{0}%
  \forestoset{@previous}{0}% previous sibling
  \forestoset{@next}{0}%     next sibling
  \forestoset{@first}{0}%  primary child
  \forestoset{@last}{0}%   last child
}
\def\forestoinit#1{%
  \pgfkeysgetvalue{/forest/#1}\forestoinit@temp
  \forestolet{#1}\forestoinit@temp
}
\newcount\forest@node@maxid
\def\forest@node@new#1{% #1 = cs receiving the new node id
  \advance\forest@node@maxid1
  \forest@fornode{\the\forest@node@maxid}{%
    \forest@node@init
    \forest@node@setname{node@\forest@cn}%
    \forest@initializefromstandardnode
    \edef#1{\forest@cn}%
  }%
}
\let\forestoinit@orig\forestoinit
\def\forest@node@copy#1#2{% #1=from node id, cs receiving the new node id
  \advance\forest@node@maxid1
  \def\forestoinit##1{\forestoletO{##1}{#1}{##1}}%
  \forest@fornode{\the\forest@node@maxid}{%
    \forest@node@init
    \forest@node@setname{\forest@copy@name@template{\forestOve{#1}{name}}}%
    \edef#2{\forest@cn}%
  }%  
  \let\forestoinit\forestoinit@orig
}
\forestset{
  copy name template/.code={\def\forest@copy@name@template##1{#1}},
  copy name template/.default={node@\the\forest@node@maxid},
  copy name template
}
\def\forest@tree@copy#1#2{% #1=from node id, #2=cs receiving the new node id
  \forest@node@copy{#1}\forest@node@copy@temp@id
  \forest@fornode{\forest@node@copy@temp@id}{%
    \expandafter\forest@tree@copy@\expandafter{\forest@node@copy@temp@id}{#1}%
    \edef#2{\forest@cn}%
  }%
}
\def\forest@tree@copy@#1#2{%
  \forest@node@Foreachchild{#2}{%
    \expandafter\forest@tree@copy\expandafter{\forest@cn}\forest@node@copy@temp@childid
    \forest@node@Append{#1}{\forest@node@copy@temp@childid}%
  }%
}
%    \end{macrocode}
% Macro |\forest@cn| holds the current node id (a number). Node 0 is a special ``null'' node which
% is used to signal the absence of a node.
%    \begin{macrocode}
\def\forest@cn{0}
\forest@node@init
%    \end{macrocode}
%
% \subsection{Tree structure}
% Node insertion/removal.
% 
% For the lowercase variants, |\forest@cn| is the parent/removed node.  For the uppercase variants,
% |#1| is the parent/removed node.  For efficiency, the public macros all expand the arguments
% before calling the internal macros.
%    \begin{macrocode}
\def\forest@node@append#1{\expandtwonumberargs\forest@node@Append{\forest@cn}{#1}}
\def\forest@node@prepend#1{\expandtwonumberargs\forest@node@Insertafter{\forest@cn}{#1}{0}}
\def\forest@node@insertafter#1#2{%
  \expandthreenumberargs\forest@node@Insertafter{\forest@cn}{#1}{#2}}
\def\forest@node@insertbefore#1#2{%
  \expandthreenumberargs\forest@node@Insertafter{\forest@cn}{#1}{\forestOve{#2}{@previous}}%
}
\def\forest@node@remove{\expandnumberarg\forest@node@Remove{\forest@cn}}
\def\forest@node@Append#1#2{\expandtwonumberargs\forest@node@Append@{#1}{#2}}
\def\forest@node@Prepend#1#2{\expandtwonumberargs\forest@node@Insertafter{#1}{#2}{0}}
\def\forest@node@Insertafter#1#2#3{% #2 is inserted after #3
  \expandthreenumberargs\forest@node@Insertafter@{#1}{#2}{#3}%
}
\def\forest@node@Insertbefore#1#2#3{% #2 is inserted before #3
  \expandthreenumberargs\forest@node@Insertafter{#1}{#2}{\forestOve{#3}{@previous}}%
}
\def\forest@node@Remove#1{\expandnumberarg\forest@node@Remove@{#1}}
\def\forest@node@Insertafter@#1#2#3{%
  \ifnum\forestOve{#2}{@parent}=0
  \else
    \PackageError{forest}{Insertafter(#1,#2,#3):
      node #2 already has a parent (\forestOve{#2}{@parent})}{}%
  \fi
  \ifnum#3=0
  \else
    \ifnum#1=\forestOve{#3}{@parent}
    \else
      \PackageError{forest}{Insertafter(#1,#2,#3): node #1 is not the parent of the
            intended sibling #3 (with parent \forestOve{#3}{@parent})}{}%
    \fi
  \fi
  \forestOeset{#2}{@parent}{#1}%
  \forestOeset{#2}{@previous}{#3}%
  \ifnum#3=0
    \forestOget{#1}{@first}\forest@node@temp
    \forestOeset{#1}{@first}{#2}%
  \else
    \forestOget{#3}{@next}\forest@node@temp
    \forestOeset{#3}{@next}{#2}%
  \fi
  \forestOeset{#2}{@next}{\forest@node@temp}%
  \ifnum\forest@node@temp=0
    \forestOeset{#1}{@last}{#2}%
  \else
    \forestOeset{\forest@node@temp}{@previous}{#2}%
  \fi
}
\def\forest@node@Append@#1#2{%
  \ifnum\forestOve{#2}{@parent}=0
  \else
    \PackageError{forest}{Append(#1,#2):
      node #2 already has a parent (\forestOve{#2}{@parent})}{}%
  \fi
  \forestOeset{#2}{@parent}{#1}%
  \forestOget{#1}{@last}\forest@node@temp
  \forestOeset{#1}{@last}{#2}%
  \forestOeset{#2}{@previous}{\forest@node@temp}%
  \ifnum\forest@node@temp=0
    \forestOeset{#1}{@first}{#2}%
  \else
    \forestOeset{\forest@node@temp}{@next}{#2}%
  \fi
}
\def\forest@node@Remove@#1{%
  \forestOget{#1}{@parent}\forest@node@temp@parent
  \ifnum\forest@node@temp@parent=0
  \else
    \forestOget{#1}{@previous}\forest@node@temp@previous
    \forestOget{#1}{@next}\forest@node@temp@next
    \ifnum\forest@node@temp@previous=0
      \forestOeset{\forest@node@temp@parent}{@first}{\forest@node@temp@next}%
    \else
      \forestOeset{\forest@node@temp@previous}{@next}{\forest@node@temp@next}%
    \fi
    \ifnum\forest@node@temp@next=0
      \forestOeset{\forest@node@temp@parent}{@last}{\forest@node@temp@previous}%
    \else
      \forestOeset{\forest@node@temp@next}{@previous}{\forest@node@temp@previous}%
    \fi
    \forestOset{#1}{@parent}{0}%
    \forestOset{#1}{@previous}{0}%
    \forestOset{#1}{@next}{0}%
  \fi
}
%    \end{macrocode}
% Looping methods.
%    \begin{macrocode}
\def\forest@forthis#1{%
  \edef\forest@node@marshal{\unexpanded{#1}\def\noexpand\forest@cn}%
  \expandafter\forest@node@marshal\expandafter{\forest@cn}%
}
\def\forest@fornode#1#2{%
  \edef\forest@node@marshal{\edef\noexpand\forest@cn{#1}\unexpanded{#2}\def\noexpand\forest@cn}%
  \expandafter\forest@node@marshal\expandafter{\forest@cn}%
}
\def\forest@fornode@ifexists#1#2{%
  \edef\forest@node@temp{#1}%
  \ifnum\forest@node@temp=0
  \else
    \@escapeif{\expandnumberarg\forest@fornode{\forest@node@temp}{#2}}%
  \fi
}
\def\forest@node@foreachchild#1{\forest@node@Foreachchild{\forest@cn}{#1}}
\def\forest@node@Foreachchild#1#2{%
  \forest@fornode{\forestOve{#1}{@first}}{\forest@node@@forselfandfollowingsiblings{#2}}%
}
\def\forest@node@@forselfandfollowingsiblings#1{%
  \ifnum\forest@cn=0
  \else
    \forest@forthis{#1}%
    \@escapeif{%
      \edef\forest@cn{\forestove{@next}}%
      \forest@node@@forselfandfollowingsiblings{#1}%
    }%
  \fi
}
\def\forest@node@foreach#1{\forest@node@Foreach{\forest@cn}{#1}}
\def\forest@node@Foreach#1#2{%
  \forest@fornode{#1}{\forest@node@@foreach{#2}}%
}
\def\forest@node@@foreach#1{%
  \forest@forthis{#1}%
  \ifnum\forestove{@first}=0
  \else\@escapeif{%
      \edef\forest@cn{\forestove{@first}}%
      \forest@node@@forselfandfollowingsiblings{\forest@node@@foreach{#1}}%
    }%
  \fi
}
\def\forest@node@foreachdescendant#1{\forest@node@Foreachdescendant{\forest@cn}{#1}}
\def\forest@node@Foreachdescendant#1#2{%
  \forest@node@Foreachchild{#1}{%
    \forest@node@foreach{#2}%
  }%
}
%    \end{macrocode}
% 
% Compute |n|, |n'|, |n children| and |level|.
%    \begin{macrocode}
\def\forest@node@Compute@numeric@ts@info@#1{%
  \forest@node@Foreach{#1}{\forest@node@@compute@numeric@ts@info}%
  \ifnum\forestOve{#1}{@parent}=0
  \else
    \fornode{#1}{\forest@node@@compute@numeric@ts@info@nbar}%
  \fi
  \forest@node@Foreachdescendant{#1}{\forest@node@@compute@numeric@ts@info@nbar}%
}
\def\forest@node@@compute@numeric@ts@info{%
  \forestoset{n children}{0}%
  %
  \edef\forest@node@temp{\forestove{@previous}}%
  \ifnum\forest@node@temp=0
    \forestoset{n}{1}%
  \else
    \forestoeset{n}{\number\numexpr\forestOve{\forest@node@temp}{n}+1}%
  \fi
  %
  \edef\forest@node@temp{\forestove{@parent}}%
  \ifnum\forest@node@temp=0
    \forestoset{n}{0}%
    \forestoset{n'}{0}%
    \forestoset{level}{0}%
  \else
    \forestOeset{\forest@node@temp}{n children}{%
      \number\numexpr\forestOve{\forest@node@temp}{n children}+1%
    }%
    \forestoeset{level}{%
      \number\numexpr\forestOve{\forest@node@temp}{level}+1%
    }%
  \fi
}
\def\forest@node@@compute@numeric@ts@info@nbar{%
  \forestoeset{n'}{\number\numexpr\forestOve{\forestove{@parent}}{n children}-\forestove{n}+1}%
}
\def\forest@node@compute@numeric@ts@info#1{%
  \expandnumberarg\forest@node@Compute@numeric@ts@info@{\forest@cn}%
}
\def\forest@node@Compute@numeric@ts@info#1{%
  \expandnumberarg\forest@node@Compute@numeric@ts@info@{#1}%
}
%    \end{macrocode}
%    
% Tree structure queries.
%    \begin{macrocode}
\def\forest@node@rootid{%
  \expandnumberarg\forest@node@Rootid{\forest@cn}%
}
\def\forest@node@Rootid#1{% #1=node
  \ifnum\forestOve{#1}{@parent}=0
    #1%
  \else
    \@escapeif{\expandnumberarg\forest@node@Rootid{\forestOve{#1}{@parent}}}%
  \fi
}
\def\forest@node@nthchildid#1{% #1=n
  \ifnum#1<1
    0%
  \else
    \expandnumberarg\forest@node@nthchildid@{\number\forestove{@first}}{#1}%
  \fi
}
\def\forest@node@nthchildid@#1#2{%
  \ifnum#1=0
    0%
  \else
    \ifnum#2>1
      \@escapeifif{\expandtwonumberargs
        \forest@node@nthchildid@{\forestOve{#1}{@next}}{\numexpr#2-1}}%
    \else
      #1%
    \fi
  \fi
}
\def\forest@node@nbarthchildid#1{% #1=n
  \expandnumberarg\forest@node@nbarthchildid@{\number\forestove{@last}}{#1}%
}
\def\forest@node@nbarthchildid@#1#2{%
  \ifnum#1=0
    0%
  \else
    \ifnum#2>1
      \@escapeifif{\expandtwonumberargs
        \forest@node@nbarthchildid@{\forestOve{#1}{@previous}}{\numexpr#2-1}}%
    \else
      #1%
    \fi
  \fi
}
\def\forest@node@nornbarthchildid#1{%
  \ifnum#1>0
    \forest@node@nthchildid{#1}%
  \else
    \ifnum#1<0
      \forest@node@nbarthchildid{-#1}%
    \else
      \forest@node@nornbarthchildid@error
    \fi
  \fi
}
\def\forest@node@nornbarthchildid@error{%
  \PackageError{forest}{In \string\forest@node@nornbarthchildid, n should !=0}{}%
}
\def\forest@node@previousleafid{%
  \expandnumberarg\forest@node@Previousleafid{\forest@cn}%
}
\def\forest@node@Previousleafid#1{%
  \ifnum\forestOve{#1}{@previous}=0
    \@escapeif{\expandnumberarg\forest@node@previousleafid@Goup{#1}}%
  \else
    \expandnumberarg\forest@node@previousleafid@Godown{\forestOve{#1}{@previous}}%
  \fi
}
\def\forest@node@previousleafid@Goup#1{%
  \ifnum\forestOve{#1}{@parent}=0
    \PackageError{forest}{get previous leaf: this is the first leaf}{}%
  \else
    \@escapeif{\expandnumberarg\forest@node@Previousleafid{\forestOve{#1}{@parent}}}%
  \fi
}
\def\forest@node@previousleafid@Godown#1{%
  \ifnum\forestOve{#1}{@last}=0
    #1%
  \else
    \@escapeif{\expandnumberarg\forest@node@previousleafid@Godown{\forestOve{#1}{@last}}}%
  \fi
}
\def\forest@node@nextleafid{%
  \expandnumberarg\forest@node@Nextleafid{\forest@cn}%
}
\def\forest@node@Nextleafid#1{%
  \ifnum\forestOve{#1}{@next}=0
    \@escapeif{\expandnumberarg\forest@node@nextleafid@Goup{#1}}%
  \else
    \expandnumberarg\forest@node@nextleafid@Godown{\forestOve{#1}{@next}}%
  \fi
}
\def\forest@node@nextleafid@Goup#1{%
  \ifnum\forestOve{#1}{@parent}=0
    \PackageError{forest}{get next leaf: this is the last leaf}{}%
  \else
    \@escapeif{\expandnumberarg\forest@node@Nextleafid{\forestOve{#1}{@parent}}}%
  \fi
}
\def\forest@node@nextleafid@Godown#1{%
  \ifnum\forestOve{#1}{@first}=0
    #1%
  \else
    \@escapeif{\expandnumberarg\forest@node@nextleafid@Godown{\forestOve{#1}{@first}}}%
  \fi
}
\def\forest@node@linearnextid{%
  \ifnum\forestove{@first}=0
    \expandafter\forest@node@linearnextnotdescendantid
  \else
    \forestove{@first}%
  \fi
}
\def\forest@node@linearnextnotdescendantid{%
  \expandnumberarg\forest@node@Linearnextnotdescendantid{\forest@cn}%
}
\def\forest@node@Linearnextnotdescendantid#1{%
  \ifnum\forestOve{#1}{@next}=0
    \@escapeif{\expandnumberarg\forest@node@Linearnextnotdescendantid{\forestOve{#1}{@parent}}}%
  \else
    \forestOve{#1}{@next}%
  \fi
}
\def\forest@node@linearpreviousid{%
  \ifnum\forestove{@previous}=0
    \forestove{@parent}%
  \else
    \forest@node@previousleafid
  \fi
}
\def\forest@ifancestorof#1{% is the current node an ancestor of #1? Yes: #2, no: #3
  \expandnumberarg\forest@ifancestorof@{\forestOve{#1}{@parent}}%
}
\def\forest@ifancestorof@#1#2#3{%
  \ifnum#1=0
    \def\forest@ifancestorof@next{\@secondoftwo}%
  \else
    \ifnum\forest@cn=#1
      \def\forest@ifancestorof@next{\@firstoftwo}%
    \else
      \def\forest@ifancestorof@next{\expandnumberarg\forest@ifancestorof@{\forestOve{#1}{@parent}}}%
    \fi
  \fi
  \forest@ifancestorof@next{#2}{#3}%
}
%    \end{macrocode}
%
%
% \subsection{Node walk}
%
%    \begin{macrocode}
\newloop\forest@nodewalk@loop
\forestset{
  @handlers@save@currentpath/.code={%
    \edef\pgfkeyscurrentkey{\pgfkeyscurrentpath}%
    \let\forest@currentkey\pgfkeyscurrentkey
    \pgfkeys@split@path
    \edef\forest@currentpath{\pgfkeyscurrentpath}%
    \let\forest@currentname\pgfkeyscurrentname
  },
  /handlers/.step 0 args/.style={
    /forest/@handlers@save@currentpath,
    \forest@currentkey/.code={#1\forestset{node walk/every step}},
    /forest/for \forest@currentname/.style/.expanded={%
      for={\forest@currentname}{####1}%
    }
  },
  /handlers/.step 1 arg/.style={%
    /forest/@handlers@save@currentpath,
    \forest@currentkey/.code={#1\forestset{node walk/every step}},
    /forest/for \forest@currentname/.style 2 args/.expanded={%
      for={\forest@currentname=####1}{####2}%
    }
  },
  node walk/.code={%
    \forestset{%
      node walk/before walk,%
      node walk/.cd,
      #1,%
      /forest/.cd,
      node walk/after walk
    }%
  },
  for/.code 2 args={%
    \forest@forthis{%
      \pgfkeysalso{%
        node walk/before walk/.style={},%
        node walk/every step/.style={},%
        node walk/after walk/.style={/forest,if id=0{}{#2}},%
        %node walk/after walk/.style={#2},%
        node walk={#1}%
      }%
    }%
  },
  node walk/.cd,  
  before walk/.code={},
  every step/.code={},
  after walk/.code={},
  current/.step 0 args={},
  current/.default=1,
  next/.step 0 args={\edef\forest@cn{\forestove{@next}}},
  next/.default=1,
  previous/.step 0 args={\edef\forest@cn{\forestove{@previous}}},
  previous/.default=1,
  parent/.step 0 args={\edef\forest@cn{\forestove{@parent}}},
  parent/.default=1,
  first/.step 0 args={\edef\forest@cn{\forestove{@first}}},
  first/.default=1,
  last/.step 0 args={\edef\forest@cn{\forestove{@last}}},
  last/.default=1,
  n/.step 1 arg={%
    \def\forest@nodewalk@temp{#1}%
    \ifx\forest@nodewalk@temp\pgfkeysnovalue@text
      \edef\forest@cn{\forestove{@next}}%
    \else
      \edef\forest@cn{\forest@node@nthchildid{#1}}%
    \fi
  },
  n'/.step 1 arg={\edef\forest@cn{\forest@node@nbarthchildid{#1}}},
  sibling/.step 0 args={%
    \edef\forest@cn{%
      \ifnum\forestove{@previous}=0
        \forestove{@next}%
      \else
        \forestove{@previous}%
      \fi
    }%
  },
  previous leaf/.step 0 args={\edef\forest@cn{\forest@node@previousleafid}},
  previous leaf/.default=1,
  next leaf/.step 0 args={\edef\forest@cn{\forest@node@nextleafid}},
  next leaf/.default=1,
  linear next/.step 0 args={\edef\forest@cn{\forest@node@linearnextid}},
  linear previous/.step 0 args={\edef\forest@cn{\forest@node@linearpreviousid}},
  first leaf/.step 0 args={%
    \forest@nodewalk@loop
      \edef\forest@cn{\forestove{@first}}%
    \unless\ifnum\forestove{@first}=0
    \forest@nodewalk@repeat
  },
  last leaf/.step 0 args={%
    \forest@nodewalk@loop
      \edef\forest@cn{\forestove{@last}}%
    \unless\ifnum\forestove{@last}=0
    \forest@nodewalk@repeat
  },
  to tier/.step 1 arg={%
    \def\forest@nodewalk@giventier{#1}%
    \forest@nodewalk@loop
      \forestoget{tier}\forest@nodewalk@tier
    \unless\ifx\forest@nodewalk@tier\forest@nodewalk@giventier
      \forestoget{@parent}\forest@cn
    \forest@nodewalk@repeat
  },
  next on tier/.step 0 args={\forest@nodewalk@nextontier},
  next on tier/.default=1,
  previous on tier/.step 0 args={\forest@nodewalk@previousontier},
  previous on tier/.default=1,
  name/.step 1 arg={\edef\forest@cn{\forest@node@Nametoid{#1}}},
  root/.step 0 args={\edef\forest@cn{\forest@node@rootid}},
  root'/.step 0 args={\edef\forest@cn{\forest@root}},
  id/.step 1 arg={\edef\forest@cn{#1}},
  % maybe it's not wise to have short-step sequences and names potentially clashing
  % .unknown/.code={%
  %   \forest@node@Ifnamedefined{\pgfkeyscurrentname}%
  %     {\pgfkeysalso{name=\pgfkeyscurrentname}}%
  %     {\expandafter\forest@nodewalk@shortsteps\pgfkeyscurrentname\forest@nodewalk@endshortsteps}%
  % }, 
  .unknown/.code={%
    \expandafter\forest@nodewalk@shortsteps\pgfkeyscurrentname\forest@nodewalk@endshortsteps
  },
  node walk/.style={/forest/node walk={#1}},
  trip/.code={\forest@forthis{\pgfkeysalso{#1}}},
  group/.code={\forest@go{#1}\forestset{node walk/every step}},
  % repeat is taken later from /forest/repeat
  p/.style={previous=1},
  %n/.style={next=1}, % defined in "long" n
  u/.style={parent=1},
  s/.style={sibling},
  c/.style={current=1},
  r/.style={root},
  P/.style={previous leaf=1},
  N/.style={next leaf=1},  
  F/.style={first leaf=1},
  L/.style={last leaf=1},  
  >/.style={next on tier=1},
  </.style={previous on tier=1},
  1/.style={n=1},
  2/.style={n=2},
  3/.style={n=3},
  4/.style={n=4},
  5/.style={n=5},
  6/.style={n=6},
  7/.style={n=7},
  8/.style={n=8},
  9/.style={n=9},
  l/.style={last=1},
  %{...} is short for group={...}
}
\def\forest@nodewalk@nextontier{%
  \forestoget{tier}\forest@nodewalk@giventier
  \edef\forest@cn{\forest@node@linearnextnotdescendantid}%
  \forest@nodewalk@loop
    \forestoget{tier}\forest@nodewalk@tier
  \unless\ifx\forest@nodewalk@tier\forest@nodewalk@giventier
    \edef\forest@cn{\forest@node@linearnextid}%
  \forest@nodewalk@repeat    
}
\def\forest@nodewalk@previousontier{%
  \forestoget{tier}\forest@nodewalk@giventier
  \forest@nodewalk@loop
    \edef\forest@cn{\forest@node@linearpreviousid}%
    \forestoget{tier}\forest@nodewalk@tier    
  \unless\ifx\forest@nodewalk@tier\forest@nodewalk@giventier
  \forest@nodewalk@repeat    
}
\def\forest@nodewalk@shortsteps{%
  \futurelet\forest@nodewalk@nexttoken\forest@nodewalk@shortsteps@
}
\def\forest@nodewalk@shortsteps@#1{%
  \ifx\forest@nodewalk@nexttoken\forest@nodewalk@endshortsteps
  \else
    \ifx\forest@nodewalk@nexttoken\bgroup
      \pgfkeysalso{group=#1}%
      \@escapeifif\forest@nodewalk@shortsteps
    \else
      \pgfkeysalso{#1}%
      \@escapeifif\forest@nodewalk@shortsteps
    \fi
  \fi
}
\def\forest@go#1{%
  {%
    \forestset{%
      node walk/before walk/.code={},%
      node walk/every step/.code={},%
      node walk/after walk/.code={},%
      node walk={#1}%
    }%
    \expandafter
  }%
  \expandafter\def\expandafter\forest@cn\expandafter{\forest@cn}%
}
%    \end{macrocode}
%
% \subsection{Node options}
%    
% \subsubsection{Option-declaration mechanism}
% 
% Common code for declaring options.
%    \begin{macrocode}
\def\forest@declarehandler#1#2#3{%#1=handler for specific type,#2=option name,#3=default value
  \pgfkeyssetvalue{/forest/#2}{#3}%
  \appto\forest@node@init{\forestoinit{#2}}%
  \forest@convert@others@to@underscores{#2}\forest@pgfmathoptionname
  \edef\forest@marshal{%
    \noexpand#1{/forest/#2}{/forest}{#2}{\forest@pgfmathoptionname}%
  }\forest@marshal
}
\def\forest@def@with@pgfeov#1#2{% \pgfeov mustn't occur in the arg of the .code handler!!!
  \long\def#1##1\pgfeov{#2}%
}
%    \end{macrocode}
% Option-declaration handlers.
%    \begin{macrocode}
\newtoks\forest@temp@toks
\def\forest@declaretoks@handler#1#2#3#4{%
  \forest@declaretoks@handler@A{#1}{#2}{#3}{#4}{}%
}
\def\forest@declarekeylist@handler#1#2#3#4{%
  \forest@declaretoks@handler@A{#1}{#2}{#3}{#4}{,}%
  \pgfkeysgetvalue{#1/.@cmd}\forest@temp
  \pgfkeyslet{#1'/.@cmd}\forest@temp
  \pgfkeyssetvalue{#1'/option@name}{#3}%
  \pgfkeysgetvalue{#1+/.@cmd}\forest@temp
  \pgfkeyslet{#1/.@cmd}\forest@temp
 }
\def\forest@declaretoks@handler@A#1#2#3#4#5{% #1=key,#2=path,#3=name,#4=pgfmathname,#5=infix
  \pgfkeysalso{%
    #1/.code={\forestOset{\forest@setter@node}{#3}{##1}},
    #1+/.code={\forestOappto{\forest@setter@node}{#3}{#5##1}},
    #1-/.code={\forestOpreto{\forest@setter@node}{#3}{##1#5}},
    #2/if #3/.code n args={3}{%
      \forestoget{#3}\forest@temp@option@value
      \edef\forest@temp@compared@value{\unexpanded{##1}}%
      \ifx\forest@temp@option@value\forest@temp@compared@value
        \pgfkeysalso{##2}%
      \else
        \pgfkeysalso{##3}%
      \fi
    },
    #2/if in #3/.code n args={3}{%
      \forestoget{#3}\forest@temp@option@value
      \edef\forest@temp@compared@value{\unexpanded{##1}}%
      \expandafter\expandafter\expandafter\pgfutil@in@\expandafter\expandafter\expandafter{\expandafter\forest@temp@compared@value\expandafter}\expandafter{\forest@temp@option@value}%
      \ifpgfutil@in@
        \pgfkeysalso{##2}%
      \else
        \pgfkeysalso{##3}%
      \fi
    },
    #2/where #3/.style n args={3}{for tree={#2/if #3={##1}{##2}{##3}}},
    #2/where in #3/.style n args={3}{for tree={#2/if in #3={##1}{##2}{##3}}}
  }%
  \pgfkeyssetvalue{#1/option@name}{#3}%  
  \pgfkeyssetvalue{#1+/option@name}{#3}%  
  \pgfmathdeclarefunction{#4}{1}{\forest@pgfmathhelper@attribute@toks{##1}{#3}}%
}
\def\forest@declareautowrappedtoks@handler#1#2#3#4{% #1=key,#2=path,#3=name,#4=pgfmathname,#5=infix
  \forest@declaretoks@handler{#1}{#2}{#3}{#4}%
  \pgfkeysgetvalue{#1/.@cmd}\forest@temp
  \pgfkeyslet{#1'/.@cmd}\forest@temp
  \pgfkeysalso{#1/.style={#1'/.wrap value={##1}}}%
  \pgfkeyssetvalue{#1'/option@name}{#3}%
  \pgfkeysgetvalue{#1+/.@cmd}\forest@temp
  \pgfkeyslet{#1+'/.@cmd}\forest@temp
  \pgfkeysalso{#1+/.style={#1+'/.wrap value={##1}}}%
  \pgfkeyssetvalue{#1+'/option@name}{#3}%
  \pgfkeysgetvalue{#1-/.@cmd}\forest@temp
  \pgfkeyslet{#1-'/.@cmd}\forest@temp
  \pgfkeysalso{#1-/.style={#1-'/.wrap value={##1}}}%
  \pgfkeyssetvalue{#1-'/option@name}{#3}%
}
\def\forest@declarereadonlydimen@handler#1#2#3#4{% #1=key,#2=path,#3=name,#4=pgfmathname
  \pgfkeysalso{%
    #2/if #3/.code n args={3}{%
      \forestoget{#3}\forest@temp@option@value
      \ifdim\forest@temp@option@value=##1\relax
        \pgfkeysalso{##2}%
      \else
        \pgfkeysalso{##3}%
      \fi
    },
    #2/where #3/.style n args={3}{for tree={#2/if #3={##1}{##2}{##3}}},
  }%
  \pgfmathdeclarefunction{#4}{1}{\forest@pgfmathhelper@attribute@dimen{##1}{#3}}%
}
\def\forest@declaredimen@handler#1#2#3#4{% #1=key,#2=path,#3=name,#4=pgfmathname
  \forest@declarereadonlydimen@handler{#1}{#2}{#3}{#4}%
  \pgfkeysalso{%
    #1/.code={%
      \pgfmathsetlengthmacro\forest@temp{##1}%
      \forestOlet{\forest@setter@node}{#3}\forest@temp
    },
    #1+/.code={%
      \pgfmathsetlengthmacro\forest@temp{##1}%
      \pgfutil@tempdima=\forestove{#3}
      \advance\pgfutil@tempdima\forest@temp\relax
      \forestOeset{\forest@setter@node}{#3}{\the\pgfutil@tempdima}%
    },
    #1-/.code={%
      \pgfmathsetlengthmacro\forest@temp{##1}%
      \pgfutil@tempdima=\forestove{#3}
      \advance\pgfutil@tempdima-\forest@temp\relax
      \forestOeset{\forest@setter@node}{#3}{\the\pgfutil@tempdima}%
    },
    #1*/.style={%
      #1={#4()*(##1)}%
    },
    #1:/.style={%
      #1={#4()/(##1)}%
    },
    #1'/.code={%
      \pgfutil@tempdima=##1\relax
      \forestOeset{\forest@setter@node}{#3}{\the\pgfutil@tempdima}%
    },
    #1'+/.code={%
      \pgfutil@tempdima=\forestove{#3}\relax
      \advance\pgfutil@tempdima##1\relax
      \forestOeset{\forest@setter@node}{#3}{\the\pgfutil@tempdima}%
    },
    #1'-/.code={%
      \pgfutil@tempdima=\forestove{#3}\relax
      \advance\pgfutil@tempdima-##1\relax
      \forestOeset{\forest@setter@node}{#3}{\the\pgfutil@tempdima}%
    },
    #1'*/.style={%
      \pgfutil@tempdima=\forestove{#3}\relax
      \multiply\pgfutil@tempdima##1\relax
      \forestOeset{\forest@setter@node}{#3}{\the\pgfutil@tempdima}%
    },
    #1':/.style={%
      \pgfutil@tempdima=\forestove{#3}\relax
      \divide\pgfutil@tempdima##1\relax
      \forestOeset{\forest@setter@node}{#3}{\the\pgfutil@tempdima}%
    },
  }%
  \pgfkeyssetvalue{#1/option@name}{#3}%  
  \pgfkeyssetvalue{#1+/option@name}{#3}%  
  \pgfkeyssetvalue{#1-/option@name}{#3}%  
  \pgfkeyssetvalue{#1*/option@name}{#3}%  
  \pgfkeyssetvalue{#1:/option@name}{#3}%  
  \pgfkeyssetvalue{#1'/option@name}{#3}%  
  \pgfkeyssetvalue{#1'+/option@name}{#3}%  
  \pgfkeyssetvalue{#1'-/option@name}{#3}%  
  \pgfkeyssetvalue{#1'*/option@name}{#3}%  
  \pgfkeyssetvalue{#1':/option@name}{#3}%  
}
\def\forest@declarereadonlycount@handler#1#2#3#4{% #1=key,#2=path,#3=name,#4=pgfmathname
  \pgfkeysalso{
    #2/if #3/.code n args={3}{%
      \forestoget{#3}\forest@temp@option@value
      \ifnum\forest@temp@option@value=##1\relax
        \pgfkeysalso{##2}%
      \else
        \pgfkeysalso{##3}%
      \fi
    },
    #2/where #3/.style n args={3}{for tree={#2/if #3={##1}{##2}{##3}}},
  }%
  \pgfmathdeclarefunction{#4}{1}{\forest@pgfmathhelper@attribute@count{##1}{#3}}%
}
\def\forest@declarecount@handler#1#2#3#4{% #1=key,#2=path,#3=name,#4=pgfmathname
  \forest@declarereadonlycount@handler{#1}{#2}{#3}{#4}%
  \pgfkeysalso{
    #1/.code={%
      \pgfmathtruncatemacro\forest@temp{##1}%
      \forestOlet{\forest@setter@node}{#3}\forest@temp
    },
    #1+/.code={%
      \pgfmathsetlengthmacro\forest@temp{##1}%
      \c@pgf@counta=\forestove{#3}\relax
      \advance\c@pgf@counta\forest@temp\relax
      \forestOeset{\forest@setter@node}{#3}{\the\c@pgf@counta}%
    },
    #1-/.code={%
      \pgfmathsetlengthmacro\forest@temp{##1}%
      \c@pgf@counta=\forestove{#3}\relax
      \advance\c@pgf@counta-\forest@temp\relax
      \forestOeset{\forest@setter@node}{#3}{\the\c@pgf@counta}%
    },
    #1*/.code={%
      \pgfmathsetlengthmacro\forest@temp{##1}%
      \c@pgf@counta=\forestove{#3}\relax
      \multiply\c@pgf@counta\forest@temp\relax
      \forestOeset{\forest@setter@node}{#3}{\the\c@pgf@counta}%
    },
    #1:/.code={%
      \pgfmathsetlengthmacro\forest@temp{##1}%
      \c@pgf@counta=\forestove{#3}\relax
      \divide\c@pgf@counta\forest@temp\relax
      \forestOeset{\forest@setter@node}{#3}{\the\c@pgf@counta}%
    },
    #1'/.code={%
      \c@pgf@counta=##1\relax
      \forestOeset{\forest@setter@node}{#3}{\the\c@pgf@counta}%
    },
    #1'+/.code={%
      \c@pgf@counta=\forestove{#3}\relax
      \advance\c@pgf@counta##1\relax
      \forestOeset{\forest@setter@node}{#3}{\the\c@pgf@counta}%
    },
    #1'-/.code={%
      \c@pgf@counta=\forestove{#3}\relax
      \advance\c@pgf@counta-##1\relax
      \forestOeset{\forest@setter@node}{#3}{\the\c@pgf@counta}%
    },
    #1'*/.style={%
      \c@pgf@counta=\forestove{#3}\relax
      \multiply\c@pgf@counta##1\relax
      \forestOeset{\forest@setter@node}{#3}{\the\c@pgf@counta}%
    },
    #1':/.style={%
      \c@pgf@counta=\forestove{#3}\relax
      \divide\c@pgf@counta##1\relax
      \forestOeset{\forest@setter@node}{#3}{\the\c@pgf@counta}%
    },
  }%
  \pgfkeyssetvalue{#1/option@name}{#3}%  
  \pgfkeyssetvalue{#1+/option@name}{#3}%  
  \pgfkeyssetvalue{#1-/option@name}{#3}%  
  \pgfkeyssetvalue{#1*/option@name}{#3}%  
  \pgfkeyssetvalue{#1:/option@name}{#3}%  
  \pgfkeyssetvalue{#1'/option@name}{#3}%  
  \pgfkeyssetvalue{#1'+/option@name}{#3}%  
  \pgfkeyssetvalue{#1'-/option@name}{#3}%  
  \pgfkeyssetvalue{#1'*/option@name}{#3}%  
  \pgfkeyssetvalue{#1':/option@name}{#3}%  
}
\def\forest@declareboolean@handler#1#2#3#4{% #1=key,#2=path,#3=name,#4=pgfmathname
  \pgfkeysalso{%
    #1/.code={%
      \ifstrequal{##1}{1}{%
        \forestOset{\forest@setter@node}{#3}{1}%
      }{%
        \pgfmathifthenelse{##1}{1}{0}%
        \forestOlet{\forest@setter@node}{#3}\pgfmathresult
      }%
    },
    #1/.default=1,
    #2/not #3/.code={\forestOset{\forest@setter@node}{#3}{0}},
    #2/if #3/.code 2 args={%
      \forestoget{#3}\forest@temp@option@value
      \ifnum\forest@temp@option@value=1
        \pgfkeysalso{##1}%
      \else
        \pgfkeysalso{##2}%
      \fi
    },
    #2/where #3/.style 2 args={for tree={#2/if #3={##1}{##2}}}
  }%
  \pgfkeyssetvalue{#1/option@name}{#3}%
  \pgfmathdeclarefunction{#4}{1}{\forest@pgfmathhelper@attribute@count{##1}{#3}}%
}
\pgfkeys{/forest,
  declare toks/.code 2 args={%
    \forest@declarehandler\forest@declaretoks@handler{#1}{#2}%
  },
  declare autowrapped toks/.code 2 args={%
    \forest@declarehandler\forest@declareautowrappedtoks@handler{#1}{#2}%
  },
  declare keylist/.code 2 args={%
    \forest@declarehandler\forest@declarekeylist@handler{#1}{#2}%
  },
  declare readonly dimen/.code={%
    \forest@declarehandler\forest@declarereadonlydimen@handler{#1}{}%
  },
  declare dimen/.code 2 args={%
    \forest@declarehandler\forest@declaredimen@handler{#1}{#2}%
  },
  declare readonly count/.code={%
    \forest@declarehandler\forest@declarereadonlycount@handler{#1}{}%
  },
  declare count/.code 2 args={%
    \forest@declarehandler\forest@declarecount@handler{#1}{#2}%
  },
  declare boolean/.code 2 args={%
    \forest@declarehandler\forest@declareboolean@handler{#1}{#2}%
  },
  /handlers/.pgfmath/.code={%
    \pgfmathparse{#1}%
    \pgfkeysalso{\pgfkeyscurrentpath/.expand once=\pgfmathresult}%
  },
  /handlers/.wrap value/.code={%
    \edef\forest@handlers@wrap@currentpath{\pgfkeyscurrentpath}%
    \pgfkeysgetvalue{\forest@handlers@wrap@currentpath/option@name}\forest@currentoptionname
    \expandafter\forestoget\expandafter{\forest@currentoptionname}\forest@option@value
    \forest@def@with@pgfeov\forest@wrap@code{#1}%
    \expandafter\edef\expandafter\forest@wrapped@value\expandafter{\expandafter\expandonce\expandafter{\expandafter\forest@wrap@code\forest@option@value\pgfeov}}%
    \pgfkeysalso{\forest@handlers@wrap@currentpath/.expand once=\forest@wrapped@value}%
  },
  /handlers/.wrap pgfmath arg/.code 2 args={%    
    \pgfmathparse{#2}\let\forest@wrap@arg@i\pgfmathresult
    \edef\forest@wrap@args{{\expandonce\forest@wrap@arg@i}}%
    \def\forest@wrap@code##1{#1}%
    \expandafter\expandafter\expandafter\forest@temp@toks\expandafter\expandafter\expandafter{\expandafter\forest@wrap@code\forest@wrap@args}%
    \pgfkeysalso{\pgfkeyscurrentpath/.expand once=\the\forest@temp@toks}%
  },
  /handlers/.wrap 2 pgfmath args/.code n args={3}{%
    \pgfmathparse{#2}\let\forest@wrap@arg@i\pgfmathresult
    \pgfmathparse{#3}\let\forest@wrap@arg@ii\pgfmathresult
    \edef\forest@wrap@args{{\expandonce\forest@wrap@arg@i}{\expandonce\forest@wrap@arg@ii}}%
    \def\forest@wrap@code##1##2{#1}%
    \expandafter\expandafter\expandafter\def\expandafter\expandafter\expandafter\forest@wrapped\expandafter\expandafter\expandafter{\expandafter\forest@wrap@code\forest@wrap@args}%
    \pgfkeysalso{\pgfkeyscurrentpath/.expand once=\forest@wrapped}%
  },
  /handlers/.wrap 3 pgfmath args/.code n args={4}{%
    \forest@wrap@n@pgfmath@args{#2}{#3}{#4}{}{}{}{}{}{3}%
    \forest@wrap@n@pgfmath@do{#1}{3}},
  /handlers/.wrap 4 pgfmath args/.code n args={5}{%
    \forest@wrap@n@pgfmath@args{#2}{#3}{#4}{#5}{}{}{}{}{4}%
    \forest@wrap@n@pgfmath@do{#1}{4}},
  /handlers/.wrap 5 pgfmath args/.code n args={6}{%
    \forest@wrap@n@pgfmath@args{#2}{#3}{#4}{#5}{#6}{}{}{}{5}%
    \forest@wrap@n@pgfmath@do{#1}{5}},
  /handlers/.wrap 6 pgfmath args/.code n args={7}{%
    \forest@wrap@n@pgfmath@args{#2}{#3}{#4}{#5}{#6}{#7}{}{}{6}%
    \forest@wrap@n@pgfmath@do{#1}{6}},
  /handlers/.wrap 7 pgfmath args/.code n args={8}{%
    \forest@wrap@n@pgfmath@args{#2}{#3}{#4}{#5}{#6}{#7}{#8}{}{7}%
    \forest@wrap@n@pgfmath@do{#1}{7}},
  /handlers/.wrap 8 pgfmath args/.code n args={9}{%
    \forest@wrap@n@pgfmath@args{#2}{#3}{#4}{#5}{#6}{#7}{#8}{#9}{8}%
    \forest@wrap@n@pgfmath@do{#1}{8}},
}
\def\forest@wrap@n@pgfmath@args#1#2#3#4#5#6#7#8#9{%
  \pgfmathparse{#1}\let\forest@wrap@arg@i\pgfmathresult
  \ifnum#9>1 \pgfmathparse{#2}\let\forest@wrap@arg@ii\pgfmathresult\fi
  \ifnum#9>2 \pgfmathparse{#3}\let\forest@wrap@arg@iii\pgfmathresult\fi
  \ifnum#9>3 \pgfmathparse{#4}\let\forest@wrap@arg@iv\pgfmathresult\fi
  \ifnum#9>4 \pgfmathparse{#5}\let\forest@wrap@arg@v\pgfmathresult\fi
  \ifnum#9>5 \pgfmathparse{#6}\let\forest@wrap@arg@vi\pgfmathresult\fi
  \ifnum#9>6 \pgfmathparse{#7}\let\forest@wrap@arg@vii\pgfmathresult\fi
  \ifnum#9>7 \pgfmathparse{#8}\let\forest@wrap@arg@viii\pgfmathresult\fi
  \edef\forest@wrap@args{%
    {\expandonce\forest@wrap@arg@i}
    \ifnum#9>1 {\expandonce\forest@wrap@arg@ii}\fi
    \ifnum#9>2 {\expandonce\forest@wrap@arg@iii}\fi
    \ifnum#9>3 {\expandonce\forest@wrap@arg@iv}\fi
    \ifnum#9>4 {\expandonce\forest@wrap@arg@v}\fi
    \ifnum#9>5 {\expandonce\forest@wrap@arg@vi}\fi
    \ifnum#9>6 {\expandonce\forest@wrap@arg@vii}\fi
    \ifnum#9>7 {\expandonce\forest@wrap@arg@viii}\fi
  }%
}
\def\forest@wrap@n@pgfmath@do#1#2{%
  \ifcase#2\relax
  \or\def\forest@wrap@code##1{#1}%
  \or\def\forest@wrap@code##1##2{#1}%
  \or\def\forest@wrap@code##1##2##3{#1}%
  \or\def\forest@wrap@code##1##2##3##4{#1}%
  \or\def\forest@wrap@code##1##2##3##4##5{#1}%
  \or\def\forest@wrap@code##1##2##3##4##5##6{#1}%
  \or\def\forest@wrap@code##1##2##3##4##5##6##7{#1}%
  \or\def\forest@wrap@code##1##2##3##4##5##6##7##8{#1}%
  \fi
  \expandafter\expandafter\expandafter\def\expandafter\expandafter\expandafter\forest@wrapped\expandafter\expandafter\expandafter{\expandafter\forest@wrap@code\forest@wrap@args}%
  \pgfkeysalso{\pgfkeyscurrentpath/.expand once=\forest@wrapped}%
}
%    \end{macrocode}
%
% \subsubsection{Declaring options}
%
%    \begin{macrocode}
\def\forest@node@setname#1{%
  \forestoeset{name}{#1}%
  \csedef{forest@id@of@#1}{\forest@cn}%
}
\def\forest@node@Nametoid#1{% #1 = name
  \csname forest@id@of@#1\endcsname
}
\def\forest@node@Ifnamedefined#1{% #1 = name, #2=true,#3=false
  \ifcsname forest@id@of@#1\endcsname
    \expandafter\@firstoftwo
  \else
    \expandafter\@secondoftwo
  \fi
}
\def\forest@node@setalias#1{%
  \csedef{forest@id@of@#1}{\forest@cn}%    
}
\def\forest@node@Setalias#1#2{%
  \csedef{forest@id@of@#2}{#1}%
}  
\forestset{
  TeX/.code={#1},
  TeX'/.code={\appto\forest@externalize@loadimages{#1}#1},
  TeX''/.code={\appto\forest@externalize@loadimages{#1}},
  declare toks={name}{},
  name/.code={% override the default setter
    \forest@node@setname{#1}%
  },
  alias/.code={\forest@node@setalias{#1}},
  begin draw/.code={\begin{tikzpicture}},
  end draw/.code={\end{tikzpicture}},
  begin forest/.code={},
  end forest/.code={},
  declare autowrapped toks={content}{},
  declare count={grow}{270},
  TeX={% a hack for grow-reversed connection, and compass-based grow specification
    \pgfkeysgetvalue{/forest/grow/.@cmd}\forest@temp
    \pgfkeyslet{/forest/grow@@/.@cmd}\forest@temp
  },
  grow/.style={grow@={#1},reversed=0},
  grow'/.style={grow@={#1},reversed=1},
  grow''/.style={grow@={#1}},
  grow@/.is choice,
  grow@/east/.style={/forest/grow@@=0},
  grow@/north east/.style={/forest/grow@@=45},
  grow@/north/.style={/forest/grow@@=90},
  grow@/north west/.style={/forest/grow@@=135},
  grow@/west/.style={/forest/grow@@=180},
  grow@/south west/.style={/forest/grow@@=225},
  grow@/south/.style={/forest/grow@@=270},
  grow@/south east/.style={/forest/grow@@=315},
  grow@/.unknown/.code={\let\forest@temp@grow\pgfkeyscurrentname
    \pgfkeysalso{/forest/grow@@/.expand once=\forest@temp@grow}},
  declare boolean={reversed}{0},
  declare toks={parent anchor}{},
  declare toks={child anchor}{},
  declare toks={anchor}{base},
  declare toks={calign}{midpoint},
  TeX={%
    \pgfkeysgetvalue{/forest/calign/.@cmd}\forest@temp
    \pgfkeyslet{/forest/calign'/.@cmd}\forest@temp
  },
  calign/.is choice,
  calign/child/.style={calign'=child},
  calign/first/.style={calign'=child,calign primary child=1},
  calign/last/.style={calign'=child,calign primary child=-1},
  calign with current/.style={for parent/.wrap pgfmath arg={calign=child,calign primary child=##1}{n}},
  calign with current edge/.style={for parent/.wrap pgfmath arg={calign=child edge,calign primary child=##1}{n}},
  calign/child edge/.style={calign'=child edge},
  calign/midpoint/.style={calign'=midpoint},
  calign/center/.style={calign'=midpoint,calign primary child=1,calign secondary child=-1},
  calign/edge midpoint/.style={calign'=edge midpoint},
  calign/fixed angles/.style={calign'=fixed angles},
  calign/fixed edge angles/.style={calign'=fixed edge angles},
  calign/.unknown/.code={\PackageError{forest}{unknown calign '\pgfkeyscurrentname'}{}},
  declare count={calign primary child}{1},
  declare count={calign secondary child}{-1},
  declare count={calign primary angle}{-35},
  declare count={calign secondary angle}{35},
  calign child/.style={calign primary child={#1}},
  calign angle/.style={calign primary angle={-#1},calign secondary angle={#1}},
  declare toks={tier}{},
  declare toks={fit}{tight},
  declare boolean={ignore}{0},
  declare boolean={ignore edge}{0},
  no edge/.style={edge'={},ignore edge},
  declare keylist={edge}{draw},
  declare toks={edge path}{%
    \noexpand\path[\forestoption{edge}]%
    (\forestOve{\forestove{@parent}}{name}.parent anchor)--(\forestove{name}.child anchor)\forestoption{edge label};},
  triangle/.style={edge path={%
      \noexpand\path[\forestoption{edge}]%
      (\forestove{name}.north east)--(\forestOve{\forestove{@parent}}{name}.south)--(\forestove{name}.north west)--(\forestove{name}.north east)\forestoption{edge label};}},
  declare toks={edge label}{},
  declare boolean={phantom}{0},
  baseline/.style={alias={forest@baseline@node}},
  declare readonly count={n},
  declare readonly count={n'},
  declare readonly count={n children},
  declare readonly count={level},
  declare dimen=x{},
  declare dimen=y{},
  declare dimen={s}{0pt},
  declare dimen={l}{6ex}, % just in case: should be set by the calibration
  declare dimen={s sep}{0.6666em},
  declare dimen={l sep}{1ex},  % just in case: calibration!
  declare keylist={node options}{},
  declare toks={tikz}{},
  afterthought/.style={tikz+={#1}},
  label/.style={tikz={\path[late options={%
        name=\forestoption{name},label={#1}}];}},
  pin/.style={tikz={\path[late options={%
        name=\forestoption{name},pin={#1}}];}},
  declare toks={content format}{\forestoption{content}},
  math content/.style={content format={\ensuremath{\forestoption{content}}}},
  declare toks={node format}{%
    \noexpand\node
    [\forestoption{node options},anchor=\forestoption{anchor}]%
    (\forestoption{name})%
    {\foresteoption{content format}};%
  },
  tabular@environment/.style={content format={%
    \noexpand\begin{tabular}[\forestoption{base}]{\forestoption{align}}%
      \forestoption{content}%
     \noexpand\end{tabular}%
  }},
  declare toks={align}{},
  TeX={\pgfkeysgetvalue{/forest/align/.@cmd}\forest@temp
    \pgfkeyslet{/forest/align'/.@cmd}\forest@temp},
  align/.is choice,
  align/.unknown/.code={%
    \edef\forest@marshal{%
      \noexpand\pgfkeysalso{%
        align'={\pgfkeyscurrentname},%
        tabular@environment
      }%
    }\forest@marshal
  },
  align/center/.style={align'={@{}c@{}},tabular@environment},
  align/left/.style={align'={@{}l@{}},tabular@environment},
  align/right/.style={align'={@{}r@{}},tabular@environment},
  declare toks={base}{t},
  TeX={\pgfkeysgetvalue{/forest/base/.@cmd}\forest@temp
    \pgfkeyslet{/forest/base'/.@cmd}\forest@temp},
  base/.is choice,
  base/top/.style={base'=t},
  base/bottom/.style={base'=b},
  base/.unknown/.style={base'/.expand once=\pgfkeyscurrentname},
  .unknown/.code={%
    \expandafter\pgfutil@in@\expandafter.\expandafter{\pgfkeyscurrentname}%
    \ifpgfutil@in@
      \expandafter\forest@relatednode@option@setter\pgfkeyscurrentname=#1\forest@END
    \else
      \edef\forest@marshal{%
        \noexpand\pgfkeysalso{node options={\pgfkeyscurrentname=\unexpanded{#1}}}%
      }\forest@marshal
    \fi
  },
  get node boundary/.code={%
    \forestoget{boundary}\forest@node@boundary
    \def#1{}%
    \forest@extendpath#1\forest@node@boundary{\pgfpoint{\forestove{x}}{\forestove{y}}}%
  },
  % get min l tree boundary/.code={%
  %   \forest@get@tree@boundary{negative}{\the\numexpr\forestove{grow}-90\relax}#1},
  % get max l tree boundary/.code={%
  %   \forest@get@tree@boundary{positive}{\the\numexpr\forestove{grow}-90\relax}#1},
  get min s tree boundary/.code={%
    \forest@get@tree@boundary{negative}{\forestove{grow}}#1},
  get max s tree boundary/.code={%
    \forest@get@tree@boundary{positive}{\forestove{grow}}#1},
  fit to tree/.code={%
    \pgfkeysalso{%
      /forest/get min s tree boundary=\forest@temp@negative@boundary,
      /forest/get max s tree boundary=\forest@temp@positive@boundary
    }%
    \edef\forest@temp@boundary{\expandonce{\forest@temp@negative@boundary}\expandonce{\forest@temp@positive@boundary}}%
    \forest@path@getboundingrectangle@xy\forest@temp@boundary
    \pgfkeysalso{inner sep=0,fit/.expanded={(\the\pgf@xa,\the\pgf@ya)(\the\pgf@xb,\the\pgf@yb)}}%
  },
  use as bounding box/.style={%
    before drawing tree={
      tikz+/.expanded={%
        \noexpand\pgfresetboundingbox
        \noexpand\useasboundingbox
        ($(.anchor)+(\forestoption{min x},\forestoption{min y})$)
        rectangle
        ($(.anchor)+(\forestoption{max x},\forestoption{max y})$)
        ;
      }
    }
  },
  use as bounding box'/.style={%
    before drawing tree={
      tikz+/.expanded={%
        \noexpand\pgfresetboundingbox
        \noexpand\useasboundingbox
        ($(.anchor)+(\forestoption{min x}+\pgfkeysvalueof{/pgf/outer xsep}/2+\pgfkeysvalueof{/pgf/inner xsep},\forestoption{min y}+\pgfkeysvalueof{/pgf/outer ysep}/2+\pgfkeysvalueof{/pgf/inner ysep})$)
        rectangle
        ($(.anchor)+(\forestoption{max x}-\pgfkeysvalueof{/pgf/outer xsep}/2-\pgfkeysvalueof{/pgf/inner xsep},\forestoption{max y}-\pgfkeysvalueof{/pgf/outer ysep}/2-\pgfkeysvalueof{/pgf/inner ysep})$)
        ;
      }
    }
  },
}%
\def\forest@get@tree@boundary#1#2#3{%#1=pos/neg,#2=grow,#3=receiving cs
  \def#3{}%
  \forest@node@getedge{#1}{#2}\forest@temp@boundary
  \forest@extendpath#3\forest@temp@boundary{\pgfpoint{\forestove{x}}{\forestove{y}}}%
}
\def\forest@setter@node{\forest@cn}%
\def\forest@relatednode@option@setter#1.#2=#3\forest@END{%
  \forest@forthis{%
    \forest@nameandgo{#1}%
    \let\forest@setter@node\forest@cn
  }%
  \pgfkeysalso{#2={#3}}%
  \def\forest@setter@node{\forest@cn}%
}%
%    \end{macrocode}
%
% \subsubsection{Option propagation}
% 
% The propagators targeting single nodes are automatically defined by node walk steps definitions. 
%
%    \begin{macrocode}
\forestset{
  for tree/.code={\forest@node@foreach{\pgfkeysalso{#1}}},
  if/.code n args={3}{%
    \pgfmathparse{#1}%
    \ifnum\pgfmathresult=0 \pgfkeysalso{#3}\else\pgfkeysalso{#2}\fi
  },
  where/.style n args={3}{for tree={if={#1}{#2}{#3}}},
  for descendants/.code={\forest@node@foreachdescendant{\pgfkeysalso{#1}}},
  for all next/.style={for next={#1,for all next={#1}}},
  for all previous/.style={for previous={#1,for all previous={#1}}},
  for siblings/.style={for all previous={#1},for all next={#1}},
  for ancestors/.style={for parent={#1,for ancestors={#1}}},
  for ancestors'/.style={#1,for ancestors={#1}},
  for children/.code={\forest@node@foreachchild{\pgfkeysalso{#1}}},
  for c-commanded={for sibling={for tree={#1}}},
  for c-commanders={for sibling={#1},for parent={for c-commanders={#1}}}
}
%    \end{macrocode}
%    
% A bit of complication to allow for nested \keyname{repeat}s without \TeX\ groups.
%    \begin{macrocode}
\newcount\forest@repeat@key@depth
\forestset{%
  repeat/.code 2 args={%
    \advance\forest@repeat@key@depth1
    \pgfmathparse{int(#1)}%
    \csedef{forest@repeat@key@\the\forest@repeat@key@depth}{\pgfmathresult}%
    \expandafter\newloop\csname forest@repeat@key@loop@\the\forest@repeat@key@depth\endcsname
    \def\forest@marshal{%
      \csname forest@repeat@key@loop@\the\forest@repeat@key@depth\endcsname
        \forest@temp@count=\csname forest@repeat@key@\the\forest@repeat@key@depth\endcsname\relax
      \ifnum\forest@temp@count>0
        \advance\forest@temp@count-1
        \csedef{forest@repeat@key@\the\forest@repeat@key@depth}{\the\forest@temp@count}%
        \pgfkeysalso{#2}%
    }%   
    \expandafter\forest@marshal\csname forest@repeat@key@repeat@\the\forest@repeat@key@depth\endcsname
    \advance\forest@repeat@key@depth-1
  },
}
\pgfkeysgetvalue{/forest/repeat/.@cmd}\forest@temp
\pgfkeyslet{/forest/node walk/repeat/.@cmd}\forest@temp
%    
%    \end{macrocode}
%
% \subsubsection{\texttt{pgfmath} extensions}
%
%    \begin{macrocode}
\pgfmathdeclarefunction{strequal}{2}{%
  \ifstrequal{#1}{#2}{\def\pgfmathresult{1}}{\def\pgfmathresult{0}}%
}
\pgfmathdeclarefunction{instr}{2}{%
  \pgfutil@in@{#1}{#2}%
  \ifpgfutil@in@\def\pgfmathresult{1}\else\def\pgfmathresult{0}\fi
}
\pgfmathdeclarefunction{strcat}{...}{%
  \edef\pgfmathresult{\forest@strip@braces{#1}}%
}
\def\forest@pgfmathhelper@attribute@toks#1#2{%
  \forest@forthis{%
    \forest@nameandgo{#1}%
    \forestoget{#2}\pgfmathresult
  }%
}
\def\forest@pgfmathhelper@attribute@dimen#1#2{%
  \forest@forthis{%
    \forest@nameandgo{#1}%
    \forestoget{#2}\forest@temp
    \pgfmathparse{+\forest@temp}%
  }%
}
\def\forest@pgfmathhelper@attribute@count#1#2{%
  \forest@forthis{%
    \forest@nameandgo{#1}%
    \forestoget{#2}\forest@temp
    \pgfmathtruncatemacro\pgfmathresult{\forest@temp}%
  }%
}
\pgfmathdeclarefunction{id}{1}{%
  \forest@forthis{%
    \forest@nameandgo{#1}%
    \let\pgfmathresult\forest@cn
  }%
}
\forestset{%
  if id/.code n args={3}{%
    \ifnum#1=\forest@cn\relax
      \pgfkeysalso{#2}%
    \else
      \pgfkeysalso{#3}%
    \fi
  },
  where id/.style n args={3}{for tree={if id={#1}{#2}{#3}}}
}
%    \end{macrocode}
%
%
% \subsection{Dynamic tree}
% \label{sec:impl:dynamic}
%
%    \begin{macrocode}
\def\forest@last@node{0}
\def\forest@nodehandleby@name@nodewalk@or@bracket#1{%
  \ifx\pgfkeysnovalue#1%
    \edef\forest@last@node{\forest@node@Nametoid{forest@last@node}}%
  \else
    \forest@nodehandleby@nnb@checkfirst#1\forest@END
  \fi
}
\def\forest@nodehandleby@nnb@checkfirst#1#2\forest@END{%
  \ifx[#1%]
    \forest@create@node{#1#2}%
  \else
    \forest@forthis{%
      \forest@nameandgo{#1#2}%
      \let\forest@last@node\forest@cn
    }%
  \fi
}
\def\forest@create@node#1{% #1=bracket representation
  \bracketParse{\forest@create@collectafterthought}%
               \forest@last@node=#1\forest@end@create@node
}
\def\forest@create@collectafterthought#1\forest@end@create@node{%
  \forestOletO{\forest@last@node}{delay}{\forest@last@node}{given options}%
  \forestOset{\forest@last@node}{given options}{}%
  \forestOeappto{\forest@last@node}{delay}{,\unexpanded{#1}}%
}
\def\forest@create@collectafterthought#1\forest@end@create@node{%
  \forest@node@Foreach{\forest@last@node}{%
    \forestoleto{delay}{given options}%
    \forestoset{given options}{}%
  }%
  \forestOeappto{\forest@last@node}{delay}{,\unexpanded{#1}}%
}
\def\forest@remove@node#1{%
  \forest@node@Remove{#1}%
}
\def\forest@append@node#1#2{%
  \forest@node@Remove{#2}%
  \forest@node@Append{#1}{#2}%
}
\def\forest@prepend@node#1#2{%
  \forest@node@Remove{#2}%
  \forest@node@Prepend{#1}{#2}%
}
\def\forest@insertafter@node#1#2{%
  \forest@node@Remove{#2}%
  \forest@node@Insertafter{\forestOve{#1}{@parent}}{#2}{#1}%
}
\def\forest@insertbefore@node#1#2{%
  \forest@node@Remove{#2}%
  \forest@node@Insertbefore{\forestOve{#1}{@parent}}{#2}{#1}%
}
\def\forest@appto@do@dynamics#1#2{%
   \forest@nodehandleby@name@nodewalk@or@bracket{#2}%
   \ifcase\forest@dynamics@copyhow\relax\or
     \forest@tree@copy{\forest@last@node}\forest@last@node
   \or
     \forest@node@copy{\forest@last@node}\forest@last@node
   \fi
   \forest@node@Ifnamedefined{forest@last@node}{%
     \forestOepreto{\forest@last@node}{delay}
       {for id={\forest@node@Nametoid{forest@last@node}}{alias=forest@last@node},}%
     }{}%
   \forest@havedelayedoptionstrue
   \edef\forest@marshal{%
     \noexpand\apptotoks\noexpand\forest@do@dynamics{%
       \noexpand#1{\forest@cn}{\forest@last@node}}%
   }\forest@marshal
}
\forestset{%
  create/.code={\forest@create@node{#1}},
  append/.code={\def\forest@dynamics@copyhow{0}\forest@appto@do@dynamics\forest@append@node{#1}},
  prepend/.code={\def\forest@dynamics@copyhow{0}\forest@appto@do@dynamics\forest@prepend@node{#1}},
  insert after/.code={\def\forest@dynamics@copyhow{0}\forest@appto@do@dynamics\forest@insertafter@node{#1}},
  insert before/.code={\def\forest@dynamics@copyhow{0}\forest@appto@do@dynamics\forest@insertbefore@node{#1}},
  append'/.code={\def\forest@dynamics@copyhow{1}\forest@appto@do@dynamics\forest@append@node{#1}},
  prepend'/.code={\def\forest@dynamics@copyhow{1}\forest@appto@do@dynamics\forest@prepend@node{#1}},
  insert after'/.code={\def\forest@dynamics@copyhow{1}\forest@appto@do@dynamics\forest@insertafter@node{#1}},
  insert before'/.code={\def\forest@dynamics@copyhow{1}\forest@appto@do@dynamics\forest@insertbefore@node{#1}},
  append''/.code={\def\forest@dynamics@copyhow{2}\forest@appto@do@dynamics\forest@append@node{#1}},
  prepend''/.code={\def\forest@dynamics@copyhow{2}\forest@appto@do@dynamics\forest@prepend@node{#1}},
  insert after''/.code={\def\forest@dynamics@copyhow{2}\forest@appto@do@dynamics\forest@insertafter@node{#1}},
  insert before''/.code={\def\forest@dynamics@copyhow{2}\forest@appto@do@dynamics\forest@insertbefore@node{#1}},
  remove/.code={%
    \pgfkeysalso{alias=forest@last@node}%
    \expandafter\apptotoks\expandafter\forest@do@dynamics\expandafter{%
      \expandafter\forest@remove@node\expandafter{\forest@cn}}%
  },
  set root/.code={%
    \forest@nodehandleby@name@nodewalk@or@bracket{#1}%
    \edef\forest@marshal{%
      \noexpand\apptotoks\noexpand\forest@do@dynamics{%
        \def\noexpand\forest@root{\forest@last@node}%
      }%
    }\forest@marshal
  },
  replace by/.code={\forest@replaceby@code{#1}{insert after}},
  replace by'/.code={\forest@replaceby@code{#1}{insert after'}},
  replace by''/.code={\forest@replaceby@code{#1}{insert after''}},
}
\def\forest@replaceby@code#1#2{%#1=node spec,#2=insert after['][']
  \ifnum\forestove{@parent}=0
    \pgfkeysalso{set root={#1}}%
  \else
    \pgfkeysalso{alias=forest@last@node,#2={#1}}%
    \eapptotoks\forest@do@dynamics{%
      \noexpand\ifnum\noexpand\forestOve{\forest@cn}{@parent}=\forestove{@parent}
        \noexpand\forest@remove@node{\forest@cn}%
      \noexpand\fi
    }%
  \fi
}
%    \end{macrocode}
%
% \section{Stages}
% 
%    \begin{macrocode}
\forestset{
  stages/.style={
    process keylist=before typesetting nodes,
    typeset nodes stage,
    process keylist=before packing,
    pack stage,
    process keylist=before computing xy,
    compute xy stage,
    process keylist=before drawing tree,
    draw tree stage,
  },
  typeset nodes stage/.style={for root'=typeset nodes},
  pack stage/.style={for root'=pack},
  compute xy stage/.style={for root'=compute xy},
  draw tree stage/.style={for root'=draw tree},
  process keylist/.code={\forest@process@hook@keylist{#1}},
  declare keylist={given options}{},
  declare keylist={before typesetting nodes}{},
  declare keylist={before packing}{},
  declare keylist={before computing xy}{},
  declare keylist={before drawing tree}{},
  declare keylist={delay}{},
  delay/.append code={\forest@havedelayedoptionstrue},
  delay n/.style 2 args={if={#1==0}{#2}{delay@n={#1}{#2}}},
  delay@n/.style 2 args={
    if={#1==1}{delay={#2}}{delay={delay@n/.wrap pgfmath arg={{##1}{#2}}{#1-1}}}
  },
  if have delayed/.code 2 args={%
    \ifforest@havedelayedoptions\pgfkeysalso{#1}\else\pgfkeysalso{#2}\fi
  },
  typeset nodes/.code={%
    \forest@drawtree@preservenodeboxes@false
    \forest@node@foreach{\forest@node@typeset}},
  typeset nodes'/.code={%
    \forest@drawtree@preservenodeboxes@true
    \forest@node@foreach{\forest@node@typeset}},
  typeset node/.code={%
    \forest@drawtree@preservenodeboxes@false
    \forest@node@typeset
  },  
  pack/.code={\forest@pack},
  pack'/.code={\forest@pack@onlythisnode},
  compute xy/.code={\forest@node@computeabsolutepositions},
  draw tree box/.store in=\forest@drawtreebox,
  draw tree box,
  draw tree/.code={%
    \forest@drawtree@preservenodeboxes@false
    \forest@node@drawtree
  },
  draw tree'/.code={%
    \forest@drawtree@preservenodeboxes@true
    \forest@node@drawtree
  },
}
\newtoks\forest@do@dynamics
\newif\ifforest@havedelayedoptions
\def\forest@process@hook@keylist#1{%
  \forest@loopa
    \forest@havedelayedoptionsfalse
    \forest@do@dynamics={}%
    \forest@fornode{\forest@root}{\forest@process@hook@keylist@{#1}}%
    \expandafter\ifstrempty\expandafter{\the\forest@do@dynamics}{}{%
      \the\forest@do@dynamics
      \forest@node@Compute@numeric@ts@info{\forest@root}%
      \forest@havedelayedoptionstrue
    }%
  \ifforest@havedelayedoptions
    \forest@node@Foreach{\forest@root}{%
      \forestoget{delay}\forest@temp@delayed
      \forestolet{#1}\forest@temp@delayed
      \forestoset{delay}{}%
    }%
  \forest@repeata
}
\def\forest@process@hook@keylist@#1{%
  \forest@node@foreach{%
    \forestoget{#1}\forest@temp@keys
    \ifdefvoid\forest@temp@keys{}{%
      \forestoset{#1}{}%
      \expandafter\forestset\expandafter{\forest@temp@keys}%
    }%
  }%
}
%    \end{macrocode}
%
%
% \subsection{Typesetting nodes}
%
%    \begin{macrocode}
\def\forest@node@typeset{%
  \let\forest@next\forest@node@typeset@
  \forestoifdefined{box}{%
    \ifforest@drawtree@preservenodeboxes@
      \let\forest@next\relax
    \fi
  }{%
    \locbox\forest@temp@box
    \forestolet{box}\forest@temp@box
  }%
  \def\forest@node@typeset@restore{}%
  \ifdefined\ifsa@tikz\forest@standalone@hack\fi
  \forest@next
  \forest@node@typeset@restore
}
\def\forest@standalone@hack{%
  \ifsa@tikz
    \let\forest@standalone@tikzpicture\tikzpicture
    \let\forest@standalone@endtikzpicture\endtikzpicture
    \let\tikzpicture\sa@orig@tikzpicture
    \let\endtikzpicture\sa@orig@endtikzpicture
    \def\forest@node@typeset@restore{%
      \let\tikzpicture\forest@standalone@tikzpicture
      \let\endtikzpicture\forest@standalone@endtikzpicture
    }%
  \fi
}
\newbox\forest@box
\def\forest@node@typeset@{%
  \forestoget{name}\forest@nodename
  \edef\forest@temp@nodeformat{\forestove{node format}}%
  \gdef\forest@smuggle{}%
  \setbox0=\hbox{%
    \begin{tikzpicture}%
      \pgfpositionnodelater{\forest@positionnodelater@save}%
      \forest@temp@nodeformat
      \pgfinterruptpath
      \pgfpointanchor{\forest@pgf@notyetpositioned\forest@nodename}{forestcomputenodeboundary}%
      \endpgfinterruptpath
      %\forest@compute@node@boundary\forest@temp
      %\xappto\forest@smuggle{\noexpand\forestoset{boundary}{\expandonce\forest@temp}}%
      \if\relax\forestove{parent anchor}\relax
        \pgfpointanchor{\forest@pgf@notyetpositioned\forest@nodename}{center}%
      \else
        \pgfpointanchor{\forest@pgf@notyetpositioned\forest@nodename}{\forestove{parent anchor}}%
      \fi
      \xappto\forest@smuggle{%
        \noexpand\forestoset{parent@anchor}{%
          \noexpand\noexpand\noexpand\pgf@x=\the\pgf@x\relax
          \noexpand\noexpand\noexpand\pgf@y=\the\pgf@y\relax}}%
      \if\relax\forestove{child anchor}\relax
        \pgfpointanchor{\forest@pgf@notyetpositioned\forest@nodename}{center}%
      \else      
        \pgfpointanchor{\forest@pgf@notyetpositioned\forest@nodename}{\forestove{child anchor}}%
      \fi
      \xappto\forest@smuggle{%
        \noexpand\forestoeset{child@anchor}{%
          \noexpand\noexpand\noexpand\pgf@x=\the\pgf@x\relax
          \noexpand\noexpand\noexpand\pgf@y=\the\pgf@y\relax}}%
      \if\relax\forestove{anchor}\relax
        \pgfpointanchor{\forest@pgf@notyetpositioned\forest@nodename}{center}%
      \else      
        \pgfpointanchor{\forest@pgf@notyetpositioned\forest@nodename}{\forestove{anchor}}%
      \fi
      \xappto\forest@smuggle{%
        \noexpand\forestoeset{@anchor}{%
          \noexpand\noexpand\noexpand\pgf@x=\the\pgf@x\relax
          \noexpand\noexpand\noexpand\pgf@y=\the\pgf@y\relax}}%
    \end{tikzpicture}%
  }%
  \setbox\forestove{box}=\box\forest@box % smuggle the box
  \forestolet{boundary}\forest@global@boundary
  \forest@smuggle % ... and the rest
}
\forestset{
  declare readonly dimen={min x},
  declare readonly dimen={min y},
  declare readonly dimen={max x},
  declare readonly dimen={max y},
}
\def\forest@patch@enormouscoordinateboxbounds@plus#1{%
  \expandafter\ifstrequal\expandafter{#1}{16000.0pt}{\def#1{0.0pt}}{}%
}
\def\forest@patch@enormouscoordinateboxbounds@minus#1{%
  \expandafter\ifstrequal\expandafter{#1}{-16000.0pt}{\def#1{0.0pt}}{}%
}
\def\forest@positionnodelater@save{%
  \global\setbox\forest@box=\box\pgfpositionnodelaterbox
  \xappto\forest@smuggle{\noexpand\forestoset{later@name}{\pgfpositionnodelatername}}%
  % a bug in pgf? ---well, here's a patch
  \forest@patch@enormouscoordinateboxbounds@plus\pgfpositionnodelaterminx
  \forest@patch@enormouscoordinateboxbounds@plus\pgfpositionnodelaterminy
  \forest@patch@enormouscoordinateboxbounds@minus\pgfpositionnodelatermaxx
  \forest@patch@enormouscoordinateboxbounds@minus\pgfpositionnodelatermaxy
  % end of patch
  \xappto\forest@smuggle{\noexpand\forestoset{min x}{\pgfpositionnodelaterminx}}%
  \xappto\forest@smuggle{\noexpand\forestoset{min y}{\pgfpositionnodelaterminy}}%
  \xappto\forest@smuggle{\noexpand\forestoset{max x}{\pgfpositionnodelatermaxx}}%
  \xappto\forest@smuggle{\noexpand\forestoset{max y}{\pgfpositionnodelatermaxy}}%
}
\def\forest@node@forest@positionnodelater@restore{%
  \ifforest@drawtree@preservenodeboxes@
    \let\forest@boxorcopy\copy
  \else
    \let\forest@boxorcopy\box
  \fi
  \forestoget{box}\forest@temp
  \setbox\pgfpositionnodelaterbox=\forest@boxorcopy\forest@temp
  \edef\pgfpositionnodelatername{\forestove{later@name}}%
  \edef\pgfpositionnodelaterminx{\forestove{min x}}%
  \edef\pgfpositionnodelaterminy{\forestove{min y}}%
  \edef\pgfpositionnodelatermaxx{\forestove{max x}}%
  \edef\pgfpositionnodelatermaxy{\forestove{max y}}%
}
%    \end{macrocode}
%
% \subsection{Packing}
% \label{imp:packing}
%
% Method |pack| should be called to calculate the positions of
% descendant nodes; the positions are stored in attributes |l| and |s|
% of these nodes, in a level/sibling coordinate system with origin at
% the parent's anchor.  
%    \begin{macrocode}
\def\forest@pack{%
  \forest@pack@computetiers
  \forest@pack@computegrowthuniformity
  \forest@@pack
}
\def\forest@@pack{%
  \ifnum\forestove{n children}>0
    \ifnum\forestove{uniform growth}>0
      \forest@pack@level@uniform
      \forest@pack@aligntiers@ofsubtree
      \forest@pack@sibling@uniform@recursive
    \else
      \forest@node@foreachchild{\forest@@pack}%
      \forest@pack@level@nonuniform
      \forest@pack@aligntiers
      \forest@pack@sibling@uniform@applyreversed
    \fi
  \fi
}
\def\forest@pack@onlythisnode{%
  \ifnum\forestove{n children}>0
    \forest@pack@computetiers
      \forest@pack@level@nonuniform
      \forest@pack@aligntiers
      \forest@pack@sibling@uniform@applyreversed  
  \fi  
}
%    \end{macrocode}
%    
% Compute growth uniformity for the subtree.  A tree grows uniformly is all its branching nodes have
% the same |grow|. 
%    \begin{macrocode}
\def\forest@pack@computegrowthuniformity{%
  \forest@node@foreachchild{\forest@pack@computegrowthuniformity}%
  \edef\forest@pack@cgu@uniformity{%
    \ifnum\forestove{n children}=0
    2\else 1\fi
  }%
  \forestoget{grow}\forest@pack@cgu@parentgrow
  \forest@node@foreachchild{%
    \ifnum\forestove{uniform growth}=0
      \def\forest@pack@cgu@uniformity{0}%
    \else
      \ifnum\forestove{uniform growth}=1
        \ifnum\forestove{grow}=\forest@pack@cgu@parentgrow\relax\else
          \def\forest@pack@cgu@uniformity{0}%
        \fi
      \fi
    \fi
  }%
  \forestolet{uniform growth}\forest@pack@cgu@uniformity
}
%    \end{macrocode}
%    
% Pack children in the level dimension in a uniform tree.
%    \begin{macrocode}
\def\forest@pack@level@uniform{%
  \let\forest@plu@minchildl\relax
  \forestoget{grow}\forest@plu@grow
  \forest@node@foreachchild{%
    \forest@node@getboundingrectangle@ls{\forest@plu@grow}%
    \advance\pgf@xa\forestove{l}\relax
    \ifx\forest@plu@minchildl\relax
      \edef\forest@plu@minchildl{\the\pgf@xa}%
    \else
      \ifdim\pgf@xa<\forest@plu@minchildl\relax
        \edef\forest@plu@minchildl{\the\pgf@xa}%
      \fi
    \fi
  }%
  \forest@node@getboundingrectangle@ls{\forest@plu@grow}%
  \pgfutil@tempdima=\pgf@xb\relax
  \advance\pgfutil@tempdima -\forest@plu@minchildl\relax
  \advance\pgfutil@tempdima \forestove{l sep}\relax
  \ifdim\pgfutil@tempdima>0pt
    \forest@node@foreachchild{%
      \forestoeset{l}{\the\dimexpr\forestove{l}+\the\pgfutil@tempdima}%
    }%
  \fi
  \forest@node@foreachchild{%
    \ifnum\forestove{n children}>0
      \forest@pack@level@uniform
    \fi
  }%
}
%    \end{macrocode}
% 
% Pack children in the level dimension in a non-uniform tree. (Expects
% the children to be fully packed.)
%    \begin{macrocode}
\def\forest@pack@level@nonuniform{%
  \let\forest@plu@minchildl\relax
  \forestoget{grow}\forest@plu@grow
  \forest@node@foreachchild{%
    \forest@node@getedge{negative}{\forest@plu@grow}{\forest@plnu@negativechildedge}%
    \forest@node@getedge{positive}{\forest@plu@grow}{\forest@plnu@positivechildedge}%
    \def\forest@plnu@childedge{\forest@plnu@negativechildedge\forest@plnu@positivechildedge}%
    \forest@path@getboundingrectangle@ls\forest@plnu@childedge{\forest@plu@grow}%
    \advance\pgf@xa\forestove{l}\relax
    \ifx\forest@plu@minchildl\relax
      \edef\forest@plu@minchildl{\the\pgf@xa}%
    \else
      \ifdim\pgf@xa<\forest@plu@minchildl\relax
        \edef\forest@plu@minchildl{\the\pgf@xa}%
      \fi
    \fi
  }%
  \forest@node@getboundingrectangle@ls{\forest@plu@grow}%
  \pgfutil@tempdima=\pgf@xb\relax
  \advance\pgfutil@tempdima -\forest@plu@minchildl\relax
  \advance\pgfutil@tempdima \forestove{l sep}\relax
  \ifdim\pgfutil@tempdima>0pt
    \forest@node@foreachchild{%
      \forestoeset{l}{\the\dimexpr\the\pgfutil@tempdima+\forestove{l}}%
    }%
  \fi
}
%    \end{macrocode}
%
% Align tiers.
%    \begin{macrocode}
\def\forest@pack@aligntiers{%
  \forestoget{grow}\forest@temp@parentgrow
  \forestoget{@tiers}\forest@temp@tiers
  \forlistloop\forest@pack@aligntier@\forest@temp@tiers
}
\def\forest@pack@aligntiers@ofsubtree{%
  \forest@node@foreach{\forest@pack@aligntiers}%
}
\def\forest@pack@aligntiers@computeabsl{%
  \forestoleto{abs@l}{l}%
  \forest@node@foreachdescendant{\forest@pack@aligntiers@computeabsl@}%
}
\def\forest@pack@aligntiers@computeabsl@{%
  \forestoeset{abs@l}{\the\dimexpr\forestove{l}+\forestOve{\forestove{@parent}}{abs@l}}%
}
\def\forest@pack@aligntier@#1{%
  \forest@pack@aligntiers@computeabsl
  \pgfutil@tempdima=-\maxdimen\relax
  \def\forest@temp@currenttier{#1}%
  \forest@node@foreach{%
    \forestoget{tier}\forest@temp@tier
    \ifx\forest@temp@currenttier\forest@temp@tier
      \ifdim\pgfutil@tempdima<\forestove{abs@l}\relax
        \pgfutil@tempdima=\forestove{abs@l}\relax
      \fi
    \fi
  }%
  \ifdim\pgfutil@tempdima=-\maxdimen\relax\else
    \forest@node@foreach{%
      \forestoget{tier}\forest@temp@tier
      \ifx\forest@temp@currenttier\forest@temp@tier
        \forestoeset{l}{\the\dimexpr\pgfutil@tempdima-\forestove{abs@l}+\forestove{l}}%
      \fi
    }%
  \fi
}
%    \end{macrocode}
% Pack children in the sibling dimension in a uniform tree:
% recursion.
%    \begin{macrocode}
\def\forest@pack@sibling@uniform@recursive{%
  \forest@node@foreachchild{\forest@pack@sibling@uniform@recursive}%
  \forest@pack@sibling@uniform@applyreversed
}
%    \end{macrocode}
% Pack children in the sibling dimension in a uniform tree: applyreversed.
%    \begin{macrocode}
\def\forest@pack@sibling@uniform@applyreversed{%
  \ifnum\forestove{n children}>1
    \ifnum\forestove{reversed}=0
      \pack@sibling@uniform@main{first}{last}{next}{previous}%
    \else
      \pack@sibling@uniform@main{last}{first}{previous}{next}%
    \fi
  \fi
}
%    \end{macrocode}
% Pack children in the sibling dimension in a uniform tree: the main
% routine.
%    \begin{macrocode}
\def\pack@sibling@uniform@main#1#2#3#4{%
%    \end{macrocode}
% Loop through the children. At each iteration, we compute the
% distance between the negative edge of the current child and the
% positive edge of the block of the previous children, and then set
% the |s| attribute of the current child accordingly.
%
% We start the loop with the second (to last) child, having
% initialized the positive edge of the previous children to the
% positive edge of the first child.
%    \begin{macrocode}
  \forestoget{@#1}\forest@child
  \edef\forest@temp{%
    \noexpand\forest@fornode{\forestove{@#1}}{%
      \noexpand\forest@node@getedge
        {positive}
        {\forestove{grow}}
        \noexpand\forest@temp@edge
    }%
  }\forest@temp
  \forest@pack@pgfpoint@childsposition\forest@child
  \let\forest@previous@positive@edge\pgfutil@empty
  \forest@extendpath\forest@previous@positive@edge\forest@temp@edge{}%
  \forestOget{\forest@child}{@#3}\forest@child
%    \end{macrocode}
% Loop until the current child is the null node.
%    \begin{macrocode}
  \edef\forest@previous@child@s{0pt}%
  \forest@loopb
  \unless\ifnum\forest@child=0
%    \end{macrocode}
% Get the negative edge of the child.
%    \begin{macrocode}
    \edef\forest@temp{%
      \noexpand\forest@fornode{\forest@child}{%
        \noexpand\forest@node@getedge
          {negative}
          {\forestove{grow}}
          \noexpand\forest@temp@edge
        }%
    }\forest@temp
%    \end{macrocode}
% Set |\pgf@x| and |\pgf@y| to the position of the child (in the
% coordinate system of this node).
%    \begin{macrocode}
    \forest@pack@pgfpoint@childsposition\forest@child
%    \end{macrocode}
% Translate the edge of the child by the child's position.
%    \begin{macrocode}
    \let\forest@child@negative@edge\pgfutil@empty
    \forest@extendpath\forest@child@negative@edge\forest@temp@edge{}%
%    \end{macrocode}
% Setup the grow line: the angle is given by this node's |grow|
% attribute. 
%    \begin{macrocode}
    \forest@setupgrowline{\forestove{grow}}%
%    \end{macrocode}
% Get the distance (wrt the grow line) between the positive edge of
% the previous children and the negative edge of the current
% child. (The distance can be negative!)
%    \begin{macrocode}
    \forest@distance@between@edge@paths\forest@previous@positive@edge\forest@child@negative@edge\forest@csdistance
%    \end{macrocode}
% If the distance is |\relax|, the projections of the edges onto the
% grow line don't overlap: do nothing. Otherwise, shift the current child so that its distance to the block
% of previous children is |s_sep|.
%    \begin{macrocode}
    \ifx\forest@csdistance\relax
      %\forestOeset{\forest@child}{s}{\forest@previous@child@s}%
    \else
      \advance\pgfutil@tempdimb-\forest@csdistance\relax
      \advance\pgfutil@tempdimb\forestove{s sep}\relax
      \forestOeset{\forest@child}{s}{\the\dimexpr\forestove{s}-\forest@csdistance+\forestove{s sep}}%
    \fi
%    \end{macrocode}
% Retain monotonicity (is this ok?).  (This problem arises when the adjacent children's |l| are too
% far apart.) 
%    \begin{macrocode}
    \ifdim\forestOve{\forest@child}{s}<\forest@previous@child@s\relax
      \forestOeset{\forest@child}{s}{\forest@previous@child@s}%
    \fi
%    \end{macrocode}
% Prepare for the next iteration: add the current child's positive
% edge to the positive edge of the previous children, and set up the
% next current child.
%    \begin{macrocode}
    \forestOget{\forest@child}{s}\forest@child@s
    \edef\forest@previous@child@s{\forest@child@s}%
    \edef\forest@temp{%
      \noexpand\forest@fornode{\forest@child}{%
        \noexpand\forest@node@getedge
          {positive}
          {\forestove{grow}}
          \noexpand\forest@temp@edge
      }%
    }\forest@temp
    \forest@pack@pgfpoint@childsposition\forest@child
    \forest@extendpath\forest@previous@positive@edge\forest@temp@edge{}%
    \forest@getpositivetightedgeofpath\forest@previous@positive@edge\forest@previous@positive@edge
    \forestOget{\forest@child}{@#3}\forest@child
  \forest@repeatb
%    \end{macrocode}
% Shift the position of all children to achieve the desired alignment
% of the parent and its children.
%    \begin{macrocode}
  \csname forest@calign@\forestove{calign}\endcsname
}
%    \end{macrocode}
% Get the position of child |#1| in the current node, in node's l-s
% coordinate system.
%    \begin{macrocode}
\def\forest@pack@pgfpoint@childsposition#1{%
  {%
    \pgftransformreset
    \pgftransformrotate{\forestove{grow}}%
    \forest@fornode{#1}{%
      \pgfpointtransformed{\pgfqpoint{\forestove{l}}{\forestove{s}}}%
    }%
  }%
}
%    \end{macrocode}
% Get the position of the node in the grow (|#1|)-rotated coordinate
% system.
%    \begin{macrocode}
\def\forest@pack@pgfpoint@positioningrow#1{%
  {%
    \pgftransformreset
    \pgftransformrotate{#1}%
    \pgfpointtransformed{\pgfqpoint{\forestove{l}}{\forestove{s}}}%
  }%
}
%    \end{macrocode}
%    
% Child alignment.
%    \begin{macrocode}
\def\forest@calign@s@shift#1{%
  \pgfutil@tempdima=#1\relax
  \forest@node@foreachchild{%
    \forestoeset{s}{\the\dimexpr\forestove{s}+\pgfutil@tempdima}%
  }%
}
\def\forest@calign@child{%
  \forest@calign@s@shift{-\forestOve{\forest@node@nornbarthchildid{\forestove{calign primary child}}}{s}}%
}
\csdef{forest@calign@child edge}{%
  {%
    \edef\forest@temp@child{\forest@node@nornbarthchildid{\forestove{calign primary child}}}%
    \pgftransformreset
    \pgftransformrotate{\forestove{grow}}%
    \pgfpointtransformed{\pgfqpoint{\forestOve{\forest@temp@child}{l}}{\forestOve{\forest@temp@child}{s}}}%
    \pgf@xa=\pgf@x\relax\pgf@ya=\pgf@y\relax
    \forestOve{\forest@temp@child}{child@anchor}%
    \advance\pgf@xa\pgf@x\relax\advance\pgf@ya\pgf@y\relax
    \forestove{parent@anchor}%
    \advance\pgf@xa-\pgf@x\relax\advance\pgf@ya-\pgf@y\relax
    \edef\forest@marshal{%
      \noexpand\pgftransformreset
      \noexpand\pgftransformrotate{-\forestove{grow}}%
      \noexpand\pgfpointtransformed{\noexpand\pgfqpoint{\the\pgf@xa}{\the\pgf@ya}}%
    }\forest@marshal
  }%
  \forest@calign@s@shift{\the\dimexpr-\the\pgf@y}%
}
\csdef{forest@calign@midpoint}{%
  \forest@calign@s@shift{\the\dimexpr 0pt -%
    (\forestOve{\forest@node@nornbarthchildid{\forestove{calign primary child}}}{s}%
    +\forestOve{\forest@node@nornbarthchildid{\forestove{calign secondary child}}}{s}%
    )/2\relax
  }%
}
\csdef{forest@calign@edge midpoint}{%
  {%
    \edef\forest@temp@firstchild{\forest@node@nornbarthchildid{\forestove{calign primary child}}}%
    \edef\forest@temp@secondchild{\forest@node@nornbarthchildid{\forestove{calign secondary child}}}%
    \pgftransformreset
    \pgftransformrotate{\forestove{grow}}%
    \pgfpointtransformed{\pgfqpoint{\forestOve{\forest@temp@firstchild}{l}}{\forestOve{\forest@temp@firstchild}{s}}}%
    \pgf@xa=\pgf@x\relax\pgf@ya=\pgf@y\relax
    \forestOve{\forest@temp@firstchild}{child@anchor}%
    \advance\pgf@xa\pgf@x\relax\advance\pgf@ya\pgf@y\relax
    \edef\forest@marshal{%
      \noexpand\pgfpointtransformed{\noexpand\pgfqpoint{\forestOve{\forest@temp@secondchild}{l}}{\forestOve{\forest@temp@secondchild}{s}}}%
    }\forest@marshal
    \advance\pgf@xa\pgf@x\relax\advance\pgf@ya\pgf@y\relax
    \forestOve{\forest@temp@secondchild}{child@anchor}%
    \advance\pgf@xa\pgf@x\relax\advance\pgf@ya\pgf@y\relax
    \divide\pgf@xa2 \divide\pgf@ya2
    \edef\forest@marshal{%
      \noexpand\pgftransformreset
      \noexpand\pgftransformrotate{-\forestove{grow}}%
      \noexpand\pgfpointtransformed{\noexpand\pgfqpoint{\the\pgf@xa}{\the\pgf@ya}}%
    }\forest@marshal
  }%
  \forest@calign@s@shift{\the\dimexpr-\the\pgf@y}%
}
%    \end{macrocode}
% Aligns the children to the center of the angles given by the options
% |calign_first_angle| and |calign_second_angle| and spreads them additionally if needed to fill the
% whole 
% space determined by the option.  The version |fixed_angles| calculates the
% angles between node anchors; the version |fixes_edge_angles| calculates the angles between the
% node edges. 
%    \begin{macrocode}
\csdef{forest@calign@fixed angles}{%
  \edef\forest@ca@first@child{\forest@node@nornbarthchildid{\forestove{calign primary child}}}%
  \edef\forest@ca@second@child{\forest@node@nornbarthchildid{\forestove{calign secondary child}}}%
  \ifnum\forestove{reversed}=1
    \let\forest@temp\forest@ca@first@child
    \let\forest@ca@first@child\forest@ca@second@child
    \let\forest@ca@second@child\forest@temp
  \fi
  \forestOget{\forest@ca@first@child}{l}\forest@ca@first@l
  \forestOget{\forest@ca@second@child}{l}\forest@ca@second@l
  \pgfmathsetlengthmacro\forest@ca@desired@s@distance{%
    tan(\forestove{calign secondary angle})*\forest@ca@second@l
    -tan(\forestove{calign primary angle})*\forest@ca@first@l
  }%
  \forestOget{\forest@ca@first@child}{s}\forest@ca@first@s
  \forestOget{\forest@ca@second@child}{s}\forest@ca@second@s
  \pgfmathsetlengthmacro\forest@ca@actual@s@distance{%
    \forest@ca@second@s-\forest@ca@first@s}%
  \ifdim\forest@ca@desired@s@distance>\forest@ca@actual@s@distance\relax
    \ifdim\forest@ca@actual@s@distance=0pt
      \pgfmathsetlength\pgfutil@tempdima{tan(\forestove{calign primary angle})*\forest@ca@second@l}%
      \pgfmathsetlength\pgfutil@tempdimb{\forest@ca@desired@s@distance/(\forestove{n children}-1)}%
      \forest@node@foreachchild{%
        \forestoeset{s}{\the\pgfutil@tempdima}%
        \advance\pgfutil@tempdima\pgfutil@tempdimb
      }%
      \def\forest@calign@anchor{0pt}%
    \else
      \pgfmathsetmacro\forest@ca@ratio{%
        \forest@ca@desired@s@distance/\forest@ca@actual@s@distance}%
      \forest@node@foreachchild{%
        \pgfmathsetlengthmacro\forest@temp{\forest@ca@ratio*\forestove{s}}%
        \forestolet{s}\forest@temp
      }%
      \pgfmathsetlengthmacro\forest@calign@anchor{%
        -tan(\forestove{calign primary angle})*\forest@ca@first@l}%
    \fi
  \else
    \ifdim\forest@ca@desired@s@distance<\forest@ca@actual@s@distance\relax
      \pgfmathsetlengthmacro\forest@ca@ratio{%
        \forest@ca@actual@s@distance/\forest@ca@desired@s@distance}%
      \forest@node@foreachchild{%
        \pgfmathsetlengthmacro\forest@temp{\forest@ca@ratio*\forestove{l}}%
        \forestolet{l}\forest@temp
      }%
      \forestOget{\forest@ca@first@child}{l}\forest@ca@first@l
      \pgfmathsetlengthmacro\forest@calign@anchor{%
        -tan(\forestove{calign primary angle})*\forest@ca@first@l}%
    \fi
  \fi
  \forest@calign@s@shift{-\forest@calign@anchor}%
}
\csdef{forest@calign@fixed edge angles}{%
  \edef\forest@ca@first@child{\forest@node@nornbarthchildid{\forestove{calign primary child}}}%
  \edef\forest@ca@second@child{\forest@node@nornbarthchildid{\forestove{calign secondary child}}}%
  \ifnum\forestove{reversed}=1
    \let\forest@temp\forest@ca@first@child
    \let\forest@ca@first@child\forest@ca@second@child
    \let\forest@ca@second@child\forest@temp
  \fi
  \forestOget{\forest@ca@first@child}{l}\forest@ca@first@l
  \forestOget{\forest@ca@second@child}{l}\forest@ca@second@l
  \forestoget{parent@anchor}\forest@ca@parent@anchor
  \forest@ca@parent@anchor
  \edef\forest@ca@parent@anchor@s{\the\pgf@x}%
  \edef\forest@ca@parent@anchor@l{\the\pgf@y}%
  \forestOget{\forest@ca@first@child}{child@anchor}\forest@ca@first@child@anchor
  \forest@ca@first@child@anchor
  \edef\forest@ca@first@child@anchor@s{\the\pgf@x}%
  \edef\forest@ca@first@child@anchor@l{\the\pgf@y}%
  \forestOget{\forest@ca@second@child}{child@anchor}\forest@ca@second@child@anchor
  \forest@ca@second@child@anchor
  \edef\forest@ca@second@child@anchor@s{\the\pgf@x}%
  \edef\forest@ca@second@child@anchor@l{\the\pgf@y}%
  \pgfmathsetlengthmacro\forest@ca@desired@second@edge@s{tan(\forestove{calign secondary angle})*%
    (\forest@ca@second@l-\forest@ca@second@child@anchor@l+\forest@ca@parent@anchor@l)}%
  \pgfmathsetlengthmacro\forest@ca@desired@first@edge@s{tan(\forestove{calign primary angle})*%
    (\forest@ca@first@l-\forest@ca@first@child@anchor@l+\forest@ca@parent@anchor@l)}
  \pgfmathsetlengthmacro\forest@ca@desired@s@distance{\forest@ca@desired@second@edge@s-\forest@ca@desired@first@edge@s}%
  \forestOget{\forest@ca@first@child}{s}\forest@ca@first@s
  \forestOget{\forest@ca@second@child}{s}\forest@ca@second@s
  \pgfmathsetlengthmacro\forest@ca@actual@s@distance{%
    \forest@ca@second@s+\forest@ca@second@child@anchor@s
    -\forest@ca@first@s-\forest@ca@first@child@anchor@s}%
  \ifdim\forest@ca@desired@s@distance>\forest@ca@actual@s@distance\relax
    \ifdim\forest@ca@actual@s@distance=0pt
      \forestoget{n children}\forest@temp@n@children
      \forest@node@foreachchild{%
        \forestoget{child@anchor}\forest@temp@child@anchor
        \forest@temp@child@anchor
        \edef\forest@temp@child@anchor@s{\the\pgf@x}%
        \pgfmathsetlengthmacro\forest@temp{%
          \forest@ca@desired@first@edge@s+(\forestove{n}-1)*\forest@ca@desired@s@distance/(\forest@temp@n@children-1)+\forest@ca@first@child@anchor@s-\forest@temp@child@anchor@s}%
        \forestolet{s}\forest@temp
      }%
      \def\forest@calign@anchor{0pt}%
    \else
      \pgfmathsetmacro\forest@ca@ratio{%
        \forest@ca@desired@s@distance/\forest@ca@actual@s@distance}%
      \forest@node@foreachchild{%
        \forestoget{child@anchor}\forest@temp@child@anchor
        \forest@temp@child@anchor
        \edef\forest@temp@child@anchor@s{\the\pgf@x}%
        \pgfmathsetlengthmacro\forest@temp{%
          \forest@ca@ratio*(%
            \forestove{s}-\forest@ca@first@s
            +\forest@temp@child@anchor@s-\forest@ca@first@child@anchor@s)%
          +\forest@ca@first@s
          +\forest@ca@first@child@anchor@s-\forest@temp@child@anchor@s}%
        \forestolet{s}\forest@temp
      }%
      \pgfmathsetlengthmacro\forest@calign@anchor{%
        -tan(\forestove{calign primary angle})*(\forest@ca@first@l-\forest@ca@first@child@anchor@l+\forest@ca@parent@anchor@l)%
        +\forest@ca@first@child@anchor@s-\forest@ca@parent@anchor@s
      }%
    \fi
  \else
    \ifdim\forest@ca@desired@s@distance<\forest@ca@actual@s@distance\relax
      \pgfmathsetlengthmacro\forest@ca@ratio{%
        \forest@ca@actual@s@distance/\forest@ca@desired@s@distance}%
      \forest@node@foreachchild{%
        \forestoget{child@anchor}\forest@temp@child@anchor
        \forest@temp@child@anchor
        \edef\forest@temp@child@anchor@l{\the\pgf@y}%
        \pgfmathsetlengthmacro\forest@temp{%
          \forest@ca@ratio*(%
            \forestove{l}+\forest@ca@parent@anchor@l-\forest@temp@child@anchor@l)
            -\forest@ca@parent@anchor@l+\forest@temp@child@anchor@l}%
        \forestolet{l}\forest@temp
      }%
      \forestOget{\forest@ca@first@child}{l}\forest@ca@first@l
      \pgfmathsetlengthmacro\forest@calign@anchor{%
        -tan(\forestove{calign primary angle})*(\forest@ca@first@l+\forest@ca@parent@anchor@l-\forest@temp@child@anchor@l)%
        +\forest@ca@first@child@anchor@s-\forest@ca@parent@anchor@s
      }%
    \fi
  \fi
  \forest@calign@s@shift{-\forest@calign@anchor}%
}
%    \end{macrocode}
%
% Get edge: |#1| = |positive|/|negative|, |#2| = grow (in degrees), |#3| = the control
% sequence receiving the resulting path.  The edge is taken from the
% cache (attribute |#1@edge@#2|) if possible; otherwise, both
% positive and negative edge are computed and stored in the cache.
%    \begin{macrocode}
\def\forest@node@getedge#1#2#3{%
  \forestoget{#1@edge@#2}#3%
  \ifx#3\relax
    \forest@node@foreachchild{%
      \forest@node@getedge{#1}{#2}{\forest@temp@edge}%
    }%
    \forest@forthis{\forest@node@getedges{#2}}%
    \forestoget{#1@edge@#2}#3%
  \fi
}
%    \end{macrocode}
% Get edges. |#1| = grow (in degrees).  The result is stored in
% attributes |negative@edge@#1| and |positive@edge@#1|.  This method
% expects that the children's edges are already cached.
%    \begin{macrocode}
\def\forest@node@getedges#1{%
%    \end{macrocode}
% Run the computation in a \TeX\ group.
%    \begin{macrocode}
  %{%
%    \end{macrocode}
% Setup the grow line.
%    \begin{macrocode}
    \forest@setupgrowline{#1}%
%    \end{macrocode}
% Get the edge of the node itself.
%    \begin{macrocode}
    \ifnum\forestove{ignore}=0
      \forestoget{boundary}\forest@node@boundary
    \else
      \def\forest@node@boundary{}%
    \fi
    \csname forest@getboth\forestove{fit}edgesofpath\endcsname
        \forest@node@boundary\forest@negative@node@edge\forest@positive@node@edge
    \forestolet{negative@edge@#1}\forest@negative@node@edge
    \forestolet{positive@edge@#1}\forest@positive@node@edge
%    \end{macrocode}
% Add the edges of the children.
%    \begin{macrocode}
    \get@edges@merge{negative}{#1}%
    \get@edges@merge{positive}{#1}%
  %}%
}
%    \end{macrocode}
% Merge the |#1| (=|negative| or |positive|) edge of the node with
% |#1| edges of the children. |#2| = grow angle.
%    \begin{macrocode}
\def\get@edges@merge#1#2{%
  \ifnum\forestove{n children}>0
    \forestoget{#1@edge@#2}\forest@node@edge
%    \end{macrocode}
% Remember the node's |parent anchor| and add it to the path (for breaking).
%    \begin{macrocode}
    \forestove{parent@anchor}%
    \edef\forest@getedge@pa@l{\the\pgf@x}%
    \edef\forest@getedge@pa@s{\the\pgf@y}%
    \eappto\forest@node@edge{\noexpand\pgfsyssoftpath@movetotoken{\forest@getedge@pa@l}{\forest@getedge@pa@s}}%
%    \end{macrocode}
% Switch to this node's |(l,s)| coordinate system (origin at the
% node's anchor).
%    \begin{macrocode}
    \pgftransformreset
    \pgftransformrotate{\forestove{grow}}%
%    \end{macrocode}
% Get the child's (cached) edge, translate it by the child's position,
% and add it to the path holding all edges.  Also add the edge from parent to the child to the path.
% This gets complicated when the child and/or parent anchor is empty, i.e.\ automatic border: we can
% get self-intersecting paths.  So we store all the parent--child edges to a safe place first,
% compute all the possible breaking points (i.e.\ all the points in node@edge path), and break the
% parent--child edges on these points.
%    \begin{macrocode}
    \def\forest@all@edges{}%
    \forest@node@foreachchild{%
      \forestoget{#1@edge@#2}\forest@temp@edge
      \pgfpointtransformed{\pgfqpoint{\forestove{l}}{\forestove{s}}}%
      \forest@extendpath\forest@node@edge\forest@temp@edge{}%
      \ifnum\forestove{ignore edge}=0
        \pgfpointadd
          {\pgfpointtransformed{\pgfqpoint{\forestove{l}}{\forestove{s}}}}%
          {\forestove{child@anchor}}%
        \pgfgetlastxy{\forest@getedge@ca@l}{\forest@getedge@ca@s}%
        \eappto\forest@all@edges{%
          \noexpand\pgfsyssoftpath@movetotoken{\forest@getedge@pa@l}{\forest@getedge@pa@s}%
          \noexpand\pgfsyssoftpath@linetotoken{\forest@getedge@ca@l}{\forest@getedge@ca@s}%
        }%
        % this deals with potential overlap of the edges:
        \eappto\forest@node@edge{\noexpand\pgfsyssoftpath@movetotoken{\forest@getedge@ca@l}{\forest@getedge@ca@s}}%
      \fi
    }%
    \ifdefempty{\forest@all@edges}{}{%
      \pgfintersectionofpaths{\pgfsetpath\forest@all@edges}{\pgfsetpath\forest@node@edge}%
      \def\forest@edgenode@intersections{}%
      \forest@merge@intersectionloop
      \eappto\forest@node@edge{\expandonce{\forest@all@edges}\expandonce{\forest@edgenode@intersections}}%
    }%
%    \end{macrocode}
% Process the path into an edge and store the edge.
%    \begin{macrocode}
    \csname forest@get#1\forestove{fit}edgeofpath\endcsname\forest@node@edge\forest@node@edge
    \forestolet{#1@edge@#2}\forest@node@edge
  \fi
}
\newloop\forest@merge@loop
\def\forest@merge@intersectionloop{%
  \c@pgf@counta=0
  \forest@merge@loop
  \ifnum\c@pgf@counta<\pgfintersectionsolutions\relax
    \advance\c@pgf@counta1
    \pgfpointintersectionsolution{\the\c@pgf@counta}%
    \eappto\forest@edgenode@intersections{\noexpand\pgfsyssoftpath@movetotoken
      {\the\pgf@x}{\the\pgf@y}}%
  \forest@merge@repeat
}
%    \end{macrocode}
% 
% Get the bounding rectangle of the node (without descendants). |#1| =
% grow.
%    \begin{macrocode}
\def\forest@node@getboundingrectangle@ls#1{%
  \forestoget{boundary}\forest@node@boundary
  \forest@path@getboundingrectangle@ls\forest@node@boundary{#1}%
}
%    \end{macrocode}
%
% Applies the current coordinate transformation to the points in the
% path |#1|. Returns via the current path (so that the coordinate
% transformation can be set up as local).
%    \begin{macrocode}
\def\forest@pgfpathtransformed#1{%
  \forest@save@pgfsyssoftpath@tokendefs
  \let\pgfsyssoftpath@movetotoken\forest@pgfpathtransformed@moveto
  \let\pgfsyssoftpath@linetotoken\forest@pgfpathtransformed@lineto
  \pgfsyssoftpath@setcurrentpath\pgfutil@empty
  #1%
  \forest@restore@pgfsyssoftpath@tokendefs
}
\def\forest@pgfpathtransformed@moveto#1#2{%
  \forest@pgfpathtransformed@op\pgfsyssoftpath@moveto{#1}{#2}%
}
\def\forest@pgfpathtransformed@lineto#1#2{%
  \forest@pgfpathtransformed@op\pgfsyssoftpath@lineto{#1}{#2}%
}
\def\forest@pgfpathtransformed@op#1#2#3{%
  \pgfpointtransformed{\pgfqpoint{#2}{#3}}%
  \edef\forest@temp{%
    \noexpand#1{\the\pgf@x}{\the\pgf@y}%
  }%
  \forest@temp
}
%    \end{macrocode}
%
% \subsubsection{Tiers}
% 
% Compute tiers to be aligned at a node. The result in saved in
% attribute |@tiers|.
%    \begin{macrocode}
\def\forest@pack@computetiers{%
  {%
    \forest@pack@tiers@getalltiersinsubtree
    \forest@pack@tiers@computetierhierarchy
    \forest@pack@tiers@findcontainers
    \forest@pack@tiers@raisecontainers
    \forest@pack@tiers@computeprocessingorder
    \gdef\forest@smuggle{}%
    \forest@pack@tiers@write
  }%
  \forest@node@foreach{\forestoset{@tiers}{}}%
  \forest@smuggle
}
%    \end{macrocode}
% Puts all tiers contained in the subtree into attribute
% |tiers|.  
%    \begin{macrocode}
\def\forest@pack@tiers@getalltiersinsubtree{%
  \ifnum\forestove{n children}>0
    \forest@node@foreachchild{\forest@pack@tiers@getalltiersinsubtree}%
  \fi
  \forestoget{tier}\forest@temp@mytier
  \def\forest@temp@mytiers{}%
  \ifdefempty\forest@temp@mytier{}{%
    \listeadd\forest@temp@mytiers\forest@temp@mytier
  }%
  \ifnum\forestove{n children}>0
    \forest@node@foreachchild{%
      \forestoget{tiers}\forest@temp@tiers
      \forlistloop\forest@pack@tiers@forhandlerA\forest@temp@tiers          
    }%
  \fi
  \forestolet{tiers}\forest@temp@mytiers
}
\def\forest@pack@tiers@forhandlerA#1{%
  \ifinlist{#1}\forest@temp@mytiers{}{%
    \listeadd\forest@temp@mytiers{#1}%
  }%
}
%    \end{macrocode}
% Compute a set of higher and lower tiers for each tier.  Tier A is
% higher than tier B iff a node on tier A is an ancestor of a
% node on tier B.
%    \begin{macrocode}
\def\forest@pack@tiers@computetierhierarchy{%
  \def\forest@tiers@ancestors{}%
  \forestoget{tiers}\forest@temp@mytiers
  \forlistloop\forest@pack@tiers@cth@init\forest@temp@mytiers
  \forest@pack@tiers@computetierhierarchy@
}
\def\forest@pack@tiers@cth@init#1{%
  \csdef{forest@tiers@higher@#1}{}%
  \csdef{forest@tiers@lower@#1}{}%
}
\def\forest@pack@tiers@computetierhierarchy@{%
  \forestoget{tier}\forest@temp@mytier
  \ifdefempty\forest@temp@mytier{}{%
    \forlistloop\forest@pack@tiers@forhandlerB\forest@tiers@ancestors
    \listeadd\forest@tiers@ancestors\forest@temp@mytier
  }%
  \forest@node@foreachchild{%
    \forest@pack@tiers@computetierhierarchy@
  }%
  \forestoget{tier}\forest@temp@mytier
  \ifdefempty\forest@temp@mytier{}{%
    \forest@listedel\forest@tiers@ancestors\forest@temp@mytier
  }%
}
\def\forest@pack@tiers@forhandlerB#1{%
  \def\forest@temp@tier{#1}%
  \ifx\forest@temp@tier\forest@temp@mytier
    \PackageError{forest}{Circular tier hierarchy (tier \forest@temp@mytier)}{}%
  \fi
  \ifinlistcs{#1}{forest@tiers@higher@\forest@temp@mytier}{}{%
    \listcsadd{forest@tiers@higher@\forest@temp@mytier}{#1}}%
  \xifinlistcs\forest@temp@mytier{forest@tiers@lower@#1}{}{%
    \listcseadd{forest@tiers@lower@#1}{\forest@temp@mytier}}%
}
\def\forest@pack@tiers@findcontainers{%
  \forestoget{tiers}\forest@temp@tiers
  \forlistloop\forest@pack@tiers@findcontainer\forest@temp@tiers
}
\def\forest@pack@tiers@findcontainer#1{%
  \def\forest@temp@tier{#1}%
  \forestoget{tier}\forest@temp@mytier
  \ifx\forest@temp@tier\forest@temp@mytier
    \csedef{forest@tiers@container@#1}{\forest@cn}%
  \else\@escapeif{%
    \forest@pack@tiers@findcontainerA{#1}%
  }\fi%
}
\def\forest@pack@tiers@findcontainerA#1{%
  \c@pgf@counta=0
  \forest@node@foreachchild{%
    \forestoget{tiers}\forest@temp@tiers
    \ifinlist{#1}\forest@temp@tiers{%
      \advance\c@pgf@counta 1
      \let\forest@temp@child\forest@cn
    }{}%
  }%
  \ifnum\c@pgf@counta>1
    \csedef{forest@tiers@container@#1}{\forest@cn}%
  \else\@escapeif{% surely =1
    \forest@fornode{\forest@temp@child}{%
      \forest@pack@tiers@findcontainer{#1}%
    }%
  }\fi
}
\def\forest@pack@tiers@raisecontainers{%
  \forestoget{tiers}\forest@temp@mytiers
  \forlistloop\forest@pack@tiers@rc@forhandlerA\forest@temp@mytiers
}
\def\forest@pack@tiers@rc@forhandlerA#1{%
  \edef\forest@tiers@temptier{#1}%
  \letcs\forest@tiers@containernodeoftier{forest@tiers@container@#1}%
  \letcs\forest@temp@lowertiers{forest@tiers@lower@#1}%
  \forlistloop\forest@pack@tiers@rc@forhandlerB\forest@temp@lowertiers
}
\def\forest@pack@tiers@rc@forhandlerB#1{%
  \letcs\forest@tiers@containernodeoflowertier{forest@tiers@container@#1}%
  \forestOget{\forest@tiers@containernodeoflowertier}{content}\lowercontent
  \forestOget{\forest@tiers@containernodeoftier}{content}\uppercontent
  \forest@fornode{\forest@tiers@containernodeoflowertier}{%
    \forest@ifancestorof
      {\forest@tiers@containernodeoftier}
      {\csletcs{forest@tiers@container@\forest@tiers@temptier}{forest@tiers@container@#1}}%
      {}%
  }%
}
\def\forest@pack@tiers@computeprocessingorder{%
  \def\forest@tiers@processingorder{}%
  \forestoget{tiers}\forest@tiers@cpo@tierstodo
  \forest@loopa
    \ifdefempty\forest@tiers@cpo@tierstodo{\forest@tempfalse}{\forest@temptrue}%
  \ifforest@temp
    \def\forest@tiers@cpo@tiersremaining{}%
    \def\forest@tiers@cpo@tiersindependent{}%
    \forlistloop\forest@pack@tiers@cpo@forhandlerA\forest@tiers@cpo@tierstodo
    \ifdefempty\forest@tiers@cpo@tiersindependent{%
      \PackageError{forest}{Circular tiers!}{}}{}%
    \forlistloop\forest@pack@tiers@cpo@forhandlerB\forest@tiers@cpo@tiersremaining
    \let\forest@tiers@cpo@tierstodo\forest@tiers@cpo@tiersremaining
  \forest@repeata
}
\def\forest@pack@tiers@cpo@forhandlerA#1{%
  \ifcsempty{forest@tiers@higher@#1}{%
    \listadd\forest@tiers@cpo@tiersindependent{#1}%
    \listadd\forest@tiers@processingorder{#1}%
  }{%
    \listadd\forest@tiers@cpo@tiersremaining{#1}%
  }%
}
\def\forest@pack@tiers@cpo@forhandlerB#1{%
  \def\forest@pack@tiers@cpo@aremainingtier{#1}%
  \forlistloop\forest@pack@tiers@cpo@forhandlerC\forest@tiers@cpo@tiersindependent
}
\def\forest@pack@tiers@cpo@forhandlerC#1{%
  \ifinlistcs{#1}{forest@tiers@higher@\forest@pack@tiers@cpo@aremainingtier}{%
    \forest@listcsdel{forest@tiers@higher@\forest@pack@tiers@cpo@aremainingtier}{#1}%
  }{}%
}
\def\forest@pack@tiers@write{%
  \forlistloop\forest@pack@tiers@write@forhandler\forest@tiers@processingorder
}
\def\forest@pack@tiers@write@forhandler#1{%
  \forest@fornode{\csname forest@tiers@container@#1\endcsname}{%
    \forest@pack@tiers@check{#1}%
  }%
  \xappto\forest@smuggle{%
    \noexpand\listadd
      \forestOm{\csname forest@tiers@container@#1\endcsname}{@tiers}%
      {#1}%
  }%
}
 % checks if the tier is compatible with growth changes and calign=node/edge angle
\def\forest@pack@tiers@check#1{%
  \def\forest@temp@currenttier{#1}%
  \forest@node@foreachdescendant{%
    \ifnum\forestove{grow}=\forestOve{\forestove{@parent}}{grow}
    \else
      \forest@pack@tiers@check@grow
    \fi
    \ifnum\forestove{n children}>1
      \forestoget{calign}\forest@temp
      \ifx\forest@temp\forest@pack@tiers@check@nodeangle
        \forest@pack@tiers@check@calign
      \fi
      \ifx\forest@temp\forest@pack@tiers@check@edgeangle
        \forest@pack@tiers@check@calign
      \fi
    \fi
  }%
}
\def\forest@pack@tiers@check@nodeangle{node angle}%
\def\forest@pack@tiers@check@edgeangle{edge angle}%
\def\forest@pack@tiers@check@grow{%
  \forestoget{content}\forest@temp@content
  \let\forest@temp@currentnode\forest@cn
  \forest@node@foreachdescendant{%
    \forestoget{tier}\forest@temp
    \ifx\forest@temp@currenttier\forest@temp
      \forest@pack@tiers@check@grow@error
    \fi
  }%
}
\def\forest@pack@tiers@check@grow@error{%
  \PackageError{forest}{Tree growth direction changes in node \forest@temp@currentnode\space
    (content: \forest@temp@content), while tier '\forest@temp' is specified for nodes both
    out- and inside the subtree rooted in node \forest@temp@currentnode.  This will not work.}{}%
}
\def\forest@pack@tiers@check@calign{%
  \forest@node@foreachchild{%
    \forestoget{tier}\forest@temp
    \ifx\forest@temp@currenttier\forest@temp
      \forest@pack@tiers@check@calign@warning
    \fi
  }%
}
\def\forest@pack@tiers@check@calign@warning{%
  \PackageWarning{forest}{Potential option conflict: node \forestove{@parent} (content:
    '\forestOve{\forestove{@parent}}{content}') was given 'calign=\forestove{calign}', while its
    child \forest@cn\space (content: '\forestove{content}') was given 'tier=\forestove{tier}'.
    The parent's 'calign' will only work if the child was the lowest node on its tier before the
    alignment.}{} 
}
%    \end{macrocode}
%
%
% \subsubsection{Node boundary}
%
% Compute the node boundary: it will be put in the pgf's current path.  The computation is done
% within a generic anchor so that the shape's saved anchors and macros are available.
%    \begin{macrocode}
\pgfdeclaregenericanchor{forestcomputenodeboundary}{%
  \letcs\forest@temp@boundary@macro{forest@compute@node@boundary@#1}%
  \ifcsname forest@compute@node@boundary@#1\endcsname
    \csname forest@compute@node@boundary@#1\endcsname
  \else
    \forest@compute@node@boundary@rectangle
  \fi
  \pgfsyssoftpath@getcurrentpath\forest@temp
  \global\let\forest@global@boundary\forest@temp
}
\def\forest@mt#1{%
  \expandafter\pgfpointanchor\expandafter{\pgfreferencednodename}{#1}%
  \pgfsyssoftpath@moveto{\the\pgf@x}{\the\pgf@y}%
}%
\def\forest@lt#1{%
  \expandafter\pgfpointanchor\expandafter{\pgfreferencednodename}{#1}%
  \pgfsyssoftpath@lineto{\the\pgf@x}{\the\pgf@y}%
}%
\def\forest@compute@node@boundary@coordinate{%
  \forest@mt{center}%
}
\def\forest@compute@node@boundary@circle{%
  \forest@mt{east}%
  \forest@lt{north east}%
  \forest@lt{north}%
  \forest@lt{north west}%
  \forest@lt{west}%
  \forest@lt{south west}%
  \forest@lt{south}%
  \forest@lt{south east}%
  \forest@lt{east}%
}
\def\forest@compute@node@boundary@rectangle{%
  \forest@mt{south west}%
  \forest@lt{south east}%
  \forest@lt{north east}%
  \forest@lt{north west}%
  \forest@lt{south west}%
}
\def\forest@compute@node@boundary@diamond{%
  \forest@mt{east}%
  \forest@lt{north}%
  \forest@lt{west}%
  \forest@lt{south}%
  \forest@lt{east}%
}
\let\forest@compute@node@boundary@ellipse\forest@compute@node@boundary@circle
\def\forest@compute@node@boundary@trapezium{%
  \forest@mt{top right corner}%
  \forest@lt{top left corner}%
  \forest@lt{bottom left corner}%
  \forest@lt{bottom right corner}%
  \forest@lt{top right corner}%
}
\def\forest@compute@node@boundary@semicircle{%
  \forest@mt{arc start}%
  \forest@lt{north}%
  \forest@lt{east}%
  \forest@lt{north east}%
  \forest@lt{apex}%
  \forest@lt{north west}%
  \forest@lt{west}%
  \forest@lt{arc end}%
  \forest@lt{arc start}%
}
\newloop\forest@computenodeboundary@loop
\csdef{forest@compute@node@boundary@regular polygon}{%
  \forest@mt{corner 1}%
  \c@pgf@counta=\sides\relax
  \forest@computenodeboundary@loop
  \ifnum\c@pgf@counta>0
    \forest@lt{corner \the\c@pgf@counta}%
    \advance\c@pgf@counta-1
  \forest@computenodeboundary@repeat
}%
\def\forest@compute@node@boundary@star{%
  \forest@mt{outer point 1}%
  \c@pgf@counta=\totalstarpoints\relax
  \divide\c@pgf@counta2
  \forest@computenodeboundary@loop
  \ifnum\c@pgf@counta>0
    \forest@lt{inner point \the\c@pgf@counta}%
    \forest@lt{outer point \the\c@pgf@counta}%
    \advance\c@pgf@counta-1
  \forest@computenodeboundary@repeat  
}%
\csdef{forest@compute@node@boundary@isosceles triangle}{%
  \forest@mt{apex}%
  \forest@lt{left corner}%
  \forest@lt{right corner}%
  \forest@lt{apex}%
}
\def\forest@compute@node@boundary@kite{%
  \forest@mt{upper vertex}%
  \forest@lt{left vertex}%
  \forest@lt{lower vertex}%
  \forest@lt{right vertex}%
  \forest@lt{upper vertex}%
}
\def\forest@compute@node@boundary@dart{%
  \forest@mt{tip}%
  \forest@lt{left tail}%
  \forest@lt{tail center}%
  \forest@lt{right tail}%
  \forest@lt{tip}%
}
\csdef{forest@compute@node@boundary@circular sector}{%
  \forest@mt{sector center}%
  \forest@lt{arc start}%
  \forest@lt{arc center}%
  \forest@lt{arc end}%
  \forest@lt{sector center}%
}  
\def\forest@compute@node@boundary@cylinder{%
  \forest@mt{top}%
  \forest@lt{after top}%
  \forest@lt{before bottom}%
  \forest@lt{bottom}%
  \forest@lt{after bottom}%
  \forest@lt{before top}%
  \forest@lt{top}%
}
\cslet{forest@compute@node@boundary@forbidden sign}\forest@compute@node@boundary@circle
\cslet{forest@compute@node@boundary@magnifying glass}\forest@compute@node@boundary@circle
\def\forest@compute@node@boundary@cloud{%
  \getradii
  \forest@mt{puff 1}%
  \c@pgf@counta=\puffs\relax
  \forest@computenodeboundary@loop
  \ifnum\c@pgf@counta>0
    \forest@lt{puff \the\c@pgf@counta}%
    \advance\c@pgf@counta-1
  \forest@computenodeboundary@repeat  
}
\def\forest@compute@node@boundary@starburst{
  \calculatestarburstpoints
  \forest@mt{outer point 1}%
  \c@pgf@counta=\totalpoints\relax
  \divide\c@pgf@counta2
  \forest@computenodeboundary@loop
  \ifnum\c@pgf@counta>0
    \forest@lt{inner point \the\c@pgf@counta}%
    \forest@lt{outer point \the\c@pgf@counta}%
    \advance\c@pgf@counta-1
  \forest@computenodeboundary@repeat  
}%
\def\forest@compute@node@boundary@signal{%
  \forest@mt{east}%
  \forest@lt{south east}%
  \forest@lt{south west}%
  \forest@lt{west}%
  \forest@lt{north west}%
  \forest@lt{north east}%
  \forest@lt{east}%
}
\def\forest@compute@node@boundary@tape{%
  \forest@mt{north east}%
  \forest@lt{60}%
  \forest@lt{north}%
  \forest@lt{120}%
  \forest@lt{north west}%
  \forest@lt{south west}%
  \forest@lt{240}%
  \forest@lt{south}%
  \forest@lt{310}%
  \forest@lt{south east}%
  \forest@lt{north east}%
}
\csdef{forest@compute@node@boundary@single arrow}{%
  \forest@mt{tip}%
  \forest@lt{after tip}%
  \forest@lt{after head}%
  \forest@lt{before tail}%
  \forest@lt{after tail}%
  \forest@lt{before head}%
  \forest@lt{before tip}%
  \forest@lt{tip}%
}
\csdef{forest@compute@node@boundary@double arrow}{%
  \forest@mt{tip 1}%
  \forest@lt{after tip 1}%
  \forest@lt{after head 1}%
  \forest@lt{before head 2}%
  \forest@lt{before tip 2}%
  \forest@mt{tip 2}%
  \forest@lt{after tip 2}%
  \forest@lt{after head 2}%
  \forest@lt{before head 1}%
  \forest@lt{before tip 1}%
  \forest@lt{tip 1}%
}
\csdef{forest@compute@node@boundary@arrow box}{%
  \forest@mt{before north arrow}%
  \forest@lt{before north arrow head}%
  \forest@lt{before north arrow tip}%
  \forest@lt{north arrow tip}%
  \forest@lt{after north arrow tip}%
  \forest@lt{after north arrow head}%
  \forest@lt{after north arrow}%
  \forest@lt{north east}%
  \forest@lt{before east arrow}%
  \forest@lt{before east arrow head}%
  \forest@lt{before east arrow tip}%
  \forest@lt{east arrow tip}%
  \forest@lt{after east arrow tip}%
  \forest@lt{after east arrow head}%
  \forest@lt{after east arrow}%
  \forest@lt{south east}%
  \forest@lt{before south arrow}%
  \forest@lt{before south arrow head}%
  \forest@lt{before south arrow tip}%
  \forest@lt{south arrow tip}%
  \forest@lt{after south arrow tip}%
  \forest@lt{after south arrow head}%
  \forest@lt{after south arrow}%
  \forest@lt{south west}% 
  \forest@lt{before west arrow}%
  \forest@lt{before west arrow head}%
  \forest@lt{before west arrow tip}%
  \forest@lt{west arrow tip}%
  \forest@lt{after west arrow tip}%
  \forest@lt{after west arrow head}%
  \forest@lt{after west arrow}%
  \forest@lt{north west}%
  \forest@lt{before north arrow}%
}
\cslet{forest@compute@node@boundary@circle split}\forest@compute@node@boundary@circle
\cslet{forest@compute@node@boundary@circle solidus}\forest@compute@node@boundary@circle
\cslet{forest@compute@node@boundary@ellipse split}\forest@compute@node@boundary@ellipse
\cslet{forest@compute@node@boundary@rectangle split}\forest@compute@node@boundary@rectangle
\def\forest@compute@node@boundary@@callout{%
  \beforecalloutpointer
  \pgfsyssoftpath@moveto{\the\pgf@x}{\the\pgf@y}%
  \calloutpointeranchor
  \pgfsyssoftpath@lineto{\the\pgf@x}{\the\pgf@y}%  
  \aftercalloutpointer
  \pgfsyssoftpath@lineto{\the\pgf@x}{\the\pgf@y}%  
}
\csdef{forest@compute@node@boundary@rectangle callout}{%
  \forest@compute@node@boundary@rectangle
  \rectanglecalloutpoints
  \forest@compute@node@boundary@@callout
}
\csdef{forest@compute@node@boundary@ellipse callout}{%
  \forest@compute@node@boundary@ellipse
  \ellipsecalloutpoints
  \forest@compute@node@boundary@@callout
}
\csdef{forest@compute@node@boundary@cloud callout}{%
  \forest@compute@node@boundary@cloud
  % at least a first approx...
  \forest@mt{center}%
  \forest@lt{pointer}%
}%
\csdef{forest@compute@node@boundary@cross out}{%
  \forest@mt{south east}%
  \forest@lt{north west}%
  \forest@mt{south west}%
  \forest@lt{north east}%
}%
\csdef{forest@compute@node@boundary@strike out}{%
  \forest@mt{north east}%
  \forest@lt{south west}%
}%  
\cslet{forest@compute@node@boundary@rounded rectangle}\forest@compute@node@boundary@rectangle
\csdef{forest@compute@node@boundary@chamfered rectangle}{%
  \forest@mt{before south west}%
  \forest@mt{after south west}%
  \forest@lt{before south east}%
  \forest@lt{after south east}%
  \forest@lt{before north east}%
  \forest@lt{after north east}%
  \forest@lt{before north west}%
  \forest@lt{after north west}%
  \forest@lt{before south west}%
}%
%    \end{macrocode}
%
%
%
%
% \subsection{Compute absolute positions}
%
% Computes absolute positions of descendants relative to this node.
% Stores the results in attributes |x| and |y|.
%    \begin{macrocode}
\def\forest@node@computeabsolutepositions{%
  \forestoset{x}{0pt}%
  \forestoset{y}{0pt}%
  \edef\forest@marshal{%
    \noexpand\forest@node@foreachchild{%
      \noexpand\forest@node@computeabsolutepositions@{0pt}{0pt}{\forestove{grow}}%
    }%
  }\forest@marshal
}
\def\forest@node@computeabsolutepositions@#1#2#3{%
  \pgfpointadd{\pgfpoint{#1}{#2}}{%
    \pgfpointadd{\pgfpolar{#3}{\forestove{l}}}{\pgfpolar{90 + #3}{\forestove{s}}}}%
  \pgfgetlastxy\forest@temp@x\forest@temp@y
  \forestolet{x}\forest@temp@x
  \forestolet{y}\forest@temp@y
  \edef\forest@marshal{%
    \noexpand\forest@node@foreachchild{%
      \noexpand\forest@node@computeabsolutepositions@{\forest@temp@x}{\forest@temp@y}{\forestove{grow}}%
    }%
  }\forest@marshal    
}
%    \end{macrocode}
%
%
% \subsection{Drawing the tree}
% \label{imp:drawing-the-tree}
%    \begin{macrocode}
\newif\ifforest@drawtree@preservenodeboxes@
\def\forest@node@drawtree{%
  \expandafter\ifstrequal\expandafter{\forest@drawtreebox}{\pgfkeysnovalue}{%
    \let\forest@drawtree@beginbox\relax
    \let\forest@drawtree@endbox\relax
  }{%
    \edef\forest@drawtree@beginbox{\global\setbox\forest@drawtreebox=\hbox\bgroup}%
    \let\forest@drawtree@endbox\egroup
  }%
  \ifforest@external@
    \ifforest@externalize@tree@
      \forest@temptrue
    \else
      \tikzifexternalizing{%
        \ifforest@was@tikzexternalwasenable
          \forest@temptrue
          \pgfkeys{/tikz/external/optimize=false}%
          \let\forest@drawtree@beginbox\relax
          \let\forest@drawtree@endbox\relax
        \else
          \forest@tempfalse
        \fi
      }{%
        \forest@tempfalse
      }%
    \fi
    \ifforest@temp
      \advance\forest@externalize@inner@n 1
      \edef\forest@externalize@filename{%
        \tikzexternalrealjob-forest-\forest@externalize@outer@n
        \ifnum\forest@externalize@inner@n=0 \else.\the\forest@externalize@inner@n\fi}%
      \expandafter\tikzsetnextfilename\expandafter{\forest@externalize@filename}%
      \tikzexternalenable
      \pgfkeysalso{/tikz/external/remake next,/tikz/external/export next}%
    \fi
    \ifforest@externalize@tree@
      \typeout{forest: Invoking a recursive call to generate the external picture
        '\forest@externalize@filename' for the following context+code:
        '\expandafter\detokenize\expandafter{\forest@externalize@id}'}%
    \fi
  \fi
  %
  \ifforesttikzcshack
    \let\forest@original@tikz@parse@node\tikz@parse@node
    \let\tikz@parse@node\forest@tikz@parse@node
  \fi
  \pgfkeysgetvalue{/forest/begin draw/.@cmd}\forest@temp@begindraw
  \pgfkeysgetvalue{/forest/end draw/.@cmd}\forest@temp@enddraw
  \edef\forest@marshal{%
    \noexpand\forest@drawtree@beginbox
    \expandonce{\forest@temp@begindraw\pgfkeysnovalue\pgfeov}%
    \noexpand\forest@node@drawtree@
    \expandonce{\forest@temp@enddraw\pgfkeysnovalue\pgfeov}%
    \noexpand\forest@drawtree@endbox
  }\forest@marshal
  \ifforesttikzcshack
    \let\tikz@parse@node\forest@original@tikz@parse@node
  \fi
  %
  \ifforest@external@
    \ifforest@externalize@tree@
      \tikzexternaldisable
      \eappto\forest@externalize@checkimages{%
        \noexpand\forest@includeexternal@check{\forest@externalize@filename}%
      }%
      \expandafter\ifstrequal\expandafter{\forest@drawtreebox}{\pgfkeysnovalue}{%
        \eappto\forest@externalize@loadimages{%
          \noexpand\forest@includeexternal{\forest@externalize@filename}%
        }%
      }{%
        \eappto\forest@externalize@loadimages{%
          \noexpand\forest@includeexternal@box\forest@drawtreebox{\forest@externalize@filename}%
        }%
      }%
    \fi
  \fi
}
\def\forest@node@drawtree@{%
  \forest@node@foreach{\forest@draw@node}%
  \forest@node@Ifnamedefined{forest@baseline@node}{%
    \edef\forest@temp{%
      \noexpand\pgfsetbaselinepointlater{%
        \noexpand\pgfpointanchor
          {\forestOve{\forest@node@Nametoid{forest@baseline@node}}{name}}
          {\forestOve{\forest@node@Nametoid{forest@baseline@node}}{anchor}}
      }%
    }\forest@temp
  }{}%
  \forest@node@foreachdescendant{\forest@draw@edge}%
  \forest@node@foreach{\forest@draw@tikz}%
}
\def\forest@draw@node{%
  \ifnum\forestove{phantom}=0
    \forest@node@forest@positionnodelater@restore
    \ifforest@drawtree@preservenodeboxes@
      \pgfnodealias{forest@temp}{\forestove{later@name}}%
    \fi
    \pgfpositionnodenow{\pgfqpoint{\forestove{x}}{\forestove{y}}}%
    \ifforest@drawtree@preservenodeboxes@
      \pgfnodealias{\forestove{later@name}}{forest@temp}%
    \fi
  \fi
}
\def\forest@draw@edge{%
  \ifnum\forestove{phantom}=0
    \ifnum\forestOve{\forestove{@parent}}{phantom}=0
      \edef\forest@temp{\forestove{edge path}}%
      \forest@temp
    \fi
  \fi
}
\def\forest@draw@tikz{%
  \forestove{tikz}%
}
%    \end{macrocode}
% A hack into \TikZ;'s coordinate parser: implements relative node names!
%    \begin{macrocode}
\def\forest@tikz@parse@node#1(#2){%
  \pgfutil@in@.{#2}%
  \ifpgfutil@in@
    \expandafter\forest@tikz@parse@node@checkiftikzname@withdot
  \else%
    \expandafter\forest@tikz@parse@node@checkiftikzname@withoutdot
  \fi%
  #1(#2)\forest@end
}
\def\forest@tikz@parse@node@checkiftikzname@withdot#1(#2.#3)\forest@end{%
  \forest@tikz@parse@node@checkiftikzname#1{#2}{.#3}}
\def\forest@tikz@parse@node@checkiftikzname@withoutdot#1(#2)\forest@end{%
  \forest@tikz@parse@node@checkiftikzname#1{#2}{}}  
\def\forest@tikz@parse@node@checkiftikzname#1#2#3{%
  \expandafter\ifx\csname pgf@sh@ns@#2\endcsname\relax
    \forest@forthis{%
      \forest@nameandgo{#2}%
      \edef\forest@temp@relativenodename{\forestove{name}}%
    }%
  \else
    \def\forest@temp@relativenodename{#2}%
  \fi
  \expandafter\forest@original@tikz@parse@node\expandafter#1\expandafter(\forest@temp@relativenodename#3)%
}
\def\forest@nameandgo#1{%
  \pgfutil@in@!{#1}%
  \ifpgfutil@in@
    \forest@nameandgo@(#1)%
  \else
    \ifstrempty{#1}{}{\edef\forest@cn{\forest@node@Nametoid{#1}}}%
  \fi
}
\def\forest@nameandgo@(#1!#2){%
  \ifstrempty{#1}{}{\edef\forest@cn{\forest@node@Nametoid{#1}}}%
  \forest@go{#2}%
}
%    \end{macrocode}
% |parent/child anchor| are generic anchors which forward to the real one.  There's a hack in there
% to deal with link pointing to the ``border'' anchor.
%    \begin{macrocode}
\pgfdeclaregenericanchor{parent anchor}{%
  \forest@generic@parent@child@anchor{parent }{#1}}
\pgfdeclaregenericanchor{child anchor}{%
  \forest@generic@parent@child@anchor{child }{#1}}
\pgfdeclaregenericanchor{anchor}{%
  \forest@generic@parent@child@anchor{}{#1}}
\def\forest@generic@parent@child@anchor#1#2{%
  \forestOget{\forest@node@Nametoid{\pgfreferencednodename}}{#1anchor}\forest@temp@parent@anchor
  \ifdefempty\forest@temp@parent@anchor{%
    \pgf@sh@reanchor{#2}{center}%
    \xdef\forest@hack@tikzshapeborder{%
      \noexpand\tikz@shapebordertrue
      \def\noexpand\tikz@shapeborder@name{\pgfreferencednodename}%
    }\aftergroup\forest@hack@tikzshapeborder
  }{%
    \pgf@sh@reanchor{#2}{\forest@temp@parent@anchor}%
  }%
}
%    \end{macrocode}
%
%
% \section{Geometry}
% \label{imp:geometry}
%
% A \emph{$\alpha$ grow line} is a line through the origin at angle
% $\alpha$. The following macro sets up the grow line, which can then
% be used by other code (the change is local to the \TeX\ group). More
% precisely, two normalized vectors are set up: one $(x_g,y_g)$ on the
% grow line, and one $(x_s,y_s)$ orthogonal to it---to get
% $(x_s,y_s$), rotate $(x_g,y_g)$ 90$^\circ$ counter-clockwise.
%    \begin{macrocode}
\newdimen\forest@xg
\newdimen\forest@yg
\newdimen\forest@xs
\newdimen\forest@ys
\def\forest@setupgrowline#1{%
  \edef\forest@grow{#1}%
  \pgfpointpolar\forest@grow{1pt}%
  \forest@xg=\pgf@x
  \forest@yg=\pgf@y
  \forest@xs=-\pgf@y
  \forest@ys=\pgf@x   
}
%    \end{macrocode}
%
% \subsection{Projections}
% \label{imp:projections}
% 
% The following macro belongs to the |\pgfpoint...| family: it
% projects point |#1| on the grow line.  (The result is returned via
% |\pgf@x| and |\pgf@y|.)  The implementation is based on code from
% |tikzlibrarycalc|, but optimized for projecting on grow lines, and
% split to optimize serial usage in |\forest@projectpath|.
%    \begin{macrocode}
\def\forest@pgfpointprojectiontogrowline#1{{%
  \pgf@process{#1}%
%    \end{macrocode}
% Calculate the scalar product of $(x,y)$ and $(x_g,y_g)$: that's the
% distance of $(x,y)$ to the grow line.
%    \begin{macrocode}
  \pgfutil@tempdima=\pgf@sys@tonumber{\pgf@x}\forest@xg%
  \advance\pgfutil@tempdima by\pgf@sys@tonumber{\pgf@y}\forest@yg%
%    \end{macrocode}
% The projection is $(x_g,y_g)$ scaled by the distance.
%    \begin{macrocode}
  \global\pgf@x=\pgf@sys@tonumber{\pgfutil@tempdima}\forest@xg%
  \global\pgf@y=\pgf@sys@tonumber{\pgfutil@tempdima}\forest@yg%
}}
%    \end{macrocode}
%
% The following macro calculates the distance of point |#2| to the
% grow line and stores the result in \TeX-dimension |#1|. The distance
% is the scalar product of the point vector and the normalized vector
% orthogonal to the grow line. 
%    \begin{macrocode}
\def\forest@distancetogrowline#1#2{%
  \pgf@process{#2}%
  #1=\pgf@sys@tonumber{\pgf@x}\forest@xs\relax
  \advance#1 by\pgf@sys@tonumber{\pgf@y}\forest@ys\relax
}
%    \end{macrocode}
% Note that the distance to the grow line is positive for points on
% one of its sides and negative for points on the other side. (It is
% positive on the side which $(x_s,y_s)$ points to.)  We thus say that
% the grow line partitions the plane into a \emph{positive} and a
% \emph{negative} side.
%
% The following macro projects all segment edges (``points'') of a
% simple\footnote{A path is \emph{simple} if it consists of only
% move-to and line-to operations.} path |#1| onto the grow line.
% The result is an array of tuples (|xo|, |yo|, |xp|, |yp|), where
% |xo| and |yo| stand for the \emph{o}riginal point, and |xp| and |yp|
% stand for its \emph{p}rojection. The prefix of the array is given by
% |#2|.  If the array already exists, the new items are appended to
% it.  The array is not sorted: the order of original points in the
% array is their order in the path.  The computation does not destroy
% the current path.  All result-macros have local scope.
%
% The macro is just a wrapper for |\forest@projectpath@process|.
%    \begin{macrocode}
\let\forest@pp@n\relax
\def\forest@projectpathtogrowline#1#2{%
  \edef\forest@pp@prefix{#2}%
  \forest@save@pgfsyssoftpath@tokendefs
  \let\pgfsyssoftpath@movetotoken\forest@projectpath@processpoint
  \let\pgfsyssoftpath@linetotoken\forest@projectpath@processpoint
  \c@pgf@counta=0
  #1%
  \csedef{#2n}{\the\c@pgf@counta}%
  \forest@restore@pgfsyssoftpath@tokendefs
}
%    \end{macrocode}
% For each point, remember the point and its projection to grow line.
%    \begin{macrocode}
\def\forest@projectpath@processpoint#1#2{%
  \pgfqpoint{#1}{#2}%
  \expandafter\edef\csname\forest@pp@prefix\the\c@pgf@counta xo\endcsname{\the\pgf@x}%
  \expandafter\edef\csname\forest@pp@prefix\the\c@pgf@counta yo\endcsname{\the\pgf@y}%
  \forest@pgfpointprojectiontogrowline{}%
  \expandafter\edef\csname\forest@pp@prefix\the\c@pgf@counta xp\endcsname{\the\pgf@x}%
  \expandafter\edef\csname\forest@pp@prefix\the\c@pgf@counta yp\endcsname{\the\pgf@y}%
  \advance\c@pgf@counta 1\relax
}
%    \end{macrocode}
% Sort the array (prefix |#1|) produced by
% |\forest@projectpathtogrowline| by |(xp,yp)|, in the ascending order.
%    \begin{macrocode}
\def\forest@sortprojections#1{%
  % todo: optimize in cases when we know that the array is actually a
  % merger of sorted arrays; when does this happen? in
  % distance_between_paths, and when merging the edges of the parent
  % and its children in a uniform growth tree
  \edef\forest@ppi@inputprefix{#1}%
  \c@pgf@counta=\csname#1n\endcsname\relax
  \advance\c@pgf@counta -1
  \forest@sort\forest@ppiraw@cmp\forest@ppiraw@let\forest@sort@ascending{0}{\the\c@pgf@counta}%
}
%    \end{macrocode}
%
% The following macro processes the data gathered by (possibly more
% than one invocation of) |\forest@projectpathtogrowline| into array
% with prefix |#1|.  The resulting data is the following.
% \begin{itemize}
% \item Array of projections (prefix |#2|)
%   \begin{itemize}
%   \item its items are tuples |(x,y)| (the array is sorted by |x|
%     and |y|), and
%   \item an inner array of original points (prefix |#2N@|, where $N$
%     is the index of the item in array |#2|.  The items of |#2N@|
%     are |x|, |y| and |d|: |x| and |y| are the coordinates of the
%     original point; |d| is its distance to the grow line. The inner
%     array is not sorted.
%   \end{itemize}
% \item A dictionary |#2|: keys are the coordinates |(x,y)| of
%   the original points; a value is the index of the original point's
%   projection in array |#2|.\footnote{At first sight, this
%   information could be cached ``at the source'': by
%   forest@pgfpointprojectiontogrowline.  However, due to imprecise
%   intersecting (in breakpath), we cheat and merge very adjacent
%   projection points, expecting that the points to project to the
%   merged projection point.  All this depends on the given path, so a
%   generic cache is not feasible.}
% \end{itemize}
%    \begin{macrocode}
\def\forest@processprojectioninfo#1#2{%
  \edef\forest@ppi@inputprefix{#1}%
%    \end{macrocode}
% Loop (counter |\c@pgf@counta|) through the sorted array of raw data.
%    \begin{macrocode}
  \c@pgf@counta=0
  \c@pgf@countb=-1
  \loop
  \ifnum\c@pgf@counta<\csname#1n\endcsname\relax
%    \end{macrocode}
% Check if the projection tuple in the current raw item equals the
% current projection.
%    \begin{macrocode}
    \letcs\forest@xo{#1\the\c@pgf@counta xo}%
    \letcs\forest@yo{#1\the\c@pgf@counta yo}%
    \letcs\forest@xp{#1\the\c@pgf@counta xp}%
    \letcs\forest@yp{#1\the\c@pgf@counta yp}%
    \ifnum\c@pgf@countb<0
      \forest@equaltotolerancefalse
    \else
      \forest@equaltotolerance
        {\pgfqpoint\forest@xp\forest@yp}%
        {\pgfqpoint
          {\csname#2\the\c@pgf@countb x\endcsname}%
          {\csname#2\the\c@pgf@countb y\endcsname}%
        }%
    \fi
    \ifforest@equaltotolerance\else
%    \end{macrocode}
% It not, we will append a new item to the outer result array.
%    \begin{macrocode}
      \advance\c@pgf@countb 1
      \cslet{#2\the\c@pgf@countb x}\forest@xp
      \cslet{#2\the\c@pgf@countb y}\forest@yp
      \csdef{#2\the\c@pgf@countb @n}{0}%
    \fi
%    \end{macrocode}
% If the projection is actually a projection of one a point in our path:
%    \begin{macrocode}
    % todo: this is ugly!
    \ifdefined\forest@xo\ifx\forest@xo\relax\else
      \ifdefined\forest@yo\ifx\forest@yo\relax\else
%    \end{macrocode}
% Append the point of the current raw item to the inner array of
% points projecting to the current projection.
%    \begin{macrocode}
        \forest@append@point@to@inner@array
          \forest@xo\forest@yo
          {#2\the\c@pgf@countb @}%
%    \end{macrocode}
% Put a new item in the dictionary: key = the original point, value =
% the projection index.
%    \begin{macrocode}
        \csedef{#2(\forest@xo,\forest@yo)}{\the\c@pgf@countb}%
      \fi\fi
    \fi\fi
%    \end{macrocode}
% Clean-up the raw array item.
%    \begin{macrocode}
    \cslet{#1\the\c@pgf@counta xo}\relax
    \cslet{#1\the\c@pgf@counta yo}\relax
    \cslet{#1\the\c@pgf@counta xp}\relax
    \cslet{#1\the\c@pgf@counta yp}\relax
    \advance\c@pgf@counta 1
  \repeat
%    \end{macrocode}
% Clean up the raw array length.
%    \begin{macrocode}
  \cslet{#1n}\relax
%    \end{macrocode}
% Store the length of the outer result array.
%    \begin{macrocode}
  \advance\c@pgf@countb 1
  \csedef{#2n}{\the\c@pgf@countb}%
}
%    \end{macrocode}
%    
% Item-exchange macro for quicksorting the raw projection data. (|#1|
% is copied into |#2|.)
%    \begin{macrocode}
\def\forest@ppiraw@let#1#2{%
  \csletcs{\forest@ppi@inputprefix#1xo}{\forest@ppi@inputprefix#2xo}%
  \csletcs{\forest@ppi@inputprefix#1yo}{\forest@ppi@inputprefix#2yo}%
  \csletcs{\forest@ppi@inputprefix#1xp}{\forest@ppi@inputprefix#2xp}%
  \csletcs{\forest@ppi@inputprefix#1yp}{\forest@ppi@inputprefix#2yp}%
}
%    \end{macrocode}
% Item comparision macro for quicksorting the raw projection data.
%    \begin{macrocode}
\def\forest@ppiraw@cmp#1#2{%
  \forest@sort@cmptwodimcs
    {\forest@ppi@inputprefix#1xp}{\forest@ppi@inputprefix#1yp}%
    {\forest@ppi@inputprefix#2xp}{\forest@ppi@inputprefix#2yp}%
}
%    \end{macrocode}
%    
% Append the point |(#1,#2)| to the (inner) array of points
% (prefix |#3|).
%    \begin{macrocode}
\def\forest@append@point@to@inner@array#1#2#3{%
  \c@pgf@countc=\csname#3n\endcsname\relax
  \csedef{#3\the\c@pgf@countc x}{#1}%
  \csedef{#3\the\c@pgf@countc y}{#2}%
  \forest@distancetogrowline\pgfutil@tempdima{\pgfqpoint#1#2}%
  \csedef{#3\the\c@pgf@countc d}{\the\pgfutil@tempdima}%
  \advance\c@pgf@countc 1
  \csedef{#3n}{\the\c@pgf@countc}%
}
%    \end{macrocode}
%    
% \subsection{Break path}
%
% The following macro computes from the given path (|#1|) a ``broken''
% path (|#3|) that contains the same points of the plane, but has
% potentially more segments, so that, for every point from a given set
% of points on the grow line, a line through this point perpendicular
% to the grow line intersects the broken path only at its edge
% segments (i.e.\ not between them).
%
% The macro works only for \emph{simple} paths, i.e.\ paths built
% using only move-to and line-to operations.  Furthermore,
% |\forest@processprojectioninfo| must be called before calling
% |\forest@breakpath|: we expect information with prefix |#2|.  The
% macro updates the information compiled by
% |\forest@processprojectioninfo| with information about points added
% by path-breaking.
%    \begin{macrocode}
\def\forest@breakpath#1#2#3{%
%    \end{macrocode}
% Store the current path in a macro and empty it, then process the
% stored path. The processing creates a new current path.
%    \begin{macrocode}
  \edef\forest@bp@prefix{#2}%
  \forest@save@pgfsyssoftpath@tokendefs
  \let\pgfsyssoftpath@movetotoken\forest@breakpath@processfirstpoint
  \let\pgfsyssoftpath@linetotoken\forest@breakpath@processfirstpoint
  %\pgfusepath{}% empty the current path. ok?
  #1%
  \forest@restore@pgfsyssoftpath@tokendefs
  \pgfsyssoftpath@getcurrentpath#3%
}
%    \end{macrocode}
% The original and the broken path start in the same way. (This code
% implicitely ``repairs'' a path that starts illegally, with a line-to
% operation.)
%    \begin{macrocode}
\def\forest@breakpath@processfirstpoint#1#2{%  
  \forest@breakpath@processmoveto{#1}{#2}%
  \let\pgfsyssoftpath@movetotoken\forest@breakpath@processmoveto
  \let\pgfsyssoftpath@linetotoken\forest@breakpath@processlineto
}
%    \end{macrocode}
% When a move-to operation is encountered, it is simply copied to the
% broken path, starting a new subpath. Then we remember the last
% point, its projection's index (the point dictionary is used here)
% and the actual projection point.
%    \begin{macrocode}
\def\forest@breakpath@processmoveto#1#2{%
  \pgfsyssoftpath@moveto{#1}{#2}%
  \def\forest@previous@x{#1}%
  \def\forest@previous@y{#2}%
  \expandafter\let\expandafter\forest@previous@i
    \csname\forest@bp@prefix(#1,#2)\endcsname
  \expandafter\let\expandafter\forest@previous@px
    \csname\forest@bp@prefix\forest@previous@i x\endcsname
  \expandafter\let\expandafter\forest@previous@py
    \csname\forest@bp@prefix\forest@previous@i y\endcsname
}
%    \end{macrocode}
%    
% This is the heart of the path-breaking procedure.
%    \begin{macrocode}
\def\forest@breakpath@processlineto#1#2{%
%    \end{macrocode}
% Usually, the broken path will continue with a line-to operation (to
% the current point |(#1,#2)|).
%    \begin{macrocode}
  \let\forest@breakpath@op\pgfsyssoftpath@lineto
%    \end{macrocode}
% Get the index of the current point's projection and the projection
% itself.  (The point dictionary is used here.)
%    \begin{macrocode}
  \expandafter\let\expandafter\forest@i
    \csname\forest@bp@prefix(#1,#2)\endcsname
  \expandafter\let\expandafter\forest@px
    \csname\forest@bp@prefix\forest@i x\endcsname
  \expandafter\let\expandafter\forest@py
    \csname\forest@bp@prefix\forest@i y\endcsname
%    \end{macrocode}
% Test whether the projections of the previous and the current point
% are the same.
%    \begin{macrocode}
  \forest@equaltotolerance
    {\pgfqpoint{\forest@previous@px}{\forest@previous@py}}%
    {\pgfqpoint{\forest@px}{\forest@py}}%
  \ifforest@equaltotolerance
%    \end{macrocode}
% If so, we are dealing with a segment, perpendicular to the grow
% line. This segment must be removed, so we change the operation to
% move-to.
%    \begin{macrocode}
    \let\forest@breakpath@op\pgfsyssoftpath@moveto
  \else
%    \end{macrocode}
% Figure out the ``direction'' of the segment: in the order of the
% array of projections, or in the reversed order? Setup the loop step
% and the test condition.
%    \begin{macrocode}
    \forest@temp@count=\forest@previous@i\relax
    \ifnum\forest@previous@i<\forest@i\relax
      \def\forest@breakpath@step{1}%
      \def\forest@breakpath@test{\forest@temp@count<\forest@i\relax}%
    \else
      \def\forest@breakpath@step{-1}%
      \def\forest@breakpath@test{\forest@temp@count>\forest@i\relax}%
    \fi
%    \end{macrocode}
% Loop through all the projections between (in the (possibly reversed)
% array order) the projections of the previous and the current point
% (both exclusive).
%    \begin{macrocode}
    \loop
      \advance\forest@temp@count\forest@breakpath@step\relax
    \expandafter\ifnum\forest@breakpath@test
%    \end{macrocode}
% Intersect the current segment with the line through the current (in
% the loop!) projection perpendicular to the grow line. (There
% \emph{will} be an intersection.)
%    \begin{macrocode}
      \pgfpointintersectionoflines
        {\pgfqpoint
          {\csname\forest@bp@prefix\the\forest@temp@count x\endcsname}%
          {\csname\forest@bp@prefix\the\forest@temp@count y\endcsname}%
        }%
        {\pgfpointadd
          {\pgfqpoint
            {\csname\forest@bp@prefix\the\forest@temp@count x\endcsname}%
            {\csname\forest@bp@prefix\the\forest@temp@count y\endcsname}%
          }%
          {\pgfqpoint{\forest@xs}{\forest@ys}}%
        }%
        {\pgfqpoint{\forest@previous@x}{\forest@previous@y}}%
        {\pgfqpoint{#1}{#2}}%
%    \end{macrocode}
% Break the segment at the intersection. 
%    \begin{macrocode}
      \pgfgetlastxy\forest@last@x\forest@last@y
      \pgfsyssoftpath@lineto\forest@last@x\forest@last@y
%    \end{macrocode}
% Append the breaking point to the inner array for the projection.
%    \begin{macrocode}
      \forest@append@point@to@inner@array
        \forest@last@x\forest@last@y
        {\forest@bp@prefix\the\forest@temp@count @}%
%    \end{macrocode}
% Cache the projection of the new segment edge.
%    \begin{macrocode}
      \csedef{\forest@bp@prefix(\the\pgf@x,\the\pgf@y)}{\the\forest@temp@count}%
    \repeat
  \fi
%    \end{macrocode}
% Add the current point.
%    \begin{macrocode}
  \forest@breakpath@op{#1}{#2}%
%    \end{macrocode}
% Setup new ``previous'' info: the segment edge, its projection's
% index, and the projection.
%    \begin{macrocode}
  \def\forest@previous@x{#1}%
  \def\forest@previous@y{#2}%
  \let\forest@previous@i\forest@i
  \let\forest@previous@px\forest@px
  \let\forest@previous@py\forest@py
}
%    \end{macrocode}
%
% \subsection{Get tight edge of path}
%
% This is one of the central algorithms of the package. Given a simple
% path and a grow line, this method computes its (negative and
% positive) ``tight edge'', which we (informally) define as follows.
%
% Imagine an infinitely long light source parallel to the grow line,
% on the grow line's negative/positive side.\footnote{For the
% definition of negative/positive side, see forest@distancetogrowline
% in \S\ref{imp:projections}} Furthermore imagine that the path is
% opaque. Then the negative/positive tight edge of the path is the
% part of the path that is illuminated.
%
% This macro takes three arguments: |#1| is the path; |#2| and |#3|
% are macros which will receive the negative and the positive edge,
% respectively.  The edges are returned in the softpath format.  Grow
% line should be set before calling this macro.
%
% Enclose the computation in a \TeX\ group. This is actually quite
% crucial: if there was no enclosure, the temporary data (the segment
% dictionary, to be precise) computed by the prior invocations of the
% macro could corrupt the computation in the current invocation.
%    \begin{macrocode}
\def\forest@getnegativetightedgeofpath#1#2{%
  \forest@get@onetightedgeofpath#1\forest@sort@ascending#2}
\def\forest@getpositivetightedgeofpath#1#2{%
  \forest@get@onetightedgeofpath#1\forest@sort@descending#2}
\def\forest@get@onetightedgeofpath#1#2#3{%
  {%
    \forest@get@one@tightedgeofpath#1#2\forest@gep@edge
    \global\let\forest@gep@global@edge\forest@gep@edge
  }%
  \let#3\forest@gep@global@edge
}
\def\forest@get@one@tightedgeofpath#1#2#3{%
%    \end{macrocode}
% Project the path to the grow line and compile some useful information.
%    \begin{macrocode}
  \forest@projectpathtogrowline#1{forest@pp@}%
  \forest@sortprojections{forest@pp@}%
  \forest@processprojectioninfo{forest@pp@}{forest@pi@}%
%    \end{macrocode}
% Break the path.
%    \begin{macrocode}
  \forest@breakpath#1{forest@pi@}\forest@brokenpath
%    \end{macrocode}
% Compile some more useful information.
%    \begin{macrocode}
  \forest@sort@inner@arrays{forest@pi@}#2%
  \forest@pathtodict\forest@brokenpath{forest@pi@}%
%    \end{macrocode}
% The auxiliary data is set up: do the work!
%    \begin{macrocode}
  \forest@gettightedgeofpath@getedge
  \pgfsyssoftpath@getcurrentpath\forest@edge
%    \end{macrocode}
% Where possible, merge line segments of the path into a single line
% segment. This is an important optimization, since the edges of the
% subtrees are computed recursively. Not simplifying the edge could
% result in a wild growth of the length of the edge (in the sense of
% the number of segments).
%    \begin{macrocode}
  \forest@simplifypath\forest@edge#3%
}
%    \end{macrocode}
% Get both negative (stored in |#2|) and positive (stored in |#3|)
% edge of the path |#1|.
%    \begin{macrocode}
\def\forest@getbothtightedgesofpath#1#2#3{%
  {%
    \forest@get@one@tightedgeofpath#1\forest@sort@ascending\forest@gep@firstedge
%    \end{macrocode}
% Reverse the order of items in the inner arrays.
%    \begin{macrocode}
    \c@pgf@counta=0
    \loop
    \ifnum\c@pgf@counta<\forest@pi@n\relax
      \forest@ppi@deflet{forest@pi@\the\c@pgf@counta @}%
      \forest@reversearray\forest@ppi@let
        {0}%
        {\csname forest@pi@\the\c@pgf@counta @n\endcsname}%
      \advance\c@pgf@counta 1
    \repeat
%    \end{macrocode}
% Calling |\forest@gettightedgeofpath@getedge| now will result in the
% positive edge.
%    \begin{macrocode}
    \forest@gettightedgeofpath@getedge
    \pgfsyssoftpath@getcurrentpath\forest@edge
    \forest@simplifypath\forest@edge\forest@gep@secondedge
%    \end{macrocode}
% Smuggle the results out of the enclosing \TeX\ group.
%    \begin{macrocode}
      \global\let\forest@gep@global@firstedge\forest@gep@firstedge
      \global\let\forest@gep@global@secondedge\forest@gep@secondedge
    }%
  \let#2\forest@gep@global@firstedge
  \let#3\forest@gep@global@secondedge
}
%    \end{macrocode}
% 
% Sort the inner arrays of original points wrt the distance to the
% grow line. |#2| =
% |\forest@sort@ascending|/|\forest@sort@descending|.  (|\forest@loopa| is
% used here because quicksort uses |\loop|.)
%    \begin{macrocode}
\def\forest@sort@inner@arrays#1#2{%
  \c@pgf@counta=0
  \forest@loopa
  \ifnum\c@pgf@counta<\csname#1n\endcsname
    \c@pgf@countb=\csname#1\the\c@pgf@counta @n\endcsname\relax
    \ifnum\c@pgf@countb>1
      \advance\c@pgf@countb -1
      \forest@ppi@deflet{#1\the\c@pgf@counta @}%
      \forest@ppi@defcmp{#1\the\c@pgf@counta @}%
      \forest@sort\forest@ppi@cmp\forest@ppi@let#2{0}{\the\c@pgf@countb}%
    \fi
    \advance\c@pgf@counta 1
  \forest@repeata
}
%    \end{macrocode}
%    
% A macro that will define the item exchange macro for quicksorting
% the inner arrays of original points.  It takes one argument: the
% prefix of the inner array.
%    \begin{macrocode}
\def\forest@ppi@deflet#1{%
  \edef\forest@ppi@let##1##2{%
    \noexpand\csletcs{#1##1x}{#1##2x}%
    \noexpand\csletcs{#1##1y}{#1##2y}%
    \noexpand\csletcs{#1##1d}{#1##2d}%
  }%
}
%    \end{macrocode}
% A macro that will define the item-compare macro for quicksorting the
% embedded arrays of original points. It takes one argument: the
% prefix of the inner array.
%    \begin{macrocode}
\def\forest@ppi@defcmp#1{%
  \edef\forest@ppi@cmp##1##2{%
    \noexpand\forest@sort@cmpdimcs{#1##1d}{#1##2d}%
  }%
}
%    \end{macrocode}
%    
% Put path segments into a ``segment dictionary'': for each segment of
% the path from $(x_1,y_1)$ to $(x_2,y_2)$ let
% |\forest@(x1,y1)--(x2,y2)| be |\forest@inpath| (which can be
% anything but |\relax|).
%    \begin{macrocode}
\let\forest@inpath\advance
%    \end{macrocode}
% This macro is just a wrapper to process the path.
%    \begin{macrocode}
\def\forest@pathtodict#1#2{%
  \edef\forest@pathtodict@prefix{#2}%
  \forest@save@pgfsyssoftpath@tokendefs
  \let\pgfsyssoftpath@movetotoken\forest@pathtodict@movetoop
  \let\pgfsyssoftpath@linetotoken\forest@pathtodict@linetoop
  \def\forest@pathtodict@subpathstart{}%
  #1%
  \forest@restore@pgfsyssoftpath@tokendefs
}
%    \end{macrocode}
% When a move-to operation is encountered:
%    \begin{macrocode}
\def\forest@pathtodict@movetoop#1#2{%
%    \end{macrocode}
% If a subpath had just started, it was a degenerate one (a point). No
% need to store that (i.e.\ no code would use this information).  So,
% just remember that a new subpath has started.
%    \begin{macrocode}
  \def\forest@pathtodict@subpathstart{(#1,#2)-}%
}
%    \end{macrocode}
% When a line-to operation is encountered:
%    \begin{macrocode}
\def\forest@pathtodict@linetoop#1#2{%
%    \end{macrocode}
% If the subpath has just started, its start is also the start of the
% current segment.
%    \begin{macrocode}
\if\relax\forest@pathtodict@subpathstart\relax\else
    \let\forest@pathtodict@from\forest@pathtodict@subpathstart
  \fi  
%    \end{macrocode}
% Mark the segment as existing.
%    \begin{macrocode}
  \expandafter\let\csname\forest@pathtodict@prefix\forest@pathtodict@from-(#1,#2)\endcsname\forest@inpath
%    \end{macrocode}
% Set the start of the next segment to the current point, and mark
% that we are in the middle of a subpath.
%    \begin{macrocode}
  \def\forest@pathtodict@from{(#1,#2)-}%
  \def\forest@pathtodict@subpathstart{}%
}
%    \end{macrocode}
%    
% In this macro, the edge is actually computed.
%    \begin{macrocode}
\def\forest@gettightedgeofpath@getedge{%
%    \end{macrocode}
% Clear the path and the last projection.
%    \begin{macrocode}
  \pgfsyssoftpath@setcurrentpath\pgfutil@empty
  \let\forest@last@x\relax
  \let\forest@last@y\relax
%    \end{macrocode}
% Loop through the (ordered) array of projections. (Since we will be
% dealing with the current and the next projection in each iteration
% of the loop, we loop the counter from the first to the
% second-to-last projection.)
%    \begin{macrocode}
  \c@pgf@counta=0
  \forest@temp@count=\forest@pi@n\relax  
  \advance\forest@temp@count -1
  \edef\forest@nminusone{\the\forest@temp@count}%
  \forest@loopa
  \ifnum\c@pgf@counta<\forest@nminusone\relax
    \forest@gettightedgeofpath@getedge@loopa
  \forest@repeata
%    \end{macrocode}
% A special case: the edge ends with a degenerate subpath (a
% point). 
%    \begin{macrocode}
  \ifnum\forest@nminusone<\forest@n\relax\else
    \ifnum\csname forest@pi@\forest@nminusone @n\endcsname>0
      \forest@gettightedgeofpath@maybemoveto{\forest@nminusone}{0}%
    \fi
  \fi
}
%    \end{macrocode}
% The body of a loop containing an embedded loop must be put in a
% separate macro because it contains the |\if...| of the embedded
% |\loop...| without the matching |\fi|: |\fi| is ``hiding'' in the
% embedded |\loop|, which has not been expanded yet.
%    \begin{macrocode}
\def\forest@gettightedgeofpath@getedge@loopa{%
    \ifnum\csname forest@pi@\the\c@pgf@counta @n\endcsname>0
%    \end{macrocode}
% Degenerate case: a subpath of the edge is a point.
%    \begin{macrocode}
      \forest@gettightedgeofpath@maybemoveto{\the\c@pgf@counta}{0}%
%    \end{macrocode}
% Loop through points projecting to the current projection. The
% preparations above guarantee that the points are ordered (either in
% the ascending or the descending order) with respect to their
% distance to the grow line.
%    \begin{macrocode}
      \c@pgf@countb=0
      \forest@loopb
      \ifnum\c@pgf@countb<\csname forest@pi@\the\c@pgf@counta @n\endcsname\relax
        \forest@gettightedgeofpath@getedge@loopb
      \forest@repeatb
    \fi
    \advance\c@pgf@counta 1
}
%    \end{macrocode}
% Loop through points projecting to the next projection.  Again, the
% points are ordered.
%    \begin{macrocode}
\def\forest@gettightedgeofpath@getedge@loopb{%
        \c@pgf@countc=0
        \advance\c@pgf@counta 1
        \edef\forest@aplusone{\the\c@pgf@counta}%
        \advance\c@pgf@counta -1
        \forest@loopc        
        \ifnum\c@pgf@countc<\csname forest@pi@\forest@aplusone @n\endcsname\relax
%    \end{macrocode}
% Test whether [the current point]--[the next point] or [the next
% point]--[the current point] is a segment in the (broken) path.  The
% first segment found is the one with the minimal/maximal distance
% (depending on the sort order of arrays of points projecting to the
% same projection) to the grow line.
%
% Note that for this to work in all cases, the original path should
% have been broken on its self-intersections.  However, a careful
% reader will probably remember that |\forest@breakpath| does
% \emph{not} break the path at its self-intersections.  This is
% omitted for performance reasons.  Given the intended use of the
% algorithm (calculating edges of subtrees), self-intersecting paths
% cannot arise anyway, if only the node boundaries are
% non-self-intersecting. So, a warning: if you develop a new shape and
% write a macro computing its boundary, make sure that the computed
% boundary path is non-self-intersecting!
%    \begin{macrocode}
          \forest@tempfalse
          \expandafter\ifx\csname forest@pi@(%
            \csname forest@pi@\the\c@pgf@counta @\the\c@pgf@countb x\endcsname,%
            \csname forest@pi@\the\c@pgf@counta @\the\c@pgf@countb y\endcsname)--(%
            \csname forest@pi@\forest@aplusone @\the\c@pgf@countc x\endcsname,%
            \csname forest@pi@\forest@aplusone @\the\c@pgf@countc y\endcsname)%
            \endcsname\forest@inpath
            \forest@temptrue
          \else
            \expandafter\ifx\csname forest@pi@(%
              \csname forest@pi@\forest@aplusone @\the\c@pgf@countc x\endcsname,%
              \csname forest@pi@\forest@aplusone @\the\c@pgf@countc y\endcsname)--(%
              \csname forest@pi@\the\c@pgf@counta @\the\c@pgf@countb x\endcsname,%
              \csname forest@pi@\the\c@pgf@counta @\the\c@pgf@countb y\endcsname)%
              \endcsname\forest@inpath
              \forest@temptrue
            \fi
          \fi
          \ifforest@temp
%    \end{macrocode}
% We have found the segment with the minimal/maximal distance to the
% grow line. So let's add it to the edge path.
%
% First, deal with the
% start point of the edge: check if the current point is the last
% point. If that is the case (this happens if the current point was
% the end point of the last segment added to the edge), nothing needs
% to be done; otherwise (this happens if the current point will start
% a new subpath of the edge), move to the current point, and update
% the last-point macros.
%    \begin{macrocode}
            \forest@gettightedgeofpath@maybemoveto{\the\c@pgf@counta}{\the\c@pgf@countb}%
%    \end{macrocode}
% Second, create a line to the end point.
%    \begin{macrocode}
            \edef\forest@last@x{%
              \csname forest@pi@\forest@aplusone @\the\c@pgf@countc x\endcsname}%
            \edef\forest@last@y{%
              \csname forest@pi@\forest@aplusone @\the\c@pgf@countc y\endcsname}%
            \pgfsyssoftpath@lineto\forest@last@x\forest@last@y
%    \end{macrocode}
% Finally, ``break'' out of the |\forest@loopc| and |\forest@loopb|.
%    \begin{macrocode}
            \c@pgf@countc=\csname forest@pi@\forest@aplusone @n\endcsname
            \c@pgf@countb=\csname forest@pi@\the\c@pgf@counta @n\endcsname
          \fi
          \advance\c@pgf@countc 1
        \forest@repeatc
        \advance\c@pgf@countb 1
}
%    \end{macrocode}
% |\forest@#1@| is an (ordered) array of points projecting to
% projection with index |#1|. Check if |#2|th point of that array
% equals the last point added to the edge: if not, add it.
%    \begin{macrocode}
\def\forest@gettightedgeofpath@maybemoveto#1#2{%
  \forest@temptrue
  \ifx\forest@last@x\relax\else
    \ifdim\forest@last@x=\csname forest@pi@#1@#2x\endcsname\relax
      \ifdim\forest@last@y=\csname forest@pi@#1@#2y\endcsname\relax
        \forest@tempfalse
      \fi
    \fi
  \fi
  \ifforest@temp
    \edef\forest@last@x{\csname forest@pi@#1@#2x\endcsname}%
    \edef\forest@last@y{\csname forest@pi@#1@#2y\endcsname}%
    \pgfsyssoftpath@moveto\forest@last@x\forest@last@y
  \fi
}
%    \end{macrocode}
%
% Simplify the resulting path by ``unbreaking'' segments where
% possible. (The macro itself is just a wrapper for path processing
% macros below.)
%    \begin{macrocode}
\def\forest@simplifypath#1#2{%
  \pgfsyssoftpath@setcurrentpath\pgfutil@empty
  \forest@save@pgfsyssoftpath@tokendefs
  \let\pgfsyssoftpath@movetotoken\forest@simplifypath@moveto
  \let\pgfsyssoftpath@linetotoken\forest@simplifypath@lineto
  \let\forest@last@x\relax
  \let\forest@last@y\relax
  \let\forest@last@atan\relax
  #1%
  \ifx\forest@last@x\relax\else
    \ifx\forest@last@atan\relax\else
      \pgfsyssoftpath@lineto\forest@last@x\forest@last@y
    \fi
  \fi  
  \forest@restore@pgfsyssoftpath@tokendefs
  \pgfsyssoftpath@getcurrentpath#2%
}
%    \end{macrocode}
% When a move-to is encountered, we flush whatever segment we were
% building, make the move, remember the last position, and set the
% slope to unknown.
%    \begin{macrocode}
\def\forest@simplifypath@moveto#1#2{%
  \ifx\forest@last@x\relax\else
    \pgfsyssoftpath@lineto\forest@last@x\forest@last@y
  \fi
  \pgfsyssoftpath@moveto{#1}{#2}%
  \def\forest@last@x{#1}%
  \def\forest@last@y{#2}%
  \let\forest@last@atan\relax
}
%    \end{macrocode}
% How much may the segment slopes differ that we can still merge them?
% (Ignore |pt|, these are degrees.)  Also, how good is this number?
%    \begin{macrocode}
\def\forest@getedgeofpath@precision{1pt}
%    \end{macrocode}
% When a line-to is encountered\dots
%    \begin{macrocode}
\def\forest@simplifypath@lineto#1#2{%
  \ifx\forest@last@x\relax
%    \end{macrocode}
% If we're not in the middle of a merger, we need to nothing but start
% it.
%    \begin{macrocode}
    \def\forest@last@x{#1}%
    \def\forest@last@y{#2}%
    \let\forest@last@atan\relax
  \else
%    \end{macrocode}
% Otherwise, we calculate the slope of the current segment (i.e.\ the
% segment between the last and the current point), \dots
%    \begin{macrocode}
    \pgfpointdiff{\pgfqpoint{#1}{#2}}{\pgfqpoint{\forest@last@x}{\forest@last@y}}%
    \ifdim\pgf@x<\pgfintersectiontolerance
      \ifdim-\pgf@x<\pgfintersectiontolerance
        \pgf@x=0pt
      \fi
    \fi
    \csname pgfmathatan2\endcsname{\pgf@x}{\pgf@y}%
    \let\forest@current@atan\pgfmathresult
    \ifx\forest@last@atan\relax
%    \end{macrocode}
% If this is the first segment in the current merger, simply remember
% the slope and the last point.
%    \begin{macrocode}
      \def\forest@last@x{#1}%
      \def\forest@last@y{#2}%      
      \let\forest@last@atan\forest@current@atan
    \else
%    \end{macrocode}
% Otherwise, compare the first and the current slope.
%    \begin{macrocode}
      \pgfutil@tempdima=\forest@current@atan pt
      \advance\pgfutil@tempdima -\forest@last@atan pt
      \ifdim\pgfutil@tempdima<0pt\relax
        \multiply\pgfutil@tempdima -1
      \fi
      \ifdim\pgfutil@tempdima<\forest@getedgeofpath@precision\relax 
      \else
%    \end{macrocode}
% If the slopes differ too much, flush the path up to the previous
% segment, and set up a new first slope.
%    \begin{macrocode}
        \pgfsyssoftpath@lineto\forest@last@x\forest@last@y        
        \let\forest@last@atan\forest@current@atan
      \fi
%    \end{macrocode}
% In any event, update the last point.
%    \begin{macrocode}
      \def\forest@last@x{#1}%
      \def\forest@last@y{#2}%
    \fi
  \fi
}
%    \end{macrocode}
%
%
% \subsection{Get rectangle/band edge}
%
%    \begin{macrocode}
\def\forest@getnegativerectangleedgeofpath#1#2{%
  \forest@getnegativerectangleorbandedgeofpath{#1}{#2}{\the\pgf@xb}}
\def\forest@getpositiverectangleedgeofpath#1#2{%
  \forest@getpositiverectangleorbandedgeofpath{#1}{#2}{\the\pgf@xb}}
\def\forest@getbothrectangleedgesofpath#1#2#3{%
  \forest@getbothrectangleorbandedgesofpath{#1}{#2}{#3}{\the\pgf@xb}}
\def\forest@bandlength{5000pt} % something large (ca. 180cm), but still manageable for TeX without producing `too large' errors
\def\forest@getnegativebandedgeofpath#1#2{%
  \forest@getnegativerectangleorbandedgeofpath{#1}{#2}{\forest@bandlength}}
\def\forest@getpositivebandedgeofpath#1#2{%
  \forest@getpositiverectangleorbandedgeofpath{#1}{#2}{\forest@bandlength}}
\def\forest@getbothbandedgesofpath#1#2#3{%
  \forest@getbothrectangleorbandedgesofpath{#1}{#2}{#3}{\forest@bandlength}}
\def\forest@getnegativerectangleorbandedgeofpath#1#2#3{%
  \forest@path@getboundingrectangle@ls#1{\forest@grow}%
  \edef\forest@gre@path{%
    \noexpand\pgfsyssoftpath@movetotoken{\the\pgf@xa}{\the\pgf@ya}%
    \noexpand\pgfsyssoftpath@linetotoken{#3}{\the\pgf@ya}%
  }%
  {%
    \pgftransformreset
    \pgftransformrotate{\forest@grow}%
    \forest@pgfpathtransformed\forest@gre@path
  }%
  \pgfsyssoftpath@getcurrentpath#2%
}
\def\forest@getpositiverectangleorbandedgeofpath#1#2#3{%
  \forest@path@getboundingrectangle@ls#1{\forest@grow}%
  \edef\forest@gre@path{%
    \noexpand\pgfsyssoftpath@movetotoken{\the\pgf@xa}{\the\pgf@yb}%
    \noexpand\pgfsyssoftpath@linetotoken{#3}{\the\pgf@yb}%
  }%
  {%
    \pgftransformreset
    \pgftransformrotate{\forest@grow}%
    \forest@pgfpathtransformed\forest@gre@path
  }%
  \pgfsyssoftpath@getcurrentpath#2%
}
\def\forest@getbothrectangleorbandedgesofpath#1#2#3#4{%
  \forest@path@getboundingrectangle@ls#1{\forest@grow}%
  \edef\forest@gre@negpath{%
    \noexpand\pgfsyssoftpath@movetotoken{\the\pgf@xa}{\the\pgf@ya}%
    \noexpand\pgfsyssoftpath@linetotoken{#4}{\the\pgf@ya}%
  }%
  \edef\forest@gre@pospath{%
    \noexpand\pgfsyssoftpath@movetotoken{\the\pgf@xa}{\the\pgf@yb}%
    \noexpand\pgfsyssoftpath@linetotoken{#4}{\the\pgf@yb}%
  }%
  {%
    \pgftransformreset
    \pgftransformrotate{\forest@grow}%
    \forest@pgfpathtransformed\forest@gre@negpath
  }%
  \pgfsyssoftpath@getcurrentpath#2%
  {%
    \pgftransformreset
    \pgftransformrotate{\forest@grow}%
    \forest@pgfpathtransformed\forest@gre@pospath
  }%
  \pgfsyssoftpath@getcurrentpath#3%
}
%    \end{macrocode}
%
% \subsection{Distance between paths}
% \label{imp:distance}
% 
% Another crucial part of the package.
%
%    \begin{macrocode}
\def\forest@distance@between@edge@paths#1#2#3{%
  % #1, #2 = (edge) paths
  %
  % project paths
  \forest@projectpathtogrowline#1{forest@p1@}%
  \forest@projectpathtogrowline#2{forest@p2@}%
  % merge projections (the lists are sorted already, because edge
  % paths are |sorted|)
  \forest@dbep@mergeprojections
    {forest@p1@}{forest@p2@}%
    {forest@P1@}{forest@P2@}%
  % process projections
  \forest@processprojectioninfo{forest@P1@}{forest@PI1@}%
  \forest@processprojectioninfo{forest@P2@}{forest@PI2@}%
  % break paths
  \forest@breakpath#1{forest@PI1@}\forest@broken@one
  \forest@breakpath#2{forest@PI2@}\forest@broken@two
  % sort inner arrays ---optimize: it's enough to find max and min
  \forest@sort@inner@arrays{forest@PI1@}\forest@sort@descending
  \forest@sort@inner@arrays{forest@PI2@}\forest@sort@ascending
  % compute the distance
  \let\forest@distance\relax
  \c@pgf@countc=0
  \loop
  \ifnum\c@pgf@countc<\csname forest@PI1@n\endcsname\relax
    \ifnum\csname forest@PI1@\the\c@pgf@countc @n\endcsname=0 \else
      \ifnum\csname forest@PI2@\the\c@pgf@countc @n\endcsname=0 \else
        \pgfutil@tempdima=\csname forest@PI2@\the\c@pgf@countc @0d\endcsname\relax
        \advance\pgfutil@tempdima -\csname forest@PI1@\the\c@pgf@countc @0d\endcsname\relax
        \ifx\forest@distance\relax
          \edef\forest@distance{\the\pgfutil@tempdima}%
        \else 
          \ifdim\pgfutil@tempdima<\forest@distance\relax
            \edef\forest@distance{\the\pgfutil@tempdima}%
          \fi
        \fi
      \fi
    \fi
    \advance\c@pgf@countc 1
  \repeat
  \let#3\forest@distance
}
  % merge projections: we need two projection arrays, both containing
  % projection points from both paths, but each with the original
  % points from only one path
\def\forest@dbep@mergeprojections#1#2#3#4{%
  % TODO: optimize: v bistvu ni treba sortirat, ker je edge path že sortiran
  \forest@sortprojections{#1}%
  \forest@sortprojections{#2}%
  \c@pgf@counta=0
  \c@pgf@countb=0
  \c@pgf@countc=0
  \edef\forest@input@prefix@one{#1}%
  \edef\forest@input@prefix@two{#2}%
  \edef\forest@output@prefix@one{#3}%
  \edef\forest@output@prefix@two{#4}%
  \forest@dbep@mp@iterate
  \csedef{#3n}{\the\c@pgf@countc}%
  \csedef{#4n}{\the\c@pgf@countc}%
}
\def\forest@dbep@mp@iterate{%
  \let\forest@dbep@mp@next\forest@dbep@mp@iterate
  \ifnum\c@pgf@counta<\csname\forest@input@prefix@one n\endcsname\relax
    \ifnum\c@pgf@countb<\csname\forest@input@prefix@two n\endcsname\relax
      \let\forest@dbep@mp@next\forest@dbep@mp@do
    \else
      \let\forest@dbep@mp@next\forest@dbep@mp@iteratefirst
    \fi
  \else
    \ifnum\c@pgf@countb<\csname\forest@input@prefix@two n\endcsname\relax
      \let\forest@dbep@mp@next\forest@dbep@mp@iteratesecond
    \else
      \let\forest@dbep@mp@next\relax
    \fi
  \fi
  \forest@dbep@mp@next
}
\def\forest@dbep@mp@do{%
  \forest@sort@cmptwodimcs%
    {\forest@input@prefix@one\the\c@pgf@counta xp}%
    {\forest@input@prefix@one\the\c@pgf@counta yp}%
    {\forest@input@prefix@two\the\c@pgf@countb xp}%
    {\forest@input@prefix@two\the\c@pgf@countb yp}%    
  \if\forest@sort@cmp@result=%
    \forest@dbep@mp@@store@p\forest@input@prefix@one\c@pgf@counta
    \forest@dbep@mp@@store@o\forest@input@prefix@one
        \c@pgf@counta\forest@output@prefix@one
    \forest@dbep@mp@@store@o\forest@input@prefix@two
        \c@pgf@countb\forest@output@prefix@two
    \advance\c@pgf@counta 1
    \advance\c@pgf@countb 1
  \else
    \if\forest@sort@cmp@result>%
      \forest@dbep@mp@@store@p\forest@input@prefix@two\c@pgf@countb
      \forest@dbep@mp@@store@o\forest@input@prefix@two
          \c@pgf@countb\forest@output@prefix@two
      \advance\c@pgf@countb 1
    \else%<
      \forest@dbep@mp@@store@p\forest@input@prefix@one\c@pgf@counta
      \forest@dbep@mp@@store@o\forest@input@prefix@one
          \c@pgf@counta\forest@output@prefix@one
      \advance\c@pgf@counta 1
    \fi
  \fi
  \advance\c@pgf@countc 1
  \forest@dbep@mp@iterate
}
\def\forest@dbep@mp@@store@p#1#2{%
  \csletcs
    {\forest@output@prefix@one\the\c@pgf@countc xp}%
    {#1\the#2xp}%
  \csletcs
    {\forest@output@prefix@one\the\c@pgf@countc yp}%
    {#1\the#2yp}%
  \csletcs
    {\forest@output@prefix@two\the\c@pgf@countc xp}%
    {#1\the#2xp}%
  \csletcs
    {\forest@output@prefix@two\the\c@pgf@countc yp}%
    {#1\the#2yp}%
}
\def\forest@dbep@mp@@store@o#1#2#3{%
  \csletcs{#3\the\c@pgf@countc xo}{#1\the#2xo}%
  \csletcs{#3\the\c@pgf@countc yo}{#1\the#2yo}%
}
\def\forest@dbep@mp@iteratefirst{%
  \forest@dbep@mp@iterateone\forest@input@prefix@one\c@pgf@counta\forest@output@prefix@one
}
\def\forest@dbep@mp@iteratesecond{%
  \forest@dbep@mp@iterateone\forest@input@prefix@two\c@pgf@countb\forest@output@prefix@two
}
\def\forest@dbep@mp@iterateone#1#2#3{%
  \loop
  \ifnum#2<\csname#1n\endcsname\relax
    \forest@dbep@mp@@store@p#1#2%
    \forest@dbep@mp@@store@o#1#2#3%
    \advance\c@pgf@countc 1
    \advance#21
  \repeat
}
%    \end{macrocode}
% 
% \subsection{Utilities}
%
% Equality test: points are considered equal if they differ less than
% |\pgfintersectiontolerance| in each coordinate.
%    \begin{macrocode}
\newif\ifforest@equaltotolerance
\def\forest@equaltotolerance#1#2{{%
  \pgfpointdiff{#1}{#2}%
  \ifdim\pgf@x<0pt \multiply\pgf@x -1 \fi
  \ifdim\pgf@y<0pt \multiply\pgf@y -1 \fi
  \global\forest@equaltotolerancefalse
  \ifdim\pgf@x<\pgfintersectiontolerance\relax
    \ifdim\pgf@y<\pgfintersectiontolerance\relax
      \global\forest@equaltotolerancetrue
    \fi
  \fi
}}
%    \end{macrocode}
%    
% Save/restore |pgf|s |\pgfsyssoftpath@...token| definitions.
%    \begin{macrocode}
\def\forest@save@pgfsyssoftpath@tokendefs{%
  \let\forest@origmovetotoken\pgfsyssoftpath@movetotoken
  \let\forest@origlinetotoken\pgfsyssoftpath@linetotoken
  \let\forest@origcurvetosupportatoken\pgfsyssoftpath@curvetosupportatoken
  \let\forest@origcurvetosupportbtoken\pgfsyssoftpath@curvetosupportbtoken
  \let\forest@origcurvetotoken\pgfsyssoftpath@curvetototoken
  \let\forest@origrectcornertoken\pgfsyssoftpath@rectcornertoken
  \let\forest@origrectsizetoken\pgfsyssoftpath@rectsizetoken
  \let\forest@origclosepathtoken\pgfsyssoftpath@closepathtoken  
  \let\pgfsyssoftpath@movetotoken\forest@badtoken
  \let\pgfsyssoftpath@linetotoken\forest@badtoken
  \let\pgfsyssoftpath@curvetosupportatoken\forest@badtoken
  \let\pgfsyssoftpath@curvetosupportbtoken\forest@badtoken
  \let\pgfsyssoftpath@curvetototoken\forest@badtoken
  \let\pgfsyssoftpath@rectcornertoken\forest@badtoken
  \let\pgfsyssoftpath@rectsizetoken\forest@badtoken
  \let\pgfsyssoftpath@closepathtoken\forest@badtoken
}
\def\forest@badtoken{%
  \PackageError{forest}{This token should not be in this path}{}%
}
\def\forest@restore@pgfsyssoftpath@tokendefs{%
  \let\pgfsyssoftpath@movetotoken\forest@origmovetotoken
  \let\pgfsyssoftpath@linetotoken\forest@origlinetotoken
  \let\pgfsyssoftpath@curvetosupportatoken\forest@origcurvetosupportatoken
  \let\pgfsyssoftpath@curvetosupportbtoken\forest@origcurvetosupportbtoken
  \let\pgfsyssoftpath@curvetototoken\forest@origcurvetotoken
  \let\pgfsyssoftpath@rectcornertoken\forest@origrectcornertoken
  \let\pgfsyssoftpath@rectsizetoken\forest@origrectsizetoken
  \let\pgfsyssoftpath@closepathtoken\forest@origclosepathtoken
}
%    \end{macrocode}
%
% Extend path |#1| with path |#2| translated by point |#3|.
%    \begin{macrocode}
\def\forest@extendpath#1#2#3{%
  \pgf@process{#3}%
  \pgfsyssoftpath@setcurrentpath#1%
  \forest@save@pgfsyssoftpath@tokendefs
  \let\pgfsyssoftpath@movetotoken\forest@extendpath@moveto
  \let\pgfsyssoftpath@linetotoken\forest@extendpath@lineto
  #2%
  \forest@restore@pgfsyssoftpath@tokendefs
  \pgfsyssoftpath@getcurrentpath#1%
}
\def\forest@extendpath@moveto#1#2{%
  \forest@extendpath@do{#1}{#2}\pgfsyssoftpath@moveto
}
\def\forest@extendpath@lineto#1#2{%
  \forest@extendpath@do{#1}{#2}\pgfsyssoftpath@lineto
}
\def\forest@extendpath@do#1#2#3{%
  {%
    \advance\pgf@x #1
    \advance\pgf@y #2
    #3{\the\pgf@x}{\the\pgf@y}%
  }%
}
%    \end{macrocode}
%
% Get bounding rectangle of the path. |#1| = the path, |#2| = grow.
% Returns (|\pgf@xa|=min x/l, |\pgf@ya|=max y/s, |\pgf@xb|=min x/l, |\pgf@yb|=max y/s).  (If path |#1|
% is empty, the result is undefined.)
%    \begin{macrocode}
\def\forest@path@getboundingrectangle@ls#1#2{%
  {%
    \pgftransformreset
    \pgftransformrotate{-(#2)}%
    \forest@pgfpathtransformed#1%
  }%
  \pgfsyssoftpath@getcurrentpath\forest@gbr@rotatedpath
  \forest@path@getboundingrectangle@xy\forest@gbr@rotatedpath
}
\def\forest@path@getboundingrectangle@xy#1{%
  \forest@save@pgfsyssoftpath@tokendefs
  \let\pgfsyssoftpath@movetotoken\forest@gbr@firstpoint
  \let\pgfsyssoftpath@linetotoken\forest@gbr@firstpoint
  #1%
  \forest@restore@pgfsyssoftpath@tokendefs
}
\def\forest@gbr@firstpoint#1#2{%
  \pgf@xa=#1 \pgf@xb=#1 \pgf@ya=#2 \pgf@yb=#2
  \let\pgfsyssoftpath@movetotoken\forest@gbr@point
  \let\pgfsyssoftpath@linetotoken\forest@gbr@point
}
\def\forest@gbr@point#1#2{%
  \ifdim#1<\pgf@xa\relax\pgf@xa=#1 \fi
  \ifdim#1>\pgf@xb\relax\pgf@xb=#1 \fi
  \ifdim#2<\pgf@ya\relax\pgf@ya=#2 \fi
  \ifdim#2>\pgf@yb\relax\pgf@yb=#2 \fi
}
%    \end{macrocode}
%
% \section{The outer UI}
% 
% \subsection{Package options}
%
%    \begin{macrocode}
\newif\ifforesttikzcshack
\foresttikzcshacktrue
\newif\ifforest@install@keys@to@tikz@path@
\forest@install@keys@to@tikz@path@true
\forestset{package@options/.cd,
  external/.is if=forest@external@,
  tikzcshack/.is if=foresttikzcshack,
  tikzinstallkeys/.is if=forest@install@keys@to@tikz@path@,
}
%    \end{macrocode}
% \subsection{Externalization}
%    \begin{macrocode}
\pgfkeys{/forest/external/.cd,
  copy command/.initial={cp "\source" "\target"},
  optimize/.is if=forest@external@optimize@,
  context/.initial={%
    \forestOve{\csname forest@id@of@standard node\endcsname}{environment@formula}},
  depends on macro/.style={context/.append/.expanded={%
      \expandafter\detokenize\expandafter{#1}}},
}
\def\forest@external@copy#1#2{%
  \pgfkeysgetvalue{/forest/external/copy command}\forest@copy@command
  \ifx\forest@copy@command\pgfkeysnovalue\else
    \IfFileExists{#1}{%
      {%
        \def\source{#1}%
        \def\target{#2}%
        \immediate\write18{\forest@copy@command}%
      }%
    }{}%
  \fi
}
\newif\ifforest@external@
\newif\ifforest@external@optimize@
\forest@external@optimize@true
\ProcessPgfPackageOptions{/forest/package@options}
\ifforest@install@keys@to@tikz@path@
  \tikzset{fit to tree/.style={/forest/fit to tree}}
\fi
\ifforest@external@
  \ifdefined\tikzexternal@tikz@replacement\else
    \usetikzlibrary{external}%
  \fi
  \pgfkeys{%
    /tikz/external/failed ref warnings for={},
    /pgf/images/aux in dpth=false,
  }%
  \tikzifexternalizing{}{%
    \forest@external@copy{\jobname.aux}{\jobname.aux.copy}%
  }%
  \AtBeginDocument{%
    \tikzifexternalizing{%
      \IfFileExists{\tikzexternalrealjob.aux.copy}{%
        \makeatletter
        \input \tikzexternalrealjob.aux.copy
        \makeatother
      }{}%
    }{%
      \newwrite\forest@auxout
      \immediate\openout\forest@auxout=\tikzexternalrealjob.for.tmp
    }%
    \IfFileExists{\tikzexternalrealjob.for}{%
      {%
        \makehashother\makeatletter
        \input \tikzexternalrealjob.for
      }%
    }{}%
  }%
  \AtEndDocument{%
    \tikzifexternalizing{}{%
      \immediate\closeout\forest@auxout
      \forest@external@copy{\jobname.for.tmp}{\jobname.for}%
    }%
  }%
\fi
%    \end{macrocode}
% 
% \subsection{The \texttt{forest} environment}
% \label{imp:forest-environment}
%
% There are three ways to invoke \foRest;: the environent and the starless and the starred version
% of the macro.  The latter creates no group.
% 
% Most of the code in this section deals with externalization. 
%
%    \begin{macrocode}
\newenvironment{forest}{\pgfkeysalso{/forest/begin forest}\Collect@Body\forest@env}{}
\long\def\Forest{\pgfkeysalso{/forest/begin forest}\@ifnextchar*{\forest@nogroup}{\forest@group}}
\def\forest@group#1{{\forest@env{#1}}}
\def\forest@nogroup*#1{\forest@env{#1}}
\newif\ifforest@externalize@tree@
\newif\ifforest@was@tikzexternalwasenable
\long\def\forest@env#1{%
  \let\forest@external@next\forest@begin
  \forest@was@tikzexternalwasenablefalse
  \ifdefined\tikzexternal@tikz@replacement
    \ifx\tikz\tikzexternal@tikz@replacement
      \forest@was@tikzexternalwasenabletrue
      \tikzexternaldisable
    \fi
  \fi
  \forest@externalize@tree@false
  \ifforest@external@
    \ifforest@was@tikzexternalwasenable
      \tikzifexternalizing{%
        \let\forest@external@next\forest@begin@externalizing
      }{%
        \let\forest@external@next\forest@begin@externalize
      }%
    \fi
  \fi
  \forest@standardnode@calibrate
  \forest@external@next{#1}%
}
%    \end{macrocode}
% We're externalizing, i.e.\ this code gets executed in the embedded call.
%    \begin{macrocode}
\long\def\forest@begin@externalizing#1{%
  \forest@external@setup{#1}%
  \let\forest@external@next\forest@begin
  \forest@externalize@inner@n=-1
  \ifforest@external@optimize@\forest@externalizing@maybeoptimize\fi
  \forest@external@next{#1}%
  \tikzexternalenable
}
\def\forest@externalizing@maybeoptimize{%
  \edef\forest@temp{\tikzexternalrealjob-forest-\forest@externalize@outer@n}%
  \edef\forest@marshal{%
    \noexpand\pgfutil@in@
      {\expandafter\detokenize\expandafter{\forest@temp}.}
      {\expandafter\detokenize\expandafter{\pgfactualjobname}.}%
  }\forest@marshal
  \ifpgfutil@in@
  \else
    \let\forest@external@next\@gobble
  \fi
}
%    \end{macrocode}
% Externalization is enabled, we're in the outer process, deciding if the picture is up-to-date.
%    \begin{macrocode}
\long\def\forest@begin@externalize#1{%
  \forest@external@setup{#1}%
  \iftikzexternal@file@isuptodate
    \setbox0=\hbox{%
      \csname forest@externalcheck@\forest@externalize@outer@n\endcsname
    }%
  \fi
  \iftikzexternal@file@isuptodate
    \csname forest@externalload@\forest@externalize@outer@n\endcsname
  \else
    \forest@externalize@tree@true
    \forest@externalize@inner@n=-1
    \forest@begin{#1}%
    \ifcsdef{forest@externalize@@\forest@externalize@id}{}{%
      \immediate\write\forest@auxout{%
        \noexpand\forest@external
        {\forest@externalize@outer@n}%
        {\expandafter\detokenize\expandafter{\forest@externalize@id}}%
        {\expandonce\forest@externalize@checkimages}%
        {\expandonce\forest@externalize@loadimages}%
      }%
    }%
  \fi
  \tikzexternalenable
}
\def\forest@includeexternal@check#1{%
  \tikzsetnextfilename{#1}%
  \tikzexternal@externalizefig@systemcall@uptodatecheck
}
\def\makehashother{\catcode`\#=12}%
\long\def\forest@external@setup#1{%
  % set up \forest@externalize@id and \forest@externalize@outer@n
  % we need to deal with #s correctly (\write doubles them)
  \setbox0=\hbox{\makehashother\makeatletter
    \scantokens{\forest@temp@toks{#1}}\expandafter
  }%
  \expandafter\forest@temp@toks\expandafter{\the\forest@temp@toks}%
  \edef\forest@temp{\pgfkeysvalueof{/forest/external/context}}%
  \edef\forest@externalize@id{%
    \expandafter\detokenize\expandafter{\forest@temp}%
    @@%
    \expandafter\detokenize\expandafter{\the\forest@temp@toks}%
  }%
  \letcs\forest@externalize@outer@n{forest@externalize@@\forest@externalize@id}%
  \ifdefined\forest@externalize@outer@n
    \global\tikzexternal@file@isuptodatetrue
  \else
    \global\advance\forest@externalize@max@outer@n 1
    \edef\forest@externalize@outer@n{\the\forest@externalize@max@outer@n}%
    \global\tikzexternal@file@isuptodatefalse
  \fi
  \def\forest@externalize@loadimages{}%
  \def\forest@externalize@checkimages{}%
}
\newcount\forest@externalize@max@outer@n
\global\forest@externalize@max@outer@n=0
\newcount\forest@externalize@inner@n
%    \end{macrocode}
% The \texttt{.for} file is a string of calls of this macro.
%    \begin{macrocode}
\long\def\forest@external#1#2#3#4{% #1=n,#2=context+source code,#3=update check code, #4=load code
  \ifnum\forest@externalize@max@outer@n<#1
    \global\forest@externalize@max@outer@n=#1
  \fi
  \global\csdef{forest@externalize@@\detokenize{#2}}{#1}%
  \global\csdef{forest@externalcheck@#1}{#3}%
  \global\csdef{forest@externalload@#1}{#4}%
  \tikzifexternalizing{}{%
    \immediate\write\forest@auxout{%
      \noexpand\forest@external{#1}%
      {\expandafter\detokenize\expandafter{#2}}%
      {\unexpanded{#3}}%
      {\unexpanded{#4}}%
    }%
  }%
}
%    \end{macrocode}
% These two macros include the external picture.
%    \begin{macrocode}
\def\forest@includeexternal#1{%
  \edef\forest@temp{\pgfkeysvalueof{/forest/external/context}}%
  \typeout{forest: Including external picture '#1' for forest context+code:
    '\expandafter\detokenize\expandafter{\forest@externalize@id}'}%
  {%
    %\def\pgf@declaredraftimage##1##2{\def\pgf@image{\hbox{}}}%
    \tikzsetnextfilename{#1}%
    \tikzexternalenable
    \tikz{}%
  }%
}
\def\forest@includeexternal@box#1#2{%
  \global\setbox#1=\hbox{\forest@includeexternal{#2}}%
}
%    \end{macrocode}
% This code runs the bracket parser and stage processing.
%    \begin{macrocode}
\long\def\forest@begin#1{%
  \iffalse{\fi\forest@parsebracket#1}%
}
\def\forest@parsebracket{%
  \bracketParse{\forest@get@root@afterthought}\forest@root=%
}
\def\forest@get@root@afterthought{%
  \expandafter\forest@get@root@afterthought@\expandafter{\iffalse}\fi
}
\long\def\forest@get@root@afterthought@#1{%
  \ifblank{#1}{}{%
    \forestOeappto{\forest@root}{given options}{,afterthought={\unexpanded{#1}}}%
  }%
  \forest@do
}
\def\forest@do{%
  \forest@node@Compute@numeric@ts@info{\forest@root}%
  \forestset{process keylist=given options}%
  \forestset{stages}%
  \pgfkeysalso{/forest/end forest}%
  \ifforest@was@tikzexternalwasenable
    \tikzexternalenable
  \fi
}
%    \end{macrocode}
%
% \subsection{Standard node}
% \label{impl:standard-node}
%
% The standard node should be calibrated when entering the forest env: ^^AAAAAAAAAAAAAAAAAAAAAAAA
% ^^A|\forestNodeHandle{standard node}.calibrate()|.  What the calibration does is defined in a call to
% ^^A|\forestStandardNode|.
% The standard node init does \emph{not} initialize options from a(nother) standard node!
%    \begin{macrocode}
\def\forest@standardnode@new{%
  \advance\forest@node@maxid1
  \forest@fornode{\the\forest@node@maxid}{%
    \forest@node@init
    \forest@node@setname{standard node}%
  }%
}
\def\forest@standardnode@calibrate{%
  \forest@fornode{\forest@node@Nametoid{standard node}}{%
    \edef\forest@environment{\forestove{environment@formula}}%
    \forestoget{previous@environment}\forest@previous@environment
    \ifx\forest@environment\forest@previous@environment\else
      \forestolet{previous@environment}\forest@environment
      \forest@node@typeset
      \forestoget{calibration@procedure}\forest@temp
      \expandafter\forestset\expandafter{\forest@temp}%
    \fi
  }%
}
%    \end{macrocode}
% Usage: |\forestStandardNode[#1]{#2}{#3}{#4}|.  |#1| = standard node specification --- specify it
% as any other node content (but without children, of course).  |#2| = the environment fingerprint:
% list the values of parameters that influence the standard node's height and depth; the standard
% will be adjusted whenever any of these parameters changes.  |#3| = the calibration procedure: a
% list of usual forest options which should calculating the values of exported options.  |#4| = a
% comma-separated list of exported options: every newly created node receives the initial values of
% exported options from the standard node.  (The standard node definition is local to the \TeX\
% group.)
%    \begin{macrocode}
\def\forestStandardNode[#1]#2#3#4{%
  \let\forest@standardnode@restoretikzexternal\relax
  \ifdefined\tikzexternaldisable
    \ifx\tikz\tikzexternal@tikz@replacement
      \tikzexternaldisable
      \let\forest@standardnode@restoretikzexternal\tikzexternalenable
    \fi
  \fi
  \forest@standardnode@new
  \forest@fornode{\forest@node@Nametoid{standard node}}{%
    \forestset{content=#1}%
    \forestoset{environment@formula}{#2}%
    \edef\forest@temp{\unexpanded{#3}}%
    \forestolet{calibration@procedure}\forest@temp
    \def\forest@calibration@initializing@code{}%
    \pgfqkeys{/forest/initializing@code}{#4}%
    \forestolet{initializing@code}\forest@calibration@initializing@code
    \forest@standardnode@restoretikzexternal
  }
}
\forestset{initializing@code/.unknown/.code={%
    \eappto\forest@calibration@initializing@code{%
      \noexpand\forestOget{\forest@node@Nametoid{standard node}}{\pgfkeyscurrentname}\noexpand\forest@temp
      \noexpand\forestolet{\pgfkeyscurrentname}\noexpand\forest@temp
    }%
  }
}
%    \end{macrocode}
% This macro is called from a new (non-standard) node's init.
%    \begin{macrocode}
\def\forest@initializefromstandardnode{%
  \forestOve{\forest@node@Nametoid{standard node}}{initializing@code}%
}
%    \end{macrocode}
% Define the default standard node.  Standard content: dj --- in Computer Modern font, d is the
% highest and j the deepest letter (not character!).  Environment fingerprint: the height of the
% strut and the values of inner and outer seps.  Calibration procedure: (i) \keyname{l sep}
% equals the height of the strut plus the value of \keyname{inner ysep}, implementing both font-size and
% inner sep dependency; (ii) The effect of \keyname{l} on the standard node should be the same as the
% effect of \keyname{l sep}, thus, we derive \keyname{l} from \keyname{l sep} by adding
% to the latter the total height of the standard node (plus the double outer sep, one for the parent
% and one for the child).  (iii) s sep is straightforward: a double inner xsep.  Exported options:
% options, calculated in the calibration.  (Tricks: to change the default anchor, set it in |#1| and
% export it; to set a non-forest node option (such as \keyname{draw} or \keyname{blue}) as default, set it
% in |#1| and export the (internal) option \keyname{node options}.)
%    \begin{macrocode}
\forestStandardNode[dj]
  {%
    \forestOve{\forest@node@Nametoid{standard node}}{content},%
    \the\ht\strutbox,\the\pgflinewidth,%
    \pgfkeysvalueof{/pgf/inner ysep},\pgfkeysvalueof{/pgf/outer ysep},%
    \pgfkeysvalueof{/pgf/inner xsep},\pgfkeysvalueof{/pgf/outer xsep}%
  }
  {
    l sep={\the\ht\strutbox+\pgfkeysvalueof{/pgf/inner ysep}},
    l={l_sep()+abs(max_y()-min_y())+2*\pgfkeysvalueof{/pgf/outer ysep}},
    s sep={2*\pgfkeysvalueof{/pgf/inner xsep}}
  }
  {l sep,l,s sep}
%    \end{macrocode}
%
%
% \subsection{\texttt{ls} coordinate system}
% \label{imp:ls-coordinates}
%
%    \begin{macrocode}
\pgfqkeys{/forest/@cs}{%
  name/.code={%
    \edef\forest@cn{\forest@node@Nametoid{#1}}%
    \forest@forestcs@resetxy},
  id/.code={%
    \edef\forest@cn{#1}%
    \forest@forestcs@resetxy},
  go/.code={%
    \forest@go{#1}%
    \forest@forestcs@resetxy},
  anchor/.code={\forest@forestcs@anchor{#1}},
  l/.code={%
    \pgfmathsetlengthmacro\forest@forestcs@l{#1}%
    \forest@forestcs@ls
  },
  s/.code={%
    \pgfmathsetlengthmacro\forest@forestcs@s{#1}%
    \forest@forestcs@ls
  },
  .unknown/.code={%
    \expandafter\pgfutil@in@\expandafter.\expandafter{\pgfkeyscurrentname}%
    \ifpgfutil@in@
      \expandafter\forest@forestcs@namegoanchor\pgfkeyscurrentname\forest@end
    \else
      \expandafter\forest@nameandgo\expandafter{\pgfkeyscurrentname}%
      \forest@forestcs@resetxy
    \fi
  }
}
\def\forest@forestcs@resetxy{%
  \ifnum\forest@cn=0
  \else
    \global\pgf@x\forestove{x}%
    \global\pgf@y\forestove{y}%
  \fi
}
\def\forest@forestcs@ls{%
  \ifdefined\forest@forestcs@l
    \ifdefined\forest@forestcs@s
      {%
        \pgftransformreset
        \pgftransformrotate{\forestove{grow}}%
        \pgfpointtransformed{\pgfpoint{\forest@forestcs@l}{\forest@forestcs@s}}%
      }%
      \global\advance\pgf@x\forestove{x}%
      \global\advance\pgf@y\forestove{y}%
    \fi
  \fi
}
\def\forest@forestcs@anchor#1{%
  \edef\forest@marshal{%
    \noexpand\forest@original@tikz@parse@node\relax
    (\forestove{name}\ifx\relax#1\relax\else.\fi#1)%
  }\forest@marshal
}
\def\forest@forestcs@namegoanchor#1.#2\forest@end{%
  \forest@nameandgo{#1}%
  \forest@forestcs@anchor{#2}%
}
\tikzdeclarecoordinatesystem{forest}{%
  \forest@forthis{%
    \forest@forestcs@resetxy
    \ifdefined\forest@forestcs@l\undef\forest@forestcs@l\fi
    \ifdefined\forest@forestcs@s\undef\forest@forestcs@s\fi
    \pgfqkeys{/forest/@cs}{#1}%
  }%
}
%    \end{macrocode}
%
% \addcontentsline{toc}{section}{References}
% \bibliography{tex}
% \bibliographystyle{plain}
%
% \newpage
% \addcontentsline{toc}{section}{Index}
% \makeatletter\c@IndexColumns=2 \makeatother
% \IndexPrologue{\section*{Index}}
% \PrintIndex
%
% \endinput
% 
% Local Variables: 
% mode: doctex
% fill-column: 100
% LaTeX-command: "latex -shell-escape"
% End: 
