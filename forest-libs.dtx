% \CheckSum{236}
% \iffalse meta-comment
% forest-libs.dtx
%% `forest-libs' is a collection of libraries for package `forest'.
%%
%% Copyright (c) 2012-2017 Saso Zivanovic
%%                    (Sa\v{s}o \v{Z}ivanovi\'{c})
%% saso.zivanovic@guest.arnes.si
%%
%% This work may be distributed and/or modified under the
%% conditions of the LaTeX Project Public License, either version 1.3
%% of this license or (at your option) any later version.
%% The latest version of this license is in
%% 
%%   http://www.latex-project.org/lppl.txt
%%   
%% and version 1.3 or later is part of all distributions of LaTeX
%% version 2005/12/01 or later.
%%
%% This work has the LPPL maintenance status `author-maintained'.
%% 
%% This file is a part of package `forest'. For the list of files
%% constituting the package see main source file of the package,
%% `forest.dtx', or the derived `forest.sty'.
%%
% \fi
%
% \section{Libraries}
%
% This chapter contains not only the reference of commands found in libraries and some examples of
% their usage, but also their definitions. This is done in the hope that these definitions, being
% mostly styles, will be useful as examples of the core features of the package.  I even managed to
% comment them a bit \dots
%
% \paragraph{Disclaimer.}  At least in the initial stages of a library's development, the function
% and interface of macros and keys defined in a library might change without backwards compatibility
% support!  Though I'll try to keep this from happening \dots
%
%    \begin{macrocode}
\RequirePackage{forest}
%    \end{macrocode}
%
%\iffalse
%<*linguistics>
%\fi
% \librarysection{linguistics}
%    \begin{macrocode}
\ProvidesForestLibrary{linguistics}[2017/01/27 v0.1.1]
%    \end{macrocode}
%
% Defaults:
%    \begin{macrocode}
\forestset{
  libraries/linguistics/defaults/.style={
    default preamble={
%    \end{macrocode}
% Edges of the children will ``meet'' under the node:
%    \begin{macrocode}
     sn edges,
%    \end{macrocode}
% The root of the tree will be aligned with the text \dots\ or, more commonly, the example number.
%    \begin{macrocode}
     baseline,
%    \end{macrocode}
% Enable (centered) multi-line nodes.
%    \begin{macrocode}
     for tree={align=center},
    },
  },
}
%    \end{macrocode}
%    
% There's no linguistics without c-command\footnote{The definition of c-command is as follows: a node c-commands its siblings and their descendants.} \dots
% \begin{syntax}
% \indexitem{step>c-commanded} Visit all the nodes c-commanded by the current node.
% \indexitem{step>c-commanders} Visit all the c-commanders of the current node, starting from the closest.
%
% \begin{forestexample}[tree bin=minipage,index={branch',c-commanded}]
%   \begin{forest} 
%     [VP
%       [DP[John]]
%       [V'
%         [V, draw, for c-commanded={draw,circle}
%           [sent]
%         ]
%         [DP[Mary]]
%         [DP[D[a]][NP[letter]]]
%       ]
%     ]
%   \end{forest}
% \end{forestexample}
%
% See how \indexex{branch'} is used to define |c-commanded|, and how \indexex{while nodewalk valid} and \indexex{nodewalk key>fake} are combined in the definition of |c-commanders|.
%    \begin{macrocode}
\forestset{
  define long step={c-commanded}{style}{branch'={siblings,descendants}},
  define long step={c-commanders}{style}{while nodewalk valid={parent}{siblings,fake=parent}},
}
%    \end{macrocode}
%
% |c-commanders| could also be defined using \indexex{branch}:
% \begin{lstlisting}
%   branch={current and ancestors, siblings}
% \end{lstlisting}
%
% \indexitem{node key>sn edges}
% 
% In linguistics, most people want the parent-child edge to go from the \emph{s}outh of the parent
% to the \emph{n}orth of the child.  This is achieved by this (badly named) style, which makes the
% entire (sub)tree have such edges.
%
% \begin{forestexample}
%   \begin{forest}
%     sn edges
%     [VP
%       [DP]
%       [V'
%         [V]
%         [DP]
%       ]
%     ]
%   \end{forest}
% \end{forestexample}
% 
%    \begin{macrocode}
\forestset{
  sn edges/.style={
    for tree={
      parent anchor=children, child anchor=parent
    }
  },
}
%    \end{macrocode}
%
% A note on implementation. Despite its name, this style does not refer to the |south| and |north|
% anchor of the parent and the child node directly.  If it did so, it would only work for trees with
% standard linguistic \index{grow}|=-90|.  So we rather use \FoRest;'s growth direction based
% anchors: \index{anchor>children} always faces the children and \index{anchor>parent} always faces
% the parent, so the edge will always be between them, and the normal, upward growing trees will
% look good as well.
%
% \begin{forestexample}[index={anchor>south,anchor>north,option>parent anchor,option>child anchor}]
%   \begin{forest}
%     [bad![VP,no edge, for tree={grow=90, edge=red},
%       for tree={parent anchor=south, child anchor=north} % bad
%     [DP][V'[V][DP]]]]
%   \end{forest}
%   \begin{forest}
%     [good![VP, no edge, for tree={grow=90, edge=green},
%                ~sn edges~                                 % good!
%     [DP][V'[V][DP]]]]
%   \end{forest}
% \end{forestexample}
%
% \indexitem{node key>roof}   Makes the edge to parent a triangular roof. 
%    \begin{macrocode}
\forestset{
  roof/.style={edge path'={%
      (.parent first)--(!u.children)--(.parent last)--cycle
    }
  },
}
%    \end{macrocode}
%
% \indexitem{node key>nice empty nodes}
%
% We often need empty nodes: tree (\ref{ex:niceemptynodes}a) shows how they look like by default:
% ugly.
%
% First, we don't want the gaps: we change the shape of empty nodes to coordinate.  We get tree
% (\ref{ex:niceemptynodes}b).
%
% Second, the empty nodes seem too close
% to the other (especially empty) nodes (this is a result of a small
% default |s_sep|).  We could use a greater \index{s sep}, but a better solution seems
% to be to use |calign=fixed_angles|.  The result is shown in (\ref{ex:niceemptynodes}c). 
%
% However, at the transitions from empty to non-empty nodes, tree (\ref{ex:niceemptynodes}c)
% seems to zigzag (although the base points of the spine nodes
% are perfectly in line), and the edge to the empty node left to VP
% seems too long (it reaches to the level of VP's base, while we'd
% prefer it to stop at the same level as the edge to VP itself).  The
% first problem is solved by substituting |fixed_angles| for
% |fixed_edge_angles|; the second one, by anchoring siblings of
% empty nodes at north.  Voilà, (\ref{ex:niceemptynodes}d)!
% 
% \begin{forestexample}[label=ex:niceemptynodes,layout=tree on bottom,index={fixed angles,fixed edge angles,calign,tree,delay,where option,content,for step,step>parent,step>children,option>anchor}]
%   \forestset{
%     xlist/.style={
%       phantom,
%       for children={no edge,replace by={[,append,
%           delay={content/.wrap pgfmath arg={\csname @alph\endcsname{##1}.}{n()+#1}}
%           ]}}
%     },
%     xlist/.default=0
%   }
%   \begin{forest}
%     [,~xlist~,
%     for tree={after packing node={s+=0.1pt}}, % hack!!!
%       [CP,                                                               %(a)
%         [][[][[][VP[DP[John]][V'[V[loves]][DP[Mary]]]]]]]
%       [CP, delay={where content={}{shape=coordinate}{}}                  %(b)
%         [][[][[][VP[DP[John]][V'[V[loves]][DP[Mary]]]]]]]
%       [CP, for tree={calign=fixed angles},                               %(c)
%            delay={where content={}{shape=coordinate}{}}
%         [][[][[][VP[DP[John]][V'[V[loves]][DP[Mary]]]]]]]
%       [CP, ~nice empty nodes~                                              %(d)
%         [][[][[][VP[DP[John]][V'[V[loves]][DP[Mary]]]]]]]
%     ]
%   \end{forest}
% \end{forestexample}
%
%    \begin{macrocode}
\forestset{
  nice empty nodes/.style={
    for tree={calign=fixed edge angles},
    delay={where content={}{shape=coordinate,
                            for current and siblings={anchor=north}}{}}
  },
}
%    \end{macrocode}
%    
%   \indexitem{node key>draw brackets}   Outputs the bracket representation of the tree.
%   \indexitem{node key>draw brackets compact}
%   \itemnosep
%   \indexitem{node key>draw brackets wide} These keys control whether the brackets have extra
%   spaces around them (|wide|) or not (|compact|).
%    \begin{macrocode}
\providecommand\text[1]{\mbox{\scriptsize#1}}
\forestset{
  draw brackets compact/.code={\let\drawbracketsspace\relax},
  draw brackets wide/.code={\let\drawbracketsspace\space},
  draw brackets/.style={
%    \end{macrocode}
% There's stuff to do both before (output the opening bracket and the content) and after (output the
% closing bracket) processing the children, so we use \indexex{for tree'}.
%    \begin{macrocode}
    for tree'={
      TeX={[%
%    \end{macrocode}
% Complication: \index{content format} must be expanded in advance, to correctly process tabular environments implicitely loaded by \index{align}|=|\index{value of=align>center}, which is the default in this library.  (Not that one would want a multiline output in the bracket representation, but it's better than crashing.)
%    \begin{macrocode}
        \edef\forestdrawbracketscontentformat{\foresteoption{content format}}%
      },
      if n children=0{
        TeX={\drawbracketsspace\forestdrawbracketscontentformat\drawbracketsspace}
      }{
        TeX={\textsubscript{\text{\forestdrawbracketscontentformat}}\drawbracketsspace}
      },
    }{
      TeX={]\drawbracketsspace},
    }
  },
  draw brackets wide
}
%    \end{macrocode}
% \end{syntax}
% 
% \subsubsection{GP1}
%
% \begin{syntax}
%   \indexitem{node key>GP1}
%   
% For Government Phonology (v1) representations. Here, the big trick
% is to evenly space $\times$s by having a large enough |outer_xsep|
% (adjustable), and then, before drawing (timing control option
% |before_drawing_tree|), setting |outer_xsep| back to 0pt.  The last step
% is important, otherwise the arrows between $\times$s won't draw!
%
% An example of an ``embedded'' |GP1| style:
% \begin{forestexample}[layout=tree on bottom,index={where option,tier,for step,step>children,content,tikz,option>l,dimen+,no edge},index>={!}]
%   \begin{forest}
%     myGP1/.style={
%       ~GP1~,
%       delay={where tier={x}{
%           for children={content=\textipa{##1}}}{}},
%       tikz={\draw[dotted](.south)--
%             (!1.north west)--(!l.north east)--cycle;},
%       for children={l+=5mm,no edge}
%     }
%     [VP[DP[John,tier=word,myGP1
%             [O[x[dZ]]]
%             [R[N[x[6]]]]
%             [O[x[n]]]
%             [R[N[x]]]
%     ]][V'[V[loves,tier=word,myGP1
%              [O[x[l]]]
%              [R[N[x[a]]]]
%              [O[x[v]]]
%              [R[N[x]]]
%              [O[x[z]]]
%              [R[N[x]]]
%     ]][DP[Mary,tier=word,myGP1
%            [O[x[m]]]
%            [R[N[x[e]]]]
%            [O[x[r]]]
%            [R[N[x[i]]]]
%     ]]]]
%   \end{forest}%
% \end{forestexample}
%
% And an example of annotations.
% \begin{forestexample}
%   \begin{forest}[,phantom,s sep=1cm
%     [{[ei]}, GP1
%       [R[N[x[A,~el~[I,~head~,~associate=N~]]][x]]]
%     ]
%     [{[mars]}, GP1
%       [O[x[m]]]
%       [R[N[x[a]]][x,~encircle~,densely dotted[r]]]
%       [O[x,~encircle~,~govern=<~[s]]]
%       [R,~fen~[N[x]]]
%     ]
%   ]\end{forest} 
% \end{forestexample}
%
%    \begin{macrocode}
\newbox\standardnodestrutbox
\setbox\standardnodestrutbox=\hbox to 0pt{\phantom{\forestOve{standard node}{content}}}
\def\standardnodestrut{\copy\standardnodestrutbox}
\forestset{
  GP1/.style 2 args={
    for n={1}{baseline},
    s sep=0pt, l sep=0pt,
    for descendants={
      l sep=0pt, l={#1},
      anchor=base,calign=first,child anchor=north,
      inner xsep=1pt,inner ysep=2pt,outer sep=0pt,s sep=0pt,
    },
    delay={
      if content={}{phantom}{for children={no edge}},
      for tree={
        if content={O}{tier=OR}{},
        if content={R}{tier=OR}{},
        if content={N}{tier=N}{},
        if content={x}{
          tier=x,content={$\times$},outer xsep={#2},
          for tree={calign=center},
          for descendants={content format={\noexpand\standardnodestrut\forestoption{content}}},
          before drawing tree={outer xsep=0pt,delay={typeset node}},
          s sep=4pt
        }{},
      },
    },
    before drawing tree={where content={}{parent anchor=center,child anchor=center}{}},
  },
  GP1/.default={5ex}{8.0pt},
  associate/.style={%
    tikz+={\draw[densely dotted](!)--(!#1);}},
  spread/.style={
    before drawing tree={tikz+={\draw[dotted](!)--(!#1);}}},
  govern/.style={
    before drawing tree={tikz+={\draw[->](!)--(!#1);}}},
  p-govern/.style={
    before drawing tree={tikz+={\draw[->](.north) to[out=150,in=30] (!#1.north);}}},
  no p-govern/.style={
    before drawing tree={tikz+={\draw[->,loosely dashed](.north) to[out=150,in=30] (!#1.north);}}},
  encircle/.style={before drawing tree={circle,draw,inner sep=0pt}},
  fen/.style={pin={[font=\footnotesize,inner sep=1pt,pin edge=<-]10:\textsc{Fen}}},
  el/.style={content=\textsc{\textbf{##1}}},
  head/.style={content=\textsc{\textbf{\underline{##1}}}}
}
%    \end{macrocode}
% \end{syntax}
%\iffalse
%</linguistics>
%\fi
%
%
%
%\iffalse
%<*edges>
%\fi
% \librarysection{edges}
%    \begin{macrocode}
\ProvidesForestLibrary{edges}[2016/12/05 v0.1.1]
%    \end{macrocode}
% 
% \begin{syntax}
% 
% \indexitem{node key>forked edge'}
%
%   Sets a forked edge to the current node.  Arbitrary growth direction and node rotation are
%   supported.
%
% \indexitem{node key>forked edge}
%
% Like \index{forked edge'}, but it also sets \index{option>parent anchor} and \index{option>child
% anchor} to the likely values of \index{anchor>children} and \index{anchor>parent}, respectively.
%
% \indexitem(tree){node key>forked edges}|=|\meta{nodewalk}
%
% Invokes \index{forked edge} for all nodes in the \meta{nodewalk}, by default the entire (sub)tree
% rooted in the current node.
%
% \indexitem{option>fork sep} The \index{forest cs>l}-distance between the parent anchor and the
% fork.
%   
% \end{syntax}
% 
% \begin{forestexample}[index={for step,tree,grow',forked edges}]
%   \begin{forest}
%     for tree={grow'=0,draw},
%     forked edges,
%     [/
%       [home
%         [saso
%           [Download]
%           [TeX]
%         ]
%         [alja]
%         [joe]
%       ]
%       [usr
%         [bin]
%         [share]
%       ]
%     ]
%   \end{forest}  
% \end{forestexample}
%
% See how growth direction based anchors \indexex{anchor>children} and \indexex{anchor>parent} are
% used in the definition below to easily take care of arbitrary \index{grow} and
% \index{rotate}.
%    \begin{macrocode}
\forestset{
  declare dimen={fork sep}{0.5em},
  forked edge'/.style={
    edge={rotate/.option=!parent.grow},
    edge path'={(!u.parent anchor) -- ++(\forestoption{fork sep},0) |- (.child anchor)},
  },
  forked edge/.style={
    on invalid={fake}{!parent.parent anchor=children},
    child anchor=parent,
    forked edge',
  },
  forked edges/.style={for nodewalk={#1}{forked edge}},
  forked edges/.default=tree,
}
%    \end{macrocode}
%
% \begin{syntax}
%   \indexitem{node key>folder} The children of the node are drawn like folders.
%
%   All growth directions are supported (well, cardinal directions work perfectly; the others await
%   the sensitivity of packing to \index{edge path}), as well as node rotation and \index{reversed}
%   order of children.
%
%   The outlook of the folder can be influenced by setting standard \foRest;'s options \index{l sep}
%   and \index{s sep} any time before packing, or \index{option>l} and \index{option>s} after
%   packing.  Setting \index{option>l} and \index{option>s} before packing will have no influence on
%   the layout of the tree.
%
%   \begin{syntax}
%     \indexitem(.45em){register>folder indent}|=|\meta{dimen}
%
%     Specifies the shift of the parent's side of the edge in the \index{forest cs>l}-direction.
%   \end{syntax}
%
% \end{syntax}
% 
% \begin{forestexample}[index={for step,tree,grow',folder}]
%   \begin{forest}
%     for tree={grow'=0,~folder~,draw}
%     [/
%       [home
%         [saso
%           [Download]
%           [TeX]
%         ]
%         [alja]
%         [joe]
%       ]
%       [usr
%         [bin]
%         [share]
%       ]
%     ]
%   \end{forest}
% \end{forestexample}
%    \begin{macrocode}
\forestset{
  declare dimen register=folder indent,
  folder indent=.45em,
  folder/.style={
    parent anchor=-children last,
    anchor=parent first,
    calign=child,
    calign primary child=1,
    for children={
      child anchor=parent,
      anchor=parent first,
      edge={rotate/.option=!parent.grow},
      edge path'/.expanded={
        ([xshift=\forestregister{folder indent}]!u.parent anchor) |- (.child anchor)
      },
    },
    after packing node={
      if n children=0{}{
        tempdiml=l_sep()-l("!1"),
        tempdims={-abs(max_s("","")-min_s("",""))-s_sep()},
        for children={
          l+=tempdiml,
          s+=tempdims()*(reversed()-0.5)*2,
        },
      },
    },
  }
}
%    \end{macrocode}
%\iffalse
%</edges>
%\fi
% \forestset{every index end/.style={}}
% \endinput
% Local Variables:
% mode: doctex
% TeX-command-default: "sty"
% fill-column: 100
% TeX-master: "forest"
% End:
